\documentclass[11pt]{article}
\usepackage{fontspec}
\usepackage{longtable, array}
\usepackage{hyperref}

\usepackage[
	a4paper,
	left=1.2cm,
	right=1.2cm,
	top=1.5cm,
	bottom=1cm,
	nohead
]{geometry}
\pagenumbering{gobble}

\hypersetup{
	colorlinks=true,
	linkcolor=cyan,
	filecolor=blue,      
	urlcolor=red,
	citecolor=green,
}

\usepackage{material-symbols}

\title{Material Symbols}
\date{August 2024}
\author{Exception0x0194}

\begin{document}

\maketitle

\section{Introduction}

The \texttt{material-symbols} package provides a straightforward way to incorporate \href{https://fonts.google.com/icons}{Google's Material Symbols} into \LaTeX\ documents. This package offers access to a broad array of icons in various styles, enhancing the visual elements of your documentation.

\section{Usage}

Access symbols via a standardized command: \texttt{\textbackslash mSymbol\{name\}}, where \texttt{name} refers to the symbol's name in \href{https://fonts.google.com/icons}{Google Fonts}, formatted in \texttt{kebab-case}.

For example, \texttt{\textbackslash mSymbol\{sports-soccer\}} renders \mSymbol{sports-soccer}.

These commands can be employed anywhere in your \LaTeX\ documents to insert icons at desired locations.

\subsection{Styles and Fillings}

Material Symbols are available in three styles: \textbf{Outlined}, \textbf{Round}, and \textbf{Sharp}; additionally, a \textbf{Filled} option is available for all symbols, providing a visually distinct and more pronounced appearance. The default style is \textbf{Outlined}, but you can specify an alternative style using an optional argument:

\begin{itemize}
    \item \texttt{\textbackslash mSymbol[outlined]\{star-rate\}} (or \texttt{\textbackslash mSymbol\{star-rate\}}) displays the icon's outlined version: \mSymbol{star-rate}.
    \item \texttt{\textbackslash mSymbol[rounded]\{star-rate\}} displays the icon's rounded version: \mSymbol[rounded]{star-rate}.
    \item \texttt{\textbackslash mSymbol[sharp]\{star-rate\}} displays the icon's sharp version: \mSymbol[sharp]{star-rate}.
\end{itemize}

For more nuanced customization, filled variants can also be combined with the outline, round, and sharp styles as follows:

\begin{itemize}
    \item \texttt{\textbackslash mSymbol[outlined-filled]\{star-rate\}} displays the icon's outlined filled version: \mSymbol[outlined-filled]{star-rate}.
    \item \texttt{\textbackslash mSymbol[rounded-filled]\{star-rate\}} displays the icon's rounded filled version: \mSymbol[rounded-filled]{star-rate}.
    \item \texttt{\textbackslash mSymbol[sharp-filled]\{star-rate\}} displays the icon's sharp filled version: \mSymbol[sharp-filled]{star-rate}.
\end{itemize}

These commands allow for a high degree of flexibility in document design, enabling users to choose the icon style that best suits the visual context of their documents.

\section{List of Icons}

Below is a table listing all available symbols in this package, complete with previews of the icons in their different styles, the corresponding commands, and their codepoints. For a more user-friendly search, visit \href{https://fonts.google.com/icons}{Google Fonts}.

\begin{longtable}{
p{0.1\textwidth}
p{0.1\textwidth}
p{0.1\textwidth}
>{\raggedright\arraybackslash}p{0.5\textwidth}
>{\raggedright\arraybackslash}p{0.2\textwidth}
}
\textbf{Outlined} & \textbf{Rounded} &\textbf{Sharp} & \textbf{Command} & \textbf{Codepoint} \\
\endfirsthead
\textbf{Outlined} & \textbf{Rounded} &\textbf{Sharp} & \textbf{Command} & \textbf{Codepoint} \\
\endhead
\mSymbol[outlined]{10k} & \mSymbol[rounded]{10k} & \mSymbol[sharp]{10k} & \texttt{\textbackslash mSymbol\{10k\}} & \texttt{E951}\\
\mSymbol[outlined]{10mp} & \mSymbol[rounded]{10mp} & \mSymbol[sharp]{10mp} & \texttt{\textbackslash mSymbol\{10mp\}} & \texttt{E952}\\
\mSymbol[outlined]{11mp} & \mSymbol[rounded]{11mp} & \mSymbol[sharp]{11mp} & \texttt{\textbackslash mSymbol\{11mp\}} & \texttt{E953}\\
\mSymbol[outlined]{123} & \mSymbol[rounded]{123} & \mSymbol[sharp]{123} & \texttt{\textbackslash mSymbol\{123\}} & \texttt{EB8D}\\
\mSymbol[outlined]{12mp} & \mSymbol[rounded]{12mp} & \mSymbol[sharp]{12mp} & \texttt{\textbackslash mSymbol\{12mp\}} & \texttt{E954}\\
\mSymbol[outlined]{13mp} & \mSymbol[rounded]{13mp} & \mSymbol[sharp]{13mp} & \texttt{\textbackslash mSymbol\{13mp\}} & \texttt{E955}\\
\mSymbol[outlined]{14mp} & \mSymbol[rounded]{14mp} & \mSymbol[sharp]{14mp} & \texttt{\textbackslash mSymbol\{14mp\}} & \texttt{E956}\\
\mSymbol[outlined]{15mp} & \mSymbol[rounded]{15mp} & \mSymbol[sharp]{15mp} & \texttt{\textbackslash mSymbol\{15mp\}} & \texttt{E957}\\
\mSymbol[outlined]{16mp} & \mSymbol[rounded]{16mp} & \mSymbol[sharp]{16mp} & \texttt{\textbackslash mSymbol\{16mp\}} & \texttt{E958}\\
\mSymbol[outlined]{17mp} & \mSymbol[rounded]{17mp} & \mSymbol[sharp]{17mp} & \texttt{\textbackslash mSymbol\{17mp\}} & \texttt{E959}\\
\mSymbol[outlined]{18-up-rating} & \mSymbol[rounded]{18-up-rating} & \mSymbol[sharp]{18-up-rating} & \texttt{\textbackslash mSymbol\{18-up-rating\}} & \texttt{F8FD}\\
\mSymbol[outlined]{18mp} & \mSymbol[rounded]{18mp} & \mSymbol[sharp]{18mp} & \texttt{\textbackslash mSymbol\{18mp\}} & \texttt{E95A}\\
\mSymbol[outlined]{19mp} & \mSymbol[rounded]{19mp} & \mSymbol[sharp]{19mp} & \texttt{\textbackslash mSymbol\{19mp\}} & \texttt{E95B}\\
\mSymbol[outlined]{1k} & \mSymbol[rounded]{1k} & \mSymbol[sharp]{1k} & \texttt{\textbackslash mSymbol\{1k\}} & \texttt{E95C}\\
\mSymbol[outlined]{1k-plus} & \mSymbol[rounded]{1k-plus} & \mSymbol[sharp]{1k-plus} & \texttt{\textbackslash mSymbol\{1k-plus\}} & \texttt{E95D}\\
\mSymbol[outlined]{1x-mobiledata} & \mSymbol[rounded]{1x-mobiledata} & \mSymbol[sharp]{1x-mobiledata} & \texttt{\textbackslash mSymbol\{1x-mobiledata\}} & \texttt{EFCD}\\
\mSymbol[outlined]{1x-mobiledata-badge} & \mSymbol[rounded]{1x-mobiledata-badge} & \mSymbol[sharp]{1x-mobiledata-badge} & \texttt{\textbackslash mSymbol\{1x-mobiledata-badge\}} & \texttt{F7F1}\\
\mSymbol[outlined]{20mp} & \mSymbol[rounded]{20mp} & \mSymbol[sharp]{20mp} & \texttt{\textbackslash mSymbol\{20mp\}} & \texttt{E95E}\\
\mSymbol[outlined]{21mp} & \mSymbol[rounded]{21mp} & \mSymbol[sharp]{21mp} & \texttt{\textbackslash mSymbol\{21mp\}} & \texttt{E95F}\\
\mSymbol[outlined]{22mp} & \mSymbol[rounded]{22mp} & \mSymbol[sharp]{22mp} & \texttt{\textbackslash mSymbol\{22mp\}} & \texttt{E960}\\
\mSymbol[outlined]{23mp} & \mSymbol[rounded]{23mp} & \mSymbol[sharp]{23mp} & \texttt{\textbackslash mSymbol\{23mp\}} & \texttt{E961}\\
\mSymbol[outlined]{24mp} & \mSymbol[rounded]{24mp} & \mSymbol[sharp]{24mp} & \texttt{\textbackslash mSymbol\{24mp\}} & \texttt{E962}\\
\mSymbol[outlined]{2d} & \mSymbol[rounded]{2d} & \mSymbol[sharp]{2d} & \texttt{\textbackslash mSymbol\{2d\}} & \texttt{EF37}\\
\mSymbol[outlined]{2k} & \mSymbol[rounded]{2k} & \mSymbol[sharp]{2k} & \texttt{\textbackslash mSymbol\{2k\}} & \texttt{E963}\\
\mSymbol[outlined]{2k-plus} & \mSymbol[rounded]{2k-plus} & \mSymbol[sharp]{2k-plus} & \texttt{\textbackslash mSymbol\{2k-plus\}} & \texttt{E964}\\
\mSymbol[outlined]{2mp} & \mSymbol[rounded]{2mp} & \mSymbol[sharp]{2mp} & \texttt{\textbackslash mSymbol\{2mp\}} & \texttt{E965}\\
\mSymbol[outlined]{30fps} & \mSymbol[rounded]{30fps} & \mSymbol[sharp]{30fps} & \texttt{\textbackslash mSymbol\{30fps\}} & \texttt{EFCE}\\
\mSymbol[outlined]{30fps-select} & \mSymbol[rounded]{30fps-select} & \mSymbol[sharp]{30fps-select} & \texttt{\textbackslash mSymbol\{30fps-select\}} & \texttt{EFCF}\\
\mSymbol[outlined]{360} & \mSymbol[rounded]{360} & \mSymbol[sharp]{360} & \texttt{\textbackslash mSymbol\{360\}} & \texttt{E577}\\
\mSymbol[outlined]{3d-rotation} & \mSymbol[rounded]{3d-rotation} & \mSymbol[sharp]{3d-rotation} & \texttt{\textbackslash mSymbol\{3d-rotation\}} & \texttt{E84D}\\
\mSymbol[outlined]{3g-mobiledata} & \mSymbol[rounded]{3g-mobiledata} & \mSymbol[sharp]{3g-mobiledata} & \texttt{\textbackslash mSymbol\{3g-mobiledata\}} & \texttt{EFD0}\\
\mSymbol[outlined]{3g-mobiledata-badge} & \mSymbol[rounded]{3g-mobiledata-badge} & \mSymbol[sharp]{3g-mobiledata-badge} & \texttt{\textbackslash mSymbol\{3g-mobiledata-badge\}} & \texttt{F7F0}\\
\mSymbol[outlined]{3k} & \mSymbol[rounded]{3k} & \mSymbol[sharp]{3k} & \texttt{\textbackslash mSymbol\{3k\}} & \texttt{E966}\\
\mSymbol[outlined]{3k-plus} & \mSymbol[rounded]{3k-plus} & \mSymbol[sharp]{3k-plus} & \texttt{\textbackslash mSymbol\{3k-plus\}} & \texttt{E967}\\
\mSymbol[outlined]{3mp} & \mSymbol[rounded]{3mp} & \mSymbol[sharp]{3mp} & \texttt{\textbackslash mSymbol\{3mp\}} & \texttt{E968}\\
\mSymbol[outlined]{3p} & \mSymbol[rounded]{3p} & \mSymbol[sharp]{3p} & \texttt{\textbackslash mSymbol\{3p\}} & \texttt{EFD1}\\
\mSymbol[outlined]{4g-mobiledata} & \mSymbol[rounded]{4g-mobiledata} & \mSymbol[sharp]{4g-mobiledata} & \texttt{\textbackslash mSymbol\{4g-mobiledata\}} & \texttt{EFD2}\\
\mSymbol[outlined]{4g-mobiledata-badge} & \mSymbol[rounded]{4g-mobiledata-badge} & \mSymbol[sharp]{4g-mobiledata-badge} & \texttt{\textbackslash mSymbol\{4g-mobiledata-badge\}} & \texttt{F7EF}\\
\mSymbol[outlined]{4g-plus-mobiledata} & \mSymbol[rounded]{4g-plus-mobiledata} & \mSymbol[sharp]{4g-plus-mobiledata} & \texttt{\textbackslash mSymbol\{4g-plus-mobiledata\}} & \texttt{EFD3}\\
\mSymbol[outlined]{4k} & \mSymbol[rounded]{4k} & \mSymbol[sharp]{4k} & \texttt{\textbackslash mSymbol\{4k\}} & \texttt{E072}\\
\mSymbol[outlined]{4k-plus} & \mSymbol[rounded]{4k-plus} & \mSymbol[sharp]{4k-plus} & \texttt{\textbackslash mSymbol\{4k-plus\}} & \texttt{E969}\\
\mSymbol[outlined]{4mp} & \mSymbol[rounded]{4mp} & \mSymbol[sharp]{4mp} & \texttt{\textbackslash mSymbol\{4mp\}} & \texttt{E96A}\\
\mSymbol[outlined]{50mp} & \mSymbol[rounded]{50mp} & \mSymbol[sharp]{50mp} & \texttt{\textbackslash mSymbol\{50mp\}} & \texttt{F6F3}\\
\mSymbol[outlined]{5g} & \mSymbol[rounded]{5g} & \mSymbol[sharp]{5g} & \texttt{\textbackslash mSymbol\{5g\}} & \texttt{EF38}\\
\mSymbol[outlined]{5g-mobiledata-badge} & \mSymbol[rounded]{5g-mobiledata-badge} & \mSymbol[sharp]{5g-mobiledata-badge} & \texttt{\textbackslash mSymbol\{5g-mobiledata-badge\}} & \texttt{F7EE}\\
\mSymbol[outlined]{5k} & \mSymbol[rounded]{5k} & \mSymbol[sharp]{5k} & \texttt{\textbackslash mSymbol\{5k\}} & \texttt{E96B}\\
\mSymbol[outlined]{5k-plus} & \mSymbol[rounded]{5k-plus} & \mSymbol[sharp]{5k-plus} & \texttt{\textbackslash mSymbol\{5k-plus\}} & \texttt{E96C}\\
\mSymbol[outlined]{5mp} & \mSymbol[rounded]{5mp} & \mSymbol[sharp]{5mp} & \texttt{\textbackslash mSymbol\{5mp\}} & \texttt{E96D}\\
\mSymbol[outlined]{60fps} & \mSymbol[rounded]{60fps} & \mSymbol[sharp]{60fps} & \texttt{\textbackslash mSymbol\{60fps\}} & \texttt{EFD4}\\
\mSymbol[outlined]{60fps-select} & \mSymbol[rounded]{60fps-select} & \mSymbol[sharp]{60fps-select} & \texttt{\textbackslash mSymbol\{60fps-select\}} & \texttt{EFD5}\\
\mSymbol[outlined]{6-ft-apart} & \mSymbol[rounded]{6-ft-apart} & \mSymbol[sharp]{6-ft-apart} & \texttt{\textbackslash mSymbol\{6-ft-apart\}} & \texttt{F21E}\\
\mSymbol[outlined]{6k} & \mSymbol[rounded]{6k} & \mSymbol[sharp]{6k} & \texttt{\textbackslash mSymbol\{6k\}} & \texttt{E96E}\\
\mSymbol[outlined]{6k-plus} & \mSymbol[rounded]{6k-plus} & \mSymbol[sharp]{6k-plus} & \texttt{\textbackslash mSymbol\{6k-plus\}} & \texttt{E96F}\\
\mSymbol[outlined]{6mp} & \mSymbol[rounded]{6mp} & \mSymbol[sharp]{6mp} & \texttt{\textbackslash mSymbol\{6mp\}} & \texttt{E970}\\
\mSymbol[outlined]{7k} & \mSymbol[rounded]{7k} & \mSymbol[sharp]{7k} & \texttt{\textbackslash mSymbol\{7k\}} & \texttt{E971}\\
\mSymbol[outlined]{7k-plus} & \mSymbol[rounded]{7k-plus} & \mSymbol[sharp]{7k-plus} & \texttt{\textbackslash mSymbol\{7k-plus\}} & \texttt{E972}\\
\mSymbol[outlined]{7mp} & \mSymbol[rounded]{7mp} & \mSymbol[sharp]{7mp} & \texttt{\textbackslash mSymbol\{7mp\}} & \texttt{E973}\\
\mSymbol[outlined]{8k} & \mSymbol[rounded]{8k} & \mSymbol[sharp]{8k} & \texttt{\textbackslash mSymbol\{8k\}} & \texttt{E974}\\
\mSymbol[outlined]{8k-plus} & \mSymbol[rounded]{8k-plus} & \mSymbol[sharp]{8k-plus} & \texttt{\textbackslash mSymbol\{8k-plus\}} & \texttt{E975}\\
\mSymbol[outlined]{8mp} & \mSymbol[rounded]{8mp} & \mSymbol[sharp]{8mp} & \texttt{\textbackslash mSymbol\{8mp\}} & \texttt{E976}\\
\mSymbol[outlined]{9k} & \mSymbol[rounded]{9k} & \mSymbol[sharp]{9k} & \texttt{\textbackslash mSymbol\{9k\}} & \texttt{E977}\\
\mSymbol[outlined]{9k-plus} & \mSymbol[rounded]{9k-plus} & \mSymbol[sharp]{9k-plus} & \texttt{\textbackslash mSymbol\{9k-plus\}} & \texttt{E978}\\
\mSymbol[outlined]{9mp} & \mSymbol[rounded]{9mp} & \mSymbol[sharp]{9mp} & \texttt{\textbackslash mSymbol\{9mp\}} & \texttt{E979}\\
\mSymbol[outlined]{abc} & \mSymbol[rounded]{abc} & \mSymbol[sharp]{abc} & \texttt{\textbackslash mSymbol\{abc\}} & \texttt{EB94}\\
\mSymbol[outlined]{ac-unit} & \mSymbol[rounded]{ac-unit} & \mSymbol[sharp]{ac-unit} & \texttt{\textbackslash mSymbol\{ac-unit\}} & \texttt{EB3B}\\
\mSymbol[outlined]{access-alarm} & \mSymbol[rounded]{access-alarm} & \mSymbol[sharp]{access-alarm} & \texttt{\textbackslash mSymbol\{access-alarm\}} & \texttt{E855}\\
\mSymbol[outlined]{access-alarms} & \mSymbol[rounded]{access-alarms} & \mSymbol[sharp]{access-alarms} & \texttt{\textbackslash mSymbol\{access-alarms\}} & \texttt{E855}\\
\mSymbol[outlined]{access-time} & \mSymbol[rounded]{access-time} & \mSymbol[sharp]{access-time} & \texttt{\textbackslash mSymbol\{access-time\}} & \texttt{EFD6}\\
\mSymbol[outlined]{access-time-filled} & \mSymbol[rounded]{access-time-filled} & \mSymbol[sharp]{access-time-filled} & \texttt{\textbackslash mSymbol\{access-time-filled\}} & \texttt{EFD6}\\
\mSymbol[outlined]{accessibility} & \mSymbol[rounded]{accessibility} & \mSymbol[sharp]{accessibility} & \texttt{\textbackslash mSymbol\{accessibility\}} & \texttt{E84E}\\
\mSymbol[outlined]{accessibility-new} & \mSymbol[rounded]{accessibility-new} & \mSymbol[sharp]{accessibility-new} & \texttt{\textbackslash mSymbol\{accessibility-new\}} & \texttt{E92C}\\
\mSymbol[outlined]{accessible} & \mSymbol[rounded]{accessible} & \mSymbol[sharp]{accessible} & \texttt{\textbackslash mSymbol\{accessible\}} & \texttt{E914}\\
\mSymbol[outlined]{accessible-forward} & \mSymbol[rounded]{accessible-forward} & \mSymbol[sharp]{accessible-forward} & \texttt{\textbackslash mSymbol\{accessible-forward\}} & \texttt{E934}\\
\mSymbol[outlined]{account-balance} & \mSymbol[rounded]{account-balance} & \mSymbol[sharp]{account-balance} & \texttt{\textbackslash mSymbol\{account-balance\}} & \texttt{E84F}\\
\mSymbol[outlined]{account-balance-wallet} & \mSymbol[rounded]{account-balance-wallet} & \mSymbol[sharp]{account-balance-wallet} & \texttt{\textbackslash mSymbol\{account-balance-wallet\}} & \texttt{E850}\\
\mSymbol[outlined]{account-box} & \mSymbol[rounded]{account-box} & \mSymbol[sharp]{account-box} & \texttt{\textbackslash mSymbol\{account-box\}} & \texttt{E851}\\
\mSymbol[outlined]{account-child} & \mSymbol[rounded]{account-child} & \mSymbol[sharp]{account-child} & \texttt{\textbackslash mSymbol\{account-child\}} & \texttt{E852}\\
\mSymbol[outlined]{account-child-invert} & \mSymbol[rounded]{account-child-invert} & \mSymbol[sharp]{account-child-invert} & \texttt{\textbackslash mSymbol\{account-child-invert\}} & \texttt{E659}\\
\mSymbol[outlined]{account-circle} & \mSymbol[rounded]{account-circle} & \mSymbol[sharp]{account-circle} & \texttt{\textbackslash mSymbol\{account-circle\}} & \texttt{F20B}\\
\mSymbol[outlined]{account-circle-filled} & \mSymbol[rounded]{account-circle-filled} & \mSymbol[sharp]{account-circle-filled} & \texttt{\textbackslash mSymbol\{account-circle-filled\}} & \texttt{F20B}\\
\mSymbol[outlined]{account-circle-off} & \mSymbol[rounded]{account-circle-off} & \mSymbol[sharp]{account-circle-off} & \texttt{\textbackslash mSymbol\{account-circle-off\}} & \texttt{F7B3}\\
\mSymbol[outlined]{account-tree} & \mSymbol[rounded]{account-tree} & \mSymbol[sharp]{account-tree} & \texttt{\textbackslash mSymbol\{account-tree\}} & \texttt{E97A}\\
\mSymbol[outlined]{action-key} & \mSymbol[rounded]{action-key} & \mSymbol[sharp]{action-key} & \texttt{\textbackslash mSymbol\{action-key\}} & \texttt{F502}\\
\mSymbol[outlined]{activity-zone} & \mSymbol[rounded]{activity-zone} & \mSymbol[sharp]{activity-zone} & \texttt{\textbackslash mSymbol\{activity-zone\}} & \texttt{E1E6}\\
\mSymbol[outlined]{acute} & \mSymbol[rounded]{acute} & \mSymbol[sharp]{acute} & \texttt{\textbackslash mSymbol\{acute\}} & \texttt{E4CB}\\
\mSymbol[outlined]{ad} & \mSymbol[rounded]{ad} & \mSymbol[sharp]{ad} & \texttt{\textbackslash mSymbol\{ad\}} & \texttt{E65A}\\
\mSymbol[outlined]{ad-group} & \mSymbol[rounded]{ad-group} & \mSymbol[sharp]{ad-group} & \texttt{\textbackslash mSymbol\{ad-group\}} & \texttt{E65B}\\
\mSymbol[outlined]{ad-group-off} & \mSymbol[rounded]{ad-group-off} & \mSymbol[sharp]{ad-group-off} & \texttt{\textbackslash mSymbol\{ad-group-off\}} & \texttt{EAE5}\\
\mSymbol[outlined]{ad-off} & \mSymbol[rounded]{ad-off} & \mSymbol[sharp]{ad-off} & \texttt{\textbackslash mSymbol\{ad-off\}} & \texttt{F7B2}\\
\mSymbol[outlined]{ad-units} & \mSymbol[rounded]{ad-units} & \mSymbol[sharp]{ad-units} & \texttt{\textbackslash mSymbol\{ad-units\}} & \texttt{EF39}\\
\mSymbol[outlined]{adaptive-audio-mic} & \mSymbol[rounded]{adaptive-audio-mic} & \mSymbol[sharp]{adaptive-audio-mic} & \texttt{\textbackslash mSymbol\{adaptive-audio-mic\}} & \texttt{F4CC}\\
\mSymbol[outlined]{adaptive-audio-mic-off} & \mSymbol[rounded]{adaptive-audio-mic-off} & \mSymbol[sharp]{adaptive-audio-mic-off} & \texttt{\textbackslash mSymbol\{adaptive-audio-mic-off\}} & \texttt{F4CB}\\
\mSymbol[outlined]{adb} & \mSymbol[rounded]{adb} & \mSymbol[sharp]{adb} & \texttt{\textbackslash mSymbol\{adb\}} & \texttt{E60E}\\
\mSymbol[outlined]{add} & \mSymbol[rounded]{add} & \mSymbol[sharp]{add} & \texttt{\textbackslash mSymbol\{add\}} & \texttt{E145}\\
\mSymbol[outlined]{add-a-photo} & \mSymbol[rounded]{add-a-photo} & \mSymbol[sharp]{add-a-photo} & \texttt{\textbackslash mSymbol\{add-a-photo\}} & \texttt{E439}\\
\mSymbol[outlined]{add-ad} & \mSymbol[rounded]{add-ad} & \mSymbol[sharp]{add-ad} & \texttt{\textbackslash mSymbol\{add-ad\}} & \texttt{E72A}\\
\mSymbol[outlined]{add-alarm} & \mSymbol[rounded]{add-alarm} & \mSymbol[sharp]{add-alarm} & \texttt{\textbackslash mSymbol\{add-alarm\}} & \texttt{E856}\\
\mSymbol[outlined]{add-alert} & \mSymbol[rounded]{add-alert} & \mSymbol[sharp]{add-alert} & \texttt{\textbackslash mSymbol\{add-alert\}} & \texttt{E003}\\
\mSymbol[outlined]{add-box} & \mSymbol[rounded]{add-box} & \mSymbol[sharp]{add-box} & \texttt{\textbackslash mSymbol\{add-box\}} & \texttt{E146}\\
\mSymbol[outlined]{add-business} & \mSymbol[rounded]{add-business} & \mSymbol[sharp]{add-business} & \texttt{\textbackslash mSymbol\{add-business\}} & \texttt{E729}\\
\mSymbol[outlined]{add-call} & \mSymbol[rounded]{add-call} & \mSymbol[sharp]{add-call} & \texttt{\textbackslash mSymbol\{add-call\}} & \texttt{F0B7}\\
\mSymbol[outlined]{add-card} & \mSymbol[rounded]{add-card} & \mSymbol[sharp]{add-card} & \texttt{\textbackslash mSymbol\{add-card\}} & \texttt{EB86}\\
\mSymbol[outlined]{add-chart} & \mSymbol[rounded]{add-chart} & \mSymbol[sharp]{add-chart} & \texttt{\textbackslash mSymbol\{add-chart\}} & \texttt{EF3C}\\
\mSymbol[outlined]{add-circle} & \mSymbol[rounded]{add-circle} & \mSymbol[sharp]{add-circle} & \texttt{\textbackslash mSymbol\{add-circle\}} & \texttt{E3BA}\\
\mSymbol[outlined]{add-circle-outline} & \mSymbol[rounded]{add-circle-outline} & \mSymbol[sharp]{add-circle-outline} & \texttt{\textbackslash mSymbol\{add-circle-outline\}} & \texttt{E3BA}\\
\mSymbol[outlined]{add-column-left} & \mSymbol[rounded]{add-column-left} & \mSymbol[sharp]{add-column-left} & \texttt{\textbackslash mSymbol\{add-column-left\}} & \texttt{F425}\\
\mSymbol[outlined]{add-column-right} & \mSymbol[rounded]{add-column-right} & \mSymbol[sharp]{add-column-right} & \texttt{\textbackslash mSymbol\{add-column-right\}} & \texttt{F424}\\
\mSymbol[outlined]{add-comment} & \mSymbol[rounded]{add-comment} & \mSymbol[sharp]{add-comment} & \texttt{\textbackslash mSymbol\{add-comment\}} & \texttt{E266}\\
\mSymbol[outlined]{add-diamond} & \mSymbol[rounded]{add-diamond} & \mSymbol[sharp]{add-diamond} & \texttt{\textbackslash mSymbol\{add-diamond\}} & \texttt{F49C}\\
\mSymbol[outlined]{add-home} & \mSymbol[rounded]{add-home} & \mSymbol[sharp]{add-home} & \texttt{\textbackslash mSymbol\{add-home\}} & \texttt{F8EB}\\
\mSymbol[outlined]{add-home-work} & \mSymbol[rounded]{add-home-work} & \mSymbol[sharp]{add-home-work} & \texttt{\textbackslash mSymbol\{add-home-work\}} & \texttt{F8ED}\\
\mSymbol[outlined]{add-ic-call} & \mSymbol[rounded]{add-ic-call} & \mSymbol[sharp]{add-ic-call} & \texttt{\textbackslash mSymbol\{add-ic-call\}} & \texttt{F0B7}\\
\mSymbol[outlined]{add-link} & \mSymbol[rounded]{add-link} & \mSymbol[sharp]{add-link} & \texttt{\textbackslash mSymbol\{add-link\}} & \texttt{E178}\\
\mSymbol[outlined]{add-location} & \mSymbol[rounded]{add-location} & \mSymbol[sharp]{add-location} & \texttt{\textbackslash mSymbol\{add-location\}} & \texttt{E567}\\
\mSymbol[outlined]{add-location-alt} & \mSymbol[rounded]{add-location-alt} & \mSymbol[sharp]{add-location-alt} & \texttt{\textbackslash mSymbol\{add-location-alt\}} & \texttt{EF3A}\\
\mSymbol[outlined]{add-moderator} & \mSymbol[rounded]{add-moderator} & \mSymbol[sharp]{add-moderator} & \texttt{\textbackslash mSymbol\{add-moderator\}} & \texttt{E97D}\\
\mSymbol[outlined]{add-notes} & \mSymbol[rounded]{add-notes} & \mSymbol[sharp]{add-notes} & \texttt{\textbackslash mSymbol\{add-notes\}} & \texttt{E091}\\
\mSymbol[outlined]{add-photo-alternate} & \mSymbol[rounded]{add-photo-alternate} & \mSymbol[sharp]{add-photo-alternate} & \texttt{\textbackslash mSymbol\{add-photo-alternate\}} & \texttt{E43E}\\
\mSymbol[outlined]{add-reaction} & \mSymbol[rounded]{add-reaction} & \mSymbol[sharp]{add-reaction} & \texttt{\textbackslash mSymbol\{add-reaction\}} & \texttt{E1D3}\\
\mSymbol[outlined]{add-road} & \mSymbol[rounded]{add-road} & \mSymbol[sharp]{add-road} & \texttt{\textbackslash mSymbol\{add-road\}} & \texttt{EF3B}\\
\mSymbol[outlined]{add-row-above} & \mSymbol[rounded]{add-row-above} & \mSymbol[sharp]{add-row-above} & \texttt{\textbackslash mSymbol\{add-row-above\}} & \texttt{F423}\\
\mSymbol[outlined]{add-row-below} & \mSymbol[rounded]{add-row-below} & \mSymbol[sharp]{add-row-below} & \texttt{\textbackslash mSymbol\{add-row-below\}} & \texttt{F422}\\
\mSymbol[outlined]{add-shopping-cart} & \mSymbol[rounded]{add-shopping-cart} & \mSymbol[sharp]{add-shopping-cart} & \texttt{\textbackslash mSymbol\{add-shopping-cart\}} & \texttt{E854}\\
\mSymbol[outlined]{add-task} & \mSymbol[rounded]{add-task} & \mSymbol[sharp]{add-task} & \texttt{\textbackslash mSymbol\{add-task\}} & \texttt{F23A}\\
\mSymbol[outlined]{add-to-drive} & \mSymbol[rounded]{add-to-drive} & \mSymbol[sharp]{add-to-drive} & \texttt{\textbackslash mSymbol\{add-to-drive\}} & \texttt{E65C}\\
\mSymbol[outlined]{add-to-home-screen} & \mSymbol[rounded]{add-to-home-screen} & \mSymbol[sharp]{add-to-home-screen} & \texttt{\textbackslash mSymbol\{add-to-home-screen\}} & \texttt{E1FE}\\
\mSymbol[outlined]{add-to-photos} & \mSymbol[rounded]{add-to-photos} & \mSymbol[sharp]{add-to-photos} & \texttt{\textbackslash mSymbol\{add-to-photos\}} & \texttt{E39D}\\
\mSymbol[outlined]{add-to-queue} & \mSymbol[rounded]{add-to-queue} & \mSymbol[sharp]{add-to-queue} & \texttt{\textbackslash mSymbol\{add-to-queue\}} & \texttt{E05C}\\
\mSymbol[outlined]{add-triangle} & \mSymbol[rounded]{add-triangle} & \mSymbol[sharp]{add-triangle} & \texttt{\textbackslash mSymbol\{add-triangle\}} & \texttt{F48E}\\
\mSymbol[outlined]{addchart} & \mSymbol[rounded]{addchart} & \mSymbol[sharp]{addchart} & \texttt{\textbackslash mSymbol\{addchart\}} & \texttt{EF3C}\\
\mSymbol[outlined]{adf-scanner} & \mSymbol[rounded]{adf-scanner} & \mSymbol[sharp]{adf-scanner} & \texttt{\textbackslash mSymbol\{adf-scanner\}} & \texttt{EADA}\\
\mSymbol[outlined]{adjust} & \mSymbol[rounded]{adjust} & \mSymbol[sharp]{adjust} & \texttt{\textbackslash mSymbol\{adjust\}} & \texttt{E39E}\\
\mSymbol[outlined]{admin-meds} & \mSymbol[rounded]{admin-meds} & \mSymbol[sharp]{admin-meds} & \texttt{\textbackslash mSymbol\{admin-meds\}} & \texttt{E48D}\\
\mSymbol[outlined]{admin-panel-settings} & \mSymbol[rounded]{admin-panel-settings} & \mSymbol[sharp]{admin-panel-settings} & \texttt{\textbackslash mSymbol\{admin-panel-settings\}} & \texttt{EF3D}\\
\mSymbol[outlined]{ads-click} & \mSymbol[rounded]{ads-click} & \mSymbol[sharp]{ads-click} & \texttt{\textbackslash mSymbol\{ads-click\}} & \texttt{E762}\\
\mSymbol[outlined]{agender} & \mSymbol[rounded]{agender} & \mSymbol[sharp]{agender} & \texttt{\textbackslash mSymbol\{agender\}} & \texttt{F888}\\
\mSymbol[outlined]{agriculture} & \mSymbol[rounded]{agriculture} & \mSymbol[sharp]{agriculture} & \texttt{\textbackslash mSymbol\{agriculture\}} & \texttt{EA79}\\
\mSymbol[outlined]{air} & \mSymbol[rounded]{air} & \mSymbol[sharp]{air} & \texttt{\textbackslash mSymbol\{air\}} & \texttt{EFD8}\\
\mSymbol[outlined]{air-freshener} & \mSymbol[rounded]{air-freshener} & \mSymbol[sharp]{air-freshener} & \texttt{\textbackslash mSymbol\{air-freshener\}} & \texttt{E2CA}\\
\mSymbol[outlined]{air-purifier} & \mSymbol[rounded]{air-purifier} & \mSymbol[sharp]{air-purifier} & \texttt{\textbackslash mSymbol\{air-purifier\}} & \texttt{E97E}\\
\mSymbol[outlined]{air-purifier-gen} & \mSymbol[rounded]{air-purifier-gen} & \mSymbol[sharp]{air-purifier-gen} & \texttt{\textbackslash mSymbol\{air-purifier-gen\}} & \texttt{E829}\\
\mSymbol[outlined]{airline-seat-flat} & \mSymbol[rounded]{airline-seat-flat} & \mSymbol[sharp]{airline-seat-flat} & \texttt{\textbackslash mSymbol\{airline-seat-flat\}} & \texttt{E630}\\
\mSymbol[outlined]{airline-seat-flat-angled} & \mSymbol[rounded]{airline-seat-flat-angled} & \mSymbol[sharp]{airline-seat-flat-angled} & \texttt{\textbackslash mSymbol\{airline-seat-flat-angled\}} & \texttt{E631}\\
\mSymbol[outlined]{airline-seat-individual-suite} & \mSymbol[rounded]{airline-seat-individual-suite} & \mSymbol[sharp]{airline-seat-individual-suite} & \texttt{\textbackslash mSymbol\{airline-seat-individual-suite\}} & \texttt{E632}\\
\mSymbol[outlined]{airline-seat-legroom-extra} & \mSymbol[rounded]{airline-seat-legroom-extra} & \mSymbol[sharp]{airline-seat-legroom-extra} & \texttt{\textbackslash mSymbol\{airline-seat-legroom-extra\}} & \texttt{E633}\\
\mSymbol[outlined]{airline-seat-legroom-normal} & \mSymbol[rounded]{airline-seat-legroom-normal} & \mSymbol[sharp]{airline-seat-legroom-normal} & \texttt{\textbackslash mSymbol\{airline-seat-legroom-normal\}} & \texttt{E634}\\
\mSymbol[outlined]{airline-seat-legroom-reduced} & \mSymbol[rounded]{airline-seat-legroom-reduced} & \mSymbol[sharp]{airline-seat-legroom-reduced} & \texttt{\textbackslash mSymbol\{airline-seat-legroom-reduced\}} & \texttt{E635}\\
\mSymbol[outlined]{airline-seat-recline-extra} & \mSymbol[rounded]{airline-seat-recline-extra} & \mSymbol[sharp]{airline-seat-recline-extra} & \texttt{\textbackslash mSymbol\{airline-seat-recline-extra\}} & \texttt{E636}\\
\mSymbol[outlined]{airline-seat-recline-normal} & \mSymbol[rounded]{airline-seat-recline-normal} & \mSymbol[sharp]{airline-seat-recline-normal} & \texttt{\textbackslash mSymbol\{airline-seat-recline-normal\}} & \texttt{E637}\\
\mSymbol[outlined]{airline-stops} & \mSymbol[rounded]{airline-stops} & \mSymbol[sharp]{airline-stops} & \texttt{\textbackslash mSymbol\{airline-stops\}} & \texttt{E7D0}\\
\mSymbol[outlined]{airlines} & \mSymbol[rounded]{airlines} & \mSymbol[sharp]{airlines} & \texttt{\textbackslash mSymbol\{airlines\}} & \texttt{E7CA}\\
\mSymbol[outlined]{airplane-ticket} & \mSymbol[rounded]{airplane-ticket} & \mSymbol[sharp]{airplane-ticket} & \texttt{\textbackslash mSymbol\{airplane-ticket\}} & \texttt{EFD9}\\
\mSymbol[outlined]{airplanemode-active} & \mSymbol[rounded]{airplanemode-active} & \mSymbol[sharp]{airplanemode-active} & \texttt{\textbackslash mSymbol\{airplanemode-active\}} & \texttt{E53D}\\
\mSymbol[outlined]{airplanemode-inactive} & \mSymbol[rounded]{airplanemode-inactive} & \mSymbol[sharp]{airplanemode-inactive} & \texttt{\textbackslash mSymbol\{airplanemode-inactive\}} & \texttt{E194}\\
\mSymbol[outlined]{airplay} & \mSymbol[rounded]{airplay} & \mSymbol[sharp]{airplay} & \texttt{\textbackslash mSymbol\{airplay\}} & \texttt{E055}\\
\mSymbol[outlined]{airport-shuttle} & \mSymbol[rounded]{airport-shuttle} & \mSymbol[sharp]{airport-shuttle} & \texttt{\textbackslash mSymbol\{airport-shuttle\}} & \texttt{EB3C}\\
\mSymbol[outlined]{airware} & \mSymbol[rounded]{airware} & \mSymbol[sharp]{airware} & \texttt{\textbackslash mSymbol\{airware\}} & \texttt{F154}\\
\mSymbol[outlined]{airwave} & \mSymbol[rounded]{airwave} & \mSymbol[sharp]{airwave} & \texttt{\textbackslash mSymbol\{airwave\}} & \texttt{F154}\\
\mSymbol[outlined]{alarm} & \mSymbol[rounded]{alarm} & \mSymbol[sharp]{alarm} & \texttt{\textbackslash mSymbol\{alarm\}} & \texttt{E855}\\
\mSymbol[outlined]{alarm-add} & \mSymbol[rounded]{alarm-add} & \mSymbol[sharp]{alarm-add} & \texttt{\textbackslash mSymbol\{alarm-add\}} & \texttt{E856}\\
\mSymbol[outlined]{alarm-off} & \mSymbol[rounded]{alarm-off} & \mSymbol[sharp]{alarm-off} & \texttt{\textbackslash mSymbol\{alarm-off\}} & \texttt{E857}\\
\mSymbol[outlined]{alarm-on} & \mSymbol[rounded]{alarm-on} & \mSymbol[sharp]{alarm-on} & \texttt{\textbackslash mSymbol\{alarm-on\}} & \texttt{E858}\\
\mSymbol[outlined]{alarm-smart-wake} & \mSymbol[rounded]{alarm-smart-wake} & \mSymbol[sharp]{alarm-smart-wake} & \texttt{\textbackslash mSymbol\{alarm-smart-wake\}} & \texttt{F6B0}\\
\mSymbol[outlined]{album} & \mSymbol[rounded]{album} & \mSymbol[sharp]{album} & \texttt{\textbackslash mSymbol\{album\}} & \texttt{E019}\\
\mSymbol[outlined]{align-center} & \mSymbol[rounded]{align-center} & \mSymbol[sharp]{align-center} & \texttt{\textbackslash mSymbol\{align-center\}} & \texttt{E356}\\
\mSymbol[outlined]{align-end} & \mSymbol[rounded]{align-end} & \mSymbol[sharp]{align-end} & \texttt{\textbackslash mSymbol\{align-end\}} & \texttt{F797}\\
\mSymbol[outlined]{align-flex-center} & \mSymbol[rounded]{align-flex-center} & \mSymbol[sharp]{align-flex-center} & \texttt{\textbackslash mSymbol\{align-flex-center\}} & \texttt{F796}\\
\mSymbol[outlined]{align-flex-end} & \mSymbol[rounded]{align-flex-end} & \mSymbol[sharp]{align-flex-end} & \texttt{\textbackslash mSymbol\{align-flex-end\}} & \texttt{F795}\\
\mSymbol[outlined]{align-flex-start} & \mSymbol[rounded]{align-flex-start} & \mSymbol[sharp]{align-flex-start} & \texttt{\textbackslash mSymbol\{align-flex-start\}} & \texttt{F794}\\
\mSymbol[outlined]{align-horizontal-center} & \mSymbol[rounded]{align-horizontal-center} & \mSymbol[sharp]{align-horizontal-center} & \texttt{\textbackslash mSymbol\{align-horizontal-center\}} & \texttt{E00F}\\
\mSymbol[outlined]{align-horizontal-left} & \mSymbol[rounded]{align-horizontal-left} & \mSymbol[sharp]{align-horizontal-left} & \texttt{\textbackslash mSymbol\{align-horizontal-left\}} & \texttt{E00D}\\
\mSymbol[outlined]{align-horizontal-right} & \mSymbol[rounded]{align-horizontal-right} & \mSymbol[sharp]{align-horizontal-right} & \texttt{\textbackslash mSymbol\{align-horizontal-right\}} & \texttt{E010}\\
\mSymbol[outlined]{align-items-stretch} & \mSymbol[rounded]{align-items-stretch} & \mSymbol[sharp]{align-items-stretch} & \texttt{\textbackslash mSymbol\{align-items-stretch\}} & \texttt{F793}\\
\mSymbol[outlined]{align-justify-center} & \mSymbol[rounded]{align-justify-center} & \mSymbol[sharp]{align-justify-center} & \texttt{\textbackslash mSymbol\{align-justify-center\}} & \texttt{F792}\\
\mSymbol[outlined]{align-justify-flex-end} & \mSymbol[rounded]{align-justify-flex-end} & \mSymbol[sharp]{align-justify-flex-end} & \texttt{\textbackslash mSymbol\{align-justify-flex-end\}} & \texttt{F791}\\
\mSymbol[outlined]{align-justify-flex-start} & \mSymbol[rounded]{align-justify-flex-start} & \mSymbol[sharp]{align-justify-flex-start} & \texttt{\textbackslash mSymbol\{align-justify-flex-start\}} & \texttt{F790}\\
\mSymbol[outlined]{align-justify-space-around} & \mSymbol[rounded]{align-justify-space-around} & \mSymbol[sharp]{align-justify-space-around} & \texttt{\textbackslash mSymbol\{align-justify-space-around\}} & \texttt{F78F}\\
\mSymbol[outlined]{align-justify-space-between} & \mSymbol[rounded]{align-justify-space-between} & \mSymbol[sharp]{align-justify-space-between} & \texttt{\textbackslash mSymbol\{align-justify-space-between\}} & \texttt{F78E}\\
\mSymbol[outlined]{align-justify-space-even} & \mSymbol[rounded]{align-justify-space-even} & \mSymbol[sharp]{align-justify-space-even} & \texttt{\textbackslash mSymbol\{align-justify-space-even\}} & \texttt{F78D}\\
\mSymbol[outlined]{align-justify-stretch} & \mSymbol[rounded]{align-justify-stretch} & \mSymbol[sharp]{align-justify-stretch} & \texttt{\textbackslash mSymbol\{align-justify-stretch\}} & \texttt{F78C}\\
\mSymbol[outlined]{align-self-stretch} & \mSymbol[rounded]{align-self-stretch} & \mSymbol[sharp]{align-self-stretch} & \texttt{\textbackslash mSymbol\{align-self-stretch\}} & \texttt{F78B}\\
\mSymbol[outlined]{align-space-around} & \mSymbol[rounded]{align-space-around} & \mSymbol[sharp]{align-space-around} & \texttt{\textbackslash mSymbol\{align-space-around\}} & \texttt{F78A}\\
\mSymbol[outlined]{align-space-between} & \mSymbol[rounded]{align-space-between} & \mSymbol[sharp]{align-space-between} & \texttt{\textbackslash mSymbol\{align-space-between\}} & \texttt{F789}\\
\mSymbol[outlined]{align-space-even} & \mSymbol[rounded]{align-space-even} & \mSymbol[sharp]{align-space-even} & \texttt{\textbackslash mSymbol\{align-space-even\}} & \texttt{F788}\\
\mSymbol[outlined]{align-start} & \mSymbol[rounded]{align-start} & \mSymbol[sharp]{align-start} & \texttt{\textbackslash mSymbol\{align-start\}} & \texttt{F787}\\
\mSymbol[outlined]{align-stretch} & \mSymbol[rounded]{align-stretch} & \mSymbol[sharp]{align-stretch} & \texttt{\textbackslash mSymbol\{align-stretch\}} & \texttt{F786}\\
\mSymbol[outlined]{align-vertical-bottom} & \mSymbol[rounded]{align-vertical-bottom} & \mSymbol[sharp]{align-vertical-bottom} & \texttt{\textbackslash mSymbol\{align-vertical-bottom\}} & \texttt{E015}\\
\mSymbol[outlined]{align-vertical-center} & \mSymbol[rounded]{align-vertical-center} & \mSymbol[sharp]{align-vertical-center} & \texttt{\textbackslash mSymbol\{align-vertical-center\}} & \texttt{E011}\\
\mSymbol[outlined]{align-vertical-top} & \mSymbol[rounded]{align-vertical-top} & \mSymbol[sharp]{align-vertical-top} & \texttt{\textbackslash mSymbol\{align-vertical-top\}} & \texttt{E00C}\\
\mSymbol[outlined]{all-inbox} & \mSymbol[rounded]{all-inbox} & \mSymbol[sharp]{all-inbox} & \texttt{\textbackslash mSymbol\{all-inbox\}} & \texttt{E97F}\\
\mSymbol[outlined]{all-inclusive} & \mSymbol[rounded]{all-inclusive} & \mSymbol[sharp]{all-inclusive} & \texttt{\textbackslash mSymbol\{all-inclusive\}} & \texttt{EB3D}\\
\mSymbol[outlined]{all-match} & \mSymbol[rounded]{all-match} & \mSymbol[sharp]{all-match} & \texttt{\textbackslash mSymbol\{all-match\}} & \texttt{E093}\\
\mSymbol[outlined]{all-out} & \mSymbol[rounded]{all-out} & \mSymbol[sharp]{all-out} & \texttt{\textbackslash mSymbol\{all-out\}} & \texttt{E90B}\\
\mSymbol[outlined]{allergies} & \mSymbol[rounded]{allergies} & \mSymbol[sharp]{allergies} & \texttt{\textbackslash mSymbol\{allergies\}} & \texttt{E094}\\
\mSymbol[outlined]{allergy} & \mSymbol[rounded]{allergy} & \mSymbol[sharp]{allergy} & \texttt{\textbackslash mSymbol\{allergy\}} & \texttt{E64E}\\
\mSymbol[outlined]{alt-route} & \mSymbol[rounded]{alt-route} & \mSymbol[sharp]{alt-route} & \texttt{\textbackslash mSymbol\{alt-route\}} & \texttt{F184}\\
\mSymbol[outlined]{alternate-email} & \mSymbol[rounded]{alternate-email} & \mSymbol[sharp]{alternate-email} & \texttt{\textbackslash mSymbol\{alternate-email\}} & \texttt{E0E6}\\
\mSymbol[outlined]{altitude} & \mSymbol[rounded]{altitude} & \mSymbol[sharp]{altitude} & \texttt{\textbackslash mSymbol\{altitude\}} & \texttt{F873}\\
\mSymbol[outlined]{ambient-screen} & \mSymbol[rounded]{ambient-screen} & \mSymbol[sharp]{ambient-screen} & \texttt{\textbackslash mSymbol\{ambient-screen\}} & \texttt{F6C4}\\
\mSymbol[outlined]{ambulance} & \mSymbol[rounded]{ambulance} & \mSymbol[sharp]{ambulance} & \texttt{\textbackslash mSymbol\{ambulance\}} & \texttt{F803}\\
\mSymbol[outlined]{amend} & \mSymbol[rounded]{amend} & \mSymbol[sharp]{amend} & \texttt{\textbackslash mSymbol\{amend\}} & \texttt{F802}\\
\mSymbol[outlined]{amp-stories} & \mSymbol[rounded]{amp-stories} & \mSymbol[sharp]{amp-stories} & \texttt{\textbackslash mSymbol\{amp-stories\}} & \texttt{EA13}\\
\mSymbol[outlined]{analytics} & \mSymbol[rounded]{analytics} & \mSymbol[sharp]{analytics} & \texttt{\textbackslash mSymbol\{analytics\}} & \texttt{EF3E}\\
\mSymbol[outlined]{anchor} & \mSymbol[rounded]{anchor} & \mSymbol[sharp]{anchor} & \texttt{\textbackslash mSymbol\{anchor\}} & \texttt{F1CD}\\
\mSymbol[outlined]{android} & \mSymbol[rounded]{android} & \mSymbol[sharp]{android} & \texttt{\textbackslash mSymbol\{android\}} & \texttt{E859}\\
\mSymbol[outlined]{animated-images} & \mSymbol[rounded]{animated-images} & \mSymbol[sharp]{animated-images} & \texttt{\textbackslash mSymbol\{animated-images\}} & \texttt{F49A}\\
\mSymbol[outlined]{animation} & \mSymbol[rounded]{animation} & \mSymbol[sharp]{animation} & \texttt{\textbackslash mSymbol\{animation\}} & \texttt{E71C}\\
\mSymbol[outlined]{announcement} & \mSymbol[rounded]{announcement} & \mSymbol[sharp]{announcement} & \texttt{\textbackslash mSymbol\{announcement\}} & \texttt{E87F}\\
\mSymbol[outlined]{aod} & \mSymbol[rounded]{aod} & \mSymbol[sharp]{aod} & \texttt{\textbackslash mSymbol\{aod\}} & \texttt{EFDA}\\
\mSymbol[outlined]{aod-tablet} & \mSymbol[rounded]{aod-tablet} & \mSymbol[sharp]{aod-tablet} & \texttt{\textbackslash mSymbol\{aod-tablet\}} & \texttt{F89F}\\
\mSymbol[outlined]{aod-watch} & \mSymbol[rounded]{aod-watch} & \mSymbol[sharp]{aod-watch} & \texttt{\textbackslash mSymbol\{aod-watch\}} & \texttt{F6AC}\\
\mSymbol[outlined]{apartment} & \mSymbol[rounded]{apartment} & \mSymbol[sharp]{apartment} & \texttt{\textbackslash mSymbol\{apartment\}} & \texttt{EA40}\\
\mSymbol[outlined]{api} & \mSymbol[rounded]{api} & \mSymbol[sharp]{api} & \texttt{\textbackslash mSymbol\{api\}} & \texttt{F1B7}\\
\mSymbol[outlined]{apk-document} & \mSymbol[rounded]{apk-document} & \mSymbol[sharp]{apk-document} & \texttt{\textbackslash mSymbol\{apk-document\}} & \texttt{F88E}\\
\mSymbol[outlined]{apk-install} & \mSymbol[rounded]{apk-install} & \mSymbol[sharp]{apk-install} & \texttt{\textbackslash mSymbol\{apk-install\}} & \texttt{F88F}\\
\mSymbol[outlined]{app-badging} & \mSymbol[rounded]{app-badging} & \mSymbol[sharp]{app-badging} & \texttt{\textbackslash mSymbol\{app-badging\}} & \texttt{F72F}\\
\mSymbol[outlined]{app-blocking} & \mSymbol[rounded]{app-blocking} & \mSymbol[sharp]{app-blocking} & \texttt{\textbackslash mSymbol\{app-blocking\}} & \texttt{EF3F}\\
\mSymbol[outlined]{app-promo} & \mSymbol[rounded]{app-promo} & \mSymbol[sharp]{app-promo} & \texttt{\textbackslash mSymbol\{app-promo\}} & \texttt{E981}\\
\mSymbol[outlined]{app-registration} & \mSymbol[rounded]{app-registration} & \mSymbol[sharp]{app-registration} & \texttt{\textbackslash mSymbol\{app-registration\}} & \texttt{EF40}\\
\mSymbol[outlined]{app-settings-alt} & \mSymbol[rounded]{app-settings-alt} & \mSymbol[sharp]{app-settings-alt} & \texttt{\textbackslash mSymbol\{app-settings-alt\}} & \texttt{EF41}\\
\mSymbol[outlined]{app-shortcut} & \mSymbol[rounded]{app-shortcut} & \mSymbol[sharp]{app-shortcut} & \texttt{\textbackslash mSymbol\{app-shortcut\}} & \texttt{EAE4}\\
\mSymbol[outlined]{apparel} & \mSymbol[rounded]{apparel} & \mSymbol[sharp]{apparel} & \texttt{\textbackslash mSymbol\{apparel\}} & \texttt{EF7B}\\
\mSymbol[outlined]{approval} & \mSymbol[rounded]{approval} & \mSymbol[sharp]{approval} & \texttt{\textbackslash mSymbol\{approval\}} & \texttt{E982}\\
\mSymbol[outlined]{approval-delegation} & \mSymbol[rounded]{approval-delegation} & \mSymbol[sharp]{approval-delegation} & \texttt{\textbackslash mSymbol\{approval-delegation\}} & \texttt{F84A}\\
\mSymbol[outlined]{apps} & \mSymbol[rounded]{apps} & \mSymbol[sharp]{apps} & \texttt{\textbackslash mSymbol\{apps\}} & \texttt{E5C3}\\
\mSymbol[outlined]{apps-outage} & \mSymbol[rounded]{apps-outage} & \mSymbol[sharp]{apps-outage} & \texttt{\textbackslash mSymbol\{apps-outage\}} & \texttt{E7CC}\\
\mSymbol[outlined]{aq} & \mSymbol[rounded]{aq} & \mSymbol[sharp]{aq} & \texttt{\textbackslash mSymbol\{aq\}} & \texttt{F55A}\\
\mSymbol[outlined]{aq-indoor} & \mSymbol[rounded]{aq-indoor} & \mSymbol[sharp]{aq-indoor} & \texttt{\textbackslash mSymbol\{aq-indoor\}} & \texttt{F55B}\\
\mSymbol[outlined]{ar-on-you} & \mSymbol[rounded]{ar-on-you} & \mSymbol[sharp]{ar-on-you} & \texttt{\textbackslash mSymbol\{ar-on-you\}} & \texttt{EF7C}\\
\mSymbol[outlined]{ar-stickers} & \mSymbol[rounded]{ar-stickers} & \mSymbol[sharp]{ar-stickers} & \texttt{\textbackslash mSymbol\{ar-stickers\}} & \texttt{E983}\\
\mSymbol[outlined]{architecture} & \mSymbol[rounded]{architecture} & \mSymbol[sharp]{architecture} & \texttt{\textbackslash mSymbol\{architecture\}} & \texttt{EA3B}\\
\mSymbol[outlined]{archive} & \mSymbol[rounded]{archive} & \mSymbol[sharp]{archive} & \texttt{\textbackslash mSymbol\{archive\}} & \texttt{E149}\\
\mSymbol[outlined]{area-chart} & \mSymbol[rounded]{area-chart} & \mSymbol[sharp]{area-chart} & \texttt{\textbackslash mSymbol\{area-chart\}} & \texttt{E770}\\
\mSymbol[outlined]{arming-countdown} & \mSymbol[rounded]{arming-countdown} & \mSymbol[sharp]{arming-countdown} & \texttt{\textbackslash mSymbol\{arming-countdown\}} & \texttt{E78A}\\
\mSymbol[outlined]{arrow-and-edge} & \mSymbol[rounded]{arrow-and-edge} & \mSymbol[sharp]{arrow-and-edge} & \texttt{\textbackslash mSymbol\{arrow-and-edge\}} & \texttt{F5D7}\\
\mSymbol[outlined]{arrow-back} & \mSymbol[rounded]{arrow-back} & \mSymbol[sharp]{arrow-back} & \texttt{\textbackslash mSymbol\{arrow-back\}} & \texttt{E5C4}\\
\mSymbol[outlined]{arrow-back-2} & \mSymbol[rounded]{arrow-back-2} & \mSymbol[sharp]{arrow-back-2} & \texttt{\textbackslash mSymbol\{arrow-back-2\}} & \texttt{F43A}\\
\mSymbol[outlined]{arrow-back-ios} & \mSymbol[rounded]{arrow-back-ios} & \mSymbol[sharp]{arrow-back-ios} & \texttt{\textbackslash mSymbol\{arrow-back-ios\}} & \texttt{E5E0}\\
\mSymbol[outlined]{arrow-back-ios-new} & \mSymbol[rounded]{arrow-back-ios-new} & \mSymbol[sharp]{arrow-back-ios-new} & \texttt{\textbackslash mSymbol\{arrow-back-ios-new\}} & \texttt{E2EA}\\
\mSymbol[outlined]{arrow-circle-down} & \mSymbol[rounded]{arrow-circle-down} & \mSymbol[sharp]{arrow-circle-down} & \texttt{\textbackslash mSymbol\{arrow-circle-down\}} & \texttt{F181}\\
\mSymbol[outlined]{arrow-circle-left} & \mSymbol[rounded]{arrow-circle-left} & \mSymbol[sharp]{arrow-circle-left} & \texttt{\textbackslash mSymbol\{arrow-circle-left\}} & \texttt{EAA7}\\
\mSymbol[outlined]{arrow-circle-right} & \mSymbol[rounded]{arrow-circle-right} & \mSymbol[sharp]{arrow-circle-right} & \texttt{\textbackslash mSymbol\{arrow-circle-right\}} & \texttt{EAAA}\\
\mSymbol[outlined]{arrow-circle-up} & \mSymbol[rounded]{arrow-circle-up} & \mSymbol[sharp]{arrow-circle-up} & \texttt{\textbackslash mSymbol\{arrow-circle-up\}} & \texttt{F182}\\
\mSymbol[outlined]{arrow-cool-down} & \mSymbol[rounded]{arrow-cool-down} & \mSymbol[sharp]{arrow-cool-down} & \texttt{\textbackslash mSymbol\{arrow-cool-down\}} & \texttt{F4B6}\\
\mSymbol[outlined]{arrow-downward} & \mSymbol[rounded]{arrow-downward} & \mSymbol[sharp]{arrow-downward} & \texttt{\textbackslash mSymbol\{arrow-downward\}} & \texttt{E5DB}\\
\mSymbol[outlined]{arrow-downward-alt} & \mSymbol[rounded]{arrow-downward-alt} & \mSymbol[sharp]{arrow-downward-alt} & \texttt{\textbackslash mSymbol\{arrow-downward-alt\}} & \texttt{E984}\\
\mSymbol[outlined]{arrow-drop-down} & \mSymbol[rounded]{arrow-drop-down} & \mSymbol[sharp]{arrow-drop-down} & \texttt{\textbackslash mSymbol\{arrow-drop-down\}} & \texttt{E5C5}\\
\mSymbol[outlined]{arrow-drop-down-circle} & \mSymbol[rounded]{arrow-drop-down-circle} & \mSymbol[sharp]{arrow-drop-down-circle} & \texttt{\textbackslash mSymbol\{arrow-drop-down-circle\}} & \texttt{E5C6}\\
\mSymbol[outlined]{arrow-drop-up} & \mSymbol[rounded]{arrow-drop-up} & \mSymbol[sharp]{arrow-drop-up} & \texttt{\textbackslash mSymbol\{arrow-drop-up\}} & \texttt{E5C7}\\
\mSymbol[outlined]{arrow-forward} & \mSymbol[rounded]{arrow-forward} & \mSymbol[sharp]{arrow-forward} & \texttt{\textbackslash mSymbol\{arrow-forward\}} & \texttt{E5C8}\\
\mSymbol[outlined]{arrow-forward-ios} & \mSymbol[rounded]{arrow-forward-ios} & \mSymbol[sharp]{arrow-forward-ios} & \texttt{\textbackslash mSymbol\{arrow-forward-ios\}} & \texttt{E5E1}\\
\mSymbol[outlined]{arrow-insert} & \mSymbol[rounded]{arrow-insert} & \mSymbol[sharp]{arrow-insert} & \texttt{\textbackslash mSymbol\{arrow-insert\}} & \texttt{F837}\\
\mSymbol[outlined]{arrow-left} & \mSymbol[rounded]{arrow-left} & \mSymbol[sharp]{arrow-left} & \texttt{\textbackslash mSymbol\{arrow-left\}} & \texttt{E5DE}\\
\mSymbol[outlined]{arrow-left-alt} & \mSymbol[rounded]{arrow-left-alt} & \mSymbol[sharp]{arrow-left-alt} & \texttt{\textbackslash mSymbol\{arrow-left-alt\}} & \texttt{EF7D}\\
\mSymbol[outlined]{arrow-or-edge} & \mSymbol[rounded]{arrow-or-edge} & \mSymbol[sharp]{arrow-or-edge} & \texttt{\textbackslash mSymbol\{arrow-or-edge\}} & \texttt{F5D6}\\
\mSymbol[outlined]{arrow-outward} & \mSymbol[rounded]{arrow-outward} & \mSymbol[sharp]{arrow-outward} & \texttt{\textbackslash mSymbol\{arrow-outward\}} & \texttt{F8CE}\\
\mSymbol[outlined]{arrow-range} & \mSymbol[rounded]{arrow-range} & \mSymbol[sharp]{arrow-range} & \texttt{\textbackslash mSymbol\{arrow-range\}} & \texttt{F69B}\\
\mSymbol[outlined]{arrow-right} & \mSymbol[rounded]{arrow-right} & \mSymbol[sharp]{arrow-right} & \texttt{\textbackslash mSymbol\{arrow-right\}} & \texttt{E5DF}\\
\mSymbol[outlined]{arrow-right-alt} & \mSymbol[rounded]{arrow-right-alt} & \mSymbol[sharp]{arrow-right-alt} & \texttt{\textbackslash mSymbol\{arrow-right-alt\}} & \texttt{E941}\\
\mSymbol[outlined]{arrow-selector-tool} & \mSymbol[rounded]{arrow-selector-tool} & \mSymbol[sharp]{arrow-selector-tool} & \texttt{\textbackslash mSymbol\{arrow-selector-tool\}} & \texttt{F82F}\\
\mSymbol[outlined]{arrow-split} & \mSymbol[rounded]{arrow-split} & \mSymbol[sharp]{arrow-split} & \texttt{\textbackslash mSymbol\{arrow-split\}} & \texttt{EA04}\\
\mSymbol[outlined]{arrow-top-left} & \mSymbol[rounded]{arrow-top-left} & \mSymbol[sharp]{arrow-top-left} & \texttt{\textbackslash mSymbol\{arrow-top-left\}} & \texttt{F72E}\\
\mSymbol[outlined]{arrow-top-right} & \mSymbol[rounded]{arrow-top-right} & \mSymbol[sharp]{arrow-top-right} & \texttt{\textbackslash mSymbol\{arrow-top-right\}} & \texttt{F72D}\\
\mSymbol[outlined]{arrow-upward} & \mSymbol[rounded]{arrow-upward} & \mSymbol[sharp]{arrow-upward} & \texttt{\textbackslash mSymbol\{arrow-upward\}} & \texttt{E5D8}\\
\mSymbol[outlined]{arrow-upward-alt} & \mSymbol[rounded]{arrow-upward-alt} & \mSymbol[sharp]{arrow-upward-alt} & \texttt{\textbackslash mSymbol\{arrow-upward-alt\}} & \texttt{E986}\\
\mSymbol[outlined]{arrow-warm-up} & \mSymbol[rounded]{arrow-warm-up} & \mSymbol[sharp]{arrow-warm-up} & \texttt{\textbackslash mSymbol\{arrow-warm-up\}} & \texttt{F4B5}\\
\mSymbol[outlined]{arrows-more-down} & \mSymbol[rounded]{arrows-more-down} & \mSymbol[sharp]{arrows-more-down} & \texttt{\textbackslash mSymbol\{arrows-more-down\}} & \texttt{F8AB}\\
\mSymbol[outlined]{arrows-more-up} & \mSymbol[rounded]{arrows-more-up} & \mSymbol[sharp]{arrows-more-up} & \texttt{\textbackslash mSymbol\{arrows-more-up\}} & \texttt{F8AC}\\
\mSymbol[outlined]{arrows-outward} & \mSymbol[rounded]{arrows-outward} & \mSymbol[sharp]{arrows-outward} & \texttt{\textbackslash mSymbol\{arrows-outward\}} & \texttt{F72C}\\
\mSymbol[outlined]{art-track} & \mSymbol[rounded]{art-track} & \mSymbol[sharp]{art-track} & \texttt{\textbackslash mSymbol\{art-track\}} & \texttt{E060}\\
\mSymbol[outlined]{article} & \mSymbol[rounded]{article} & \mSymbol[sharp]{article} & \texttt{\textbackslash mSymbol\{article\}} & \texttt{EF42}\\
\mSymbol[outlined]{article-shortcut} & \mSymbol[rounded]{article-shortcut} & \mSymbol[sharp]{article-shortcut} & \texttt{\textbackslash mSymbol\{article-shortcut\}} & \texttt{F587}\\
\mSymbol[outlined]{artist} & \mSymbol[rounded]{artist} & \mSymbol[sharp]{artist} & \texttt{\textbackslash mSymbol\{artist\}} & \texttt{E01A}\\
\mSymbol[outlined]{aspect-ratio} & \mSymbol[rounded]{aspect-ratio} & \mSymbol[sharp]{aspect-ratio} & \texttt{\textbackslash mSymbol\{aspect-ratio\}} & \texttt{E85B}\\
\mSymbol[outlined]{assessment} & \mSymbol[rounded]{assessment} & \mSymbol[sharp]{assessment} & \texttt{\textbackslash mSymbol\{assessment\}} & \texttt{F0CC}\\
\mSymbol[outlined]{assignment} & \mSymbol[rounded]{assignment} & \mSymbol[sharp]{assignment} & \texttt{\textbackslash mSymbol\{assignment\}} & \texttt{E85D}\\
\mSymbol[outlined]{assignment-add} & \mSymbol[rounded]{assignment-add} & \mSymbol[sharp]{assignment-add} & \texttt{\textbackslash mSymbol\{assignment-add\}} & \texttt{F848}\\
\mSymbol[outlined]{assignment-ind} & \mSymbol[rounded]{assignment-ind} & \mSymbol[sharp]{assignment-ind} & \texttt{\textbackslash mSymbol\{assignment-ind\}} & \texttt{E85E}\\
\mSymbol[outlined]{assignment-late} & \mSymbol[rounded]{assignment-late} & \mSymbol[sharp]{assignment-late} & \texttt{\textbackslash mSymbol\{assignment-late\}} & \texttt{E85F}\\
\mSymbol[outlined]{assignment-return} & \mSymbol[rounded]{assignment-return} & \mSymbol[sharp]{assignment-return} & \texttt{\textbackslash mSymbol\{assignment-return\}} & \texttt{E860}\\
\mSymbol[outlined]{assignment-returned} & \mSymbol[rounded]{assignment-returned} & \mSymbol[sharp]{assignment-returned} & \texttt{\textbackslash mSymbol\{assignment-returned\}} & \texttt{E861}\\
\mSymbol[outlined]{assignment-turned-in} & \mSymbol[rounded]{assignment-turned-in} & \mSymbol[sharp]{assignment-turned-in} & \texttt{\textbackslash mSymbol\{assignment-turned-in\}} & \texttt{E862}\\
\mSymbol[outlined]{assist-walker} & \mSymbol[rounded]{assist-walker} & \mSymbol[sharp]{assist-walker} & \texttt{\textbackslash mSymbol\{assist-walker\}} & \texttt{F8D5}\\
\mSymbol[outlined]{assistant} & \mSymbol[rounded]{assistant} & \mSymbol[sharp]{assistant} & \texttt{\textbackslash mSymbol\{assistant\}} & \texttt{E39F}\\
\mSymbol[outlined]{assistant-device} & \mSymbol[rounded]{assistant-device} & \mSymbol[sharp]{assistant-device} & \texttt{\textbackslash mSymbol\{assistant-device\}} & \texttt{E987}\\
\mSymbol[outlined]{assistant-direction} & \mSymbol[rounded]{assistant-direction} & \mSymbol[sharp]{assistant-direction} & \texttt{\textbackslash mSymbol\{assistant-direction\}} & \texttt{E988}\\
\mSymbol[outlined]{assistant-navigation} & \mSymbol[rounded]{assistant-navigation} & \mSymbol[sharp]{assistant-navigation} & \texttt{\textbackslash mSymbol\{assistant-navigation\}} & \texttt{E989}\\
\mSymbol[outlined]{assistant-on-hub} & \mSymbol[rounded]{assistant-on-hub} & \mSymbol[sharp]{assistant-on-hub} & \texttt{\textbackslash mSymbol\{assistant-on-hub\}} & \texttt{F6C1}\\
\mSymbol[outlined]{assistant-photo} & \mSymbol[rounded]{assistant-photo} & \mSymbol[sharp]{assistant-photo} & \texttt{\textbackslash mSymbol\{assistant-photo\}} & \texttt{F0C6}\\
\mSymbol[outlined]{assured-workload} & \mSymbol[rounded]{assured-workload} & \mSymbol[sharp]{assured-workload} & \texttt{\textbackslash mSymbol\{assured-workload\}} & \texttt{EB6F}\\
\mSymbol[outlined]{asterisk} & \mSymbol[rounded]{asterisk} & \mSymbol[sharp]{asterisk} & \texttt{\textbackslash mSymbol\{asterisk\}} & \texttt{F525}\\
\mSymbol[outlined]{astrophotography-auto} & \mSymbol[rounded]{astrophotography-auto} & \mSymbol[sharp]{astrophotography-auto} & \texttt{\textbackslash mSymbol\{astrophotography-auto\}} & \texttt{F1D9}\\
\mSymbol[outlined]{astrophotography-off} & \mSymbol[rounded]{astrophotography-off} & \mSymbol[sharp]{astrophotography-off} & \texttt{\textbackslash mSymbol\{astrophotography-off\}} & \texttt{F1DA}\\
\mSymbol[outlined]{atm} & \mSymbol[rounded]{atm} & \mSymbol[sharp]{atm} & \texttt{\textbackslash mSymbol\{atm\}} & \texttt{E573}\\
\mSymbol[outlined]{atr} & \mSymbol[rounded]{atr} & \mSymbol[sharp]{atr} & \texttt{\textbackslash mSymbol\{atr\}} & \texttt{EBC7}\\
\mSymbol[outlined]{attach-email} & \mSymbol[rounded]{attach-email} & \mSymbol[sharp]{attach-email} & \texttt{\textbackslash mSymbol\{attach-email\}} & \texttt{EA5E}\\
\mSymbol[outlined]{attach-file} & \mSymbol[rounded]{attach-file} & \mSymbol[sharp]{attach-file} & \texttt{\textbackslash mSymbol\{attach-file\}} & \texttt{E226}\\
\mSymbol[outlined]{attach-file-add} & \mSymbol[rounded]{attach-file-add} & \mSymbol[sharp]{attach-file-add} & \texttt{\textbackslash mSymbol\{attach-file-add\}} & \texttt{F841}\\
\mSymbol[outlined]{attach-file-off} & \mSymbol[rounded]{attach-file-off} & \mSymbol[sharp]{attach-file-off} & \texttt{\textbackslash mSymbol\{attach-file-off\}} & \texttt{F4D9}\\
\mSymbol[outlined]{attach-money} & \mSymbol[rounded]{attach-money} & \mSymbol[sharp]{attach-money} & \texttt{\textbackslash mSymbol\{attach-money\}} & \texttt{E227}\\
\mSymbol[outlined]{attachment} & \mSymbol[rounded]{attachment} & \mSymbol[sharp]{attachment} & \texttt{\textbackslash mSymbol\{attachment\}} & \texttt{E2BC}\\
\mSymbol[outlined]{attractions} & \mSymbol[rounded]{attractions} & \mSymbol[sharp]{attractions} & \texttt{\textbackslash mSymbol\{attractions\}} & \texttt{EA52}\\
\mSymbol[outlined]{attribution} & \mSymbol[rounded]{attribution} & \mSymbol[sharp]{attribution} & \texttt{\textbackslash mSymbol\{attribution\}} & \texttt{EFDB}\\
\mSymbol[outlined]{audio-description} & \mSymbol[rounded]{audio-description} & \mSymbol[sharp]{audio-description} & \texttt{\textbackslash mSymbol\{audio-description\}} & \texttt{F58C}\\
\mSymbol[outlined]{audio-file} & \mSymbol[rounded]{audio-file} & \mSymbol[sharp]{audio-file} & \texttt{\textbackslash mSymbol\{audio-file\}} & \texttt{EB82}\\
\mSymbol[outlined]{audio-video-receiver} & \mSymbol[rounded]{audio-video-receiver} & \mSymbol[sharp]{audio-video-receiver} & \texttt{\textbackslash mSymbol\{audio-video-receiver\}} & \texttt{F5D3}\\
\mSymbol[outlined]{audiotrack} & \mSymbol[rounded]{audiotrack} & \mSymbol[sharp]{audiotrack} & \texttt{\textbackslash mSymbol\{audiotrack\}} & \texttt{E405}\\
\mSymbol[outlined]{auto-activity-zone} & \mSymbol[rounded]{auto-activity-zone} & \mSymbol[sharp]{auto-activity-zone} & \texttt{\textbackslash mSymbol\{auto-activity-zone\}} & \texttt{F8AD}\\
\mSymbol[outlined]{auto-awesome} & \mSymbol[rounded]{auto-awesome} & \mSymbol[sharp]{auto-awesome} & \texttt{\textbackslash mSymbol\{auto-awesome\}} & \texttt{E65F}\\
\mSymbol[outlined]{auto-awesome-mosaic} & \mSymbol[rounded]{auto-awesome-mosaic} & \mSymbol[sharp]{auto-awesome-mosaic} & \texttt{\textbackslash mSymbol\{auto-awesome-mosaic\}} & \texttt{E660}\\
\mSymbol[outlined]{auto-awesome-motion} & \mSymbol[rounded]{auto-awesome-motion} & \mSymbol[sharp]{auto-awesome-motion} & \texttt{\textbackslash mSymbol\{auto-awesome-motion\}} & \texttt{E661}\\
\mSymbol[outlined]{auto-delete} & \mSymbol[rounded]{auto-delete} & \mSymbol[sharp]{auto-delete} & \texttt{\textbackslash mSymbol\{auto-delete\}} & \texttt{EA4C}\\
\mSymbol[outlined]{auto-detect-voice} & \mSymbol[rounded]{auto-detect-voice} & \mSymbol[sharp]{auto-detect-voice} & \texttt{\textbackslash mSymbol\{auto-detect-voice\}} & \texttt{F83E}\\
\mSymbol[outlined]{auto-draw-solid} & \mSymbol[rounded]{auto-draw-solid} & \mSymbol[sharp]{auto-draw-solid} & \texttt{\textbackslash mSymbol\{auto-draw-solid\}} & \texttt{E98A}\\
\mSymbol[outlined]{auto-fix} & \mSymbol[rounded]{auto-fix} & \mSymbol[sharp]{auto-fix} & \texttt{\textbackslash mSymbol\{auto-fix\}} & \texttt{E663}\\
\mSymbol[outlined]{auto-fix-high} & \mSymbol[rounded]{auto-fix-high} & \mSymbol[sharp]{auto-fix-high} & \texttt{\textbackslash mSymbol\{auto-fix-high\}} & \texttt{E663}\\
\mSymbol[outlined]{auto-fix-normal} & \mSymbol[rounded]{auto-fix-normal} & \mSymbol[sharp]{auto-fix-normal} & \texttt{\textbackslash mSymbol\{auto-fix-normal\}} & \texttt{E664}\\
\mSymbol[outlined]{auto-fix-off} & \mSymbol[rounded]{auto-fix-off} & \mSymbol[sharp]{auto-fix-off} & \texttt{\textbackslash mSymbol\{auto-fix-off\}} & \texttt{E665}\\
\mSymbol[outlined]{auto-graph} & \mSymbol[rounded]{auto-graph} & \mSymbol[sharp]{auto-graph} & \texttt{\textbackslash mSymbol\{auto-graph\}} & \texttt{E4FB}\\
\mSymbol[outlined]{auto-label} & \mSymbol[rounded]{auto-label} & \mSymbol[sharp]{auto-label} & \texttt{\textbackslash mSymbol\{auto-label\}} & \texttt{F6BE}\\
\mSymbol[outlined]{auto-meeting-room} & \mSymbol[rounded]{auto-meeting-room} & \mSymbol[sharp]{auto-meeting-room} & \texttt{\textbackslash mSymbol\{auto-meeting-room\}} & \texttt{F6BF}\\
\mSymbol[outlined]{auto-mode} & \mSymbol[rounded]{auto-mode} & \mSymbol[sharp]{auto-mode} & \texttt{\textbackslash mSymbol\{auto-mode\}} & \texttt{EC20}\\
\mSymbol[outlined]{auto-read-pause} & \mSymbol[rounded]{auto-read-pause} & \mSymbol[sharp]{auto-read-pause} & \texttt{\textbackslash mSymbol\{auto-read-pause\}} & \texttt{F219}\\
\mSymbol[outlined]{auto-read-play} & \mSymbol[rounded]{auto-read-play} & \mSymbol[sharp]{auto-read-play} & \texttt{\textbackslash mSymbol\{auto-read-play\}} & \texttt{F216}\\
\mSymbol[outlined]{auto-schedule} & \mSymbol[rounded]{auto-schedule} & \mSymbol[sharp]{auto-schedule} & \texttt{\textbackslash mSymbol\{auto-schedule\}} & \texttt{E214}\\
\mSymbol[outlined]{auto-stories} & \mSymbol[rounded]{auto-stories} & \mSymbol[sharp]{auto-stories} & \texttt{\textbackslash mSymbol\{auto-stories\}} & \texttt{E666}\\
\mSymbol[outlined]{auto-timer} & \mSymbol[rounded]{auto-timer} & \mSymbol[sharp]{auto-timer} & \texttt{\textbackslash mSymbol\{auto-timer\}} & \texttt{EF7F}\\
\mSymbol[outlined]{auto-towing} & \mSymbol[rounded]{auto-towing} & \mSymbol[sharp]{auto-towing} & \texttt{\textbackslash mSymbol\{auto-towing\}} & \texttt{E71E}\\
\mSymbol[outlined]{auto-transmission} & \mSymbol[rounded]{auto-transmission} & \mSymbol[sharp]{auto-transmission} & \texttt{\textbackslash mSymbol\{auto-transmission\}} & \texttt{F53F}\\
\mSymbol[outlined]{auto-videocam} & \mSymbol[rounded]{auto-videocam} & \mSymbol[sharp]{auto-videocam} & \texttt{\textbackslash mSymbol\{auto-videocam\}} & \texttt{F6C0}\\
\mSymbol[outlined]{autofps-select} & \mSymbol[rounded]{autofps-select} & \mSymbol[sharp]{autofps-select} & \texttt{\textbackslash mSymbol\{autofps-select\}} & \texttt{EFDC}\\
\mSymbol[outlined]{automation} & \mSymbol[rounded]{automation} & \mSymbol[sharp]{automation} & \texttt{\textbackslash mSymbol\{automation\}} & \texttt{F421}\\
\mSymbol[outlined]{autopause} & \mSymbol[rounded]{autopause} & \mSymbol[sharp]{autopause} & \texttt{\textbackslash mSymbol\{autopause\}} & \texttt{F6B6}\\
\mSymbol[outlined]{autopay} & \mSymbol[rounded]{autopay} & \mSymbol[sharp]{autopay} & \texttt{\textbackslash mSymbol\{autopay\}} & \texttt{F84B}\\
\mSymbol[outlined]{autoplay} & \mSymbol[rounded]{autoplay} & \mSymbol[sharp]{autoplay} & \texttt{\textbackslash mSymbol\{autoplay\}} & \texttt{F6B5}\\
\mSymbol[outlined]{autorenew} & \mSymbol[rounded]{autorenew} & \mSymbol[sharp]{autorenew} & \texttt{\textbackslash mSymbol\{autorenew\}} & \texttt{E863}\\
\mSymbol[outlined]{autostop} & \mSymbol[rounded]{autostop} & \mSymbol[sharp]{autostop} & \texttt{\textbackslash mSymbol\{autostop\}} & \texttt{F682}\\
\mSymbol[outlined]{av1} & \mSymbol[rounded]{av1} & \mSymbol[sharp]{av1} & \texttt{\textbackslash mSymbol\{av1\}} & \texttt{F4B0}\\
\mSymbol[outlined]{av-timer} & \mSymbol[rounded]{av-timer} & \mSymbol[sharp]{av-timer} & \texttt{\textbackslash mSymbol\{av-timer\}} & \texttt{E01B}\\
\mSymbol[outlined]{avc} & \mSymbol[rounded]{avc} & \mSymbol[sharp]{avc} & \texttt{\textbackslash mSymbol\{avc\}} & \texttt{F4AF}\\
\mSymbol[outlined]{avg-pace} & \mSymbol[rounded]{avg-pace} & \mSymbol[sharp]{avg-pace} & \texttt{\textbackslash mSymbol\{avg-pace\}} & \texttt{F6BB}\\
\mSymbol[outlined]{avg-time} & \mSymbol[rounded]{avg-time} & \mSymbol[sharp]{avg-time} & \texttt{\textbackslash mSymbol\{avg-time\}} & \texttt{F813}\\
\mSymbol[outlined]{award-star} & \mSymbol[rounded]{award-star} & \mSymbol[sharp]{award-star} & \texttt{\textbackslash mSymbol\{award-star\}} & \texttt{F612}\\
\mSymbol[outlined]{azm} & \mSymbol[rounded]{azm} & \mSymbol[sharp]{azm} & \texttt{\textbackslash mSymbol\{azm\}} & \texttt{F6EC}\\
\mSymbol[outlined]{baby-changing-station} & \mSymbol[rounded]{baby-changing-station} & \mSymbol[sharp]{baby-changing-station} & \texttt{\textbackslash mSymbol\{baby-changing-station\}} & \texttt{F19B}\\
\mSymbol[outlined]{back-hand} & \mSymbol[rounded]{back-hand} & \mSymbol[sharp]{back-hand} & \texttt{\textbackslash mSymbol\{back-hand\}} & \texttt{E764}\\
\mSymbol[outlined]{back-to-tab} & \mSymbol[rounded]{back-to-tab} & \mSymbol[sharp]{back-to-tab} & \texttt{\textbackslash mSymbol\{back-to-tab\}} & \texttt{F72B}\\
\mSymbol[outlined]{background-dot-large} & \mSymbol[rounded]{background-dot-large} & \mSymbol[sharp]{background-dot-large} & \texttt{\textbackslash mSymbol\{background-dot-large\}} & \texttt{F79E}\\
\mSymbol[outlined]{background-dot-small} & \mSymbol[rounded]{background-dot-small} & \mSymbol[sharp]{background-dot-small} & \texttt{\textbackslash mSymbol\{background-dot-small\}} & \texttt{F514}\\
\mSymbol[outlined]{background-grid-small} & \mSymbol[rounded]{background-grid-small} & \mSymbol[sharp]{background-grid-small} & \texttt{\textbackslash mSymbol\{background-grid-small\}} & \texttt{F79D}\\
\mSymbol[outlined]{background-replace} & \mSymbol[rounded]{background-replace} & \mSymbol[sharp]{background-replace} & \texttt{\textbackslash mSymbol\{background-replace\}} & \texttt{F20A}\\
\mSymbol[outlined]{backlight-high} & \mSymbol[rounded]{backlight-high} & \mSymbol[sharp]{backlight-high} & \texttt{\textbackslash mSymbol\{backlight-high\}} & \texttt{F7ED}\\
\mSymbol[outlined]{backlight-high-off} & \mSymbol[rounded]{backlight-high-off} & \mSymbol[sharp]{backlight-high-off} & \texttt{\textbackslash mSymbol\{backlight-high-off\}} & \texttt{F4EF}\\
\mSymbol[outlined]{backlight-low} & \mSymbol[rounded]{backlight-low} & \mSymbol[sharp]{backlight-low} & \texttt{\textbackslash mSymbol\{backlight-low\}} & \texttt{F7EC}\\
\mSymbol[outlined]{backpack} & \mSymbol[rounded]{backpack} & \mSymbol[sharp]{backpack} & \texttt{\textbackslash mSymbol\{backpack\}} & \texttt{F19C}\\
\mSymbol[outlined]{backspace} & \mSymbol[rounded]{backspace} & \mSymbol[sharp]{backspace} & \texttt{\textbackslash mSymbol\{backspace\}} & \texttt{E14A}\\
\mSymbol[outlined]{backup} & \mSymbol[rounded]{backup} & \mSymbol[sharp]{backup} & \texttt{\textbackslash mSymbol\{backup\}} & \texttt{E864}\\
\mSymbol[outlined]{backup-table} & \mSymbol[rounded]{backup-table} & \mSymbol[sharp]{backup-table} & \texttt{\textbackslash mSymbol\{backup-table\}} & \texttt{EF43}\\
\mSymbol[outlined]{badge} & \mSymbol[rounded]{badge} & \mSymbol[sharp]{badge} & \texttt{\textbackslash mSymbol\{badge\}} & \texttt{EA67}\\
\mSymbol[outlined]{badge-critical-battery} & \mSymbol[rounded]{badge-critical-battery} & \mSymbol[sharp]{badge-critical-battery} & \texttt{\textbackslash mSymbol\{badge-critical-battery\}} & \texttt{F156}\\
\mSymbol[outlined]{bakery-dining} & \mSymbol[rounded]{bakery-dining} & \mSymbol[sharp]{bakery-dining} & \texttt{\textbackslash mSymbol\{bakery-dining\}} & \texttt{EA53}\\
\mSymbol[outlined]{balance} & \mSymbol[rounded]{balance} & \mSymbol[sharp]{balance} & \texttt{\textbackslash mSymbol\{balance\}} & \texttt{EAF6}\\
\mSymbol[outlined]{balcony} & \mSymbol[rounded]{balcony} & \mSymbol[sharp]{balcony} & \texttt{\textbackslash mSymbol\{balcony\}} & \texttt{E58F}\\
\mSymbol[outlined]{ballot} & \mSymbol[rounded]{ballot} & \mSymbol[sharp]{ballot} & \texttt{\textbackslash mSymbol\{ballot\}} & \texttt{E172}\\
\mSymbol[outlined]{bar-chart} & \mSymbol[rounded]{bar-chart} & \mSymbol[sharp]{bar-chart} & \texttt{\textbackslash mSymbol\{bar-chart\}} & \texttt{E26B}\\
\mSymbol[outlined]{bar-chart-4-bars} & \mSymbol[rounded]{bar-chart-4-bars} & \mSymbol[sharp]{bar-chart-4-bars} & \texttt{\textbackslash mSymbol\{bar-chart-4-bars\}} & \texttt{F681}\\
\mSymbol[outlined]{bar-chart-off} & \mSymbol[rounded]{bar-chart-off} & \mSymbol[sharp]{bar-chart-off} & \texttt{\textbackslash mSymbol\{bar-chart-off\}} & \texttt{F411}\\
\mSymbol[outlined]{barcode} & \mSymbol[rounded]{barcode} & \mSymbol[sharp]{barcode} & \texttt{\textbackslash mSymbol\{barcode\}} & \texttt{E70B}\\
\mSymbol[outlined]{barcode-reader} & \mSymbol[rounded]{barcode-reader} & \mSymbol[sharp]{barcode-reader} & \texttt{\textbackslash mSymbol\{barcode-reader\}} & \texttt{F85C}\\
\mSymbol[outlined]{barcode-scanner} & \mSymbol[rounded]{barcode-scanner} & \mSymbol[sharp]{barcode-scanner} & \texttt{\textbackslash mSymbol\{barcode-scanner\}} & \texttt{E70C}\\
\mSymbol[outlined]{barefoot} & \mSymbol[rounded]{barefoot} & \mSymbol[sharp]{barefoot} & \texttt{\textbackslash mSymbol\{barefoot\}} & \texttt{F871}\\
\mSymbol[outlined]{batch-prediction} & \mSymbol[rounded]{batch-prediction} & \mSymbol[sharp]{batch-prediction} & \texttt{\textbackslash mSymbol\{batch-prediction\}} & \texttt{F0F5}\\
\mSymbol[outlined]{bath-outdoor} & \mSymbol[rounded]{bath-outdoor} & \mSymbol[sharp]{bath-outdoor} & \texttt{\textbackslash mSymbol\{bath-outdoor\}} & \texttt{F6FB}\\
\mSymbol[outlined]{bath-private} & \mSymbol[rounded]{bath-private} & \mSymbol[sharp]{bath-private} & \texttt{\textbackslash mSymbol\{bath-private\}} & \texttt{F6FA}\\
\mSymbol[outlined]{bath-public-large} & \mSymbol[rounded]{bath-public-large} & \mSymbol[sharp]{bath-public-large} & \texttt{\textbackslash mSymbol\{bath-public-large\}} & \texttt{F6F9}\\
\mSymbol[outlined]{bathroom} & \mSymbol[rounded]{bathroom} & \mSymbol[sharp]{bathroom} & \texttt{\textbackslash mSymbol\{bathroom\}} & \texttt{EFDD}\\
\mSymbol[outlined]{bathtub} & \mSymbol[rounded]{bathtub} & \mSymbol[sharp]{bathtub} & \texttt{\textbackslash mSymbol\{bathtub\}} & \texttt{EA41}\\
\mSymbol[outlined]{battery-0-bar} & \mSymbol[rounded]{battery-0-bar} & \mSymbol[sharp]{battery-0-bar} & \texttt{\textbackslash mSymbol\{battery-0-bar\}} & \texttt{EBDC}\\
\mSymbol[outlined]{battery-1-bar} & \mSymbol[rounded]{battery-1-bar} & \mSymbol[sharp]{battery-1-bar} & \texttt{\textbackslash mSymbol\{battery-1-bar\}} & \texttt{F09C}\\
\mSymbol[outlined]{battery-20} & \mSymbol[rounded]{battery-20} & \mSymbol[sharp]{battery-20} & \texttt{\textbackslash mSymbol\{battery-20\}} & \texttt{F09C}\\
\mSymbol[outlined]{battery-2-bar} & \mSymbol[rounded]{battery-2-bar} & \mSymbol[sharp]{battery-2-bar} & \texttt{\textbackslash mSymbol\{battery-2-bar\}} & \texttt{F09D}\\
\mSymbol[outlined]{battery-30} & \mSymbol[rounded]{battery-30} & \mSymbol[sharp]{battery-30} & \texttt{\textbackslash mSymbol\{battery-30\}} & \texttt{F09D}\\
\mSymbol[outlined]{battery-3-bar} & \mSymbol[rounded]{battery-3-bar} & \mSymbol[sharp]{battery-3-bar} & \texttt{\textbackslash mSymbol\{battery-3-bar\}} & \texttt{F09E}\\
\mSymbol[outlined]{battery-4-bar} & \mSymbol[rounded]{battery-4-bar} & \mSymbol[sharp]{battery-4-bar} & \texttt{\textbackslash mSymbol\{battery-4-bar\}} & \texttt{F09F}\\
\mSymbol[outlined]{battery-50} & \mSymbol[rounded]{battery-50} & \mSymbol[sharp]{battery-50} & \texttt{\textbackslash mSymbol\{battery-50\}} & \texttt{F09E}\\
\mSymbol[outlined]{battery-5-bar} & \mSymbol[rounded]{battery-5-bar} & \mSymbol[sharp]{battery-5-bar} & \texttt{\textbackslash mSymbol\{battery-5-bar\}} & \texttt{F0A0}\\
\mSymbol[outlined]{battery-60} & \mSymbol[rounded]{battery-60} & \mSymbol[sharp]{battery-60} & \texttt{\textbackslash mSymbol\{battery-60\}} & \texttt{F09F}\\
\mSymbol[outlined]{battery-6-bar} & \mSymbol[rounded]{battery-6-bar} & \mSymbol[sharp]{battery-6-bar} & \texttt{\textbackslash mSymbol\{battery-6-bar\}} & \texttt{F0A1}\\
\mSymbol[outlined]{battery-80} & \mSymbol[rounded]{battery-80} & \mSymbol[sharp]{battery-80} & \texttt{\textbackslash mSymbol\{battery-80\}} & \texttt{F0A0}\\
\mSymbol[outlined]{battery-90} & \mSymbol[rounded]{battery-90} & \mSymbol[sharp]{battery-90} & \texttt{\textbackslash mSymbol\{battery-90\}} & \texttt{F0A1}\\
\mSymbol[outlined]{battery-alert} & \mSymbol[rounded]{battery-alert} & \mSymbol[sharp]{battery-alert} & \texttt{\textbackslash mSymbol\{battery-alert\}} & \texttt{E19C}\\
\mSymbol[outlined]{battery-change} & \mSymbol[rounded]{battery-change} & \mSymbol[sharp]{battery-change} & \texttt{\textbackslash mSymbol\{battery-change\}} & \texttt{F7EB}\\
\mSymbol[outlined]{battery-charging-20} & \mSymbol[rounded]{battery-charging-20} & \mSymbol[sharp]{battery-charging-20} & \texttt{\textbackslash mSymbol\{battery-charging-20\}} & \texttt{F0A2}\\
\mSymbol[outlined]{battery-charging-30} & \mSymbol[rounded]{battery-charging-30} & \mSymbol[sharp]{battery-charging-30} & \texttt{\textbackslash mSymbol\{battery-charging-30\}} & \texttt{F0A3}\\
\mSymbol[outlined]{battery-charging-50} & \mSymbol[rounded]{battery-charging-50} & \mSymbol[sharp]{battery-charging-50} & \texttt{\textbackslash mSymbol\{battery-charging-50\}} & \texttt{F0A4}\\
\mSymbol[outlined]{battery-charging-60} & \mSymbol[rounded]{battery-charging-60} & \mSymbol[sharp]{battery-charging-60} & \texttt{\textbackslash mSymbol\{battery-charging-60\}} & \texttt{F0A5}\\
\mSymbol[outlined]{battery-charging-80} & \mSymbol[rounded]{battery-charging-80} & \mSymbol[sharp]{battery-charging-80} & \texttt{\textbackslash mSymbol\{battery-charging-80\}} & \texttt{F0A6}\\
\mSymbol[outlined]{battery-charging-90} & \mSymbol[rounded]{battery-charging-90} & \mSymbol[sharp]{battery-charging-90} & \texttt{\textbackslash mSymbol\{battery-charging-90\}} & \texttt{F0A7}\\
\mSymbol[outlined]{battery-charging-full} & \mSymbol[rounded]{battery-charging-full} & \mSymbol[sharp]{battery-charging-full} & \texttt{\textbackslash mSymbol\{battery-charging-full\}} & \texttt{E1A3}\\
\mSymbol[outlined]{battery-error} & \mSymbol[rounded]{battery-error} & \mSymbol[sharp]{battery-error} & \texttt{\textbackslash mSymbol\{battery-error\}} & \texttt{F7EA}\\
\mSymbol[outlined]{battery-full} & \mSymbol[rounded]{battery-full} & \mSymbol[sharp]{battery-full} & \texttt{\textbackslash mSymbol\{battery-full\}} & \texttt{E1A5}\\
\mSymbol[outlined]{battery-full-alt} & \mSymbol[rounded]{battery-full-alt} & \mSymbol[sharp]{battery-full-alt} & \texttt{\textbackslash mSymbol\{battery-full-alt\}} & \texttt{F13B}\\
\mSymbol[outlined]{battery-horiz-000} & \mSymbol[rounded]{battery-horiz-000} & \mSymbol[sharp]{battery-horiz-000} & \texttt{\textbackslash mSymbol\{battery-horiz-000\}} & \texttt{F8AE}\\
\mSymbol[outlined]{battery-horiz-050} & \mSymbol[rounded]{battery-horiz-050} & \mSymbol[sharp]{battery-horiz-050} & \texttt{\textbackslash mSymbol\{battery-horiz-050\}} & \texttt{F8AF}\\
\mSymbol[outlined]{battery-horiz-075} & \mSymbol[rounded]{battery-horiz-075} & \mSymbol[sharp]{battery-horiz-075} & \texttt{\textbackslash mSymbol\{battery-horiz-075\}} & \texttt{F8B0}\\
\mSymbol[outlined]{battery-low} & \mSymbol[rounded]{battery-low} & \mSymbol[sharp]{battery-low} & \texttt{\textbackslash mSymbol\{battery-low\}} & \texttt{F155}\\
\mSymbol[outlined]{battery-plus} & \mSymbol[rounded]{battery-plus} & \mSymbol[sharp]{battery-plus} & \texttt{\textbackslash mSymbol\{battery-plus\}} & \texttt{F7E9}\\
\mSymbol[outlined]{battery-profile} & \mSymbol[rounded]{battery-profile} & \mSymbol[sharp]{battery-profile} & \texttt{\textbackslash mSymbol\{battery-profile\}} & \texttt{E206}\\
\mSymbol[outlined]{battery-saver} & \mSymbol[rounded]{battery-saver} & \mSymbol[sharp]{battery-saver} & \texttt{\textbackslash mSymbol\{battery-saver\}} & \texttt{EFDE}\\
\mSymbol[outlined]{battery-share} & \mSymbol[rounded]{battery-share} & \mSymbol[sharp]{battery-share} & \texttt{\textbackslash mSymbol\{battery-share\}} & \texttt{F67E}\\
\mSymbol[outlined]{battery-status-good} & \mSymbol[rounded]{battery-status-good} & \mSymbol[sharp]{battery-status-good} & \texttt{\textbackslash mSymbol\{battery-status-good\}} & \texttt{F67D}\\
\mSymbol[outlined]{battery-std} & \mSymbol[rounded]{battery-std} & \mSymbol[sharp]{battery-std} & \texttt{\textbackslash mSymbol\{battery-std\}} & \texttt{E1A5}\\
\mSymbol[outlined]{battery-unknown} & \mSymbol[rounded]{battery-unknown} & \mSymbol[sharp]{battery-unknown} & \texttt{\textbackslash mSymbol\{battery-unknown\}} & \texttt{E1A6}\\
\mSymbol[outlined]{battery-vert-005} & \mSymbol[rounded]{battery-vert-005} & \mSymbol[sharp]{battery-vert-005} & \texttt{\textbackslash mSymbol\{battery-vert-005\}} & \texttt{F8B1}\\
\mSymbol[outlined]{battery-vert-020} & \mSymbol[rounded]{battery-vert-020} & \mSymbol[sharp]{battery-vert-020} & \texttt{\textbackslash mSymbol\{battery-vert-020\}} & \texttt{F8B2}\\
\mSymbol[outlined]{battery-vert-050} & \mSymbol[rounded]{battery-vert-050} & \mSymbol[sharp]{battery-vert-050} & \texttt{\textbackslash mSymbol\{battery-vert-050\}} & \texttt{F8B3}\\
\mSymbol[outlined]{battery-very-low} & \mSymbol[rounded]{battery-very-low} & \mSymbol[sharp]{battery-very-low} & \texttt{\textbackslash mSymbol\{battery-very-low\}} & \texttt{F156}\\
\mSymbol[outlined]{beach-access} & \mSymbol[rounded]{beach-access} & \mSymbol[sharp]{beach-access} & \texttt{\textbackslash mSymbol\{beach-access\}} & \texttt{EB3E}\\
\mSymbol[outlined]{bed} & \mSymbol[rounded]{bed} & \mSymbol[sharp]{bed} & \texttt{\textbackslash mSymbol\{bed\}} & \texttt{EFDF}\\
\mSymbol[outlined]{bedroom-baby} & \mSymbol[rounded]{bedroom-baby} & \mSymbol[sharp]{bedroom-baby} & \texttt{\textbackslash mSymbol\{bedroom-baby\}} & \texttt{EFE0}\\
\mSymbol[outlined]{bedroom-child} & \mSymbol[rounded]{bedroom-child} & \mSymbol[sharp]{bedroom-child} & \texttt{\textbackslash mSymbol\{bedroom-child\}} & \texttt{EFE1}\\
\mSymbol[outlined]{bedroom-parent} & \mSymbol[rounded]{bedroom-parent} & \mSymbol[sharp]{bedroom-parent} & \texttt{\textbackslash mSymbol\{bedroom-parent\}} & \texttt{EFE2}\\
\mSymbol[outlined]{bedtime} & \mSymbol[rounded]{bedtime} & \mSymbol[sharp]{bedtime} & \texttt{\textbackslash mSymbol\{bedtime\}} & \texttt{EF44}\\
\mSymbol[outlined]{bedtime-off} & \mSymbol[rounded]{bedtime-off} & \mSymbol[sharp]{bedtime-off} & \texttt{\textbackslash mSymbol\{bedtime-off\}} & \texttt{EB76}\\
\mSymbol[outlined]{beenhere} & \mSymbol[rounded]{beenhere} & \mSymbol[sharp]{beenhere} & \texttt{\textbackslash mSymbol\{beenhere\}} & \texttt{E52D}\\
\mSymbol[outlined]{bento} & \mSymbol[rounded]{bento} & \mSymbol[sharp]{bento} & \texttt{\textbackslash mSymbol\{bento\}} & \texttt{F1F4}\\
\mSymbol[outlined]{bia} & \mSymbol[rounded]{bia} & \mSymbol[sharp]{bia} & \texttt{\textbackslash mSymbol\{bia\}} & \texttt{F6EB}\\
\mSymbol[outlined]{bid-landscape} & \mSymbol[rounded]{bid-landscape} & \mSymbol[sharp]{bid-landscape} & \texttt{\textbackslash mSymbol\{bid-landscape\}} & \texttt{E678}\\
\mSymbol[outlined]{bid-landscape-disabled} & \mSymbol[rounded]{bid-landscape-disabled} & \mSymbol[sharp]{bid-landscape-disabled} & \texttt{\textbackslash mSymbol\{bid-landscape-disabled\}} & \texttt{EF81}\\
\mSymbol[outlined]{bigtop-updates} & \mSymbol[rounded]{bigtop-updates} & \mSymbol[sharp]{bigtop-updates} & \texttt{\textbackslash mSymbol\{bigtop-updates\}} & \texttt{E669}\\
\mSymbol[outlined]{bike-dock} & \mSymbol[rounded]{bike-dock} & \mSymbol[sharp]{bike-dock} & \texttt{\textbackslash mSymbol\{bike-dock\}} & \texttt{F47B}\\
\mSymbol[outlined]{bike-lane} & \mSymbol[rounded]{bike-lane} & \mSymbol[sharp]{bike-lane} & \texttt{\textbackslash mSymbol\{bike-lane\}} & \texttt{F47A}\\
\mSymbol[outlined]{bike-scooter} & \mSymbol[rounded]{bike-scooter} & \mSymbol[sharp]{bike-scooter} & \texttt{\textbackslash mSymbol\{bike-scooter\}} & \texttt{EF45}\\
\mSymbol[outlined]{biotech} & \mSymbol[rounded]{biotech} & \mSymbol[sharp]{biotech} & \texttt{\textbackslash mSymbol\{biotech\}} & \texttt{EA3A}\\
\mSymbol[outlined]{blanket} & \mSymbol[rounded]{blanket} & \mSymbol[sharp]{blanket} & \texttt{\textbackslash mSymbol\{blanket\}} & \texttt{E828}\\
\mSymbol[outlined]{blender} & \mSymbol[rounded]{blender} & \mSymbol[sharp]{blender} & \texttt{\textbackslash mSymbol\{blender\}} & \texttt{EFE3}\\
\mSymbol[outlined]{blind} & \mSymbol[rounded]{blind} & \mSymbol[sharp]{blind} & \texttt{\textbackslash mSymbol\{blind\}} & \texttt{F8D6}\\
\mSymbol[outlined]{blinds} & \mSymbol[rounded]{blinds} & \mSymbol[sharp]{blinds} & \texttt{\textbackslash mSymbol\{blinds\}} & \texttt{E286}\\
\mSymbol[outlined]{blinds-closed} & \mSymbol[rounded]{blinds-closed} & \mSymbol[sharp]{blinds-closed} & \texttt{\textbackslash mSymbol\{blinds-closed\}} & \texttt{EC1F}\\
\mSymbol[outlined]{block} & \mSymbol[rounded]{block} & \mSymbol[sharp]{block} & \texttt{\textbackslash mSymbol\{block\}} & \texttt{F08C}\\
\mSymbol[outlined]{blood-pressure} & \mSymbol[rounded]{blood-pressure} & \mSymbol[sharp]{blood-pressure} & \texttt{\textbackslash mSymbol\{blood-pressure\}} & \texttt{E097}\\
\mSymbol[outlined]{bloodtype} & \mSymbol[rounded]{bloodtype} & \mSymbol[sharp]{bloodtype} & \texttt{\textbackslash mSymbol\{bloodtype\}} & \texttt{EFE4}\\
\mSymbol[outlined]{bluetooth} & \mSymbol[rounded]{bluetooth} & \mSymbol[sharp]{bluetooth} & \texttt{\textbackslash mSymbol\{bluetooth\}} & \texttt{E1A7}\\
\mSymbol[outlined]{bluetooth-audio} & \mSymbol[rounded]{bluetooth-audio} & \mSymbol[sharp]{bluetooth-audio} & \texttt{\textbackslash mSymbol\{bluetooth-audio\}} & \texttt{E60F}\\
\mSymbol[outlined]{bluetooth-connected} & \mSymbol[rounded]{bluetooth-connected} & \mSymbol[sharp]{bluetooth-connected} & \texttt{\textbackslash mSymbol\{bluetooth-connected\}} & \texttt{E1A8}\\
\mSymbol[outlined]{bluetooth-disabled} & \mSymbol[rounded]{bluetooth-disabled} & \mSymbol[sharp]{bluetooth-disabled} & \texttt{\textbackslash mSymbol\{bluetooth-disabled\}} & \texttt{E1A9}\\
\mSymbol[outlined]{bluetooth-drive} & \mSymbol[rounded]{bluetooth-drive} & \mSymbol[sharp]{bluetooth-drive} & \texttt{\textbackslash mSymbol\{bluetooth-drive\}} & \texttt{EFE5}\\
\mSymbol[outlined]{bluetooth-searching} & \mSymbol[rounded]{bluetooth-searching} & \mSymbol[sharp]{bluetooth-searching} & \texttt{\textbackslash mSymbol\{bluetooth-searching\}} & \texttt{E60F}\\
\mSymbol[outlined]{blur-circular} & \mSymbol[rounded]{blur-circular} & \mSymbol[sharp]{blur-circular} & \texttt{\textbackslash mSymbol\{blur-circular\}} & \texttt{E3A2}\\
\mSymbol[outlined]{blur-linear} & \mSymbol[rounded]{blur-linear} & \mSymbol[sharp]{blur-linear} & \texttt{\textbackslash mSymbol\{blur-linear\}} & \texttt{E3A3}\\
\mSymbol[outlined]{blur-medium} & \mSymbol[rounded]{blur-medium} & \mSymbol[sharp]{blur-medium} & \texttt{\textbackslash mSymbol\{blur-medium\}} & \texttt{E84C}\\
\mSymbol[outlined]{blur-off} & \mSymbol[rounded]{blur-off} & \mSymbol[sharp]{blur-off} & \texttt{\textbackslash mSymbol\{blur-off\}} & \texttt{E3A4}\\
\mSymbol[outlined]{blur-on} & \mSymbol[rounded]{blur-on} & \mSymbol[sharp]{blur-on} & \texttt{\textbackslash mSymbol\{blur-on\}} & \texttt{E3A5}\\
\mSymbol[outlined]{blur-short} & \mSymbol[rounded]{blur-short} & \mSymbol[sharp]{blur-short} & \texttt{\textbackslash mSymbol\{blur-short\}} & \texttt{E8CF}\\
\mSymbol[outlined]{body-fat} & \mSymbol[rounded]{body-fat} & \mSymbol[sharp]{body-fat} & \texttt{\textbackslash mSymbol\{body-fat\}} & \texttt{E098}\\
\mSymbol[outlined]{body-system} & \mSymbol[rounded]{body-system} & \mSymbol[sharp]{body-system} & \texttt{\textbackslash mSymbol\{body-system\}} & \texttt{E099}\\
\mSymbol[outlined]{bolt} & \mSymbol[rounded]{bolt} & \mSymbol[sharp]{bolt} & \texttt{\textbackslash mSymbol\{bolt\}} & \texttt{EA0B}\\
\mSymbol[outlined]{bomb} & \mSymbol[rounded]{bomb} & \mSymbol[sharp]{bomb} & \texttt{\textbackslash mSymbol\{bomb\}} & \texttt{F568}\\
\mSymbol[outlined]{book} & \mSymbol[rounded]{book} & \mSymbol[sharp]{book} & \texttt{\textbackslash mSymbol\{book\}} & \texttt{E86E}\\
\mSymbol[outlined]{book-2} & \mSymbol[rounded]{book-2} & \mSymbol[sharp]{book-2} & \texttt{\textbackslash mSymbol\{book-2\}} & \texttt{F53E}\\
\mSymbol[outlined]{book-3} & \mSymbol[rounded]{book-3} & \mSymbol[sharp]{book-3} & \texttt{\textbackslash mSymbol\{book-3\}} & \texttt{F53D}\\
\mSymbol[outlined]{book-4} & \mSymbol[rounded]{book-4} & \mSymbol[sharp]{book-4} & \texttt{\textbackslash mSymbol\{book-4\}} & \texttt{F53C}\\
\mSymbol[outlined]{book-5} & \mSymbol[rounded]{book-5} & \mSymbol[sharp]{book-5} & \texttt{\textbackslash mSymbol\{book-5\}} & \texttt{F53B}\\
\mSymbol[outlined]{book-online} & \mSymbol[rounded]{book-online} & \mSymbol[sharp]{book-online} & \texttt{\textbackslash mSymbol\{book-online\}} & \texttt{F217}\\
\mSymbol[outlined]{bookmark} & \mSymbol[rounded]{bookmark} & \mSymbol[sharp]{bookmark} & \texttt{\textbackslash mSymbol\{bookmark\}} & \texttt{E8E7}\\
\mSymbol[outlined]{bookmark-add} & \mSymbol[rounded]{bookmark-add} & \mSymbol[sharp]{bookmark-add} & \texttt{\textbackslash mSymbol\{bookmark-add\}} & \texttt{E598}\\
\mSymbol[outlined]{bookmark-added} & \mSymbol[rounded]{bookmark-added} & \mSymbol[sharp]{bookmark-added} & \texttt{\textbackslash mSymbol\{bookmark-added\}} & \texttt{E599}\\
\mSymbol[outlined]{bookmark-bag} & \mSymbol[rounded]{bookmark-bag} & \mSymbol[sharp]{bookmark-bag} & \texttt{\textbackslash mSymbol\{bookmark-bag\}} & \texttt{F410}\\
\mSymbol[outlined]{bookmark-border} & \mSymbol[rounded]{bookmark-border} & \mSymbol[sharp]{bookmark-border} & \texttt{\textbackslash mSymbol\{bookmark-border\}} & \texttt{E8E7}\\
\mSymbol[outlined]{bookmark-check} & \mSymbol[rounded]{bookmark-check} & \mSymbol[sharp]{bookmark-check} & \texttt{\textbackslash mSymbol\{bookmark-check\}} & \texttt{F457}\\
\mSymbol[outlined]{bookmark-flag} & \mSymbol[rounded]{bookmark-flag} & \mSymbol[sharp]{bookmark-flag} & \texttt{\textbackslash mSymbol\{bookmark-flag\}} & \texttt{F456}\\
\mSymbol[outlined]{bookmark-heart} & \mSymbol[rounded]{bookmark-heart} & \mSymbol[sharp]{bookmark-heart} & \texttt{\textbackslash mSymbol\{bookmark-heart\}} & \texttt{F455}\\
\mSymbol[outlined]{bookmark-manager} & \mSymbol[rounded]{bookmark-manager} & \mSymbol[sharp]{bookmark-manager} & \texttt{\textbackslash mSymbol\{bookmark-manager\}} & \texttt{F7B1}\\
\mSymbol[outlined]{bookmark-remove} & \mSymbol[rounded]{bookmark-remove} & \mSymbol[sharp]{bookmark-remove} & \texttt{\textbackslash mSymbol\{bookmark-remove\}} & \texttt{E59A}\\
\mSymbol[outlined]{bookmark-star} & \mSymbol[rounded]{bookmark-star} & \mSymbol[sharp]{bookmark-star} & \texttt{\textbackslash mSymbol\{bookmark-star\}} & \texttt{F454}\\
\mSymbol[outlined]{bookmarks} & \mSymbol[rounded]{bookmarks} & \mSymbol[sharp]{bookmarks} & \texttt{\textbackslash mSymbol\{bookmarks\}} & \texttt{E98B}\\
\mSymbol[outlined]{border-all} & \mSymbol[rounded]{border-all} & \mSymbol[sharp]{border-all} & \texttt{\textbackslash mSymbol\{border-all\}} & \texttt{E228}\\
\mSymbol[outlined]{border-bottom} & \mSymbol[rounded]{border-bottom} & \mSymbol[sharp]{border-bottom} & \texttt{\textbackslash mSymbol\{border-bottom\}} & \texttt{E229}\\
\mSymbol[outlined]{border-clear} & \mSymbol[rounded]{border-clear} & \mSymbol[sharp]{border-clear} & \texttt{\textbackslash mSymbol\{border-clear\}} & \texttt{E22A}\\
\mSymbol[outlined]{border-color} & \mSymbol[rounded]{border-color} & \mSymbol[sharp]{border-color} & \texttt{\textbackslash mSymbol\{border-color\}} & \texttt{E22B}\\
\mSymbol[outlined]{border-horizontal} & \mSymbol[rounded]{border-horizontal} & \mSymbol[sharp]{border-horizontal} & \texttt{\textbackslash mSymbol\{border-horizontal\}} & \texttt{E22C}\\
\mSymbol[outlined]{border-inner} & \mSymbol[rounded]{border-inner} & \mSymbol[sharp]{border-inner} & \texttt{\textbackslash mSymbol\{border-inner\}} & \texttt{E22D}\\
\mSymbol[outlined]{border-left} & \mSymbol[rounded]{border-left} & \mSymbol[sharp]{border-left} & \texttt{\textbackslash mSymbol\{border-left\}} & \texttt{E22E}\\
\mSymbol[outlined]{border-outer} & \mSymbol[rounded]{border-outer} & \mSymbol[sharp]{border-outer} & \texttt{\textbackslash mSymbol\{border-outer\}} & \texttt{E22F}\\
\mSymbol[outlined]{border-right} & \mSymbol[rounded]{border-right} & \mSymbol[sharp]{border-right} & \texttt{\textbackslash mSymbol\{border-right\}} & \texttt{E230}\\
\mSymbol[outlined]{border-style} & \mSymbol[rounded]{border-style} & \mSymbol[sharp]{border-style} & \texttt{\textbackslash mSymbol\{border-style\}} & \texttt{E231}\\
\mSymbol[outlined]{border-top} & \mSymbol[rounded]{border-top} & \mSymbol[sharp]{border-top} & \texttt{\textbackslash mSymbol\{border-top\}} & \texttt{E232}\\
\mSymbol[outlined]{border-vertical} & \mSymbol[rounded]{border-vertical} & \mSymbol[sharp]{border-vertical} & \texttt{\textbackslash mSymbol\{border-vertical\}} & \texttt{E233}\\
\mSymbol[outlined]{borg} & \mSymbol[rounded]{borg} & \mSymbol[sharp]{borg} & \texttt{\textbackslash mSymbol\{borg\}} & \texttt{F40D}\\
\mSymbol[outlined]{bottom-app-bar} & \mSymbol[rounded]{bottom-app-bar} & \mSymbol[sharp]{bottom-app-bar} & \texttt{\textbackslash mSymbol\{bottom-app-bar\}} & \texttt{E730}\\
\mSymbol[outlined]{bottom-drawer} & \mSymbol[rounded]{bottom-drawer} & \mSymbol[sharp]{bottom-drawer} & \texttt{\textbackslash mSymbol\{bottom-drawer\}} & \texttt{E72D}\\
\mSymbol[outlined]{bottom-navigation} & \mSymbol[rounded]{bottom-navigation} & \mSymbol[sharp]{bottom-navigation} & \texttt{\textbackslash mSymbol\{bottom-navigation\}} & \texttt{E98C}\\
\mSymbol[outlined]{bottom-panel-close} & \mSymbol[rounded]{bottom-panel-close} & \mSymbol[sharp]{bottom-panel-close} & \texttt{\textbackslash mSymbol\{bottom-panel-close\}} & \texttt{F72A}\\
\mSymbol[outlined]{bottom-panel-open} & \mSymbol[rounded]{bottom-panel-open} & \mSymbol[sharp]{bottom-panel-open} & \texttt{\textbackslash mSymbol\{bottom-panel-open\}} & \texttt{F729}\\
\mSymbol[outlined]{bottom-right-click} & \mSymbol[rounded]{bottom-right-click} & \mSymbol[sharp]{bottom-right-click} & \texttt{\textbackslash mSymbol\{bottom-right-click\}} & \texttt{F684}\\
\mSymbol[outlined]{bottom-sheets} & \mSymbol[rounded]{bottom-sheets} & \mSymbol[sharp]{bottom-sheets} & \texttt{\textbackslash mSymbol\{bottom-sheets\}} & \texttt{E98D}\\
\mSymbol[outlined]{box} & \mSymbol[rounded]{box} & \mSymbol[sharp]{box} & \texttt{\textbackslash mSymbol\{box\}} & \texttt{F5A4}\\
\mSymbol[outlined]{box-add} & \mSymbol[rounded]{box-add} & \mSymbol[sharp]{box-add} & \texttt{\textbackslash mSymbol\{box-add\}} & \texttt{F5A5}\\
\mSymbol[outlined]{box-edit} & \mSymbol[rounded]{box-edit} & \mSymbol[sharp]{box-edit} & \texttt{\textbackslash mSymbol\{box-edit\}} & \texttt{F5A6}\\
\mSymbol[outlined]{boy} & \mSymbol[rounded]{boy} & \mSymbol[sharp]{boy} & \texttt{\textbackslash mSymbol\{boy\}} & \texttt{EB67}\\
\mSymbol[outlined]{brand-awareness} & \mSymbol[rounded]{brand-awareness} & \mSymbol[sharp]{brand-awareness} & \texttt{\textbackslash mSymbol\{brand-awareness\}} & \texttt{E98E}\\
\mSymbol[outlined]{brand-family} & \mSymbol[rounded]{brand-family} & \mSymbol[sharp]{brand-family} & \texttt{\textbackslash mSymbol\{brand-family\}} & \texttt{F4F1}\\
\mSymbol[outlined]{branding-watermark} & \mSymbol[rounded]{branding-watermark} & \mSymbol[sharp]{branding-watermark} & \texttt{\textbackslash mSymbol\{branding-watermark\}} & \texttt{E06B}\\
\mSymbol[outlined]{breakfast-dining} & \mSymbol[rounded]{breakfast-dining} & \mSymbol[sharp]{breakfast-dining} & \texttt{\textbackslash mSymbol\{breakfast-dining\}} & \texttt{EA54}\\
\mSymbol[outlined]{breaking-news} & \mSymbol[rounded]{breaking-news} & \mSymbol[sharp]{breaking-news} & \texttt{\textbackslash mSymbol\{breaking-news\}} & \texttt{EA08}\\
\mSymbol[outlined]{breaking-news-alt-1} & \mSymbol[rounded]{breaking-news-alt-1} & \mSymbol[sharp]{breaking-news-alt-1} & \texttt{\textbackslash mSymbol\{breaking-news-alt-1\}} & \texttt{F0BA}\\
\mSymbol[outlined]{breastfeeding} & \mSymbol[rounded]{breastfeeding} & \mSymbol[sharp]{breastfeeding} & \texttt{\textbackslash mSymbol\{breastfeeding\}} & \texttt{F856}\\
\mSymbol[outlined]{brightness-1} & \mSymbol[rounded]{brightness-1} & \mSymbol[sharp]{brightness-1} & \texttt{\textbackslash mSymbol\{brightness-1\}} & \texttt{E3FA}\\
\mSymbol[outlined]{brightness-2} & \mSymbol[rounded]{brightness-2} & \mSymbol[sharp]{brightness-2} & \texttt{\textbackslash mSymbol\{brightness-2\}} & \texttt{F036}\\
\mSymbol[outlined]{brightness-3} & \mSymbol[rounded]{brightness-3} & \mSymbol[sharp]{brightness-3} & \texttt{\textbackslash mSymbol\{brightness-3\}} & \texttt{E3A8}\\
\mSymbol[outlined]{brightness-4} & \mSymbol[rounded]{brightness-4} & \mSymbol[sharp]{brightness-4} & \texttt{\textbackslash mSymbol\{brightness-4\}} & \texttt{E3A9}\\
\mSymbol[outlined]{brightness-5} & \mSymbol[rounded]{brightness-5} & \mSymbol[sharp]{brightness-5} & \texttt{\textbackslash mSymbol\{brightness-5\}} & \texttt{E3AA}\\
\mSymbol[outlined]{brightness-6} & \mSymbol[rounded]{brightness-6} & \mSymbol[sharp]{brightness-6} & \texttt{\textbackslash mSymbol\{brightness-6\}} & \texttt{E3AB}\\
\mSymbol[outlined]{brightness-7} & \mSymbol[rounded]{brightness-7} & \mSymbol[sharp]{brightness-7} & \texttt{\textbackslash mSymbol\{brightness-7\}} & \texttt{E3AC}\\
\mSymbol[outlined]{brightness-alert} & \mSymbol[rounded]{brightness-alert} & \mSymbol[sharp]{brightness-alert} & \texttt{\textbackslash mSymbol\{brightness-alert\}} & \texttt{F5CF}\\
\mSymbol[outlined]{brightness-auto} & \mSymbol[rounded]{brightness-auto} & \mSymbol[sharp]{brightness-auto} & \texttt{\textbackslash mSymbol\{brightness-auto\}} & \texttt{E1AB}\\
\mSymbol[outlined]{brightness-empty} & \mSymbol[rounded]{brightness-empty} & \mSymbol[sharp]{brightness-empty} & \texttt{\textbackslash mSymbol\{brightness-empty\}} & \texttt{F7E8}\\
\mSymbol[outlined]{brightness-high} & \mSymbol[rounded]{brightness-high} & \mSymbol[sharp]{brightness-high} & \texttt{\textbackslash mSymbol\{brightness-high\}} & \texttt{E1AC}\\
\mSymbol[outlined]{brightness-low} & \mSymbol[rounded]{brightness-low} & \mSymbol[sharp]{brightness-low} & \texttt{\textbackslash mSymbol\{brightness-low\}} & \texttt{E1AD}\\
\mSymbol[outlined]{brightness-medium} & \mSymbol[rounded]{brightness-medium} & \mSymbol[sharp]{brightness-medium} & \texttt{\textbackslash mSymbol\{brightness-medium\}} & \texttt{E1AE}\\
\mSymbol[outlined]{bring-your-own-ip} & \mSymbol[rounded]{bring-your-own-ip} & \mSymbol[sharp]{bring-your-own-ip} & \texttt{\textbackslash mSymbol\{bring-your-own-ip\}} & \texttt{E016}\\
\mSymbol[outlined]{broadcast-on-home} & \mSymbol[rounded]{broadcast-on-home} & \mSymbol[sharp]{broadcast-on-home} & \texttt{\textbackslash mSymbol\{broadcast-on-home\}} & \texttt{F8F8}\\
\mSymbol[outlined]{broadcast-on-personal} & \mSymbol[rounded]{broadcast-on-personal} & \mSymbol[sharp]{broadcast-on-personal} & \texttt{\textbackslash mSymbol\{broadcast-on-personal\}} & \texttt{F8F9}\\
\mSymbol[outlined]{broken-image} & \mSymbol[rounded]{broken-image} & \mSymbol[sharp]{broken-image} & \texttt{\textbackslash mSymbol\{broken-image\}} & \texttt{E3AD}\\
\mSymbol[outlined]{browse} & \mSymbol[rounded]{browse} & \mSymbol[sharp]{browse} & \texttt{\textbackslash mSymbol\{browse\}} & \texttt{EB13}\\
\mSymbol[outlined]{browse-activity} & \mSymbol[rounded]{browse-activity} & \mSymbol[sharp]{browse-activity} & \texttt{\textbackslash mSymbol\{browse-activity\}} & \texttt{F8A5}\\
\mSymbol[outlined]{browse-gallery} & \mSymbol[rounded]{browse-gallery} & \mSymbol[sharp]{browse-gallery} & \texttt{\textbackslash mSymbol\{browse-gallery\}} & \texttt{EBD1}\\
\mSymbol[outlined]{browser-not-supported} & \mSymbol[rounded]{browser-not-supported} & \mSymbol[sharp]{browser-not-supported} & \texttt{\textbackslash mSymbol\{browser-not-supported\}} & \texttt{EF47}\\
\mSymbol[outlined]{browser-updated} & \mSymbol[rounded]{browser-updated} & \mSymbol[sharp]{browser-updated} & \texttt{\textbackslash mSymbol\{browser-updated\}} & \texttt{E7CF}\\
\mSymbol[outlined]{brunch-dining} & \mSymbol[rounded]{brunch-dining} & \mSymbol[sharp]{brunch-dining} & \texttt{\textbackslash mSymbol\{brunch-dining\}} & \texttt{EA73}\\
\mSymbol[outlined]{brush} & \mSymbol[rounded]{brush} & \mSymbol[sharp]{brush} & \texttt{\textbackslash mSymbol\{brush\}} & \texttt{E3AE}\\
\mSymbol[outlined]{bubble} & \mSymbol[rounded]{bubble} & \mSymbol[sharp]{bubble} & \texttt{\textbackslash mSymbol\{bubble\}} & \texttt{EF83}\\
\mSymbol[outlined]{bubble-chart} & \mSymbol[rounded]{bubble-chart} & \mSymbol[sharp]{bubble-chart} & \texttt{\textbackslash mSymbol\{bubble-chart\}} & \texttt{E6DD}\\
\mSymbol[outlined]{bubbles} & \mSymbol[rounded]{bubbles} & \mSymbol[sharp]{bubbles} & \texttt{\textbackslash mSymbol\{bubbles\}} & \texttt{F64E}\\
\mSymbol[outlined]{bug-report} & \mSymbol[rounded]{bug-report} & \mSymbol[sharp]{bug-report} & \texttt{\textbackslash mSymbol\{bug-report\}} & \texttt{E868}\\
\mSymbol[outlined]{build} & \mSymbol[rounded]{build} & \mSymbol[sharp]{build} & \texttt{\textbackslash mSymbol\{build\}} & \texttt{F8CD}\\
\mSymbol[outlined]{build-circle} & \mSymbol[rounded]{build-circle} & \mSymbol[sharp]{build-circle} & \texttt{\textbackslash mSymbol\{build-circle\}} & \texttt{EF48}\\
\mSymbol[outlined]{bungalow} & \mSymbol[rounded]{bungalow} & \mSymbol[sharp]{bungalow} & \texttt{\textbackslash mSymbol\{bungalow\}} & \texttt{E591}\\
\mSymbol[outlined]{burst-mode} & \mSymbol[rounded]{burst-mode} & \mSymbol[sharp]{burst-mode} & \texttt{\textbackslash mSymbol\{burst-mode\}} & \texttt{E43C}\\
\mSymbol[outlined]{bus-alert} & \mSymbol[rounded]{bus-alert} & \mSymbol[sharp]{bus-alert} & \texttt{\textbackslash mSymbol\{bus-alert\}} & \texttt{E98F}\\
\mSymbol[outlined]{business} & \mSymbol[rounded]{business} & \mSymbol[sharp]{business} & \texttt{\textbackslash mSymbol\{business\}} & \texttt{E7EE}\\
\mSymbol[outlined]{business-center} & \mSymbol[rounded]{business-center} & \mSymbol[sharp]{business-center} & \texttt{\textbackslash mSymbol\{business-center\}} & \texttt{EB3F}\\
\mSymbol[outlined]{business-chip} & \mSymbol[rounded]{business-chip} & \mSymbol[sharp]{business-chip} & \texttt{\textbackslash mSymbol\{business-chip\}} & \texttt{F84C}\\
\mSymbol[outlined]{business-messages} & \mSymbol[rounded]{business-messages} & \mSymbol[sharp]{business-messages} & \texttt{\textbackslash mSymbol\{business-messages\}} & \texttt{EF84}\\
\mSymbol[outlined]{buttons-alt} & \mSymbol[rounded]{buttons-alt} & \mSymbol[sharp]{buttons-alt} & \texttt{\textbackslash mSymbol\{buttons-alt\}} & \texttt{E72F}\\
\mSymbol[outlined]{cabin} & \mSymbol[rounded]{cabin} & \mSymbol[sharp]{cabin} & \texttt{\textbackslash mSymbol\{cabin\}} & \texttt{E589}\\
\mSymbol[outlined]{cable} & \mSymbol[rounded]{cable} & \mSymbol[sharp]{cable} & \texttt{\textbackslash mSymbol\{cable\}} & \texttt{EFE6}\\
\mSymbol[outlined]{cable-car} & \mSymbol[rounded]{cable-car} & \mSymbol[sharp]{cable-car} & \texttt{\textbackslash mSymbol\{cable-car\}} & \texttt{F479}\\
\mSymbol[outlined]{cached} & \mSymbol[rounded]{cached} & \mSymbol[sharp]{cached} & \texttt{\textbackslash mSymbol\{cached\}} & \texttt{E86A}\\
\mSymbol[outlined]{cadence} & \mSymbol[rounded]{cadence} & \mSymbol[sharp]{cadence} & \texttt{\textbackslash mSymbol\{cadence\}} & \texttt{F4B4}\\
\mSymbol[outlined]{cake} & \mSymbol[rounded]{cake} & \mSymbol[sharp]{cake} & \texttt{\textbackslash mSymbol\{cake\}} & \texttt{E7E9}\\
\mSymbol[outlined]{cake-add} & \mSymbol[rounded]{cake-add} & \mSymbol[sharp]{cake-add} & \texttt{\textbackslash mSymbol\{cake-add\}} & \texttt{F85B}\\
\mSymbol[outlined]{calculate} & \mSymbol[rounded]{calculate} & \mSymbol[sharp]{calculate} & \texttt{\textbackslash mSymbol\{calculate\}} & \texttt{EA5F}\\
\mSymbol[outlined]{calendar-add-on} & \mSymbol[rounded]{calendar-add-on} & \mSymbol[sharp]{calendar-add-on} & \texttt{\textbackslash mSymbol\{calendar-add-on\}} & \texttt{EF85}\\
\mSymbol[outlined]{calendar-apps-script} & \mSymbol[rounded]{calendar-apps-script} & \mSymbol[sharp]{calendar-apps-script} & \texttt{\textbackslash mSymbol\{calendar-apps-script\}} & \texttt{F0BB}\\
\mSymbol[outlined]{calendar-clock} & \mSymbol[rounded]{calendar-clock} & \mSymbol[sharp]{calendar-clock} & \texttt{\textbackslash mSymbol\{calendar-clock\}} & \texttt{F540}\\
\mSymbol[outlined]{calendar-month} & \mSymbol[rounded]{calendar-month} & \mSymbol[sharp]{calendar-month} & \texttt{\textbackslash mSymbol\{calendar-month\}} & \texttt{EBCC}\\
\mSymbol[outlined]{calendar-today} & \mSymbol[rounded]{calendar-today} & \mSymbol[sharp]{calendar-today} & \texttt{\textbackslash mSymbol\{calendar-today\}} & \texttt{E935}\\
\mSymbol[outlined]{calendar-view-day} & \mSymbol[rounded]{calendar-view-day} & \mSymbol[sharp]{calendar-view-day} & \texttt{\textbackslash mSymbol\{calendar-view-day\}} & \texttt{E936}\\
\mSymbol[outlined]{calendar-view-month} & \mSymbol[rounded]{calendar-view-month} & \mSymbol[sharp]{calendar-view-month} & \texttt{\textbackslash mSymbol\{calendar-view-month\}} & \texttt{EFE7}\\
\mSymbol[outlined]{calendar-view-week} & \mSymbol[rounded]{calendar-view-week} & \mSymbol[sharp]{calendar-view-week} & \texttt{\textbackslash mSymbol\{calendar-view-week\}} & \texttt{EFE8}\\
\mSymbol[outlined]{call} & \mSymbol[rounded]{call} & \mSymbol[sharp]{call} & \texttt{\textbackslash mSymbol\{call\}} & \texttt{F0D4}\\
\mSymbol[outlined]{call-end} & \mSymbol[rounded]{call-end} & \mSymbol[sharp]{call-end} & \texttt{\textbackslash mSymbol\{call-end\}} & \texttt{F0BC}\\
\mSymbol[outlined]{call-end-alt} & \mSymbol[rounded]{call-end-alt} & \mSymbol[sharp]{call-end-alt} & \texttt{\textbackslash mSymbol\{call-end-alt\}} & \texttt{F0BC}\\
\mSymbol[outlined]{call-log} & \mSymbol[rounded]{call-log} & \mSymbol[sharp]{call-log} & \texttt{\textbackslash mSymbol\{call-log\}} & \texttt{E08E}\\
\mSymbol[outlined]{call-made} & \mSymbol[rounded]{call-made} & \mSymbol[sharp]{call-made} & \texttt{\textbackslash mSymbol\{call-made\}} & \texttt{E0B2}\\
\mSymbol[outlined]{call-merge} & \mSymbol[rounded]{call-merge} & \mSymbol[sharp]{call-merge} & \texttt{\textbackslash mSymbol\{call-merge\}} & \texttt{E0B3}\\
\mSymbol[outlined]{call-missed} & \mSymbol[rounded]{call-missed} & \mSymbol[sharp]{call-missed} & \texttt{\textbackslash mSymbol\{call-missed\}} & \texttt{E0B4}\\
\mSymbol[outlined]{call-missed-outgoing} & \mSymbol[rounded]{call-missed-outgoing} & \mSymbol[sharp]{call-missed-outgoing} & \texttt{\textbackslash mSymbol\{call-missed-outgoing\}} & \texttt{E0E4}\\
\mSymbol[outlined]{call-quality} & \mSymbol[rounded]{call-quality} & \mSymbol[sharp]{call-quality} & \texttt{\textbackslash mSymbol\{call-quality\}} & \texttt{F652}\\
\mSymbol[outlined]{call-received} & \mSymbol[rounded]{call-received} & \mSymbol[sharp]{call-received} & \texttt{\textbackslash mSymbol\{call-received\}} & \texttt{E0B5}\\
\mSymbol[outlined]{call-split} & \mSymbol[rounded]{call-split} & \mSymbol[sharp]{call-split} & \texttt{\textbackslash mSymbol\{call-split\}} & \texttt{E0B6}\\
\mSymbol[outlined]{call-to-action} & \mSymbol[rounded]{call-to-action} & \mSymbol[sharp]{call-to-action} & \texttt{\textbackslash mSymbol\{call-to-action\}} & \texttt{E06C}\\
\mSymbol[outlined]{camera} & \mSymbol[rounded]{camera} & \mSymbol[sharp]{camera} & \texttt{\textbackslash mSymbol\{camera\}} & \texttt{E3AF}\\
\mSymbol[outlined]{camera-alt} & \mSymbol[rounded]{camera-alt} & \mSymbol[sharp]{camera-alt} & \texttt{\textbackslash mSymbol\{camera-alt\}} & \texttt{E412}\\
\mSymbol[outlined]{camera-enhance} & \mSymbol[rounded]{camera-enhance} & \mSymbol[sharp]{camera-enhance} & \texttt{\textbackslash mSymbol\{camera-enhance\}} & \texttt{E8FC}\\
\mSymbol[outlined]{camera-front} & \mSymbol[rounded]{camera-front} & \mSymbol[sharp]{camera-front} & \texttt{\textbackslash mSymbol\{camera-front\}} & \texttt{E3B1}\\
\mSymbol[outlined]{camera-indoor} & \mSymbol[rounded]{camera-indoor} & \mSymbol[sharp]{camera-indoor} & \texttt{\textbackslash mSymbol\{camera-indoor\}} & \texttt{EFE9}\\
\mSymbol[outlined]{camera-outdoor} & \mSymbol[rounded]{camera-outdoor} & \mSymbol[sharp]{camera-outdoor} & \texttt{\textbackslash mSymbol\{camera-outdoor\}} & \texttt{EFEA}\\
\mSymbol[outlined]{camera-rear} & \mSymbol[rounded]{camera-rear} & \mSymbol[sharp]{camera-rear} & \texttt{\textbackslash mSymbol\{camera-rear\}} & \texttt{E3B2}\\
\mSymbol[outlined]{camera-roll} & \mSymbol[rounded]{camera-roll} & \mSymbol[sharp]{camera-roll} & \texttt{\textbackslash mSymbol\{camera-roll\}} & \texttt{E3B3}\\
\mSymbol[outlined]{camera-video} & \mSymbol[rounded]{camera-video} & \mSymbol[sharp]{camera-video} & \texttt{\textbackslash mSymbol\{camera-video\}} & \texttt{F7A6}\\
\mSymbol[outlined]{cameraswitch} & \mSymbol[rounded]{cameraswitch} & \mSymbol[sharp]{cameraswitch} & \texttt{\textbackslash mSymbol\{cameraswitch\}} & \texttt{EFEB}\\
\mSymbol[outlined]{campaign} & \mSymbol[rounded]{campaign} & \mSymbol[sharp]{campaign} & \texttt{\textbackslash mSymbol\{campaign\}} & \texttt{EF49}\\
\mSymbol[outlined]{camping} & \mSymbol[rounded]{camping} & \mSymbol[sharp]{camping} & \texttt{\textbackslash mSymbol\{camping\}} & \texttt{F8A2}\\
\mSymbol[outlined]{cancel} & \mSymbol[rounded]{cancel} & \mSymbol[sharp]{cancel} & \texttt{\textbackslash mSymbol\{cancel\}} & \texttt{E888}\\
\mSymbol[outlined]{cancel-presentation} & \mSymbol[rounded]{cancel-presentation} & \mSymbol[sharp]{cancel-presentation} & \texttt{\textbackslash mSymbol\{cancel-presentation\}} & \texttt{E0E9}\\
\mSymbol[outlined]{cancel-schedule-send} & \mSymbol[rounded]{cancel-schedule-send} & \mSymbol[sharp]{cancel-schedule-send} & \texttt{\textbackslash mSymbol\{cancel-schedule-send\}} & \texttt{EA39}\\
\mSymbol[outlined]{candle} & \mSymbol[rounded]{candle} & \mSymbol[sharp]{candle} & \texttt{\textbackslash mSymbol\{candle\}} & \texttt{F588}\\
\mSymbol[outlined]{candlestick-chart} & \mSymbol[rounded]{candlestick-chart} & \mSymbol[sharp]{candlestick-chart} & \texttt{\textbackslash mSymbol\{candlestick-chart\}} & \texttt{EAD4}\\
\mSymbol[outlined]{captive-portal} & \mSymbol[rounded]{captive-portal} & \mSymbol[sharp]{captive-portal} & \texttt{\textbackslash mSymbol\{captive-portal\}} & \texttt{F728}\\
\mSymbol[outlined]{capture} & \mSymbol[rounded]{capture} & \mSymbol[sharp]{capture} & \texttt{\textbackslash mSymbol\{capture\}} & \texttt{F727}\\
\mSymbol[outlined]{car-crash} & \mSymbol[rounded]{car-crash} & \mSymbol[sharp]{car-crash} & \texttt{\textbackslash mSymbol\{car-crash\}} & \texttt{EBF2}\\
\mSymbol[outlined]{car-rental} & \mSymbol[rounded]{car-rental} & \mSymbol[sharp]{car-rental} & \texttt{\textbackslash mSymbol\{car-rental\}} & \texttt{EA55}\\
\mSymbol[outlined]{car-repair} & \mSymbol[rounded]{car-repair} & \mSymbol[sharp]{car-repair} & \texttt{\textbackslash mSymbol\{car-repair\}} & \texttt{EA56}\\
\mSymbol[outlined]{car-tag} & \mSymbol[rounded]{car-tag} & \mSymbol[sharp]{car-tag} & \texttt{\textbackslash mSymbol\{car-tag\}} & \texttt{F4E3}\\
\mSymbol[outlined]{card-giftcard} & \mSymbol[rounded]{card-giftcard} & \mSymbol[sharp]{card-giftcard} & \texttt{\textbackslash mSymbol\{card-giftcard\}} & \texttt{E8F6}\\
\mSymbol[outlined]{card-membership} & \mSymbol[rounded]{card-membership} & \mSymbol[sharp]{card-membership} & \texttt{\textbackslash mSymbol\{card-membership\}} & \texttt{E8F7}\\
\mSymbol[outlined]{card-travel} & \mSymbol[rounded]{card-travel} & \mSymbol[sharp]{card-travel} & \texttt{\textbackslash mSymbol\{card-travel\}} & \texttt{E8F8}\\
\mSymbol[outlined]{cardio-load} & \mSymbol[rounded]{cardio-load} & \mSymbol[sharp]{cardio-load} & \texttt{\textbackslash mSymbol\{cardio-load\}} & \texttt{F4B9}\\
\mSymbol[outlined]{cardiology} & \mSymbol[rounded]{cardiology} & \mSymbol[sharp]{cardiology} & \texttt{\textbackslash mSymbol\{cardiology\}} & \texttt{E09C}\\
\mSymbol[outlined]{cards} & \mSymbol[rounded]{cards} & \mSymbol[sharp]{cards} & \texttt{\textbackslash mSymbol\{cards\}} & \texttt{E991}\\
\mSymbol[outlined]{carpenter} & \mSymbol[rounded]{carpenter} & \mSymbol[sharp]{carpenter} & \texttt{\textbackslash mSymbol\{carpenter\}} & \texttt{F1F8}\\
\mSymbol[outlined]{carry-on-bag} & \mSymbol[rounded]{carry-on-bag} & \mSymbol[sharp]{carry-on-bag} & \texttt{\textbackslash mSymbol\{carry-on-bag\}} & \texttt{EB08}\\
\mSymbol[outlined]{carry-on-bag-checked} & \mSymbol[rounded]{carry-on-bag-checked} & \mSymbol[sharp]{carry-on-bag-checked} & \texttt{\textbackslash mSymbol\{carry-on-bag-checked\}} & \texttt{EB0B}\\
\mSymbol[outlined]{carry-on-bag-inactive} & \mSymbol[rounded]{carry-on-bag-inactive} & \mSymbol[sharp]{carry-on-bag-inactive} & \texttt{\textbackslash mSymbol\{carry-on-bag-inactive\}} & \texttt{EB0A}\\
\mSymbol[outlined]{carry-on-bag-question} & \mSymbol[rounded]{carry-on-bag-question} & \mSymbol[sharp]{carry-on-bag-question} & \texttt{\textbackslash mSymbol\{carry-on-bag-question\}} & \texttt{EB09}\\
\mSymbol[outlined]{cases} & \mSymbol[rounded]{cases} & \mSymbol[sharp]{cases} & \texttt{\textbackslash mSymbol\{cases\}} & \texttt{E992}\\
\mSymbol[outlined]{casino} & \mSymbol[rounded]{casino} & \mSymbol[sharp]{casino} & \texttt{\textbackslash mSymbol\{casino\}} & \texttt{EB40}\\
\mSymbol[outlined]{cast} & \mSymbol[rounded]{cast} & \mSymbol[sharp]{cast} & \texttt{\textbackslash mSymbol\{cast\}} & \texttt{E307}\\
\mSymbol[outlined]{cast-connected} & \mSymbol[rounded]{cast-connected} & \mSymbol[sharp]{cast-connected} & \texttt{\textbackslash mSymbol\{cast-connected\}} & \texttt{E308}\\
\mSymbol[outlined]{cast-for-education} & \mSymbol[rounded]{cast-for-education} & \mSymbol[sharp]{cast-for-education} & \texttt{\textbackslash mSymbol\{cast-for-education\}} & \texttt{EFEC}\\
\mSymbol[outlined]{cast-pause} & \mSymbol[rounded]{cast-pause} & \mSymbol[sharp]{cast-pause} & \texttt{\textbackslash mSymbol\{cast-pause\}} & \texttt{F5F0}\\
\mSymbol[outlined]{cast-warning} & \mSymbol[rounded]{cast-warning} & \mSymbol[sharp]{cast-warning} & \texttt{\textbackslash mSymbol\{cast-warning\}} & \texttt{F5EF}\\
\mSymbol[outlined]{castle} & \mSymbol[rounded]{castle} & \mSymbol[sharp]{castle} & \texttt{\textbackslash mSymbol\{castle\}} & \texttt{EAB1}\\
\mSymbol[outlined]{category} & \mSymbol[rounded]{category} & \mSymbol[sharp]{category} & \texttt{\textbackslash mSymbol\{category\}} & \texttt{E574}\\
\mSymbol[outlined]{category-search} & \mSymbol[rounded]{category-search} & \mSymbol[sharp]{category-search} & \texttt{\textbackslash mSymbol\{category-search\}} & \texttt{F437}\\
\mSymbol[outlined]{celebration} & \mSymbol[rounded]{celebration} & \mSymbol[sharp]{celebration} & \texttt{\textbackslash mSymbol\{celebration\}} & \texttt{EA65}\\
\mSymbol[outlined]{cell-merge} & \mSymbol[rounded]{cell-merge} & \mSymbol[sharp]{cell-merge} & \texttt{\textbackslash mSymbol\{cell-merge\}} & \texttt{F82E}\\
\mSymbol[outlined]{cell-tower} & \mSymbol[rounded]{cell-tower} & \mSymbol[sharp]{cell-tower} & \texttt{\textbackslash mSymbol\{cell-tower\}} & \texttt{EBBA}\\
\mSymbol[outlined]{cell-wifi} & \mSymbol[rounded]{cell-wifi} & \mSymbol[sharp]{cell-wifi} & \texttt{\textbackslash mSymbol\{cell-wifi\}} & \texttt{E0EC}\\
\mSymbol[outlined]{center-focus-strong} & \mSymbol[rounded]{center-focus-strong} & \mSymbol[sharp]{center-focus-strong} & \texttt{\textbackslash mSymbol\{center-focus-strong\}} & \texttt{E3B4}\\
\mSymbol[outlined]{center-focus-weak} & \mSymbol[rounded]{center-focus-weak} & \mSymbol[sharp]{center-focus-weak} & \texttt{\textbackslash mSymbol\{center-focus-weak\}} & \texttt{E3B5}\\
\mSymbol[outlined]{chair} & \mSymbol[rounded]{chair} & \mSymbol[sharp]{chair} & \texttt{\textbackslash mSymbol\{chair\}} & \texttt{EFED}\\
\mSymbol[outlined]{chair-alt} & \mSymbol[rounded]{chair-alt} & \mSymbol[sharp]{chair-alt} & \texttt{\textbackslash mSymbol\{chair-alt\}} & \texttt{EFEE}\\
\mSymbol[outlined]{chalet} & \mSymbol[rounded]{chalet} & \mSymbol[sharp]{chalet} & \texttt{\textbackslash mSymbol\{chalet\}} & \texttt{E585}\\
\mSymbol[outlined]{change-circle} & \mSymbol[rounded]{change-circle} & \mSymbol[sharp]{change-circle} & \texttt{\textbackslash mSymbol\{change-circle\}} & \texttt{E2E7}\\
\mSymbol[outlined]{change-history} & \mSymbol[rounded]{change-history} & \mSymbol[sharp]{change-history} & \texttt{\textbackslash mSymbol\{change-history\}} & \texttt{E86B}\\
\mSymbol[outlined]{charger} & \mSymbol[rounded]{charger} & \mSymbol[sharp]{charger} & \texttt{\textbackslash mSymbol\{charger\}} & \texttt{E2AE}\\
\mSymbol[outlined]{charging-station} & \mSymbol[rounded]{charging-station} & \mSymbol[sharp]{charging-station} & \texttt{\textbackslash mSymbol\{charging-station\}} & \texttt{F19D}\\
\mSymbol[outlined]{chart-data} & \mSymbol[rounded]{chart-data} & \mSymbol[sharp]{chart-data} & \texttt{\textbackslash mSymbol\{chart-data\}} & \texttt{E473}\\
\mSymbol[outlined]{chat} & \mSymbol[rounded]{chat} & \mSymbol[sharp]{chat} & \texttt{\textbackslash mSymbol\{chat\}} & \texttt{E0C9}\\
\mSymbol[outlined]{chat-add-on} & \mSymbol[rounded]{chat-add-on} & \mSymbol[sharp]{chat-add-on} & \texttt{\textbackslash mSymbol\{chat-add-on\}} & \texttt{F0F3}\\
\mSymbol[outlined]{chat-apps-script} & \mSymbol[rounded]{chat-apps-script} & \mSymbol[sharp]{chat-apps-script} & \texttt{\textbackslash mSymbol\{chat-apps-script\}} & \texttt{F0BD}\\
\mSymbol[outlined]{chat-bubble} & \mSymbol[rounded]{chat-bubble} & \mSymbol[sharp]{chat-bubble} & \texttt{\textbackslash mSymbol\{chat-bubble\}} & \texttt{E0CB}\\
\mSymbol[outlined]{chat-bubble-outline} & \mSymbol[rounded]{chat-bubble-outline} & \mSymbol[sharp]{chat-bubble-outline} & \texttt{\textbackslash mSymbol\{chat-bubble-outline\}} & \texttt{E0CB}\\
\mSymbol[outlined]{chat-error} & \mSymbol[rounded]{chat-error} & \mSymbol[sharp]{chat-error} & \texttt{\textbackslash mSymbol\{chat-error\}} & \texttt{F7AC}\\
\mSymbol[outlined]{chat-info} & \mSymbol[rounded]{chat-info} & \mSymbol[sharp]{chat-info} & \texttt{\textbackslash mSymbol\{chat-info\}} & \texttt{F52B}\\
\mSymbol[outlined]{chat-paste-go} & \mSymbol[rounded]{chat-paste-go} & \mSymbol[sharp]{chat-paste-go} & \texttt{\textbackslash mSymbol\{chat-paste-go\}} & \texttt{F6BD}\\
\mSymbol[outlined]{check} & \mSymbol[rounded]{check} & \mSymbol[sharp]{check} & \texttt{\textbackslash mSymbol\{check\}} & \texttt{E5CA}\\
\mSymbol[outlined]{check-box} & \mSymbol[rounded]{check-box} & \mSymbol[sharp]{check-box} & \texttt{\textbackslash mSymbol\{check-box\}} & \texttt{E834}\\
\mSymbol[outlined]{check-box-outline-blank} & \mSymbol[rounded]{check-box-outline-blank} & \mSymbol[sharp]{check-box-outline-blank} & \texttt{\textbackslash mSymbol\{check-box-outline-blank\}} & \texttt{E835}\\
\mSymbol[outlined]{check-circle} & \mSymbol[rounded]{check-circle} & \mSymbol[sharp]{check-circle} & \texttt{\textbackslash mSymbol\{check-circle\}} & \texttt{F0BE}\\
\mSymbol[outlined]{check-circle-filled} & \mSymbol[rounded]{check-circle-filled} & \mSymbol[sharp]{check-circle-filled} & \texttt{\textbackslash mSymbol\{check-circle-filled\}} & \texttt{F0BE}\\
\mSymbol[outlined]{check-circle-outline} & \mSymbol[rounded]{check-circle-outline} & \mSymbol[sharp]{check-circle-outline} & \texttt{\textbackslash mSymbol\{check-circle-outline\}} & \texttt{F0BE}\\
\mSymbol[outlined]{check-in-out} & \mSymbol[rounded]{check-in-out} & \mSymbol[sharp]{check-in-out} & \texttt{\textbackslash mSymbol\{check-in-out\}} & \texttt{F6F6}\\
\mSymbol[outlined]{check-indeterminate-small} & \mSymbol[rounded]{check-indeterminate-small} & \mSymbol[sharp]{check-indeterminate-small} & \texttt{\textbackslash mSymbol\{check-indeterminate-small\}} & \texttt{F88A}\\
\mSymbol[outlined]{check-small} & \mSymbol[rounded]{check-small} & \mSymbol[sharp]{check-small} & \texttt{\textbackslash mSymbol\{check-small\}} & \texttt{F88B}\\
\mSymbol[outlined]{checkbook} & \mSymbol[rounded]{checkbook} & \mSymbol[sharp]{checkbook} & \texttt{\textbackslash mSymbol\{checkbook\}} & \texttt{E70D}\\
\mSymbol[outlined]{checked-bag} & \mSymbol[rounded]{checked-bag} & \mSymbol[sharp]{checked-bag} & \texttt{\textbackslash mSymbol\{checked-bag\}} & \texttt{EB0C}\\
\mSymbol[outlined]{checked-bag-question} & \mSymbol[rounded]{checked-bag-question} & \mSymbol[sharp]{checked-bag-question} & \texttt{\textbackslash mSymbol\{checked-bag-question\}} & \texttt{EB0D}\\
\mSymbol[outlined]{checklist} & \mSymbol[rounded]{checklist} & \mSymbol[sharp]{checklist} & \texttt{\textbackslash mSymbol\{checklist\}} & \texttt{E6B1}\\
\mSymbol[outlined]{checklist-rtl} & \mSymbol[rounded]{checklist-rtl} & \mSymbol[sharp]{checklist-rtl} & \texttt{\textbackslash mSymbol\{checklist-rtl\}} & \texttt{E6B3}\\
\mSymbol[outlined]{checkroom} & \mSymbol[rounded]{checkroom} & \mSymbol[sharp]{checkroom} & \texttt{\textbackslash mSymbol\{checkroom\}} & \texttt{F19E}\\
\mSymbol[outlined]{cheer} & \mSymbol[rounded]{cheer} & \mSymbol[sharp]{cheer} & \texttt{\textbackslash mSymbol\{cheer\}} & \texttt{F6A8}\\
\mSymbol[outlined]{chess} & \mSymbol[rounded]{chess} & \mSymbol[sharp]{chess} & \texttt{\textbackslash mSymbol\{chess\}} & \texttt{F5E7}\\
\mSymbol[outlined]{chevron-backward} & \mSymbol[rounded]{chevron-backward} & \mSymbol[sharp]{chevron-backward} & \texttt{\textbackslash mSymbol\{chevron-backward\}} & \texttt{F46B}\\
\mSymbol[outlined]{chevron-forward} & \mSymbol[rounded]{chevron-forward} & \mSymbol[sharp]{chevron-forward} & \texttt{\textbackslash mSymbol\{chevron-forward\}} & \texttt{F46A}\\
\mSymbol[outlined]{chevron-left} & \mSymbol[rounded]{chevron-left} & \mSymbol[sharp]{chevron-left} & \texttt{\textbackslash mSymbol\{chevron-left\}} & \texttt{E5CB}\\
\mSymbol[outlined]{chevron-right} & \mSymbol[rounded]{chevron-right} & \mSymbol[sharp]{chevron-right} & \texttt{\textbackslash mSymbol\{chevron-right\}} & \texttt{E5CC}\\
\mSymbol[outlined]{child-care} & \mSymbol[rounded]{child-care} & \mSymbol[sharp]{child-care} & \texttt{\textbackslash mSymbol\{child-care\}} & \texttt{EB41}\\
\mSymbol[outlined]{child-friendly} & \mSymbol[rounded]{child-friendly} & \mSymbol[sharp]{child-friendly} & \texttt{\textbackslash mSymbol\{child-friendly\}} & \texttt{EB42}\\
\mSymbol[outlined]{chip-extraction} & \mSymbol[rounded]{chip-extraction} & \mSymbol[sharp]{chip-extraction} & \texttt{\textbackslash mSymbol\{chip-extraction\}} & \texttt{F821}\\
\mSymbol[outlined]{chips} & \mSymbol[rounded]{chips} & \mSymbol[sharp]{chips} & \texttt{\textbackslash mSymbol\{chips\}} & \texttt{E993}\\
\mSymbol[outlined]{chrome-reader-mode} & \mSymbol[rounded]{chrome-reader-mode} & \mSymbol[sharp]{chrome-reader-mode} & \texttt{\textbackslash mSymbol\{chrome-reader-mode\}} & \texttt{E86D}\\
\mSymbol[outlined]{chromecast-2} & \mSymbol[rounded]{chromecast-2} & \mSymbol[sharp]{chromecast-2} & \texttt{\textbackslash mSymbol\{chromecast-2\}} & \texttt{F17B}\\
\mSymbol[outlined]{chromecast-device} & \mSymbol[rounded]{chromecast-device} & \mSymbol[sharp]{chromecast-device} & \texttt{\textbackslash mSymbol\{chromecast-device\}} & \texttt{E83C}\\
\mSymbol[outlined]{chronic} & \mSymbol[rounded]{chronic} & \mSymbol[sharp]{chronic} & \texttt{\textbackslash mSymbol\{chronic\}} & \texttt{EBB2}\\
\mSymbol[outlined]{church} & \mSymbol[rounded]{church} & \mSymbol[sharp]{church} & \texttt{\textbackslash mSymbol\{church\}} & \texttt{EAAE}\\
\mSymbol[outlined]{cinematic-blur} & \mSymbol[rounded]{cinematic-blur} & \mSymbol[sharp]{cinematic-blur} & \texttt{\textbackslash mSymbol\{cinematic-blur\}} & \texttt{F853}\\
\mSymbol[outlined]{circle} & \mSymbol[rounded]{circle} & \mSymbol[sharp]{circle} & \texttt{\textbackslash mSymbol\{circle\}} & \texttt{EF4A}\\
\mSymbol[outlined]{circle-notifications} & \mSymbol[rounded]{circle-notifications} & \mSymbol[sharp]{circle-notifications} & \texttt{\textbackslash mSymbol\{circle-notifications\}} & \texttt{E994}\\
\mSymbol[outlined]{circles} & \mSymbol[rounded]{circles} & \mSymbol[sharp]{circles} & \texttt{\textbackslash mSymbol\{circles\}} & \texttt{E7EA}\\
\mSymbol[outlined]{circles-ext} & \mSymbol[rounded]{circles-ext} & \mSymbol[sharp]{circles-ext} & \texttt{\textbackslash mSymbol\{circles-ext\}} & \texttt{E7EC}\\
\mSymbol[outlined]{clarify} & \mSymbol[rounded]{clarify} & \mSymbol[sharp]{clarify} & \texttt{\textbackslash mSymbol\{clarify\}} & \texttt{F0BF}\\
\mSymbol[outlined]{class} & \mSymbol[rounded]{class} & \mSymbol[sharp]{class} & \texttt{\textbackslash mSymbol\{class\}} & \texttt{E86E}\\
\mSymbol[outlined]{clean-hands} & \mSymbol[rounded]{clean-hands} & \mSymbol[sharp]{clean-hands} & \texttt{\textbackslash mSymbol\{clean-hands\}} & \texttt{F21F}\\
\mSymbol[outlined]{cleaning} & \mSymbol[rounded]{cleaning} & \mSymbol[sharp]{cleaning} & \texttt{\textbackslash mSymbol\{cleaning\}} & \texttt{E995}\\
\mSymbol[outlined]{cleaning-bucket} & \mSymbol[rounded]{cleaning-bucket} & \mSymbol[sharp]{cleaning-bucket} & \texttt{\textbackslash mSymbol\{cleaning-bucket\}} & \texttt{F8B4}\\
\mSymbol[outlined]{cleaning-services} & \mSymbol[rounded]{cleaning-services} & \mSymbol[sharp]{cleaning-services} & \texttt{\textbackslash mSymbol\{cleaning-services\}} & \texttt{F0FF}\\
\mSymbol[outlined]{clear} & \mSymbol[rounded]{clear} & \mSymbol[sharp]{clear} & \texttt{\textbackslash mSymbol\{clear\}} & \texttt{E5CD}\\
\mSymbol[outlined]{clear-all} & \mSymbol[rounded]{clear-all} & \mSymbol[sharp]{clear-all} & \texttt{\textbackslash mSymbol\{clear-all\}} & \texttt{E0B8}\\
\mSymbol[outlined]{clear-day} & \mSymbol[rounded]{clear-day} & \mSymbol[sharp]{clear-day} & \texttt{\textbackslash mSymbol\{clear-day\}} & \texttt{F157}\\
\mSymbol[outlined]{clear-night} & \mSymbol[rounded]{clear-night} & \mSymbol[sharp]{clear-night} & \texttt{\textbackslash mSymbol\{clear-night\}} & \texttt{F159}\\
\mSymbol[outlined]{climate-mini-split} & \mSymbol[rounded]{climate-mini-split} & \mSymbol[sharp]{climate-mini-split} & \texttt{\textbackslash mSymbol\{climate-mini-split\}} & \texttt{F8B5}\\
\mSymbol[outlined]{clinical-notes} & \mSymbol[rounded]{clinical-notes} & \mSymbol[sharp]{clinical-notes} & \texttt{\textbackslash mSymbol\{clinical-notes\}} & \texttt{E09E}\\
\mSymbol[outlined]{clock-loader-10} & \mSymbol[rounded]{clock-loader-10} & \mSymbol[sharp]{clock-loader-10} & \texttt{\textbackslash mSymbol\{clock-loader-10\}} & \texttt{F726}\\
\mSymbol[outlined]{clock-loader-20} & \mSymbol[rounded]{clock-loader-20} & \mSymbol[sharp]{clock-loader-20} & \texttt{\textbackslash mSymbol\{clock-loader-20\}} & \texttt{F725}\\
\mSymbol[outlined]{clock-loader-40} & \mSymbol[rounded]{clock-loader-40} & \mSymbol[sharp]{clock-loader-40} & \texttt{\textbackslash mSymbol\{clock-loader-40\}} & \texttt{F724}\\
\mSymbol[outlined]{clock-loader-60} & \mSymbol[rounded]{clock-loader-60} & \mSymbol[sharp]{clock-loader-60} & \texttt{\textbackslash mSymbol\{clock-loader-60\}} & \texttt{F723}\\
\mSymbol[outlined]{clock-loader-80} & \mSymbol[rounded]{clock-loader-80} & \mSymbol[sharp]{clock-loader-80} & \texttt{\textbackslash mSymbol\{clock-loader-80\}} & \texttt{F722}\\
\mSymbol[outlined]{clock-loader-90} & \mSymbol[rounded]{clock-loader-90} & \mSymbol[sharp]{clock-loader-90} & \texttt{\textbackslash mSymbol\{clock-loader-90\}} & \texttt{F721}\\
\mSymbol[outlined]{close} & \mSymbol[rounded]{close} & \mSymbol[sharp]{close} & \texttt{\textbackslash mSymbol\{close\}} & \texttt{E5CD}\\
\mSymbol[outlined]{close-fullscreen} & \mSymbol[rounded]{close-fullscreen} & \mSymbol[sharp]{close-fullscreen} & \texttt{\textbackslash mSymbol\{close-fullscreen\}} & \texttt{F1CF}\\
\mSymbol[outlined]{close-small} & \mSymbol[rounded]{close-small} & \mSymbol[sharp]{close-small} & \texttt{\textbackslash mSymbol\{close-small\}} & \texttt{F508}\\
\mSymbol[outlined]{closed-caption} & \mSymbol[rounded]{closed-caption} & \mSymbol[sharp]{closed-caption} & \texttt{\textbackslash mSymbol\{closed-caption\}} & \texttt{E996}\\
\mSymbol[outlined]{closed-caption-add} & \mSymbol[rounded]{closed-caption-add} & \mSymbol[sharp]{closed-caption-add} & \texttt{\textbackslash mSymbol\{closed-caption-add\}} & \texttt{F4AE}\\
\mSymbol[outlined]{closed-caption-disabled} & \mSymbol[rounded]{closed-caption-disabled} & \mSymbol[sharp]{closed-caption-disabled} & \texttt{\textbackslash mSymbol\{closed-caption-disabled\}} & \texttt{F1DC}\\
\mSymbol[outlined]{closed-caption-off} & \mSymbol[rounded]{closed-caption-off} & \mSymbol[sharp]{closed-caption-off} & \texttt{\textbackslash mSymbol\{closed-caption-off\}} & \texttt{E996}\\
\mSymbol[outlined]{cloud} & \mSymbol[rounded]{cloud} & \mSymbol[sharp]{cloud} & \texttt{\textbackslash mSymbol\{cloud\}} & \texttt{F15C}\\
\mSymbol[outlined]{cloud-circle} & \mSymbol[rounded]{cloud-circle} & \mSymbol[sharp]{cloud-circle} & \texttt{\textbackslash mSymbol\{cloud-circle\}} & \texttt{E2BE}\\
\mSymbol[outlined]{cloud-done} & \mSymbol[rounded]{cloud-done} & \mSymbol[sharp]{cloud-done} & \texttt{\textbackslash mSymbol\{cloud-done\}} & \texttt{E2BF}\\
\mSymbol[outlined]{cloud-download} & \mSymbol[rounded]{cloud-download} & \mSymbol[sharp]{cloud-download} & \texttt{\textbackslash mSymbol\{cloud-download\}} & \texttt{E2C0}\\
\mSymbol[outlined]{cloud-off} & \mSymbol[rounded]{cloud-off} & \mSymbol[sharp]{cloud-off} & \texttt{\textbackslash mSymbol\{cloud-off\}} & \texttt{E2C1}\\
\mSymbol[outlined]{cloud-queue} & \mSymbol[rounded]{cloud-queue} & \mSymbol[sharp]{cloud-queue} & \texttt{\textbackslash mSymbol\{cloud-queue\}} & \texttt{F15C}\\
\mSymbol[outlined]{cloud-sync} & \mSymbol[rounded]{cloud-sync} & \mSymbol[sharp]{cloud-sync} & \texttt{\textbackslash mSymbol\{cloud-sync\}} & \texttt{EB5A}\\
\mSymbol[outlined]{cloud-upload} & \mSymbol[rounded]{cloud-upload} & \mSymbol[sharp]{cloud-upload} & \texttt{\textbackslash mSymbol\{cloud-upload\}} & \texttt{E2C3}\\
\mSymbol[outlined]{cloudy} & \mSymbol[rounded]{cloudy} & \mSymbol[sharp]{cloudy} & \texttt{\textbackslash mSymbol\{cloudy\}} & \texttt{F15C}\\
\mSymbol[outlined]{cloudy-filled} & \mSymbol[rounded]{cloudy-filled} & \mSymbol[sharp]{cloudy-filled} & \texttt{\textbackslash mSymbol\{cloudy-filled\}} & \texttt{F15C}\\
\mSymbol[outlined]{cloudy-snowing} & \mSymbol[rounded]{cloudy-snowing} & \mSymbol[sharp]{cloudy-snowing} & \texttt{\textbackslash mSymbol\{cloudy-snowing\}} & \texttt{E810}\\
\mSymbol[outlined]{co2} & \mSymbol[rounded]{co2} & \mSymbol[sharp]{co2} & \texttt{\textbackslash mSymbol\{co2\}} & \texttt{E7B0}\\
\mSymbol[outlined]{co-present} & \mSymbol[rounded]{co-present} & \mSymbol[sharp]{co-present} & \texttt{\textbackslash mSymbol\{co-present\}} & \texttt{EAF0}\\
\mSymbol[outlined]{code} & \mSymbol[rounded]{code} & \mSymbol[sharp]{code} & \texttt{\textbackslash mSymbol\{code\}} & \texttt{E86F}\\
\mSymbol[outlined]{code-blocks} & \mSymbol[rounded]{code-blocks} & \mSymbol[sharp]{code-blocks} & \texttt{\textbackslash mSymbol\{code-blocks\}} & \texttt{F84D}\\
\mSymbol[outlined]{code-off} & \mSymbol[rounded]{code-off} & \mSymbol[sharp]{code-off} & \texttt{\textbackslash mSymbol\{code-off\}} & \texttt{E4F3}\\
\mSymbol[outlined]{coffee} & \mSymbol[rounded]{coffee} & \mSymbol[sharp]{coffee} & \texttt{\textbackslash mSymbol\{coffee\}} & \texttt{EFEF}\\
\mSymbol[outlined]{coffee-maker} & \mSymbol[rounded]{coffee-maker} & \mSymbol[sharp]{coffee-maker} & \texttt{\textbackslash mSymbol\{coffee-maker\}} & \texttt{EFF0}\\
\mSymbol[outlined]{cognition} & \mSymbol[rounded]{cognition} & \mSymbol[sharp]{cognition} & \texttt{\textbackslash mSymbol\{cognition\}} & \texttt{E09F}\\
\mSymbol[outlined]{collapse-all} & \mSymbol[rounded]{collapse-all} & \mSymbol[sharp]{collapse-all} & \texttt{\textbackslash mSymbol\{collapse-all\}} & \texttt{E944}\\
\mSymbol[outlined]{collapse-content} & \mSymbol[rounded]{collapse-content} & \mSymbol[sharp]{collapse-content} & \texttt{\textbackslash mSymbol\{collapse-content\}} & \texttt{F507}\\
\mSymbol[outlined]{collections} & \mSymbol[rounded]{collections} & \mSymbol[sharp]{collections} & \texttt{\textbackslash mSymbol\{collections\}} & \texttt{E3D3}\\
\mSymbol[outlined]{collections-bookmark} & \mSymbol[rounded]{collections-bookmark} & \mSymbol[sharp]{collections-bookmark} & \texttt{\textbackslash mSymbol\{collections-bookmark\}} & \texttt{E431}\\
\mSymbol[outlined]{color-lens} & \mSymbol[rounded]{color-lens} & \mSymbol[sharp]{color-lens} & \texttt{\textbackslash mSymbol\{color-lens\}} & \texttt{E40A}\\
\mSymbol[outlined]{colorize} & \mSymbol[rounded]{colorize} & \mSymbol[sharp]{colorize} & \texttt{\textbackslash mSymbol\{colorize\}} & \texttt{E3B8}\\
\mSymbol[outlined]{colors} & \mSymbol[rounded]{colors} & \mSymbol[sharp]{colors} & \texttt{\textbackslash mSymbol\{colors\}} & \texttt{E997}\\
\mSymbol[outlined]{combine-columns} & \mSymbol[rounded]{combine-columns} & \mSymbol[sharp]{combine-columns} & \texttt{\textbackslash mSymbol\{combine-columns\}} & \texttt{F420}\\
\mSymbol[outlined]{comedy-mask} & \mSymbol[rounded]{comedy-mask} & \mSymbol[sharp]{comedy-mask} & \texttt{\textbackslash mSymbol\{comedy-mask\}} & \texttt{F4D6}\\
\mSymbol[outlined]{comic-bubble} & \mSymbol[rounded]{comic-bubble} & \mSymbol[sharp]{comic-bubble} & \texttt{\textbackslash mSymbol\{comic-bubble\}} & \texttt{F5DD}\\
\mSymbol[outlined]{comment} & \mSymbol[rounded]{comment} & \mSymbol[sharp]{comment} & \texttt{\textbackslash mSymbol\{comment\}} & \texttt{E24C}\\
\mSymbol[outlined]{comment-bank} & \mSymbol[rounded]{comment-bank} & \mSymbol[sharp]{comment-bank} & \texttt{\textbackslash mSymbol\{comment-bank\}} & \texttt{EA4E}\\
\mSymbol[outlined]{comments-disabled} & \mSymbol[rounded]{comments-disabled} & \mSymbol[sharp]{comments-disabled} & \texttt{\textbackslash mSymbol\{comments-disabled\}} & \texttt{E7A2}\\
\mSymbol[outlined]{commit} & \mSymbol[rounded]{commit} & \mSymbol[sharp]{commit} & \texttt{\textbackslash mSymbol\{commit\}} & \texttt{EAF5}\\
\mSymbol[outlined]{communication} & \mSymbol[rounded]{communication} & \mSymbol[sharp]{communication} & \texttt{\textbackslash mSymbol\{communication\}} & \texttt{E27C}\\
\mSymbol[outlined]{communities} & \mSymbol[rounded]{communities} & \mSymbol[sharp]{communities} & \texttt{\textbackslash mSymbol\{communities\}} & \texttt{EB16}\\
\mSymbol[outlined]{communities-filled} & \mSymbol[rounded]{communities-filled} & \mSymbol[sharp]{communities-filled} & \texttt{\textbackslash mSymbol\{communities-filled\}} & \texttt{EB16}\\
\mSymbol[outlined]{commute} & \mSymbol[rounded]{commute} & \mSymbol[sharp]{commute} & \texttt{\textbackslash mSymbol\{commute\}} & \texttt{E940}\\
\mSymbol[outlined]{compare} & \mSymbol[rounded]{compare} & \mSymbol[sharp]{compare} & \texttt{\textbackslash mSymbol\{compare\}} & \texttt{E3B9}\\
\mSymbol[outlined]{compare-arrows} & \mSymbol[rounded]{compare-arrows} & \mSymbol[sharp]{compare-arrows} & \texttt{\textbackslash mSymbol\{compare-arrows\}} & \texttt{E915}\\
\mSymbol[outlined]{compass-calibration} & \mSymbol[rounded]{compass-calibration} & \mSymbol[sharp]{compass-calibration} & \texttt{\textbackslash mSymbol\{compass-calibration\}} & \texttt{E57C}\\
\mSymbol[outlined]{component-exchange} & \mSymbol[rounded]{component-exchange} & \mSymbol[sharp]{component-exchange} & \texttt{\textbackslash mSymbol\{component-exchange\}} & \texttt{F1E7}\\
\mSymbol[outlined]{compost} & \mSymbol[rounded]{compost} & \mSymbol[sharp]{compost} & \texttt{\textbackslash mSymbol\{compost\}} & \texttt{E761}\\
\mSymbol[outlined]{compress} & \mSymbol[rounded]{compress} & \mSymbol[sharp]{compress} & \texttt{\textbackslash mSymbol\{compress\}} & \texttt{E94D}\\
\mSymbol[outlined]{computer} & \mSymbol[rounded]{computer} & \mSymbol[sharp]{computer} & \texttt{\textbackslash mSymbol\{computer\}} & \texttt{E31E}\\
\mSymbol[outlined]{concierge} & \mSymbol[rounded]{concierge} & \mSymbol[sharp]{concierge} & \texttt{\textbackslash mSymbol\{concierge\}} & \texttt{F561}\\
\mSymbol[outlined]{conditions} & \mSymbol[rounded]{conditions} & \mSymbol[sharp]{conditions} & \texttt{\textbackslash mSymbol\{conditions\}} & \texttt{E0A0}\\
\mSymbol[outlined]{confirmation-number} & \mSymbol[rounded]{confirmation-number} & \mSymbol[sharp]{confirmation-number} & \texttt{\textbackslash mSymbol\{confirmation-number\}} & \texttt{E638}\\
\mSymbol[outlined]{congenital} & \mSymbol[rounded]{congenital} & \mSymbol[sharp]{congenital} & \texttt{\textbackslash mSymbol\{congenital\}} & \texttt{E0A1}\\
\mSymbol[outlined]{connect-without-contact} & \mSymbol[rounded]{connect-without-contact} & \mSymbol[sharp]{connect-without-contact} & \texttt{\textbackslash mSymbol\{connect-without-contact\}} & \texttt{F223}\\
\mSymbol[outlined]{connected-tv} & \mSymbol[rounded]{connected-tv} & \mSymbol[sharp]{connected-tv} & \texttt{\textbackslash mSymbol\{connected-tv\}} & \texttt{E998}\\
\mSymbol[outlined]{connecting-airports} & \mSymbol[rounded]{connecting-airports} & \mSymbol[sharp]{connecting-airports} & \texttt{\textbackslash mSymbol\{connecting-airports\}} & \texttt{E7C9}\\
\mSymbol[outlined]{construction} & \mSymbol[rounded]{construction} & \mSymbol[sharp]{construction} & \texttt{\textbackslash mSymbol\{construction\}} & \texttt{EA3C}\\
\mSymbol[outlined]{contact-emergency} & \mSymbol[rounded]{contact-emergency} & \mSymbol[sharp]{contact-emergency} & \texttt{\textbackslash mSymbol\{contact-emergency\}} & \texttt{F8D1}\\
\mSymbol[outlined]{contact-mail} & \mSymbol[rounded]{contact-mail} & \mSymbol[sharp]{contact-mail} & \texttt{\textbackslash mSymbol\{contact-mail\}} & \texttt{E0D0}\\
\mSymbol[outlined]{contact-page} & \mSymbol[rounded]{contact-page} & \mSymbol[sharp]{contact-page} & \texttt{\textbackslash mSymbol\{contact-page\}} & \texttt{F22E}\\
\mSymbol[outlined]{contact-phone} & \mSymbol[rounded]{contact-phone} & \mSymbol[sharp]{contact-phone} & \texttt{\textbackslash mSymbol\{contact-phone\}} & \texttt{F0C0}\\
\mSymbol[outlined]{contact-phone-filled} & \mSymbol[rounded]{contact-phone-filled} & \mSymbol[sharp]{contact-phone-filled} & \texttt{\textbackslash mSymbol\{contact-phone-filled\}} & \texttt{F0C0}\\
\mSymbol[outlined]{contact-support} & \mSymbol[rounded]{contact-support} & \mSymbol[sharp]{contact-support} & \texttt{\textbackslash mSymbol\{contact-support\}} & \texttt{E94C}\\
\mSymbol[outlined]{contactless} & \mSymbol[rounded]{contactless} & \mSymbol[sharp]{contactless} & \texttt{\textbackslash mSymbol\{contactless\}} & \texttt{EA71}\\
\mSymbol[outlined]{contactless-off} & \mSymbol[rounded]{contactless-off} & \mSymbol[sharp]{contactless-off} & \texttt{\textbackslash mSymbol\{contactless-off\}} & \texttt{F858}\\
\mSymbol[outlined]{contacts} & \mSymbol[rounded]{contacts} & \mSymbol[sharp]{contacts} & \texttt{\textbackslash mSymbol\{contacts\}} & \texttt{E0BA}\\
\mSymbol[outlined]{contacts-product} & \mSymbol[rounded]{contacts-product} & \mSymbol[sharp]{contacts-product} & \texttt{\textbackslash mSymbol\{contacts-product\}} & \texttt{E999}\\
\mSymbol[outlined]{content-copy} & \mSymbol[rounded]{content-copy} & \mSymbol[sharp]{content-copy} & \texttt{\textbackslash mSymbol\{content-copy\}} & \texttt{E14D}\\
\mSymbol[outlined]{content-cut} & \mSymbol[rounded]{content-cut} & \mSymbol[sharp]{content-cut} & \texttt{\textbackslash mSymbol\{content-cut\}} & \texttt{E14E}\\
\mSymbol[outlined]{content-paste} & \mSymbol[rounded]{content-paste} & \mSymbol[sharp]{content-paste} & \texttt{\textbackslash mSymbol\{content-paste\}} & \texttt{E14F}\\
\mSymbol[outlined]{content-paste-go} & \mSymbol[rounded]{content-paste-go} & \mSymbol[sharp]{content-paste-go} & \texttt{\textbackslash mSymbol\{content-paste-go\}} & \texttt{EA8E}\\
\mSymbol[outlined]{content-paste-off} & \mSymbol[rounded]{content-paste-off} & \mSymbol[sharp]{content-paste-off} & \texttt{\textbackslash mSymbol\{content-paste-off\}} & \texttt{E4F8}\\
\mSymbol[outlined]{content-paste-search} & \mSymbol[rounded]{content-paste-search} & \mSymbol[sharp]{content-paste-search} & \texttt{\textbackslash mSymbol\{content-paste-search\}} & \texttt{EA9B}\\
\mSymbol[outlined]{contextual-token} & \mSymbol[rounded]{contextual-token} & \mSymbol[sharp]{contextual-token} & \texttt{\textbackslash mSymbol\{contextual-token\}} & \texttt{F486}\\
\mSymbol[outlined]{contextual-token-add} & \mSymbol[rounded]{contextual-token-add} & \mSymbol[sharp]{contextual-token-add} & \texttt{\textbackslash mSymbol\{contextual-token-add\}} & \texttt{F485}\\
\mSymbol[outlined]{contract} & \mSymbol[rounded]{contract} & \mSymbol[sharp]{contract} & \texttt{\textbackslash mSymbol\{contract\}} & \texttt{F5A0}\\
\mSymbol[outlined]{contract-delete} & \mSymbol[rounded]{contract-delete} & \mSymbol[sharp]{contract-delete} & \texttt{\textbackslash mSymbol\{contract-delete\}} & \texttt{F5A2}\\
\mSymbol[outlined]{contract-edit} & \mSymbol[rounded]{contract-edit} & \mSymbol[sharp]{contract-edit} & \texttt{\textbackslash mSymbol\{contract-edit\}} & \texttt{F5A1}\\
\mSymbol[outlined]{contrast} & \mSymbol[rounded]{contrast} & \mSymbol[sharp]{contrast} & \texttt{\textbackslash mSymbol\{contrast\}} & \texttt{EB37}\\
\mSymbol[outlined]{contrast-circle} & \mSymbol[rounded]{contrast-circle} & \mSymbol[sharp]{contrast-circle} & \texttt{\textbackslash mSymbol\{contrast-circle\}} & \texttt{F49F}\\
\mSymbol[outlined]{contrast-rtl-off} & \mSymbol[rounded]{contrast-rtl-off} & \mSymbol[sharp]{contrast-rtl-off} & \texttt{\textbackslash mSymbol\{contrast-rtl-off\}} & \texttt{EC72}\\
\mSymbol[outlined]{contrast-square} & \mSymbol[rounded]{contrast-square} & \mSymbol[sharp]{contrast-square} & \texttt{\textbackslash mSymbol\{contrast-square\}} & \texttt{F4A0}\\
\mSymbol[outlined]{control-camera} & \mSymbol[rounded]{control-camera} & \mSymbol[sharp]{control-camera} & \texttt{\textbackslash mSymbol\{control-camera\}} & \texttt{E074}\\
\mSymbol[outlined]{control-point} & \mSymbol[rounded]{control-point} & \mSymbol[sharp]{control-point} & \texttt{\textbackslash mSymbol\{control-point\}} & \texttt{E3BA}\\
\mSymbol[outlined]{control-point-duplicate} & \mSymbol[rounded]{control-point-duplicate} & \mSymbol[sharp]{control-point-duplicate} & \texttt{\textbackslash mSymbol\{control-point-duplicate\}} & \texttt{E3BB}\\
\mSymbol[outlined]{controller-gen} & \mSymbol[rounded]{controller-gen} & \mSymbol[sharp]{controller-gen} & \texttt{\textbackslash mSymbol\{controller-gen\}} & \texttt{E83D}\\
\mSymbol[outlined]{conversion-path} & \mSymbol[rounded]{conversion-path} & \mSymbol[sharp]{conversion-path} & \texttt{\textbackslash mSymbol\{conversion-path\}} & \texttt{F0C1}\\
\mSymbol[outlined]{conversion-path-off} & \mSymbol[rounded]{conversion-path-off} & \mSymbol[sharp]{conversion-path-off} & \texttt{\textbackslash mSymbol\{conversion-path-off\}} & \texttt{F7B4}\\
\mSymbol[outlined]{convert-to-text} & \mSymbol[rounded]{convert-to-text} & \mSymbol[sharp]{convert-to-text} & \texttt{\textbackslash mSymbol\{convert-to-text\}} & \texttt{F41F}\\
\mSymbol[outlined]{conveyor-belt} & \mSymbol[rounded]{conveyor-belt} & \mSymbol[sharp]{conveyor-belt} & \texttt{\textbackslash mSymbol\{conveyor-belt\}} & \texttt{F867}\\
\mSymbol[outlined]{cookie} & \mSymbol[rounded]{cookie} & \mSymbol[sharp]{cookie} & \texttt{\textbackslash mSymbol\{cookie\}} & \texttt{EAAC}\\
\mSymbol[outlined]{cookie-off} & \mSymbol[rounded]{cookie-off} & \mSymbol[sharp]{cookie-off} & \texttt{\textbackslash mSymbol\{cookie-off\}} & \texttt{F79A}\\
\mSymbol[outlined]{cooking} & \mSymbol[rounded]{cooking} & \mSymbol[sharp]{cooking} & \texttt{\textbackslash mSymbol\{cooking\}} & \texttt{E2B6}\\
\mSymbol[outlined]{cool-to-dry} & \mSymbol[rounded]{cool-to-dry} & \mSymbol[sharp]{cool-to-dry} & \texttt{\textbackslash mSymbol\{cool-to-dry\}} & \texttt{E276}\\
\mSymbol[outlined]{copy-all} & \mSymbol[rounded]{copy-all} & \mSymbol[sharp]{copy-all} & \texttt{\textbackslash mSymbol\{copy-all\}} & \texttt{E2EC}\\
\mSymbol[outlined]{copyright} & \mSymbol[rounded]{copyright} & \mSymbol[sharp]{copyright} & \texttt{\textbackslash mSymbol\{copyright\}} & \texttt{E90C}\\
\mSymbol[outlined]{coronavirus} & \mSymbol[rounded]{coronavirus} & \mSymbol[sharp]{coronavirus} & \texttt{\textbackslash mSymbol\{coronavirus\}} & \texttt{F221}\\
\mSymbol[outlined]{corporate-fare} & \mSymbol[rounded]{corporate-fare} & \mSymbol[sharp]{corporate-fare} & \texttt{\textbackslash mSymbol\{corporate-fare\}} & \texttt{F1D0}\\
\mSymbol[outlined]{cottage} & \mSymbol[rounded]{cottage} & \mSymbol[sharp]{cottage} & \texttt{\textbackslash mSymbol\{cottage\}} & \texttt{E587}\\
\mSymbol[outlined]{counter-0} & \mSymbol[rounded]{counter-0} & \mSymbol[sharp]{counter-0} & \texttt{\textbackslash mSymbol\{counter-0\}} & \texttt{F785}\\
\mSymbol[outlined]{counter-1} & \mSymbol[rounded]{counter-1} & \mSymbol[sharp]{counter-1} & \texttt{\textbackslash mSymbol\{counter-1\}} & \texttt{F784}\\
\mSymbol[outlined]{counter-2} & \mSymbol[rounded]{counter-2} & \mSymbol[sharp]{counter-2} & \texttt{\textbackslash mSymbol\{counter-2\}} & \texttt{F783}\\
\mSymbol[outlined]{counter-3} & \mSymbol[rounded]{counter-3} & \mSymbol[sharp]{counter-3} & \texttt{\textbackslash mSymbol\{counter-3\}} & \texttt{F782}\\
\mSymbol[outlined]{counter-4} & \mSymbol[rounded]{counter-4} & \mSymbol[sharp]{counter-4} & \texttt{\textbackslash mSymbol\{counter-4\}} & \texttt{F781}\\
\mSymbol[outlined]{counter-5} & \mSymbol[rounded]{counter-5} & \mSymbol[sharp]{counter-5} & \texttt{\textbackslash mSymbol\{counter-5\}} & \texttt{F780}\\
\mSymbol[outlined]{counter-6} & \mSymbol[rounded]{counter-6} & \mSymbol[sharp]{counter-6} & \texttt{\textbackslash mSymbol\{counter-6\}} & \texttt{F77F}\\
\mSymbol[outlined]{counter-7} & \mSymbol[rounded]{counter-7} & \mSymbol[sharp]{counter-7} & \texttt{\textbackslash mSymbol\{counter-7\}} & \texttt{F77E}\\
\mSymbol[outlined]{counter-8} & \mSymbol[rounded]{counter-8} & \mSymbol[sharp]{counter-8} & \texttt{\textbackslash mSymbol\{counter-8\}} & \texttt{F77D}\\
\mSymbol[outlined]{counter-9} & \mSymbol[rounded]{counter-9} & \mSymbol[sharp]{counter-9} & \texttt{\textbackslash mSymbol\{counter-9\}} & \texttt{F77C}\\
\mSymbol[outlined]{countertops} & \mSymbol[rounded]{countertops} & \mSymbol[sharp]{countertops} & \texttt{\textbackslash mSymbol\{countertops\}} & \texttt{F1F7}\\
\mSymbol[outlined]{create} & \mSymbol[rounded]{create} & \mSymbol[sharp]{create} & \texttt{\textbackslash mSymbol\{create\}} & \texttt{F097}\\
\mSymbol[outlined]{create-new-folder} & \mSymbol[rounded]{create-new-folder} & \mSymbol[sharp]{create-new-folder} & \texttt{\textbackslash mSymbol\{create-new-folder\}} & \texttt{E2CC}\\
\mSymbol[outlined]{credit-card} & \mSymbol[rounded]{credit-card} & \mSymbol[sharp]{credit-card} & \texttt{\textbackslash mSymbol\{credit-card\}} & \texttt{E8A1}\\
\mSymbol[outlined]{credit-card-clock} & \mSymbol[rounded]{credit-card-clock} & \mSymbol[sharp]{credit-card-clock} & \texttt{\textbackslash mSymbol\{credit-card-clock\}} & \texttt{F438}\\
\mSymbol[outlined]{credit-card-gear} & \mSymbol[rounded]{credit-card-gear} & \mSymbol[sharp]{credit-card-gear} & \texttt{\textbackslash mSymbol\{credit-card-gear\}} & \texttt{F52D}\\
\mSymbol[outlined]{credit-card-heart} & \mSymbol[rounded]{credit-card-heart} & \mSymbol[sharp]{credit-card-heart} & \texttt{\textbackslash mSymbol\{credit-card-heart\}} & \texttt{F52C}\\
\mSymbol[outlined]{credit-card-off} & \mSymbol[rounded]{credit-card-off} & \mSymbol[sharp]{credit-card-off} & \texttt{\textbackslash mSymbol\{credit-card-off\}} & \texttt{E4F4}\\
\mSymbol[outlined]{credit-score} & \mSymbol[rounded]{credit-score} & \mSymbol[sharp]{credit-score} & \texttt{\textbackslash mSymbol\{credit-score\}} & \texttt{EFF1}\\
\mSymbol[outlined]{crib} & \mSymbol[rounded]{crib} & \mSymbol[sharp]{crib} & \texttt{\textbackslash mSymbol\{crib\}} & \texttt{E588}\\
\mSymbol[outlined]{crisis-alert} & \mSymbol[rounded]{crisis-alert} & \mSymbol[sharp]{crisis-alert} & \texttt{\textbackslash mSymbol\{crisis-alert\}} & \texttt{EBE9}\\
\mSymbol[outlined]{crop} & \mSymbol[rounded]{crop} & \mSymbol[sharp]{crop} & \texttt{\textbackslash mSymbol\{crop\}} & \texttt{E3BE}\\
\mSymbol[outlined]{crop-16-9} & \mSymbol[rounded]{crop-16-9} & \mSymbol[sharp]{crop-16-9} & \texttt{\textbackslash mSymbol\{crop-16-9\}} & \texttt{E3BC}\\
\mSymbol[outlined]{crop-3-2} & \mSymbol[rounded]{crop-3-2} & \mSymbol[sharp]{crop-3-2} & \texttt{\textbackslash mSymbol\{crop-3-2\}} & \texttt{E3BD}\\
\mSymbol[outlined]{crop-5-4} & \mSymbol[rounded]{crop-5-4} & \mSymbol[sharp]{crop-5-4} & \texttt{\textbackslash mSymbol\{crop-5-4\}} & \texttt{E3BF}\\
\mSymbol[outlined]{crop-7-5} & \mSymbol[rounded]{crop-7-5} & \mSymbol[sharp]{crop-7-5} & \texttt{\textbackslash mSymbol\{crop-7-5\}} & \texttt{E3C0}\\
\mSymbol[outlined]{crop-9-16} & \mSymbol[rounded]{crop-9-16} & \mSymbol[sharp]{crop-9-16} & \texttt{\textbackslash mSymbol\{crop-9-16\}} & \texttt{F549}\\
\mSymbol[outlined]{crop-din} & \mSymbol[rounded]{crop-din} & \mSymbol[sharp]{crop-din} & \texttt{\textbackslash mSymbol\{crop-din\}} & \texttt{E3C6}\\
\mSymbol[outlined]{crop-free} & \mSymbol[rounded]{crop-free} & \mSymbol[sharp]{crop-free} & \texttt{\textbackslash mSymbol\{crop-free\}} & \texttt{E3C2}\\
\mSymbol[outlined]{crop-landscape} & \mSymbol[rounded]{crop-landscape} & \mSymbol[sharp]{crop-landscape} & \texttt{\textbackslash mSymbol\{crop-landscape\}} & \texttt{E3C3}\\
\mSymbol[outlined]{crop-original} & \mSymbol[rounded]{crop-original} & \mSymbol[sharp]{crop-original} & \texttt{\textbackslash mSymbol\{crop-original\}} & \texttt{E3F4}\\
\mSymbol[outlined]{crop-portrait} & \mSymbol[rounded]{crop-portrait} & \mSymbol[sharp]{crop-portrait} & \texttt{\textbackslash mSymbol\{crop-portrait\}} & \texttt{E3C5}\\
\mSymbol[outlined]{crop-rotate} & \mSymbol[rounded]{crop-rotate} & \mSymbol[sharp]{crop-rotate} & \texttt{\textbackslash mSymbol\{crop-rotate\}} & \texttt{E437}\\
\mSymbol[outlined]{crop-square} & \mSymbol[rounded]{crop-square} & \mSymbol[sharp]{crop-square} & \texttt{\textbackslash mSymbol\{crop-square\}} & \texttt{E3C6}\\
\mSymbol[outlined]{crossword} & \mSymbol[rounded]{crossword} & \mSymbol[sharp]{crossword} & \texttt{\textbackslash mSymbol\{crossword\}} & \texttt{F5E5}\\
\mSymbol[outlined]{crowdsource} & \mSymbol[rounded]{crowdsource} & \mSymbol[sharp]{crowdsource} & \texttt{\textbackslash mSymbol\{crowdsource\}} & \texttt{EB18}\\
\mSymbol[outlined]{cruelty-free} & \mSymbol[rounded]{cruelty-free} & \mSymbol[sharp]{cruelty-free} & \texttt{\textbackslash mSymbol\{cruelty-free\}} & \texttt{E799}\\
\mSymbol[outlined]{css} & \mSymbol[rounded]{css} & \mSymbol[sharp]{css} & \texttt{\textbackslash mSymbol\{css\}} & \texttt{EB93}\\
\mSymbol[outlined]{csv} & \mSymbol[rounded]{csv} & \mSymbol[sharp]{csv} & \texttt{\textbackslash mSymbol\{csv\}} & \texttt{E6CF}\\
\mSymbol[outlined]{currency-bitcoin} & \mSymbol[rounded]{currency-bitcoin} & \mSymbol[sharp]{currency-bitcoin} & \texttt{\textbackslash mSymbol\{currency-bitcoin\}} & \texttt{EBC5}\\
\mSymbol[outlined]{currency-exchange} & \mSymbol[rounded]{currency-exchange} & \mSymbol[sharp]{currency-exchange} & \texttt{\textbackslash mSymbol\{currency-exchange\}} & \texttt{EB70}\\
\mSymbol[outlined]{currency-franc} & \mSymbol[rounded]{currency-franc} & \mSymbol[sharp]{currency-franc} & \texttt{\textbackslash mSymbol\{currency-franc\}} & \texttt{EAFA}\\
\mSymbol[outlined]{currency-lira} & \mSymbol[rounded]{currency-lira} & \mSymbol[sharp]{currency-lira} & \texttt{\textbackslash mSymbol\{currency-lira\}} & \texttt{EAEF}\\
\mSymbol[outlined]{currency-pound} & \mSymbol[rounded]{currency-pound} & \mSymbol[sharp]{currency-pound} & \texttt{\textbackslash mSymbol\{currency-pound\}} & \texttt{EAF1}\\
\mSymbol[outlined]{currency-ruble} & \mSymbol[rounded]{currency-ruble} & \mSymbol[sharp]{currency-ruble} & \texttt{\textbackslash mSymbol\{currency-ruble\}} & \texttt{EAEC}\\
\mSymbol[outlined]{currency-rupee} & \mSymbol[rounded]{currency-rupee} & \mSymbol[sharp]{currency-rupee} & \texttt{\textbackslash mSymbol\{currency-rupee\}} & \texttt{EAF7}\\
\mSymbol[outlined]{currency-rupee-circle} & \mSymbol[rounded]{currency-rupee-circle} & \mSymbol[sharp]{currency-rupee-circle} & \texttt{\textbackslash mSymbol\{currency-rupee-circle\}} & \texttt{F460}\\
\mSymbol[outlined]{currency-yen} & \mSymbol[rounded]{currency-yen} & \mSymbol[sharp]{currency-yen} & \texttt{\textbackslash mSymbol\{currency-yen\}} & \texttt{EAFB}\\
\mSymbol[outlined]{currency-yuan} & \mSymbol[rounded]{currency-yuan} & \mSymbol[sharp]{currency-yuan} & \texttt{\textbackslash mSymbol\{currency-yuan\}} & \texttt{EAF9}\\
\mSymbol[outlined]{curtains} & \mSymbol[rounded]{curtains} & \mSymbol[sharp]{curtains} & \texttt{\textbackslash mSymbol\{curtains\}} & \texttt{EC1E}\\
\mSymbol[outlined]{curtains-closed} & \mSymbol[rounded]{curtains-closed} & \mSymbol[sharp]{curtains-closed} & \texttt{\textbackslash mSymbol\{curtains-closed\}} & \texttt{EC1D}\\
\mSymbol[outlined]{custom-typography} & \mSymbol[rounded]{custom-typography} & \mSymbol[sharp]{custom-typography} & \texttt{\textbackslash mSymbol\{custom-typography\}} & \texttt{E732}\\
\mSymbol[outlined]{cut} & \mSymbol[rounded]{cut} & \mSymbol[sharp]{cut} & \texttt{\textbackslash mSymbol\{cut\}} & \texttt{F08B}\\
\mSymbol[outlined]{cycle} & \mSymbol[rounded]{cycle} & \mSymbol[sharp]{cycle} & \texttt{\textbackslash mSymbol\{cycle\}} & \texttt{F854}\\
\mSymbol[outlined]{cyclone} & \mSymbol[rounded]{cyclone} & \mSymbol[sharp]{cyclone} & \texttt{\textbackslash mSymbol\{cyclone\}} & \texttt{EBD5}\\
\mSymbol[outlined]{dangerous} & \mSymbol[rounded]{dangerous} & \mSymbol[sharp]{dangerous} & \texttt{\textbackslash mSymbol\{dangerous\}} & \texttt{E99A}\\
\mSymbol[outlined]{dark-mode} & \mSymbol[rounded]{dark-mode} & \mSymbol[sharp]{dark-mode} & \texttt{\textbackslash mSymbol\{dark-mode\}} & \texttt{E51C}\\
\mSymbol[outlined]{dashboard} & \mSymbol[rounded]{dashboard} & \mSymbol[sharp]{dashboard} & \texttt{\textbackslash mSymbol\{dashboard\}} & \texttt{E871}\\
\mSymbol[outlined]{dashboard-customize} & \mSymbol[rounded]{dashboard-customize} & \mSymbol[sharp]{dashboard-customize} & \texttt{\textbackslash mSymbol\{dashboard-customize\}} & \texttt{E99B}\\
\mSymbol[outlined]{data-alert} & \mSymbol[rounded]{data-alert} & \mSymbol[sharp]{data-alert} & \texttt{\textbackslash mSymbol\{data-alert\}} & \texttt{F7F6}\\
\mSymbol[outlined]{data-array} & \mSymbol[rounded]{data-array} & \mSymbol[sharp]{data-array} & \texttt{\textbackslash mSymbol\{data-array\}} & \texttt{EAD1}\\
\mSymbol[outlined]{data-check} & \mSymbol[rounded]{data-check} & \mSymbol[sharp]{data-check} & \texttt{\textbackslash mSymbol\{data-check\}} & \texttt{F7F2}\\
\mSymbol[outlined]{data-exploration} & \mSymbol[rounded]{data-exploration} & \mSymbol[sharp]{data-exploration} & \texttt{\textbackslash mSymbol\{data-exploration\}} & \texttt{E76F}\\
\mSymbol[outlined]{data-info-alert} & \mSymbol[rounded]{data-info-alert} & \mSymbol[sharp]{data-info-alert} & \texttt{\textbackslash mSymbol\{data-info-alert\}} & \texttt{F7F5}\\
\mSymbol[outlined]{data-loss-prevention} & \mSymbol[rounded]{data-loss-prevention} & \mSymbol[sharp]{data-loss-prevention} & \texttt{\textbackslash mSymbol\{data-loss-prevention\}} & \texttt{E2DC}\\
\mSymbol[outlined]{data-object} & \mSymbol[rounded]{data-object} & \mSymbol[sharp]{data-object} & \texttt{\textbackslash mSymbol\{data-object\}} & \texttt{EAD3}\\
\mSymbol[outlined]{data-saver-off} & \mSymbol[rounded]{data-saver-off} & \mSymbol[sharp]{data-saver-off} & \texttt{\textbackslash mSymbol\{data-saver-off\}} & \texttt{EFF2}\\
\mSymbol[outlined]{data-saver-on} & \mSymbol[rounded]{data-saver-on} & \mSymbol[sharp]{data-saver-on} & \texttt{\textbackslash mSymbol\{data-saver-on\}} & \texttt{EFF3}\\
\mSymbol[outlined]{data-table} & \mSymbol[rounded]{data-table} & \mSymbol[sharp]{data-table} & \texttt{\textbackslash mSymbol\{data-table\}} & \texttt{E99C}\\
\mSymbol[outlined]{data-thresholding} & \mSymbol[rounded]{data-thresholding} & \mSymbol[sharp]{data-thresholding} & \texttt{\textbackslash mSymbol\{data-thresholding\}} & \texttt{EB9F}\\
\mSymbol[outlined]{data-usage} & \mSymbol[rounded]{data-usage} & \mSymbol[sharp]{data-usage} & \texttt{\textbackslash mSymbol\{data-usage\}} & \texttt{EFF2}\\
\mSymbol[outlined]{database} & \mSymbol[rounded]{database} & \mSymbol[sharp]{database} & \texttt{\textbackslash mSymbol\{database\}} & \texttt{F20E}\\
\mSymbol[outlined]{database-off} & \mSymbol[rounded]{database-off} & \mSymbol[sharp]{database-off} & \texttt{\textbackslash mSymbol\{database-off\}} & \texttt{F414}\\
\mSymbol[outlined]{dataset} & \mSymbol[rounded]{dataset} & \mSymbol[sharp]{dataset} & \texttt{\textbackslash mSymbol\{dataset\}} & \texttt{F8EE}\\
\mSymbol[outlined]{dataset-linked} & \mSymbol[rounded]{dataset-linked} & \mSymbol[sharp]{dataset-linked} & \texttt{\textbackslash mSymbol\{dataset-linked\}} & \texttt{F8EF}\\
\mSymbol[outlined]{date-range} & \mSymbol[rounded]{date-range} & \mSymbol[sharp]{date-range} & \texttt{\textbackslash mSymbol\{date-range\}} & \texttt{E916}\\
\mSymbol[outlined]{deblur} & \mSymbol[rounded]{deblur} & \mSymbol[sharp]{deblur} & \texttt{\textbackslash mSymbol\{deblur\}} & \texttt{EB77}\\
\mSymbol[outlined]{deceased} & \mSymbol[rounded]{deceased} & \mSymbol[sharp]{deceased} & \texttt{\textbackslash mSymbol\{deceased\}} & \texttt{E0A5}\\
\mSymbol[outlined]{decimal-decrease} & \mSymbol[rounded]{decimal-decrease} & \mSymbol[sharp]{decimal-decrease} & \texttt{\textbackslash mSymbol\{decimal-decrease\}} & \texttt{F82D}\\
\mSymbol[outlined]{decimal-increase} & \mSymbol[rounded]{decimal-increase} & \mSymbol[sharp]{decimal-increase} & \texttt{\textbackslash mSymbol\{decimal-increase\}} & \texttt{F82C}\\
\mSymbol[outlined]{deck} & \mSymbol[rounded]{deck} & \mSymbol[sharp]{deck} & \texttt{\textbackslash mSymbol\{deck\}} & \texttt{EA42}\\
\mSymbol[outlined]{dehaze} & \mSymbol[rounded]{dehaze} & \mSymbol[sharp]{dehaze} & \texttt{\textbackslash mSymbol\{dehaze\}} & \texttt{E3C7}\\
\mSymbol[outlined]{delete} & \mSymbol[rounded]{delete} & \mSymbol[sharp]{delete} & \texttt{\textbackslash mSymbol\{delete\}} & \texttt{E92E}\\
\mSymbol[outlined]{delete-forever} & \mSymbol[rounded]{delete-forever} & \mSymbol[sharp]{delete-forever} & \texttt{\textbackslash mSymbol\{delete-forever\}} & \texttt{E92B}\\
\mSymbol[outlined]{delete-history} & \mSymbol[rounded]{delete-history} & \mSymbol[sharp]{delete-history} & \texttt{\textbackslash mSymbol\{delete-history\}} & \texttt{F518}\\
\mSymbol[outlined]{delete-outline} & \mSymbol[rounded]{delete-outline} & \mSymbol[sharp]{delete-outline} & \texttt{\textbackslash mSymbol\{delete-outline\}} & \texttt{E92E}\\
\mSymbol[outlined]{delete-sweep} & \mSymbol[rounded]{delete-sweep} & \mSymbol[sharp]{delete-sweep} & \texttt{\textbackslash mSymbol\{delete-sweep\}} & \texttt{E16C}\\
\mSymbol[outlined]{demography} & \mSymbol[rounded]{demography} & \mSymbol[sharp]{demography} & \texttt{\textbackslash mSymbol\{demography\}} & \texttt{E489}\\
\mSymbol[outlined]{density-large} & \mSymbol[rounded]{density-large} & \mSymbol[sharp]{density-large} & \texttt{\textbackslash mSymbol\{density-large\}} & \texttt{EBA9}\\
\mSymbol[outlined]{density-medium} & \mSymbol[rounded]{density-medium} & \mSymbol[sharp]{density-medium} & \texttt{\textbackslash mSymbol\{density-medium\}} & \texttt{EB9E}\\
\mSymbol[outlined]{density-small} & \mSymbol[rounded]{density-small} & \mSymbol[sharp]{density-small} & \texttt{\textbackslash mSymbol\{density-small\}} & \texttt{EBA8}\\
\mSymbol[outlined]{dentistry} & \mSymbol[rounded]{dentistry} & \mSymbol[sharp]{dentistry} & \texttt{\textbackslash mSymbol\{dentistry\}} & \texttt{E0A6}\\
\mSymbol[outlined]{departure-board} & \mSymbol[rounded]{departure-board} & \mSymbol[sharp]{departure-board} & \texttt{\textbackslash mSymbol\{departure-board\}} & \texttt{E576}\\
\mSymbol[outlined]{deployed-code} & \mSymbol[rounded]{deployed-code} & \mSymbol[sharp]{deployed-code} & \texttt{\textbackslash mSymbol\{deployed-code\}} & \texttt{F720}\\
\mSymbol[outlined]{deployed-code-account} & \mSymbol[rounded]{deployed-code-account} & \mSymbol[sharp]{deployed-code-account} & \texttt{\textbackslash mSymbol\{deployed-code-account\}} & \texttt{F51B}\\
\mSymbol[outlined]{deployed-code-alert} & \mSymbol[rounded]{deployed-code-alert} & \mSymbol[sharp]{deployed-code-alert} & \texttt{\textbackslash mSymbol\{deployed-code-alert\}} & \texttt{F5F2}\\
\mSymbol[outlined]{deployed-code-history} & \mSymbol[rounded]{deployed-code-history} & \mSymbol[sharp]{deployed-code-history} & \texttt{\textbackslash mSymbol\{deployed-code-history\}} & \texttt{F5F3}\\
\mSymbol[outlined]{deployed-code-update} & \mSymbol[rounded]{deployed-code-update} & \mSymbol[sharp]{deployed-code-update} & \texttt{\textbackslash mSymbol\{deployed-code-update\}} & \texttt{F5F4}\\
\mSymbol[outlined]{dermatology} & \mSymbol[rounded]{dermatology} & \mSymbol[sharp]{dermatology} & \texttt{\textbackslash mSymbol\{dermatology\}} & \texttt{E0A7}\\
\mSymbol[outlined]{description} & \mSymbol[rounded]{description} & \mSymbol[sharp]{description} & \texttt{\textbackslash mSymbol\{description\}} & \texttt{E873}\\
\mSymbol[outlined]{deselect} & \mSymbol[rounded]{deselect} & \mSymbol[sharp]{deselect} & \texttt{\textbackslash mSymbol\{deselect\}} & \texttt{EBB6}\\
\mSymbol[outlined]{design-services} & \mSymbol[rounded]{design-services} & \mSymbol[sharp]{design-services} & \texttt{\textbackslash mSymbol\{design-services\}} & \texttt{F10A}\\
\mSymbol[outlined]{desk} & \mSymbol[rounded]{desk} & \mSymbol[sharp]{desk} & \texttt{\textbackslash mSymbol\{desk\}} & \texttt{F8F4}\\
\mSymbol[outlined]{deskphone} & \mSymbol[rounded]{deskphone} & \mSymbol[sharp]{deskphone} & \texttt{\textbackslash mSymbol\{deskphone\}} & \texttt{F7FA}\\
\mSymbol[outlined]{desktop-access-disabled} & \mSymbol[rounded]{desktop-access-disabled} & \mSymbol[sharp]{desktop-access-disabled} & \texttt{\textbackslash mSymbol\{desktop-access-disabled\}} & \texttt{E99D}\\
\mSymbol[outlined]{desktop-landscape} & \mSymbol[rounded]{desktop-landscape} & \mSymbol[sharp]{desktop-landscape} & \texttt{\textbackslash mSymbol\{desktop-landscape\}} & \texttt{F45E}\\
\mSymbol[outlined]{desktop-landscape-add} & \mSymbol[rounded]{desktop-landscape-add} & \mSymbol[sharp]{desktop-landscape-add} & \texttt{\textbackslash mSymbol\{desktop-landscape-add\}} & \texttt{F439}\\
\mSymbol[outlined]{desktop-mac} & \mSymbol[rounded]{desktop-mac} & \mSymbol[sharp]{desktop-mac} & \texttt{\textbackslash mSymbol\{desktop-mac\}} & \texttt{E30B}\\
\mSymbol[outlined]{desktop-portrait} & \mSymbol[rounded]{desktop-portrait} & \mSymbol[sharp]{desktop-portrait} & \texttt{\textbackslash mSymbol\{desktop-portrait\}} & \texttt{F45D}\\
\mSymbol[outlined]{desktop-windows} & \mSymbol[rounded]{desktop-windows} & \mSymbol[sharp]{desktop-windows} & \texttt{\textbackslash mSymbol\{desktop-windows\}} & \texttt{E30C}\\
\mSymbol[outlined]{destruction} & \mSymbol[rounded]{destruction} & \mSymbol[sharp]{destruction} & \texttt{\textbackslash mSymbol\{destruction\}} & \texttt{F585}\\
\mSymbol[outlined]{details} & \mSymbol[rounded]{details} & \mSymbol[sharp]{details} & \texttt{\textbackslash mSymbol\{details\}} & \texttt{E3C8}\\
\mSymbol[outlined]{detection-and-zone} & \mSymbol[rounded]{detection-and-zone} & \mSymbol[sharp]{detection-and-zone} & \texttt{\textbackslash mSymbol\{detection-and-zone\}} & \texttt{E29F}\\
\mSymbol[outlined]{detector} & \mSymbol[rounded]{detector} & \mSymbol[sharp]{detector} & \texttt{\textbackslash mSymbol\{detector\}} & \texttt{E282}\\
\mSymbol[outlined]{detector-alarm} & \mSymbol[rounded]{detector-alarm} & \mSymbol[sharp]{detector-alarm} & \texttt{\textbackslash mSymbol\{detector-alarm\}} & \texttt{E1F7}\\
\mSymbol[outlined]{detector-battery} & \mSymbol[rounded]{detector-battery} & \mSymbol[sharp]{detector-battery} & \texttt{\textbackslash mSymbol\{detector-battery\}} & \texttt{E204}\\
\mSymbol[outlined]{detector-co} & \mSymbol[rounded]{detector-co} & \mSymbol[sharp]{detector-co} & \texttt{\textbackslash mSymbol\{detector-co\}} & \texttt{E2AF}\\
\mSymbol[outlined]{detector-offline} & \mSymbol[rounded]{detector-offline} & \mSymbol[sharp]{detector-offline} & \texttt{\textbackslash mSymbol\{detector-offline\}} & \texttt{E223}\\
\mSymbol[outlined]{detector-smoke} & \mSymbol[rounded]{detector-smoke} & \mSymbol[sharp]{detector-smoke} & \texttt{\textbackslash mSymbol\{detector-smoke\}} & \texttt{E285}\\
\mSymbol[outlined]{detector-status} & \mSymbol[rounded]{detector-status} & \mSymbol[sharp]{detector-status} & \texttt{\textbackslash mSymbol\{detector-status\}} & \texttt{E1E8}\\
\mSymbol[outlined]{developer-board} & \mSymbol[rounded]{developer-board} & \mSymbol[sharp]{developer-board} & \texttt{\textbackslash mSymbol\{developer-board\}} & \texttt{E30D}\\
\mSymbol[outlined]{developer-board-off} & \mSymbol[rounded]{developer-board-off} & \mSymbol[sharp]{developer-board-off} & \texttt{\textbackslash mSymbol\{developer-board-off\}} & \texttt{E4FF}\\
\mSymbol[outlined]{developer-guide} & \mSymbol[rounded]{developer-guide} & \mSymbol[sharp]{developer-guide} & \texttt{\textbackslash mSymbol\{developer-guide\}} & \texttt{E99E}\\
\mSymbol[outlined]{developer-mode} & \mSymbol[rounded]{developer-mode} & \mSymbol[sharp]{developer-mode} & \texttt{\textbackslash mSymbol\{developer-mode\}} & \texttt{E1B0}\\
\mSymbol[outlined]{developer-mode-tv} & \mSymbol[rounded]{developer-mode-tv} & \mSymbol[sharp]{developer-mode-tv} & \texttt{\textbackslash mSymbol\{developer-mode-tv\}} & \texttt{E874}\\
\mSymbol[outlined]{device-hub} & \mSymbol[rounded]{device-hub} & \mSymbol[sharp]{device-hub} & \texttt{\textbackslash mSymbol\{device-hub\}} & \texttt{E335}\\
\mSymbol[outlined]{device-reset} & \mSymbol[rounded]{device-reset} & \mSymbol[sharp]{device-reset} & \texttt{\textbackslash mSymbol\{device-reset\}} & \texttt{E8B3}\\
\mSymbol[outlined]{device-thermostat} & \mSymbol[rounded]{device-thermostat} & \mSymbol[sharp]{device-thermostat} & \texttt{\textbackslash mSymbol\{device-thermostat\}} & \texttt{E1FF}\\
\mSymbol[outlined]{device-unknown} & \mSymbol[rounded]{device-unknown} & \mSymbol[sharp]{device-unknown} & \texttt{\textbackslash mSymbol\{device-unknown\}} & \texttt{E339}\\
\mSymbol[outlined]{devices} & \mSymbol[rounded]{devices} & \mSymbol[sharp]{devices} & \texttt{\textbackslash mSymbol\{devices\}} & \texttt{E326}\\
\mSymbol[outlined]{devices-fold} & \mSymbol[rounded]{devices-fold} & \mSymbol[sharp]{devices-fold} & \texttt{\textbackslash mSymbol\{devices-fold\}} & \texttt{EBDE}\\
\mSymbol[outlined]{devices-off} & \mSymbol[rounded]{devices-off} & \mSymbol[sharp]{devices-off} & \texttt{\textbackslash mSymbol\{devices-off\}} & \texttt{F7A5}\\
\mSymbol[outlined]{devices-other} & \mSymbol[rounded]{devices-other} & \mSymbol[sharp]{devices-other} & \texttt{\textbackslash mSymbol\{devices-other\}} & \texttt{E337}\\
\mSymbol[outlined]{devices-wearables} & \mSymbol[rounded]{devices-wearables} & \mSymbol[sharp]{devices-wearables} & \texttt{\textbackslash mSymbol\{devices-wearables\}} & \texttt{F6AB}\\
\mSymbol[outlined]{dew-point} & \mSymbol[rounded]{dew-point} & \mSymbol[sharp]{dew-point} & \texttt{\textbackslash mSymbol\{dew-point\}} & \texttt{F879}\\
\mSymbol[outlined]{diagnosis} & \mSymbol[rounded]{diagnosis} & \mSymbol[sharp]{diagnosis} & \texttt{\textbackslash mSymbol\{diagnosis\}} & \texttt{E0A8}\\
\mSymbol[outlined]{diagonal-line} & \mSymbol[rounded]{diagonal-line} & \mSymbol[sharp]{diagonal-line} & \texttt{\textbackslash mSymbol\{diagonal-line\}} & \texttt{F41E}\\
\mSymbol[outlined]{dialer-sip} & \mSymbol[rounded]{dialer-sip} & \mSymbol[sharp]{dialer-sip} & \texttt{\textbackslash mSymbol\{dialer-sip\}} & \texttt{E0BB}\\
\mSymbol[outlined]{dialogs} & \mSymbol[rounded]{dialogs} & \mSymbol[sharp]{dialogs} & \texttt{\textbackslash mSymbol\{dialogs\}} & \texttt{E99F}\\
\mSymbol[outlined]{dialpad} & \mSymbol[rounded]{dialpad} & \mSymbol[sharp]{dialpad} & \texttt{\textbackslash mSymbol\{dialpad\}} & \texttt{E0BC}\\
\mSymbol[outlined]{diamond} & \mSymbol[rounded]{diamond} & \mSymbol[sharp]{diamond} & \texttt{\textbackslash mSymbol\{diamond\}} & \texttt{EAD5}\\
\mSymbol[outlined]{dictionary} & \mSymbol[rounded]{dictionary} & \mSymbol[sharp]{dictionary} & \texttt{\textbackslash mSymbol\{dictionary\}} & \texttt{F539}\\
\mSymbol[outlined]{difference} & \mSymbol[rounded]{difference} & \mSymbol[sharp]{difference} & \texttt{\textbackslash mSymbol\{difference\}} & \texttt{EB7D}\\
\mSymbol[outlined]{digital-out-of-home} & \mSymbol[rounded]{digital-out-of-home} & \mSymbol[sharp]{digital-out-of-home} & \texttt{\textbackslash mSymbol\{digital-out-of-home\}} & \texttt{F1DE}\\
\mSymbol[outlined]{digital-wellbeing} & \mSymbol[rounded]{digital-wellbeing} & \mSymbol[sharp]{digital-wellbeing} & \texttt{\textbackslash mSymbol\{digital-wellbeing\}} & \texttt{EF86}\\
\mSymbol[outlined]{dining} & \mSymbol[rounded]{dining} & \mSymbol[sharp]{dining} & \texttt{\textbackslash mSymbol\{dining\}} & \texttt{EFF4}\\
\mSymbol[outlined]{dinner-dining} & \mSymbol[rounded]{dinner-dining} & \mSymbol[sharp]{dinner-dining} & \texttt{\textbackslash mSymbol\{dinner-dining\}} & \texttt{EA57}\\
\mSymbol[outlined]{directions} & \mSymbol[rounded]{directions} & \mSymbol[sharp]{directions} & \texttt{\textbackslash mSymbol\{directions\}} & \texttt{E52E}\\
\mSymbol[outlined]{directions-alt} & \mSymbol[rounded]{directions-alt} & \mSymbol[sharp]{directions-alt} & \texttt{\textbackslash mSymbol\{directions-alt\}} & \texttt{F880}\\
\mSymbol[outlined]{directions-alt-off} & \mSymbol[rounded]{directions-alt-off} & \mSymbol[sharp]{directions-alt-off} & \texttt{\textbackslash mSymbol\{directions-alt-off\}} & \texttt{F881}\\
\mSymbol[outlined]{directions-bike} & \mSymbol[rounded]{directions-bike} & \mSymbol[sharp]{directions-bike} & \texttt{\textbackslash mSymbol\{directions-bike\}} & \texttt{E52F}\\
\mSymbol[outlined]{directions-boat} & \mSymbol[rounded]{directions-boat} & \mSymbol[sharp]{directions-boat} & \texttt{\textbackslash mSymbol\{directions-boat\}} & \texttt{EFF5}\\
\mSymbol[outlined]{directions-boat-filled} & \mSymbol[rounded]{directions-boat-filled} & \mSymbol[sharp]{directions-boat-filled} & \texttt{\textbackslash mSymbol\{directions-boat-filled\}} & \texttt{EFF5}\\
\mSymbol[outlined]{directions-bus} & \mSymbol[rounded]{directions-bus} & \mSymbol[sharp]{directions-bus} & \texttt{\textbackslash mSymbol\{directions-bus\}} & \texttt{EFF6}\\
\mSymbol[outlined]{directions-bus-filled} & \mSymbol[rounded]{directions-bus-filled} & \mSymbol[sharp]{directions-bus-filled} & \texttt{\textbackslash mSymbol\{directions-bus-filled\}} & \texttt{EFF6}\\
\mSymbol[outlined]{directions-car} & \mSymbol[rounded]{directions-car} & \mSymbol[sharp]{directions-car} & \texttt{\textbackslash mSymbol\{directions-car\}} & \texttt{EFF7}\\
\mSymbol[outlined]{directions-car-filled} & \mSymbol[rounded]{directions-car-filled} & \mSymbol[sharp]{directions-car-filled} & \texttt{\textbackslash mSymbol\{directions-car-filled\}} & \texttt{EFF7}\\
\mSymbol[outlined]{directions-off} & \mSymbol[rounded]{directions-off} & \mSymbol[sharp]{directions-off} & \texttt{\textbackslash mSymbol\{directions-off\}} & \texttt{F10F}\\
\mSymbol[outlined]{directions-railway} & \mSymbol[rounded]{directions-railway} & \mSymbol[sharp]{directions-railway} & \texttt{\textbackslash mSymbol\{directions-railway\}} & \texttt{EFF8}\\
\mSymbol[outlined]{directions-railway-2} & \mSymbol[rounded]{directions-railway-2} & \mSymbol[sharp]{directions-railway-2} & \texttt{\textbackslash mSymbol\{directions-railway-2\}} & \texttt{F462}\\
\mSymbol[outlined]{directions-railway-filled} & \mSymbol[rounded]{directions-railway-filled} & \mSymbol[sharp]{directions-railway-filled} & \texttt{\textbackslash mSymbol\{directions-railway-filled\}} & \texttt{EFF8}\\
\mSymbol[outlined]{directions-run} & \mSymbol[rounded]{directions-run} & \mSymbol[sharp]{directions-run} & \texttt{\textbackslash mSymbol\{directions-run\}} & \texttt{E566}\\
\mSymbol[outlined]{directions-subway} & \mSymbol[rounded]{directions-subway} & \mSymbol[sharp]{directions-subway} & \texttt{\textbackslash mSymbol\{directions-subway\}} & \texttt{EFFA}\\
\mSymbol[outlined]{directions-subway-filled} & \mSymbol[rounded]{directions-subway-filled} & \mSymbol[sharp]{directions-subway-filled} & \texttt{\textbackslash mSymbol\{directions-subway-filled\}} & \texttt{EFFA}\\
\mSymbol[outlined]{directions-transit} & \mSymbol[rounded]{directions-transit} & \mSymbol[sharp]{directions-transit} & \texttt{\textbackslash mSymbol\{directions-transit\}} & \texttt{EFFA}\\
\mSymbol[outlined]{directions-transit-filled} & \mSymbol[rounded]{directions-transit-filled} & \mSymbol[sharp]{directions-transit-filled} & \texttt{\textbackslash mSymbol\{directions-transit-filled\}} & \texttt{EFFA}\\
\mSymbol[outlined]{directions-walk} & \mSymbol[rounded]{directions-walk} & \mSymbol[sharp]{directions-walk} & \texttt{\textbackslash mSymbol\{directions-walk\}} & \texttt{E536}\\
\mSymbol[outlined]{directory-sync} & \mSymbol[rounded]{directory-sync} & \mSymbol[sharp]{directory-sync} & \texttt{\textbackslash mSymbol\{directory-sync\}} & \texttt{E394}\\
\mSymbol[outlined]{dirty-lens} & \mSymbol[rounded]{dirty-lens} & \mSymbol[sharp]{dirty-lens} & \texttt{\textbackslash mSymbol\{dirty-lens\}} & \texttt{EF4B}\\
\mSymbol[outlined]{disabled-by-default} & \mSymbol[rounded]{disabled-by-default} & \mSymbol[sharp]{disabled-by-default} & \texttt{\textbackslash mSymbol\{disabled-by-default\}} & \texttt{F230}\\
\mSymbol[outlined]{disabled-visible} & \mSymbol[rounded]{disabled-visible} & \mSymbol[sharp]{disabled-visible} & \texttt{\textbackslash mSymbol\{disabled-visible\}} & \texttt{E76E}\\
\mSymbol[outlined]{disc-full} & \mSymbol[rounded]{disc-full} & \mSymbol[sharp]{disc-full} & \texttt{\textbackslash mSymbol\{disc-full\}} & \texttt{E610}\\
\mSymbol[outlined]{discover-tune} & \mSymbol[rounded]{discover-tune} & \mSymbol[sharp]{discover-tune} & \texttt{\textbackslash mSymbol\{discover-tune\}} & \texttt{E018}\\
\mSymbol[outlined]{dishwasher} & \mSymbol[rounded]{dishwasher} & \mSymbol[sharp]{dishwasher} & \texttt{\textbackslash mSymbol\{dishwasher\}} & \texttt{E9A0}\\
\mSymbol[outlined]{dishwasher-gen} & \mSymbol[rounded]{dishwasher-gen} & \mSymbol[sharp]{dishwasher-gen} & \texttt{\textbackslash mSymbol\{dishwasher-gen\}} & \texttt{E832}\\
\mSymbol[outlined]{display-external-input} & \mSymbol[rounded]{display-external-input} & \mSymbol[sharp]{display-external-input} & \texttt{\textbackslash mSymbol\{display-external-input\}} & \texttt{F7E7}\\
\mSymbol[outlined]{display-settings} & \mSymbol[rounded]{display-settings} & \mSymbol[sharp]{display-settings} & \texttt{\textbackslash mSymbol\{display-settings\}} & \texttt{EB97}\\
\mSymbol[outlined]{distance} & \mSymbol[rounded]{distance} & \mSymbol[sharp]{distance} & \texttt{\textbackslash mSymbol\{distance\}} & \texttt{F6EA}\\
\mSymbol[outlined]{diversity-1} & \mSymbol[rounded]{diversity-1} & \mSymbol[sharp]{diversity-1} & \texttt{\textbackslash mSymbol\{diversity-1\}} & \texttt{F8D7}\\
\mSymbol[outlined]{diversity-2} & \mSymbol[rounded]{diversity-2} & \mSymbol[sharp]{diversity-2} & \texttt{\textbackslash mSymbol\{diversity-2\}} & \texttt{F8D8}\\
\mSymbol[outlined]{diversity-3} & \mSymbol[rounded]{diversity-3} & \mSymbol[sharp]{diversity-3} & \texttt{\textbackslash mSymbol\{diversity-3\}} & \texttt{F8D9}\\
\mSymbol[outlined]{diversity-4} & \mSymbol[rounded]{diversity-4} & \mSymbol[sharp]{diversity-4} & \texttt{\textbackslash mSymbol\{diversity-4\}} & \texttt{F857}\\
\mSymbol[outlined]{dns} & \mSymbol[rounded]{dns} & \mSymbol[sharp]{dns} & \texttt{\textbackslash mSymbol\{dns\}} & \texttt{E875}\\
\mSymbol[outlined]{do-disturb} & \mSymbol[rounded]{do-disturb} & \mSymbol[sharp]{do-disturb} & \texttt{\textbackslash mSymbol\{do-disturb\}} & \texttt{F08C}\\
\mSymbol[outlined]{do-disturb-alt} & \mSymbol[rounded]{do-disturb-alt} & \mSymbol[sharp]{do-disturb-alt} & \texttt{\textbackslash mSymbol\{do-disturb-alt\}} & \texttt{F08D}\\
\mSymbol[outlined]{do-disturb-off} & \mSymbol[rounded]{do-disturb-off} & \mSymbol[sharp]{do-disturb-off} & \texttt{\textbackslash mSymbol\{do-disturb-off\}} & \texttt{F08E}\\
\mSymbol[outlined]{do-disturb-on} & \mSymbol[rounded]{do-disturb-on} & \mSymbol[sharp]{do-disturb-on} & \texttt{\textbackslash mSymbol\{do-disturb-on\}} & \texttt{F08F}\\
\mSymbol[outlined]{do-not-disturb} & \mSymbol[rounded]{do-not-disturb} & \mSymbol[sharp]{do-not-disturb} & \texttt{\textbackslash mSymbol\{do-not-disturb\}} & \texttt{F08D}\\
\mSymbol[outlined]{do-not-disturb-alt} & \mSymbol[rounded]{do-not-disturb-alt} & \mSymbol[sharp]{do-not-disturb-alt} & \texttt{\textbackslash mSymbol\{do-not-disturb-alt\}} & \texttt{F08C}\\
\mSymbol[outlined]{do-not-disturb-off} & \mSymbol[rounded]{do-not-disturb-off} & \mSymbol[sharp]{do-not-disturb-off} & \texttt{\textbackslash mSymbol\{do-not-disturb-off\}} & \texttt{F08E}\\
\mSymbol[outlined]{do-not-disturb-on} & \mSymbol[rounded]{do-not-disturb-on} & \mSymbol[sharp]{do-not-disturb-on} & \texttt{\textbackslash mSymbol\{do-not-disturb-on\}} & \texttt{F08F}\\
\mSymbol[outlined]{do-not-disturb-on-total-silence} & \mSymbol[rounded]{do-not-disturb-on-total-silence} & \mSymbol[sharp]{do-not-disturb-on-total-silence} & \texttt{\textbackslash mSymbol\{do-not-disturb-on-total-silence\}} & \texttt{EFFB}\\
\mSymbol[outlined]{do-not-step} & \mSymbol[rounded]{do-not-step} & \mSymbol[sharp]{do-not-step} & \texttt{\textbackslash mSymbol\{do-not-step\}} & \texttt{F19F}\\
\mSymbol[outlined]{do-not-touch} & \mSymbol[rounded]{do-not-touch} & \mSymbol[sharp]{do-not-touch} & \texttt{\textbackslash mSymbol\{do-not-touch\}} & \texttt{F1B0}\\
\mSymbol[outlined]{dock} & \mSymbol[rounded]{dock} & \mSymbol[sharp]{dock} & \texttt{\textbackslash mSymbol\{dock\}} & \texttt{E30E}\\
\mSymbol[outlined]{dock-to-bottom} & \mSymbol[rounded]{dock-to-bottom} & \mSymbol[sharp]{dock-to-bottom} & \texttt{\textbackslash mSymbol\{dock-to-bottom\}} & \texttt{F7E6}\\
\mSymbol[outlined]{dock-to-left} & \mSymbol[rounded]{dock-to-left} & \mSymbol[sharp]{dock-to-left} & \texttt{\textbackslash mSymbol\{dock-to-left\}} & \texttt{F7E5}\\
\mSymbol[outlined]{dock-to-right} & \mSymbol[rounded]{dock-to-right} & \mSymbol[sharp]{dock-to-right} & \texttt{\textbackslash mSymbol\{dock-to-right\}} & \texttt{F7E4}\\
\mSymbol[outlined]{docs-add-on} & \mSymbol[rounded]{docs-add-on} & \mSymbol[sharp]{docs-add-on} & \texttt{\textbackslash mSymbol\{docs-add-on\}} & \texttt{F0C2}\\
\mSymbol[outlined]{docs-apps-script} & \mSymbol[rounded]{docs-apps-script} & \mSymbol[sharp]{docs-apps-script} & \texttt{\textbackslash mSymbol\{docs-apps-script\}} & \texttt{F0C3}\\
\mSymbol[outlined]{document-scanner} & \mSymbol[rounded]{document-scanner} & \mSymbol[sharp]{document-scanner} & \texttt{\textbackslash mSymbol\{document-scanner\}} & \texttt{E5FA}\\
\mSymbol[outlined]{domain} & \mSymbol[rounded]{domain} & \mSymbol[sharp]{domain} & \texttt{\textbackslash mSymbol\{domain\}} & \texttt{E7EE}\\
\mSymbol[outlined]{domain-add} & \mSymbol[rounded]{domain-add} & \mSymbol[sharp]{domain-add} & \texttt{\textbackslash mSymbol\{domain-add\}} & \texttt{EB62}\\
\mSymbol[outlined]{domain-disabled} & \mSymbol[rounded]{domain-disabled} & \mSymbol[sharp]{domain-disabled} & \texttt{\textbackslash mSymbol\{domain-disabled\}} & \texttt{E0EF}\\
\mSymbol[outlined]{domain-verification} & \mSymbol[rounded]{domain-verification} & \mSymbol[sharp]{domain-verification} & \texttt{\textbackslash mSymbol\{domain-verification\}} & \texttt{EF4C}\\
\mSymbol[outlined]{domain-verification-off} & \mSymbol[rounded]{domain-verification-off} & \mSymbol[sharp]{domain-verification-off} & \texttt{\textbackslash mSymbol\{domain-verification-off\}} & \texttt{F7B0}\\
\mSymbol[outlined]{domino-mask} & \mSymbol[rounded]{domino-mask} & \mSymbol[sharp]{domino-mask} & \texttt{\textbackslash mSymbol\{domino-mask\}} & \texttt{F5E4}\\
\mSymbol[outlined]{done} & \mSymbol[rounded]{done} & \mSymbol[sharp]{done} & \texttt{\textbackslash mSymbol\{done\}} & \texttt{E876}\\
\mSymbol[outlined]{done-all} & \mSymbol[rounded]{done-all} & \mSymbol[sharp]{done-all} & \texttt{\textbackslash mSymbol\{done-all\}} & \texttt{E877}\\
\mSymbol[outlined]{done-outline} & \mSymbol[rounded]{done-outline} & \mSymbol[sharp]{done-outline} & \texttt{\textbackslash mSymbol\{done-outline\}} & \texttt{E92F}\\
\mSymbol[outlined]{donut-large} & \mSymbol[rounded]{donut-large} & \mSymbol[sharp]{donut-large} & \texttt{\textbackslash mSymbol\{donut-large\}} & \texttt{E917}\\
\mSymbol[outlined]{donut-small} & \mSymbol[rounded]{donut-small} & \mSymbol[sharp]{donut-small} & \texttt{\textbackslash mSymbol\{donut-small\}} & \texttt{E918}\\
\mSymbol[outlined]{door-back} & \mSymbol[rounded]{door-back} & \mSymbol[sharp]{door-back} & \texttt{\textbackslash mSymbol\{door-back\}} & \texttt{EFFC}\\
\mSymbol[outlined]{door-front} & \mSymbol[rounded]{door-front} & \mSymbol[sharp]{door-front} & \texttt{\textbackslash mSymbol\{door-front\}} & \texttt{EFFD}\\
\mSymbol[outlined]{door-open} & \mSymbol[rounded]{door-open} & \mSymbol[sharp]{door-open} & \texttt{\textbackslash mSymbol\{door-open\}} & \texttt{E77C}\\
\mSymbol[outlined]{door-sensor} & \mSymbol[rounded]{door-sensor} & \mSymbol[sharp]{door-sensor} & \texttt{\textbackslash mSymbol\{door-sensor\}} & \texttt{E28A}\\
\mSymbol[outlined]{door-sliding} & \mSymbol[rounded]{door-sliding} & \mSymbol[sharp]{door-sliding} & \texttt{\textbackslash mSymbol\{door-sliding\}} & \texttt{EFFE}\\
\mSymbol[outlined]{doorbell} & \mSymbol[rounded]{doorbell} & \mSymbol[sharp]{doorbell} & \texttt{\textbackslash mSymbol\{doorbell\}} & \texttt{EFFF}\\
\mSymbol[outlined]{doorbell-3p} & \mSymbol[rounded]{doorbell-3p} & \mSymbol[sharp]{doorbell-3p} & \texttt{\textbackslash mSymbol\{doorbell-3p\}} & \texttt{E1E7}\\
\mSymbol[outlined]{doorbell-chime} & \mSymbol[rounded]{doorbell-chime} & \mSymbol[sharp]{doorbell-chime} & \texttt{\textbackslash mSymbol\{doorbell-chime\}} & \texttt{E1F3}\\
\mSymbol[outlined]{double-arrow} & \mSymbol[rounded]{double-arrow} & \mSymbol[sharp]{double-arrow} & \texttt{\textbackslash mSymbol\{double-arrow\}} & \texttt{EA50}\\
\mSymbol[outlined]{downhill-skiing} & \mSymbol[rounded]{downhill-skiing} & \mSymbol[sharp]{downhill-skiing} & \texttt{\textbackslash mSymbol\{downhill-skiing\}} & \texttt{E509}\\
\mSymbol[outlined]{download} & \mSymbol[rounded]{download} & \mSymbol[sharp]{download} & \texttt{\textbackslash mSymbol\{download\}} & \texttt{F090}\\
\mSymbol[outlined]{download-2} & \mSymbol[rounded]{download-2} & \mSymbol[sharp]{download-2} & \texttt{\textbackslash mSymbol\{download-2\}} & \texttt{F523}\\
\mSymbol[outlined]{download-done} & \mSymbol[rounded]{download-done} & \mSymbol[sharp]{download-done} & \texttt{\textbackslash mSymbol\{download-done\}} & \texttt{F091}\\
\mSymbol[outlined]{download-for-offline} & \mSymbol[rounded]{download-for-offline} & \mSymbol[sharp]{download-for-offline} & \texttt{\textbackslash mSymbol\{download-for-offline\}} & \texttt{F000}\\
\mSymbol[outlined]{downloading} & \mSymbol[rounded]{downloading} & \mSymbol[sharp]{downloading} & \texttt{\textbackslash mSymbol\{downloading\}} & \texttt{F001}\\
\mSymbol[outlined]{draft} & \mSymbol[rounded]{draft} & \mSymbol[sharp]{draft} & \texttt{\textbackslash mSymbol\{draft\}} & \texttt{E66D}\\
\mSymbol[outlined]{draft-orders} & \mSymbol[rounded]{draft-orders} & \mSymbol[sharp]{draft-orders} & \texttt{\textbackslash mSymbol\{draft-orders\}} & \texttt{E7B3}\\
\mSymbol[outlined]{drafts} & \mSymbol[rounded]{drafts} & \mSymbol[sharp]{drafts} & \texttt{\textbackslash mSymbol\{drafts\}} & \texttt{E151}\\
\mSymbol[outlined]{drag-click} & \mSymbol[rounded]{drag-click} & \mSymbol[sharp]{drag-click} & \texttt{\textbackslash mSymbol\{drag-click\}} & \texttt{F71F}\\
\mSymbol[outlined]{drag-handle} & \mSymbol[rounded]{drag-handle} & \mSymbol[sharp]{drag-handle} & \texttt{\textbackslash mSymbol\{drag-handle\}} & \texttt{E25D}\\
\mSymbol[outlined]{drag-indicator} & \mSymbol[rounded]{drag-indicator} & \mSymbol[sharp]{drag-indicator} & \texttt{\textbackslash mSymbol\{drag-indicator\}} & \texttt{E945}\\
\mSymbol[outlined]{drag-pan} & \mSymbol[rounded]{drag-pan} & \mSymbol[sharp]{drag-pan} & \texttt{\textbackslash mSymbol\{drag-pan\}} & \texttt{F71E}\\
\mSymbol[outlined]{draw} & \mSymbol[rounded]{draw} & \mSymbol[sharp]{draw} & \texttt{\textbackslash mSymbol\{draw\}} & \texttt{E746}\\
\mSymbol[outlined]{draw-abstract} & \mSymbol[rounded]{draw-abstract} & \mSymbol[sharp]{draw-abstract} & \texttt{\textbackslash mSymbol\{draw-abstract\}} & \texttt{F7F8}\\
\mSymbol[outlined]{draw-collage} & \mSymbol[rounded]{draw-collage} & \mSymbol[sharp]{draw-collage} & \texttt{\textbackslash mSymbol\{draw-collage\}} & \texttt{F7F7}\\
\mSymbol[outlined]{drawing-recognition} & \mSymbol[rounded]{drawing-recognition} & \mSymbol[sharp]{drawing-recognition} & \texttt{\textbackslash mSymbol\{drawing-recognition\}} & \texttt{EB00}\\
\mSymbol[outlined]{dresser} & \mSymbol[rounded]{dresser} & \mSymbol[sharp]{dresser} & \texttt{\textbackslash mSymbol\{dresser\}} & \texttt{E210}\\
\mSymbol[outlined]{drive-eta} & \mSymbol[rounded]{drive-eta} & \mSymbol[sharp]{drive-eta} & \texttt{\textbackslash mSymbol\{drive-eta\}} & \texttt{EFF7}\\
\mSymbol[outlined]{drive-export} & \mSymbol[rounded]{drive-export} & \mSymbol[sharp]{drive-export} & \texttt{\textbackslash mSymbol\{drive-export\}} & \texttt{F41D}\\
\mSymbol[outlined]{drive-file-move} & \mSymbol[rounded]{drive-file-move} & \mSymbol[sharp]{drive-file-move} & \texttt{\textbackslash mSymbol\{drive-file-move\}} & \texttt{E9A1}\\
\mSymbol[outlined]{drive-file-move-outline} & \mSymbol[rounded]{drive-file-move-outline} & \mSymbol[sharp]{drive-file-move-outline} & \texttt{\textbackslash mSymbol\{drive-file-move-outline\}} & \texttt{E9A1}\\
\mSymbol[outlined]{drive-file-move-rtl} & \mSymbol[rounded]{drive-file-move-rtl} & \mSymbol[sharp]{drive-file-move-rtl} & \texttt{\textbackslash mSymbol\{drive-file-move-rtl\}} & \texttt{E9A1}\\
\mSymbol[outlined]{drive-file-rename-outline} & \mSymbol[rounded]{drive-file-rename-outline} & \mSymbol[sharp]{drive-file-rename-outline} & \texttt{\textbackslash mSymbol\{drive-file-rename-outline\}} & \texttt{E9A2}\\
\mSymbol[outlined]{drive-folder-upload} & \mSymbol[rounded]{drive-folder-upload} & \mSymbol[sharp]{drive-folder-upload} & \texttt{\textbackslash mSymbol\{drive-folder-upload\}} & \texttt{E9A3}\\
\mSymbol[outlined]{drive-fusiontable} & \mSymbol[rounded]{drive-fusiontable} & \mSymbol[sharp]{drive-fusiontable} & \texttt{\textbackslash mSymbol\{drive-fusiontable\}} & \texttt{E678}\\
\mSymbol[outlined]{dropdown} & \mSymbol[rounded]{dropdown} & \mSymbol[sharp]{dropdown} & \texttt{\textbackslash mSymbol\{dropdown\}} & \texttt{E9A4}\\
\mSymbol[outlined]{dry} & \mSymbol[rounded]{dry} & \mSymbol[sharp]{dry} & \texttt{\textbackslash mSymbol\{dry\}} & \texttt{F1B3}\\
\mSymbol[outlined]{dry-cleaning} & \mSymbol[rounded]{dry-cleaning} & \mSymbol[sharp]{dry-cleaning} & \texttt{\textbackslash mSymbol\{dry-cleaning\}} & \texttt{EA58}\\
\mSymbol[outlined]{dual-screen} & \mSymbol[rounded]{dual-screen} & \mSymbol[sharp]{dual-screen} & \texttt{\textbackslash mSymbol\{dual-screen\}} & \texttt{F6CF}\\
\mSymbol[outlined]{duo} & \mSymbol[rounded]{duo} & \mSymbol[sharp]{duo} & \texttt{\textbackslash mSymbol\{duo\}} & \texttt{E9A5}\\
\mSymbol[outlined]{dvr} & \mSymbol[rounded]{dvr} & \mSymbol[sharp]{dvr} & \texttt{\textbackslash mSymbol\{dvr\}} & \texttt{E1B2}\\
\mSymbol[outlined]{dynamic-feed} & \mSymbol[rounded]{dynamic-feed} & \mSymbol[sharp]{dynamic-feed} & \texttt{\textbackslash mSymbol\{dynamic-feed\}} & \texttt{EA14}\\
\mSymbol[outlined]{dynamic-form} & \mSymbol[rounded]{dynamic-form} & \mSymbol[sharp]{dynamic-form} & \texttt{\textbackslash mSymbol\{dynamic-form\}} & \texttt{F1BF}\\
\mSymbol[outlined]{e911-avatar} & \mSymbol[rounded]{e911-avatar} & \mSymbol[sharp]{e911-avatar} & \texttt{\textbackslash mSymbol\{e911-avatar\}} & \texttt{F11A}\\
\mSymbol[outlined]{e911-emergency} & \mSymbol[rounded]{e911-emergency} & \mSymbol[sharp]{e911-emergency} & \texttt{\textbackslash mSymbol\{e911-emergency\}} & \texttt{F119}\\
\mSymbol[outlined]{e-mobiledata} & \mSymbol[rounded]{e-mobiledata} & \mSymbol[sharp]{e-mobiledata} & \texttt{\textbackslash mSymbol\{e-mobiledata\}} & \texttt{F002}\\
\mSymbol[outlined]{e-mobiledata-badge} & \mSymbol[rounded]{e-mobiledata-badge} & \mSymbol[sharp]{e-mobiledata-badge} & \texttt{\textbackslash mSymbol\{e-mobiledata-badge\}} & \texttt{F7E3}\\
\mSymbol[outlined]{earbuds} & \mSymbol[rounded]{earbuds} & \mSymbol[sharp]{earbuds} & \texttt{\textbackslash mSymbol\{earbuds\}} & \texttt{F003}\\
\mSymbol[outlined]{earbuds-battery} & \mSymbol[rounded]{earbuds-battery} & \mSymbol[sharp]{earbuds-battery} & \texttt{\textbackslash mSymbol\{earbuds-battery\}} & \texttt{F004}\\
\mSymbol[outlined]{early-on} & \mSymbol[rounded]{early-on} & \mSymbol[sharp]{early-on} & \texttt{\textbackslash mSymbol\{early-on\}} & \texttt{E2BA}\\
\mSymbol[outlined]{earthquake} & \mSymbol[rounded]{earthquake} & \mSymbol[sharp]{earthquake} & \texttt{\textbackslash mSymbol\{earthquake\}} & \texttt{F64F}\\
\mSymbol[outlined]{east} & \mSymbol[rounded]{east} & \mSymbol[sharp]{east} & \texttt{\textbackslash mSymbol\{east\}} & \texttt{F1DF}\\
\mSymbol[outlined]{ecg} & \mSymbol[rounded]{ecg} & \mSymbol[sharp]{ecg} & \texttt{\textbackslash mSymbol\{ecg\}} & \texttt{F80F}\\
\mSymbol[outlined]{ecg-heart} & \mSymbol[rounded]{ecg-heart} & \mSymbol[sharp]{ecg-heart} & \texttt{\textbackslash mSymbol\{ecg-heart\}} & \texttt{F6E9}\\
\mSymbol[outlined]{eco} & \mSymbol[rounded]{eco} & \mSymbol[sharp]{eco} & \texttt{\textbackslash mSymbol\{eco\}} & \texttt{EA35}\\
\mSymbol[outlined]{eda} & \mSymbol[rounded]{eda} & \mSymbol[sharp]{eda} & \texttt{\textbackslash mSymbol\{eda\}} & \texttt{F6E8}\\
\mSymbol[outlined]{edgesensor-high} & \mSymbol[rounded]{edgesensor-high} & \mSymbol[sharp]{edgesensor-high} & \texttt{\textbackslash mSymbol\{edgesensor-high\}} & \texttt{F005}\\
\mSymbol[outlined]{edgesensor-low} & \mSymbol[rounded]{edgesensor-low} & \mSymbol[sharp]{edgesensor-low} & \texttt{\textbackslash mSymbol\{edgesensor-low\}} & \texttt{F006}\\
\mSymbol[outlined]{edit} & \mSymbol[rounded]{edit} & \mSymbol[sharp]{edit} & \texttt{\textbackslash mSymbol\{edit\}} & \texttt{F097}\\
\mSymbol[outlined]{edit-attributes} & \mSymbol[rounded]{edit-attributes} & \mSymbol[sharp]{edit-attributes} & \texttt{\textbackslash mSymbol\{edit-attributes\}} & \texttt{E578}\\
\mSymbol[outlined]{edit-audio} & \mSymbol[rounded]{edit-audio} & \mSymbol[sharp]{edit-audio} & \texttt{\textbackslash mSymbol\{edit-audio\}} & \texttt{F42D}\\
\mSymbol[outlined]{edit-calendar} & \mSymbol[rounded]{edit-calendar} & \mSymbol[sharp]{edit-calendar} & \texttt{\textbackslash mSymbol\{edit-calendar\}} & \texttt{E742}\\
\mSymbol[outlined]{edit-document} & \mSymbol[rounded]{edit-document} & \mSymbol[sharp]{edit-document} & \texttt{\textbackslash mSymbol\{edit-document\}} & \texttt{F88C}\\
\mSymbol[outlined]{edit-location} & \mSymbol[rounded]{edit-location} & \mSymbol[sharp]{edit-location} & \texttt{\textbackslash mSymbol\{edit-location\}} & \texttt{E568}\\
\mSymbol[outlined]{edit-location-alt} & \mSymbol[rounded]{edit-location-alt} & \mSymbol[sharp]{edit-location-alt} & \texttt{\textbackslash mSymbol\{edit-location-alt\}} & \texttt{E1C5}\\
\mSymbol[outlined]{edit-note} & \mSymbol[rounded]{edit-note} & \mSymbol[sharp]{edit-note} & \texttt{\textbackslash mSymbol\{edit-note\}} & \texttt{E745}\\
\mSymbol[outlined]{edit-notifications} & \mSymbol[rounded]{edit-notifications} & \mSymbol[sharp]{edit-notifications} & \texttt{\textbackslash mSymbol\{edit-notifications\}} & \texttt{E525}\\
\mSymbol[outlined]{edit-off} & \mSymbol[rounded]{edit-off} & \mSymbol[sharp]{edit-off} & \texttt{\textbackslash mSymbol\{edit-off\}} & \texttt{E950}\\
\mSymbol[outlined]{edit-road} & \mSymbol[rounded]{edit-road} & \mSymbol[sharp]{edit-road} & \texttt{\textbackslash mSymbol\{edit-road\}} & \texttt{EF4D}\\
\mSymbol[outlined]{edit-square} & \mSymbol[rounded]{edit-square} & \mSymbol[sharp]{edit-square} & \texttt{\textbackslash mSymbol\{edit-square\}} & \texttt{F88D}\\
\mSymbol[outlined]{editor-choice} & \mSymbol[rounded]{editor-choice} & \mSymbol[sharp]{editor-choice} & \texttt{\textbackslash mSymbol\{editor-choice\}} & \texttt{F528}\\
\mSymbol[outlined]{egg} & \mSymbol[rounded]{egg} & \mSymbol[sharp]{egg} & \texttt{\textbackslash mSymbol\{egg\}} & \texttt{EACC}\\
\mSymbol[outlined]{egg-alt} & \mSymbol[rounded]{egg-alt} & \mSymbol[sharp]{egg-alt} & \texttt{\textbackslash mSymbol\{egg-alt\}} & \texttt{EAC8}\\
\mSymbol[outlined]{eject} & \mSymbol[rounded]{eject} & \mSymbol[sharp]{eject} & \texttt{\textbackslash mSymbol\{eject\}} & \texttt{E8FB}\\
\mSymbol[outlined]{elderly} & \mSymbol[rounded]{elderly} & \mSymbol[sharp]{elderly} & \texttt{\textbackslash mSymbol\{elderly\}} & \texttt{F21A}\\
\mSymbol[outlined]{elderly-woman} & \mSymbol[rounded]{elderly-woman} & \mSymbol[sharp]{elderly-woman} & \texttt{\textbackslash mSymbol\{elderly-woman\}} & \texttt{EB69}\\
\mSymbol[outlined]{electric-bike} & \mSymbol[rounded]{electric-bike} & \mSymbol[sharp]{electric-bike} & \texttt{\textbackslash mSymbol\{electric-bike\}} & \texttt{EB1B}\\
\mSymbol[outlined]{electric-bolt} & \mSymbol[rounded]{electric-bolt} & \mSymbol[sharp]{electric-bolt} & \texttt{\textbackslash mSymbol\{electric-bolt\}} & \texttt{EC1C}\\
\mSymbol[outlined]{electric-car} & \mSymbol[rounded]{electric-car} & \mSymbol[sharp]{electric-car} & \texttt{\textbackslash mSymbol\{electric-car\}} & \texttt{EB1C}\\
\mSymbol[outlined]{electric-meter} & \mSymbol[rounded]{electric-meter} & \mSymbol[sharp]{electric-meter} & \texttt{\textbackslash mSymbol\{electric-meter\}} & \texttt{EC1B}\\
\mSymbol[outlined]{electric-moped} & \mSymbol[rounded]{electric-moped} & \mSymbol[sharp]{electric-moped} & \texttt{\textbackslash mSymbol\{electric-moped\}} & \texttt{EB1D}\\
\mSymbol[outlined]{electric-rickshaw} & \mSymbol[rounded]{electric-rickshaw} & \mSymbol[sharp]{electric-rickshaw} & \texttt{\textbackslash mSymbol\{electric-rickshaw\}} & \texttt{EB1E}\\
\mSymbol[outlined]{electric-scooter} & \mSymbol[rounded]{electric-scooter} & \mSymbol[sharp]{electric-scooter} & \texttt{\textbackslash mSymbol\{electric-scooter\}} & \texttt{EB1F}\\
\mSymbol[outlined]{electrical-services} & \mSymbol[rounded]{electrical-services} & \mSymbol[sharp]{electrical-services} & \texttt{\textbackslash mSymbol\{electrical-services\}} & \texttt{F102}\\
\mSymbol[outlined]{elevation} & \mSymbol[rounded]{elevation} & \mSymbol[sharp]{elevation} & \texttt{\textbackslash mSymbol\{elevation\}} & \texttt{F6E7}\\
\mSymbol[outlined]{elevator} & \mSymbol[rounded]{elevator} & \mSymbol[sharp]{elevator} & \texttt{\textbackslash mSymbol\{elevator\}} & \texttt{F1A0}\\
\mSymbol[outlined]{email} & \mSymbol[rounded]{email} & \mSymbol[sharp]{email} & \texttt{\textbackslash mSymbol\{email\}} & \texttt{E159}\\
\mSymbol[outlined]{emergency} & \mSymbol[rounded]{emergency} & \mSymbol[sharp]{emergency} & \texttt{\textbackslash mSymbol\{emergency\}} & \texttt{E1EB}\\
\mSymbol[outlined]{emergency-heat} & \mSymbol[rounded]{emergency-heat} & \mSymbol[sharp]{emergency-heat} & \texttt{\textbackslash mSymbol\{emergency-heat\}} & \texttt{F15D}\\
\mSymbol[outlined]{emergency-heat-2} & \mSymbol[rounded]{emergency-heat-2} & \mSymbol[sharp]{emergency-heat-2} & \texttt{\textbackslash mSymbol\{emergency-heat-2\}} & \texttt{F4E5}\\
\mSymbol[outlined]{emergency-home} & \mSymbol[rounded]{emergency-home} & \mSymbol[sharp]{emergency-home} & \texttt{\textbackslash mSymbol\{emergency-home\}} & \texttt{E82A}\\
\mSymbol[outlined]{emergency-recording} & \mSymbol[rounded]{emergency-recording} & \mSymbol[sharp]{emergency-recording} & \texttt{\textbackslash mSymbol\{emergency-recording\}} & \texttt{EBF4}\\
\mSymbol[outlined]{emergency-share} & \mSymbol[rounded]{emergency-share} & \mSymbol[sharp]{emergency-share} & \texttt{\textbackslash mSymbol\{emergency-share\}} & \texttt{EBF6}\\
\mSymbol[outlined]{emergency-share-off} & \mSymbol[rounded]{emergency-share-off} & \mSymbol[sharp]{emergency-share-off} & \texttt{\textbackslash mSymbol\{emergency-share-off\}} & \texttt{F59E}\\
\mSymbol[outlined]{emoji-emotions} & \mSymbol[rounded]{emoji-emotions} & \mSymbol[sharp]{emoji-emotions} & \texttt{\textbackslash mSymbol\{emoji-emotions\}} & \texttt{EA22}\\
\mSymbol[outlined]{emoji-events} & \mSymbol[rounded]{emoji-events} & \mSymbol[sharp]{emoji-events} & \texttt{\textbackslash mSymbol\{emoji-events\}} & \texttt{EA23}\\
\mSymbol[outlined]{emoji-flags} & \mSymbol[rounded]{emoji-flags} & \mSymbol[sharp]{emoji-flags} & \texttt{\textbackslash mSymbol\{emoji-flags\}} & \texttt{F0C6}\\
\mSymbol[outlined]{emoji-food-beverage} & \mSymbol[rounded]{emoji-food-beverage} & \mSymbol[sharp]{emoji-food-beverage} & \texttt{\textbackslash mSymbol\{emoji-food-beverage\}} & \texttt{EA1B}\\
\mSymbol[outlined]{emoji-language} & \mSymbol[rounded]{emoji-language} & \mSymbol[sharp]{emoji-language} & \texttt{\textbackslash mSymbol\{emoji-language\}} & \texttt{F4CD}\\
\mSymbol[outlined]{emoji-nature} & \mSymbol[rounded]{emoji-nature} & \mSymbol[sharp]{emoji-nature} & \texttt{\textbackslash mSymbol\{emoji-nature\}} & \texttt{EA1C}\\
\mSymbol[outlined]{emoji-objects} & \mSymbol[rounded]{emoji-objects} & \mSymbol[sharp]{emoji-objects} & \texttt{\textbackslash mSymbol\{emoji-objects\}} & \texttt{EA24}\\
\mSymbol[outlined]{emoji-people} & \mSymbol[rounded]{emoji-people} & \mSymbol[sharp]{emoji-people} & \texttt{\textbackslash mSymbol\{emoji-people\}} & \texttt{EA1D}\\
\mSymbol[outlined]{emoji-symbols} & \mSymbol[rounded]{emoji-symbols} & \mSymbol[sharp]{emoji-symbols} & \texttt{\textbackslash mSymbol\{emoji-symbols\}} & \texttt{EA1E}\\
\mSymbol[outlined]{emoji-transportation} & \mSymbol[rounded]{emoji-transportation} & \mSymbol[sharp]{emoji-transportation} & \texttt{\textbackslash mSymbol\{emoji-transportation\}} & \texttt{EA1F}\\
\mSymbol[outlined]{emoticon} & \mSymbol[rounded]{emoticon} & \mSymbol[sharp]{emoticon} & \texttt{\textbackslash mSymbol\{emoticon\}} & \texttt{E5F3}\\
\mSymbol[outlined]{empty-dashboard} & \mSymbol[rounded]{empty-dashboard} & \mSymbol[sharp]{empty-dashboard} & \texttt{\textbackslash mSymbol\{empty-dashboard\}} & \texttt{F844}\\
\mSymbol[outlined]{enable} & \mSymbol[rounded]{enable} & \mSymbol[sharp]{enable} & \texttt{\textbackslash mSymbol\{enable\}} & \texttt{F188}\\
\mSymbol[outlined]{encrypted} & \mSymbol[rounded]{encrypted} & \mSymbol[sharp]{encrypted} & \texttt{\textbackslash mSymbol\{encrypted\}} & \texttt{E593}\\
\mSymbol[outlined]{encrypted-add} & \mSymbol[rounded]{encrypted-add} & \mSymbol[sharp]{encrypted-add} & \texttt{\textbackslash mSymbol\{encrypted-add\}} & \texttt{F429}\\
\mSymbol[outlined]{encrypted-add-circle} & \mSymbol[rounded]{encrypted-add-circle} & \mSymbol[sharp]{encrypted-add-circle} & \texttt{\textbackslash mSymbol\{encrypted-add-circle\}} & \texttt{F42A}\\
\mSymbol[outlined]{encrypted-minus-circle} & \mSymbol[rounded]{encrypted-minus-circle} & \mSymbol[sharp]{encrypted-minus-circle} & \texttt{\textbackslash mSymbol\{encrypted-minus-circle\}} & \texttt{F428}\\
\mSymbol[outlined]{encrypted-off} & \mSymbol[rounded]{encrypted-off} & \mSymbol[sharp]{encrypted-off} & \texttt{\textbackslash mSymbol\{encrypted-off\}} & \texttt{F427}\\
\mSymbol[outlined]{endocrinology} & \mSymbol[rounded]{endocrinology} & \mSymbol[sharp]{endocrinology} & \texttt{\textbackslash mSymbol\{endocrinology\}} & \texttt{E0A9}\\
\mSymbol[outlined]{energy} & \mSymbol[rounded]{energy} & \mSymbol[sharp]{energy} & \texttt{\textbackslash mSymbol\{energy\}} & \texttt{E9A6}\\
\mSymbol[outlined]{energy-program-saving} & \mSymbol[rounded]{energy-program-saving} & \mSymbol[sharp]{energy-program-saving} & \texttt{\textbackslash mSymbol\{energy-program-saving\}} & \texttt{F15F}\\
\mSymbol[outlined]{energy-program-time-used} & \mSymbol[rounded]{energy-program-time-used} & \mSymbol[sharp]{energy-program-time-used} & \texttt{\textbackslash mSymbol\{energy-program-time-used\}} & \texttt{F161}\\
\mSymbol[outlined]{energy-savings-leaf} & \mSymbol[rounded]{energy-savings-leaf} & \mSymbol[sharp]{energy-savings-leaf} & \texttt{\textbackslash mSymbol\{energy-savings-leaf\}} & \texttt{EC1A}\\
\mSymbol[outlined]{engineering} & \mSymbol[rounded]{engineering} & \mSymbol[sharp]{engineering} & \texttt{\textbackslash mSymbol\{engineering\}} & \texttt{EA3D}\\
\mSymbol[outlined]{enhanced-encryption} & \mSymbol[rounded]{enhanced-encryption} & \mSymbol[sharp]{enhanced-encryption} & \texttt{\textbackslash mSymbol\{enhanced-encryption\}} & \texttt{E63F}\\
\mSymbol[outlined]{ent} & \mSymbol[rounded]{ent} & \mSymbol[sharp]{ent} & \texttt{\textbackslash mSymbol\{ent\}} & \texttt{E0AA}\\
\mSymbol[outlined]{enterprise} & \mSymbol[rounded]{enterprise} & \mSymbol[sharp]{enterprise} & \texttt{\textbackslash mSymbol\{enterprise\}} & \texttt{E70E}\\
\mSymbol[outlined]{enterprise-off} & \mSymbol[rounded]{enterprise-off} & \mSymbol[sharp]{enterprise-off} & \texttt{\textbackslash mSymbol\{enterprise-off\}} & \texttt{EB4D}\\
\mSymbol[outlined]{equal} & \mSymbol[rounded]{equal} & \mSymbol[sharp]{equal} & \texttt{\textbackslash mSymbol\{equal\}} & \texttt{F77B}\\
\mSymbol[outlined]{equalizer} & \mSymbol[rounded]{equalizer} & \mSymbol[sharp]{equalizer} & \texttt{\textbackslash mSymbol\{equalizer\}} & \texttt{E01D}\\
\mSymbol[outlined]{error} & \mSymbol[rounded]{error} & \mSymbol[sharp]{error} & \texttt{\textbackslash mSymbol\{error\}} & \texttt{F8B6}\\
\mSymbol[outlined]{error-circle-rounded} & \mSymbol[rounded]{error-circle-rounded} & \mSymbol[sharp]{error-circle-rounded} & \texttt{\textbackslash mSymbol\{error-circle-rounded\}} & \texttt{F8B6}\\
\mSymbol[outlined]{error-med} & \mSymbol[rounded]{error-med} & \mSymbol[sharp]{error-med} & \texttt{\textbackslash mSymbol\{error-med\}} & \texttt{E49B}\\
\mSymbol[outlined]{error-outline} & \mSymbol[rounded]{error-outline} & \mSymbol[sharp]{error-outline} & \texttt{\textbackslash mSymbol\{error-outline\}} & \texttt{F8B6}\\
\mSymbol[outlined]{escalator} & \mSymbol[rounded]{escalator} & \mSymbol[sharp]{escalator} & \texttt{\textbackslash mSymbol\{escalator\}} & \texttt{F1A1}\\
\mSymbol[outlined]{escalator-warning} & \mSymbol[rounded]{escalator-warning} & \mSymbol[sharp]{escalator-warning} & \texttt{\textbackslash mSymbol\{escalator-warning\}} & \texttt{F1AC}\\
\mSymbol[outlined]{euro} & \mSymbol[rounded]{euro} & \mSymbol[sharp]{euro} & \texttt{\textbackslash mSymbol\{euro\}} & \texttt{EA15}\\
\mSymbol[outlined]{euro-symbol} & \mSymbol[rounded]{euro-symbol} & \mSymbol[sharp]{euro-symbol} & \texttt{\textbackslash mSymbol\{euro-symbol\}} & \texttt{E926}\\
\mSymbol[outlined]{ev-charger} & \mSymbol[rounded]{ev-charger} & \mSymbol[sharp]{ev-charger} & \texttt{\textbackslash mSymbol\{ev-charger\}} & \texttt{E56D}\\
\mSymbol[outlined]{ev-mobiledata-badge} & \mSymbol[rounded]{ev-mobiledata-badge} & \mSymbol[sharp]{ev-mobiledata-badge} & \texttt{\textbackslash mSymbol\{ev-mobiledata-badge\}} & \texttt{F7E2}\\
\mSymbol[outlined]{ev-shadow} & \mSymbol[rounded]{ev-shadow} & \mSymbol[sharp]{ev-shadow} & \texttt{\textbackslash mSymbol\{ev-shadow\}} & \texttt{EF8F}\\
\mSymbol[outlined]{ev-shadow-add} & \mSymbol[rounded]{ev-shadow-add} & \mSymbol[sharp]{ev-shadow-add} & \texttt{\textbackslash mSymbol\{ev-shadow-add\}} & \texttt{F580}\\
\mSymbol[outlined]{ev-shadow-minus} & \mSymbol[rounded]{ev-shadow-minus} & \mSymbol[sharp]{ev-shadow-minus} & \texttt{\textbackslash mSymbol\{ev-shadow-minus\}} & \texttt{F57F}\\
\mSymbol[outlined]{ev-station} & \mSymbol[rounded]{ev-station} & \mSymbol[sharp]{ev-station} & \texttt{\textbackslash mSymbol\{ev-station\}} & \texttt{E56D}\\
\mSymbol[outlined]{event} & \mSymbol[rounded]{event} & \mSymbol[sharp]{event} & \texttt{\textbackslash mSymbol\{event\}} & \texttt{E878}\\
\mSymbol[outlined]{event-available} & \mSymbol[rounded]{event-available} & \mSymbol[sharp]{event-available} & \texttt{\textbackslash mSymbol\{event-available\}} & \texttt{E614}\\
\mSymbol[outlined]{event-busy} & \mSymbol[rounded]{event-busy} & \mSymbol[sharp]{event-busy} & \texttt{\textbackslash mSymbol\{event-busy\}} & \texttt{E615}\\
\mSymbol[outlined]{event-list} & \mSymbol[rounded]{event-list} & \mSymbol[sharp]{event-list} & \texttt{\textbackslash mSymbol\{event-list\}} & \texttt{F683}\\
\mSymbol[outlined]{event-note} & \mSymbol[rounded]{event-note} & \mSymbol[sharp]{event-note} & \texttt{\textbackslash mSymbol\{event-note\}} & \texttt{E616}\\
\mSymbol[outlined]{event-repeat} & \mSymbol[rounded]{event-repeat} & \mSymbol[sharp]{event-repeat} & \texttt{\textbackslash mSymbol\{event-repeat\}} & \texttt{EB7B}\\
\mSymbol[outlined]{event-seat} & \mSymbol[rounded]{event-seat} & \mSymbol[sharp]{event-seat} & \texttt{\textbackslash mSymbol\{event-seat\}} & \texttt{E903}\\
\mSymbol[outlined]{event-upcoming} & \mSymbol[rounded]{event-upcoming} & \mSymbol[sharp]{event-upcoming} & \texttt{\textbackslash mSymbol\{event-upcoming\}} & \texttt{F238}\\
\mSymbol[outlined]{exclamation} & \mSymbol[rounded]{exclamation} & \mSymbol[sharp]{exclamation} & \texttt{\textbackslash mSymbol\{exclamation\}} & \texttt{F22F}\\
\mSymbol[outlined]{exercise} & \mSymbol[rounded]{exercise} & \mSymbol[sharp]{exercise} & \texttt{\textbackslash mSymbol\{exercise\}} & \texttt{F6E6}\\
\mSymbol[outlined]{exit-to-app} & \mSymbol[rounded]{exit-to-app} & \mSymbol[sharp]{exit-to-app} & \texttt{\textbackslash mSymbol\{exit-to-app\}} & \texttt{E879}\\
\mSymbol[outlined]{expand} & \mSymbol[rounded]{expand} & \mSymbol[sharp]{expand} & \texttt{\textbackslash mSymbol\{expand\}} & \texttt{E94F}\\
\mSymbol[outlined]{expand-all} & \mSymbol[rounded]{expand-all} & \mSymbol[sharp]{expand-all} & \texttt{\textbackslash mSymbol\{expand-all\}} & \texttt{E946}\\
\mSymbol[outlined]{expand-circle-down} & \mSymbol[rounded]{expand-circle-down} & \mSymbol[sharp]{expand-circle-down} & \texttt{\textbackslash mSymbol\{expand-circle-down\}} & \texttt{E7CD}\\
\mSymbol[outlined]{expand-circle-right} & \mSymbol[rounded]{expand-circle-right} & \mSymbol[sharp]{expand-circle-right} & \texttt{\textbackslash mSymbol\{expand-circle-right\}} & \texttt{F591}\\
\mSymbol[outlined]{expand-circle-up} & \mSymbol[rounded]{expand-circle-up} & \mSymbol[sharp]{expand-circle-up} & \texttt{\textbackslash mSymbol\{expand-circle-up\}} & \texttt{F5D2}\\
\mSymbol[outlined]{expand-content} & \mSymbol[rounded]{expand-content} & \mSymbol[sharp]{expand-content} & \texttt{\textbackslash mSymbol\{expand-content\}} & \texttt{F830}\\
\mSymbol[outlined]{expand-less} & \mSymbol[rounded]{expand-less} & \mSymbol[sharp]{expand-less} & \texttt{\textbackslash mSymbol\{expand-less\}} & \texttt{E5CE}\\
\mSymbol[outlined]{expand-more} & \mSymbol[rounded]{expand-more} & \mSymbol[sharp]{expand-more} & \texttt{\textbackslash mSymbol\{expand-more\}} & \texttt{E5CF}\\
\mSymbol[outlined]{experiment} & \mSymbol[rounded]{experiment} & \mSymbol[sharp]{experiment} & \texttt{\textbackslash mSymbol\{experiment\}} & \texttt{E686}\\
\mSymbol[outlined]{explicit} & \mSymbol[rounded]{explicit} & \mSymbol[sharp]{explicit} & \texttt{\textbackslash mSymbol\{explicit\}} & \texttt{E01E}\\
\mSymbol[outlined]{explore} & \mSymbol[rounded]{explore} & \mSymbol[sharp]{explore} & \texttt{\textbackslash mSymbol\{explore\}} & \texttt{E87A}\\
\mSymbol[outlined]{explore-nearby} & \mSymbol[rounded]{explore-nearby} & \mSymbol[sharp]{explore-nearby} & \texttt{\textbackslash mSymbol\{explore-nearby\}} & \texttt{E538}\\
\mSymbol[outlined]{explore-off} & \mSymbol[rounded]{explore-off} & \mSymbol[sharp]{explore-off} & \texttt{\textbackslash mSymbol\{explore-off\}} & \texttt{E9A8}\\
\mSymbol[outlined]{explosion} & \mSymbol[rounded]{explosion} & \mSymbol[sharp]{explosion} & \texttt{\textbackslash mSymbol\{explosion\}} & \texttt{F685}\\
\mSymbol[outlined]{export-notes} & \mSymbol[rounded]{export-notes} & \mSymbol[sharp]{export-notes} & \texttt{\textbackslash mSymbol\{export-notes\}} & \texttt{E0AC}\\
\mSymbol[outlined]{exposure} & \mSymbol[rounded]{exposure} & \mSymbol[sharp]{exposure} & \texttt{\textbackslash mSymbol\{exposure\}} & \texttt{E3F6}\\
\mSymbol[outlined]{exposure-neg-1} & \mSymbol[rounded]{exposure-neg-1} & \mSymbol[sharp]{exposure-neg-1} & \texttt{\textbackslash mSymbol\{exposure-neg-1\}} & \texttt{E3CB}\\
\mSymbol[outlined]{exposure-neg-2} & \mSymbol[rounded]{exposure-neg-2} & \mSymbol[sharp]{exposure-neg-2} & \texttt{\textbackslash mSymbol\{exposure-neg-2\}} & \texttt{E3CC}\\
\mSymbol[outlined]{exposure-plus-1} & \mSymbol[rounded]{exposure-plus-1} & \mSymbol[sharp]{exposure-plus-1} & \texttt{\textbackslash mSymbol\{exposure-plus-1\}} & \texttt{E800}\\
\mSymbol[outlined]{exposure-plus-2} & \mSymbol[rounded]{exposure-plus-2} & \mSymbol[sharp]{exposure-plus-2} & \texttt{\textbackslash mSymbol\{exposure-plus-2\}} & \texttt{E3CE}\\
\mSymbol[outlined]{exposure-zero} & \mSymbol[rounded]{exposure-zero} & \mSymbol[sharp]{exposure-zero} & \texttt{\textbackslash mSymbol\{exposure-zero\}} & \texttt{E3CF}\\
\mSymbol[outlined]{extension} & \mSymbol[rounded]{extension} & \mSymbol[sharp]{extension} & \texttt{\textbackslash mSymbol\{extension\}} & \texttt{E87B}\\
\mSymbol[outlined]{extension-off} & \mSymbol[rounded]{extension-off} & \mSymbol[sharp]{extension-off} & \texttt{\textbackslash mSymbol\{extension-off\}} & \texttt{E4F5}\\
\mSymbol[outlined]{eye-tracking} & \mSymbol[rounded]{eye-tracking} & \mSymbol[sharp]{eye-tracking} & \texttt{\textbackslash mSymbol\{eye-tracking\}} & \texttt{F4C9}\\
\mSymbol[outlined]{eyeglasses} & \mSymbol[rounded]{eyeglasses} & \mSymbol[sharp]{eyeglasses} & \texttt{\textbackslash mSymbol\{eyeglasses\}} & \texttt{F6EE}\\
\mSymbol[outlined]{face} & \mSymbol[rounded]{face} & \mSymbol[sharp]{face} & \texttt{\textbackslash mSymbol\{face\}} & \texttt{F008}\\
\mSymbol[outlined]{face-2} & \mSymbol[rounded]{face-2} & \mSymbol[sharp]{face-2} & \texttt{\textbackslash mSymbol\{face-2\}} & \texttt{F8DA}\\
\mSymbol[outlined]{face-3} & \mSymbol[rounded]{face-3} & \mSymbol[sharp]{face-3} & \texttt{\textbackslash mSymbol\{face-3\}} & \texttt{F8DB}\\
\mSymbol[outlined]{face-4} & \mSymbol[rounded]{face-4} & \mSymbol[sharp]{face-4} & \texttt{\textbackslash mSymbol\{face-4\}} & \texttt{F8DC}\\
\mSymbol[outlined]{face-5} & \mSymbol[rounded]{face-5} & \mSymbol[sharp]{face-5} & \texttt{\textbackslash mSymbol\{face-5\}} & \texttt{F8DD}\\
\mSymbol[outlined]{face-6} & \mSymbol[rounded]{face-6} & \mSymbol[sharp]{face-6} & \texttt{\textbackslash mSymbol\{face-6\}} & \texttt{F8DE}\\
\mSymbol[outlined]{face-retouching-natural} & \mSymbol[rounded]{face-retouching-natural} & \mSymbol[sharp]{face-retouching-natural} & \texttt{\textbackslash mSymbol\{face-retouching-natural\}} & \texttt{EF4E}\\
\mSymbol[outlined]{face-retouching-off} & \mSymbol[rounded]{face-retouching-off} & \mSymbol[sharp]{face-retouching-off} & \texttt{\textbackslash mSymbol\{face-retouching-off\}} & \texttt{F007}\\
\mSymbol[outlined]{face-unlock} & \mSymbol[rounded]{face-unlock} & \mSymbol[sharp]{face-unlock} & \texttt{\textbackslash mSymbol\{face-unlock\}} & \texttt{F008}\\
\mSymbol[outlined]{fact-check} & \mSymbol[rounded]{fact-check} & \mSymbol[sharp]{fact-check} & \texttt{\textbackslash mSymbol\{fact-check\}} & \texttt{F0C5}\\
\mSymbol[outlined]{factory} & \mSymbol[rounded]{factory} & \mSymbol[sharp]{factory} & \texttt{\textbackslash mSymbol\{factory\}} & \texttt{EBBC}\\
\mSymbol[outlined]{falling} & \mSymbol[rounded]{falling} & \mSymbol[sharp]{falling} & \texttt{\textbackslash mSymbol\{falling\}} & \texttt{F60D}\\
\mSymbol[outlined]{familiar-face-and-zone} & \mSymbol[rounded]{familiar-face-and-zone} & \mSymbol[sharp]{familiar-face-and-zone} & \texttt{\textbackslash mSymbol\{familiar-face-and-zone\}} & \texttt{E21C}\\
\mSymbol[outlined]{family-history} & \mSymbol[rounded]{family-history} & \mSymbol[sharp]{family-history} & \texttt{\textbackslash mSymbol\{family-history\}} & \texttt{E0AD}\\
\mSymbol[outlined]{family-home} & \mSymbol[rounded]{family-home} & \mSymbol[sharp]{family-home} & \texttt{\textbackslash mSymbol\{family-home\}} & \texttt{EB26}\\
\mSymbol[outlined]{family-link} & \mSymbol[rounded]{family-link} & \mSymbol[sharp]{family-link} & \texttt{\textbackslash mSymbol\{family-link\}} & \texttt{EB19}\\
\mSymbol[outlined]{family-restroom} & \mSymbol[rounded]{family-restroom} & \mSymbol[sharp]{family-restroom} & \texttt{\textbackslash mSymbol\{family-restroom\}} & \texttt{F1A2}\\
\mSymbol[outlined]{family-star} & \mSymbol[rounded]{family-star} & \mSymbol[sharp]{family-star} & \texttt{\textbackslash mSymbol\{family-star\}} & \texttt{F527}\\
\mSymbol[outlined]{farsight-digital} & \mSymbol[rounded]{farsight-digital} & \mSymbol[sharp]{farsight-digital} & \texttt{\textbackslash mSymbol\{farsight-digital\}} & \texttt{F559}\\
\mSymbol[outlined]{fast-forward} & \mSymbol[rounded]{fast-forward} & \mSymbol[sharp]{fast-forward} & \texttt{\textbackslash mSymbol\{fast-forward\}} & \texttt{E01F}\\
\mSymbol[outlined]{fast-rewind} & \mSymbol[rounded]{fast-rewind} & \mSymbol[sharp]{fast-rewind} & \texttt{\textbackslash mSymbol\{fast-rewind\}} & \texttt{E020}\\
\mSymbol[outlined]{fastfood} & \mSymbol[rounded]{fastfood} & \mSymbol[sharp]{fastfood} & \texttt{\textbackslash mSymbol\{fastfood\}} & \texttt{E57A}\\
\mSymbol[outlined]{faucet} & \mSymbol[rounded]{faucet} & \mSymbol[sharp]{faucet} & \texttt{\textbackslash mSymbol\{faucet\}} & \texttt{E278}\\
\mSymbol[outlined]{favorite} & \mSymbol[rounded]{favorite} & \mSymbol[sharp]{favorite} & \texttt{\textbackslash mSymbol\{favorite\}} & \texttt{E87E}\\
\mSymbol[outlined]{favorite-border} & \mSymbol[rounded]{favorite-border} & \mSymbol[sharp]{favorite-border} & \texttt{\textbackslash mSymbol\{favorite-border\}} & \texttt{E87E}\\
\mSymbol[outlined]{fax} & \mSymbol[rounded]{fax} & \mSymbol[sharp]{fax} & \texttt{\textbackslash mSymbol\{fax\}} & \texttt{EAD8}\\
\mSymbol[outlined]{feature-search} & \mSymbol[rounded]{feature-search} & \mSymbol[sharp]{feature-search} & \texttt{\textbackslash mSymbol\{feature-search\}} & \texttt{E9A9}\\
\mSymbol[outlined]{featured-play-list} & \mSymbol[rounded]{featured-play-list} & \mSymbol[sharp]{featured-play-list} & \texttt{\textbackslash mSymbol\{featured-play-list\}} & \texttt{E06D}\\
\mSymbol[outlined]{featured-seasonal-and-gifts} & \mSymbol[rounded]{featured-seasonal-and-gifts} & \mSymbol[sharp]{featured-seasonal-and-gifts} & \texttt{\textbackslash mSymbol\{featured-seasonal-and-gifts\}} & \texttt{EF91}\\
\mSymbol[outlined]{featured-video} & \mSymbol[rounded]{featured-video} & \mSymbol[sharp]{featured-video} & \texttt{\textbackslash mSymbol\{featured-video\}} & \texttt{E06E}\\
\mSymbol[outlined]{feed} & \mSymbol[rounded]{feed} & \mSymbol[sharp]{feed} & \texttt{\textbackslash mSymbol\{feed\}} & \texttt{F009}\\
\mSymbol[outlined]{feedback} & \mSymbol[rounded]{feedback} & \mSymbol[sharp]{feedback} & \texttt{\textbackslash mSymbol\{feedback\}} & \texttt{E87F}\\
\mSymbol[outlined]{female} & \mSymbol[rounded]{female} & \mSymbol[sharp]{female} & \texttt{\textbackslash mSymbol\{female\}} & \texttt{E590}\\
\mSymbol[outlined]{femur} & \mSymbol[rounded]{femur} & \mSymbol[sharp]{femur} & \texttt{\textbackslash mSymbol\{femur\}} & \texttt{F891}\\
\mSymbol[outlined]{femur-alt} & \mSymbol[rounded]{femur-alt} & \mSymbol[sharp]{femur-alt} & \texttt{\textbackslash mSymbol\{femur-alt\}} & \texttt{F892}\\
\mSymbol[outlined]{fence} & \mSymbol[rounded]{fence} & \mSymbol[sharp]{fence} & \texttt{\textbackslash mSymbol\{fence\}} & \texttt{F1F6}\\
\mSymbol[outlined]{fertile} & \mSymbol[rounded]{fertile} & \mSymbol[sharp]{fertile} & \texttt{\textbackslash mSymbol\{fertile\}} & \texttt{F6E5}\\
\mSymbol[outlined]{festival} & \mSymbol[rounded]{festival} & \mSymbol[sharp]{festival} & \texttt{\textbackslash mSymbol\{festival\}} & \texttt{EA68}\\
\mSymbol[outlined]{fiber-dvr} & \mSymbol[rounded]{fiber-dvr} & \mSymbol[sharp]{fiber-dvr} & \texttt{\textbackslash mSymbol\{fiber-dvr\}} & \texttt{E05D}\\
\mSymbol[outlined]{fiber-manual-record} & \mSymbol[rounded]{fiber-manual-record} & \mSymbol[sharp]{fiber-manual-record} & \texttt{\textbackslash mSymbol\{fiber-manual-record\}} & \texttt{E061}\\
\mSymbol[outlined]{fiber-new} & \mSymbol[rounded]{fiber-new} & \mSymbol[sharp]{fiber-new} & \texttt{\textbackslash mSymbol\{fiber-new\}} & \texttt{E05E}\\
\mSymbol[outlined]{fiber-pin} & \mSymbol[rounded]{fiber-pin} & \mSymbol[sharp]{fiber-pin} & \texttt{\textbackslash mSymbol\{fiber-pin\}} & \texttt{E06A}\\
\mSymbol[outlined]{fiber-smart-record} & \mSymbol[rounded]{fiber-smart-record} & \mSymbol[sharp]{fiber-smart-record} & \texttt{\textbackslash mSymbol\{fiber-smart-record\}} & \texttt{E062}\\
\mSymbol[outlined]{file-copy} & \mSymbol[rounded]{file-copy} & \mSymbol[sharp]{file-copy} & \texttt{\textbackslash mSymbol\{file-copy\}} & \texttt{E173}\\
\mSymbol[outlined]{file-copy-off} & \mSymbol[rounded]{file-copy-off} & \mSymbol[sharp]{file-copy-off} & \texttt{\textbackslash mSymbol\{file-copy-off\}} & \texttt{F4D8}\\
\mSymbol[outlined]{file-download} & \mSymbol[rounded]{file-download} & \mSymbol[sharp]{file-download} & \texttt{\textbackslash mSymbol\{file-download\}} & \texttt{F090}\\
\mSymbol[outlined]{file-download-done} & \mSymbol[rounded]{file-download-done} & \mSymbol[sharp]{file-download-done} & \texttt{\textbackslash mSymbol\{file-download-done\}} & \texttt{F091}\\
\mSymbol[outlined]{file-download-off} & \mSymbol[rounded]{file-download-off} & \mSymbol[sharp]{file-download-off} & \texttt{\textbackslash mSymbol\{file-download-off\}} & \texttt{E4FE}\\
\mSymbol[outlined]{file-map} & \mSymbol[rounded]{file-map} & \mSymbol[sharp]{file-map} & \texttt{\textbackslash mSymbol\{file-map\}} & \texttt{E2C5}\\
\mSymbol[outlined]{file-open} & \mSymbol[rounded]{file-open} & \mSymbol[sharp]{file-open} & \texttt{\textbackslash mSymbol\{file-open\}} & \texttt{EAF3}\\
\mSymbol[outlined]{file-present} & \mSymbol[rounded]{file-present} & \mSymbol[sharp]{file-present} & \texttt{\textbackslash mSymbol\{file-present\}} & \texttt{EA0E}\\
\mSymbol[outlined]{file-save} & \mSymbol[rounded]{file-save} & \mSymbol[sharp]{file-save} & \texttt{\textbackslash mSymbol\{file-save\}} & \texttt{F17F}\\
\mSymbol[outlined]{file-save-off} & \mSymbol[rounded]{file-save-off} & \mSymbol[sharp]{file-save-off} & \texttt{\textbackslash mSymbol\{file-save-off\}} & \texttt{E505}\\
\mSymbol[outlined]{file-upload} & \mSymbol[rounded]{file-upload} & \mSymbol[sharp]{file-upload} & \texttt{\textbackslash mSymbol\{file-upload\}} & \texttt{F09B}\\
\mSymbol[outlined]{file-upload-off} & \mSymbol[rounded]{file-upload-off} & \mSymbol[sharp]{file-upload-off} & \texttt{\textbackslash mSymbol\{file-upload-off\}} & \texttt{F886}\\
\mSymbol[outlined]{filter} & \mSymbol[rounded]{filter} & \mSymbol[sharp]{filter} & \texttt{\textbackslash mSymbol\{filter\}} & \texttt{E3D3}\\
\mSymbol[outlined]{filter-1} & \mSymbol[rounded]{filter-1} & \mSymbol[sharp]{filter-1} & \texttt{\textbackslash mSymbol\{filter-1\}} & \texttt{E3D0}\\
\mSymbol[outlined]{filter-2} & \mSymbol[rounded]{filter-2} & \mSymbol[sharp]{filter-2} & \texttt{\textbackslash mSymbol\{filter-2\}} & \texttt{E3D1}\\
\mSymbol[outlined]{filter-3} & \mSymbol[rounded]{filter-3} & \mSymbol[sharp]{filter-3} & \texttt{\textbackslash mSymbol\{filter-3\}} & \texttt{E3D2}\\
\mSymbol[outlined]{filter-4} & \mSymbol[rounded]{filter-4} & \mSymbol[sharp]{filter-4} & \texttt{\textbackslash mSymbol\{filter-4\}} & \texttt{E3D4}\\
\mSymbol[outlined]{filter-5} & \mSymbol[rounded]{filter-5} & \mSymbol[sharp]{filter-5} & \texttt{\textbackslash mSymbol\{filter-5\}} & \texttt{E3D5}\\
\mSymbol[outlined]{filter-6} & \mSymbol[rounded]{filter-6} & \mSymbol[sharp]{filter-6} & \texttt{\textbackslash mSymbol\{filter-6\}} & \texttt{E3D6}\\
\mSymbol[outlined]{filter-7} & \mSymbol[rounded]{filter-7} & \mSymbol[sharp]{filter-7} & \texttt{\textbackslash mSymbol\{filter-7\}} & \texttt{E3D7}\\
\mSymbol[outlined]{filter-8} & \mSymbol[rounded]{filter-8} & \mSymbol[sharp]{filter-8} & \texttt{\textbackslash mSymbol\{filter-8\}} & \texttt{E3D8}\\
\mSymbol[outlined]{filter-9} & \mSymbol[rounded]{filter-9} & \mSymbol[sharp]{filter-9} & \texttt{\textbackslash mSymbol\{filter-9\}} & \texttt{E3D9}\\
\mSymbol[outlined]{filter-9-plus} & \mSymbol[rounded]{filter-9-plus} & \mSymbol[sharp]{filter-9-plus} & \texttt{\textbackslash mSymbol\{filter-9-plus\}} & \texttt{E3DA}\\
\mSymbol[outlined]{filter-alt} & \mSymbol[rounded]{filter-alt} & \mSymbol[sharp]{filter-alt} & \texttt{\textbackslash mSymbol\{filter-alt\}} & \texttt{EF4F}\\
\mSymbol[outlined]{filter-alt-off} & \mSymbol[rounded]{filter-alt-off} & \mSymbol[sharp]{filter-alt-off} & \texttt{\textbackslash mSymbol\{filter-alt-off\}} & \texttt{EB32}\\
\mSymbol[outlined]{filter-b-and-w} & \mSymbol[rounded]{filter-b-and-w} & \mSymbol[sharp]{filter-b-and-w} & \texttt{\textbackslash mSymbol\{filter-b-and-w\}} & \texttt{E3DB}\\
\mSymbol[outlined]{filter-center-focus} & \mSymbol[rounded]{filter-center-focus} & \mSymbol[sharp]{filter-center-focus} & \texttt{\textbackslash mSymbol\{filter-center-focus\}} & \texttt{E3DC}\\
\mSymbol[outlined]{filter-drama} & \mSymbol[rounded]{filter-drama} & \mSymbol[sharp]{filter-drama} & \texttt{\textbackslash mSymbol\{filter-drama\}} & \texttt{E3DD}\\
\mSymbol[outlined]{filter-frames} & \mSymbol[rounded]{filter-frames} & \mSymbol[sharp]{filter-frames} & \texttt{\textbackslash mSymbol\{filter-frames\}} & \texttt{E3DE}\\
\mSymbol[outlined]{filter-hdr} & \mSymbol[rounded]{filter-hdr} & \mSymbol[sharp]{filter-hdr} & \texttt{\textbackslash mSymbol\{filter-hdr\}} & \texttt{E3DF}\\
\mSymbol[outlined]{filter-list} & \mSymbol[rounded]{filter-list} & \mSymbol[sharp]{filter-list} & \texttt{\textbackslash mSymbol\{filter-list\}} & \texttt{E152}\\
\mSymbol[outlined]{filter-list-alt} & \mSymbol[rounded]{filter-list-alt} & \mSymbol[sharp]{filter-list-alt} & \texttt{\textbackslash mSymbol\{filter-list-alt\}} & \texttt{E94E}\\
\mSymbol[outlined]{filter-list-off} & \mSymbol[rounded]{filter-list-off} & \mSymbol[sharp]{filter-list-off} & \texttt{\textbackslash mSymbol\{filter-list-off\}} & \texttt{EB57}\\
\mSymbol[outlined]{filter-none} & \mSymbol[rounded]{filter-none} & \mSymbol[sharp]{filter-none} & \texttt{\textbackslash mSymbol\{filter-none\}} & \texttt{E3E0}\\
\mSymbol[outlined]{filter-retrolux} & \mSymbol[rounded]{filter-retrolux} & \mSymbol[sharp]{filter-retrolux} & \texttt{\textbackslash mSymbol\{filter-retrolux\}} & \texttt{E3E1}\\
\mSymbol[outlined]{filter-tilt-shift} & \mSymbol[rounded]{filter-tilt-shift} & \mSymbol[sharp]{filter-tilt-shift} & \texttt{\textbackslash mSymbol\{filter-tilt-shift\}} & \texttt{E3E2}\\
\mSymbol[outlined]{filter-vintage} & \mSymbol[rounded]{filter-vintage} & \mSymbol[sharp]{filter-vintage} & \texttt{\textbackslash mSymbol\{filter-vintage\}} & \texttt{E3E3}\\
\mSymbol[outlined]{finance} & \mSymbol[rounded]{finance} & \mSymbol[sharp]{finance} & \texttt{\textbackslash mSymbol\{finance\}} & \texttt{E6BF}\\
\mSymbol[outlined]{finance-chip} & \mSymbol[rounded]{finance-chip} & \mSymbol[sharp]{finance-chip} & \texttt{\textbackslash mSymbol\{finance-chip\}} & \texttt{F84E}\\
\mSymbol[outlined]{finance-mode} & \mSymbol[rounded]{finance-mode} & \mSymbol[sharp]{finance-mode} & \texttt{\textbackslash mSymbol\{finance-mode\}} & \texttt{EF92}\\
\mSymbol[outlined]{find-in-page} & \mSymbol[rounded]{find-in-page} & \mSymbol[sharp]{find-in-page} & \texttt{\textbackslash mSymbol\{find-in-page\}} & \texttt{E880}\\
\mSymbol[outlined]{find-replace} & \mSymbol[rounded]{find-replace} & \mSymbol[sharp]{find-replace} & \texttt{\textbackslash mSymbol\{find-replace\}} & \texttt{E881}\\
\mSymbol[outlined]{fingerprint} & \mSymbol[rounded]{fingerprint} & \mSymbol[sharp]{fingerprint} & \texttt{\textbackslash mSymbol\{fingerprint\}} & \texttt{E90D}\\
\mSymbol[outlined]{fingerprint-off} & \mSymbol[rounded]{fingerprint-off} & \mSymbol[sharp]{fingerprint-off} & \texttt{\textbackslash mSymbol\{fingerprint-off\}} & \texttt{F49D}\\
\mSymbol[outlined]{fire-extinguisher} & \mSymbol[rounded]{fire-extinguisher} & \mSymbol[sharp]{fire-extinguisher} & \texttt{\textbackslash mSymbol\{fire-extinguisher\}} & \texttt{F1D8}\\
\mSymbol[outlined]{fire-hydrant} & \mSymbol[rounded]{fire-hydrant} & \mSymbol[sharp]{fire-hydrant} & \texttt{\textbackslash mSymbol\{fire-hydrant\}} & \texttt{F1A3}\\
\mSymbol[outlined]{fire-truck} & \mSymbol[rounded]{fire-truck} & \mSymbol[sharp]{fire-truck} & \texttt{\textbackslash mSymbol\{fire-truck\}} & \texttt{F8F2}\\
\mSymbol[outlined]{fireplace} & \mSymbol[rounded]{fireplace} & \mSymbol[sharp]{fireplace} & \texttt{\textbackslash mSymbol\{fireplace\}} & \texttt{EA43}\\
\mSymbol[outlined]{first-page} & \mSymbol[rounded]{first-page} & \mSymbol[sharp]{first-page} & \texttt{\textbackslash mSymbol\{first-page\}} & \texttt{E5DC}\\
\mSymbol[outlined]{fit-page} & \mSymbol[rounded]{fit-page} & \mSymbol[sharp]{fit-page} & \texttt{\textbackslash mSymbol\{fit-page\}} & \texttt{F77A}\\
\mSymbol[outlined]{fit-screen} & \mSymbol[rounded]{fit-screen} & \mSymbol[sharp]{fit-screen} & \texttt{\textbackslash mSymbol\{fit-screen\}} & \texttt{EA10}\\
\mSymbol[outlined]{fit-width} & \mSymbol[rounded]{fit-width} & \mSymbol[sharp]{fit-width} & \texttt{\textbackslash mSymbol\{fit-width\}} & \texttt{F779}\\
\mSymbol[outlined]{fitness-center} & \mSymbol[rounded]{fitness-center} & \mSymbol[sharp]{fitness-center} & \texttt{\textbackslash mSymbol\{fitness-center\}} & \texttt{EB43}\\
\mSymbol[outlined]{fitness-tracker} & \mSymbol[rounded]{fitness-tracker} & \mSymbol[sharp]{fitness-tracker} & \texttt{\textbackslash mSymbol\{fitness-tracker\}} & \texttt{F463}\\
\mSymbol[outlined]{flag} & \mSymbol[rounded]{flag} & \mSymbol[sharp]{flag} & \texttt{\textbackslash mSymbol\{flag\}} & \texttt{F0C6}\\
\mSymbol[outlined]{flag-2} & \mSymbol[rounded]{flag-2} & \mSymbol[sharp]{flag-2} & \texttt{\textbackslash mSymbol\{flag-2\}} & \texttt{F40F}\\
\mSymbol[outlined]{flag-circle} & \mSymbol[rounded]{flag-circle} & \mSymbol[sharp]{flag-circle} & \texttt{\textbackslash mSymbol\{flag-circle\}} & \texttt{EAF8}\\
\mSymbol[outlined]{flag-filled} & \mSymbol[rounded]{flag-filled} & \mSymbol[sharp]{flag-filled} & \texttt{\textbackslash mSymbol\{flag-filled\}} & \texttt{F0C6}\\
\mSymbol[outlined]{flaky} & \mSymbol[rounded]{flaky} & \mSymbol[sharp]{flaky} & \texttt{\textbackslash mSymbol\{flaky\}} & \texttt{EF50}\\
\mSymbol[outlined]{flare} & \mSymbol[rounded]{flare} & \mSymbol[sharp]{flare} & \texttt{\textbackslash mSymbol\{flare\}} & \texttt{E3E4}\\
\mSymbol[outlined]{flash-auto} & \mSymbol[rounded]{flash-auto} & \mSymbol[sharp]{flash-auto} & \texttt{\textbackslash mSymbol\{flash-auto\}} & \texttt{E3E5}\\
\mSymbol[outlined]{flash-off} & \mSymbol[rounded]{flash-off} & \mSymbol[sharp]{flash-off} & \texttt{\textbackslash mSymbol\{flash-off\}} & \texttt{E3E6}\\
\mSymbol[outlined]{flash-on} & \mSymbol[rounded]{flash-on} & \mSymbol[sharp]{flash-on} & \texttt{\textbackslash mSymbol\{flash-on\}} & \texttt{E3E7}\\
\mSymbol[outlined]{flashlight-off} & \mSymbol[rounded]{flashlight-off} & \mSymbol[sharp]{flashlight-off} & \texttt{\textbackslash mSymbol\{flashlight-off\}} & \texttt{F00A}\\
\mSymbol[outlined]{flashlight-on} & \mSymbol[rounded]{flashlight-on} & \mSymbol[sharp]{flashlight-on} & \texttt{\textbackslash mSymbol\{flashlight-on\}} & \texttt{F00B}\\
\mSymbol[outlined]{flatware} & \mSymbol[rounded]{flatware} & \mSymbol[sharp]{flatware} & \texttt{\textbackslash mSymbol\{flatware\}} & \texttt{F00C}\\
\mSymbol[outlined]{flex-direction} & \mSymbol[rounded]{flex-direction} & \mSymbol[sharp]{flex-direction} & \texttt{\textbackslash mSymbol\{flex-direction\}} & \texttt{F778}\\
\mSymbol[outlined]{flex-no-wrap} & \mSymbol[rounded]{flex-no-wrap} & \mSymbol[sharp]{flex-no-wrap} & \texttt{\textbackslash mSymbol\{flex-no-wrap\}} & \texttt{F777}\\
\mSymbol[outlined]{flex-wrap} & \mSymbol[rounded]{flex-wrap} & \mSymbol[sharp]{flex-wrap} & \texttt{\textbackslash mSymbol\{flex-wrap\}} & \texttt{F776}\\
\mSymbol[outlined]{flight} & \mSymbol[rounded]{flight} & \mSymbol[sharp]{flight} & \texttt{\textbackslash mSymbol\{flight\}} & \texttt{E539}\\
\mSymbol[outlined]{flight-class} & \mSymbol[rounded]{flight-class} & \mSymbol[sharp]{flight-class} & \texttt{\textbackslash mSymbol\{flight-class\}} & \texttt{E7CB}\\
\mSymbol[outlined]{flight-land} & \mSymbol[rounded]{flight-land} & \mSymbol[sharp]{flight-land} & \texttt{\textbackslash mSymbol\{flight-land\}} & \texttt{E904}\\
\mSymbol[outlined]{flight-takeoff} & \mSymbol[rounded]{flight-takeoff} & \mSymbol[sharp]{flight-takeoff} & \texttt{\textbackslash mSymbol\{flight-takeoff\}} & \texttt{E905}\\
\mSymbol[outlined]{flights-and-hotels} & \mSymbol[rounded]{flights-and-hotels} & \mSymbol[sharp]{flights-and-hotels} & \texttt{\textbackslash mSymbol\{flights-and-hotels\}} & \texttt{E9AB}\\
\mSymbol[outlined]{flightsmode} & \mSymbol[rounded]{flightsmode} & \mSymbol[sharp]{flightsmode} & \texttt{\textbackslash mSymbol\{flightsmode\}} & \texttt{EF93}\\
\mSymbol[outlined]{flip} & \mSymbol[rounded]{flip} & \mSymbol[sharp]{flip} & \texttt{\textbackslash mSymbol\{flip\}} & \texttt{E3E8}\\
\mSymbol[outlined]{flip-camera-android} & \mSymbol[rounded]{flip-camera-android} & \mSymbol[sharp]{flip-camera-android} & \texttt{\textbackslash mSymbol\{flip-camera-android\}} & \texttt{EA37}\\
\mSymbol[outlined]{flip-camera-ios} & \mSymbol[rounded]{flip-camera-ios} & \mSymbol[sharp]{flip-camera-ios} & \texttt{\textbackslash mSymbol\{flip-camera-ios\}} & \texttt{EA38}\\
\mSymbol[outlined]{flip-to-back} & \mSymbol[rounded]{flip-to-back} & \mSymbol[sharp]{flip-to-back} & \texttt{\textbackslash mSymbol\{flip-to-back\}} & \texttt{E882}\\
\mSymbol[outlined]{flip-to-front} & \mSymbol[rounded]{flip-to-front} & \mSymbol[sharp]{flip-to-front} & \texttt{\textbackslash mSymbol\{flip-to-front\}} & \texttt{E883}\\
\mSymbol[outlined]{float-landscape-2} & \mSymbol[rounded]{float-landscape-2} & \mSymbol[sharp]{float-landscape-2} & \texttt{\textbackslash mSymbol\{float-landscape-2\}} & \texttt{F45C}\\
\mSymbol[outlined]{float-portrait-2} & \mSymbol[rounded]{float-portrait-2} & \mSymbol[sharp]{float-portrait-2} & \texttt{\textbackslash mSymbol\{float-portrait-2\}} & \texttt{F45B}\\
\mSymbol[outlined]{flood} & \mSymbol[rounded]{flood} & \mSymbol[sharp]{flood} & \texttt{\textbackslash mSymbol\{flood\}} & \texttt{EBE6}\\
\mSymbol[outlined]{floor} & \mSymbol[rounded]{floor} & \mSymbol[sharp]{floor} & \texttt{\textbackslash mSymbol\{floor\}} & \texttt{F6E4}\\
\mSymbol[outlined]{floor-lamp} & \mSymbol[rounded]{floor-lamp} & \mSymbol[sharp]{floor-lamp} & \texttt{\textbackslash mSymbol\{floor-lamp\}} & \texttt{E21E}\\
\mSymbol[outlined]{flourescent} & \mSymbol[rounded]{flourescent} & \mSymbol[sharp]{flourescent} & \texttt{\textbackslash mSymbol\{flourescent\}} & \texttt{F07D}\\
\mSymbol[outlined]{flowsheet} & \mSymbol[rounded]{flowsheet} & \mSymbol[sharp]{flowsheet} & \texttt{\textbackslash mSymbol\{flowsheet\}} & \texttt{E0AE}\\
\mSymbol[outlined]{fluid} & \mSymbol[rounded]{fluid} & \mSymbol[sharp]{fluid} & \texttt{\textbackslash mSymbol\{fluid\}} & \texttt{E483}\\
\mSymbol[outlined]{fluid-balance} & \mSymbol[rounded]{fluid-balance} & \mSymbol[sharp]{fluid-balance} & \texttt{\textbackslash mSymbol\{fluid-balance\}} & \texttt{F80D}\\
\mSymbol[outlined]{fluid-med} & \mSymbol[rounded]{fluid-med} & \mSymbol[sharp]{fluid-med} & \texttt{\textbackslash mSymbol\{fluid-med\}} & \texttt{F80C}\\
\mSymbol[outlined]{fluorescent} & \mSymbol[rounded]{fluorescent} & \mSymbol[sharp]{fluorescent} & \texttt{\textbackslash mSymbol\{fluorescent\}} & \texttt{F07D}\\
\mSymbol[outlined]{flutter} & \mSymbol[rounded]{flutter} & \mSymbol[sharp]{flutter} & \texttt{\textbackslash mSymbol\{flutter\}} & \texttt{F1DD}\\
\mSymbol[outlined]{flutter-dash} & \mSymbol[rounded]{flutter-dash} & \mSymbol[sharp]{flutter-dash} & \texttt{\textbackslash mSymbol\{flutter-dash\}} & \texttt{E00B}\\
\mSymbol[outlined]{flyover} & \mSymbol[rounded]{flyover} & \mSymbol[sharp]{flyover} & \texttt{\textbackslash mSymbol\{flyover\}} & \texttt{F478}\\
\mSymbol[outlined]{fmd-bad} & \mSymbol[rounded]{fmd-bad} & \mSymbol[sharp]{fmd-bad} & \texttt{\textbackslash mSymbol\{fmd-bad\}} & \texttt{F00E}\\
\mSymbol[outlined]{fmd-good} & \mSymbol[rounded]{fmd-good} & \mSymbol[sharp]{fmd-good} & \texttt{\textbackslash mSymbol\{fmd-good\}} & \texttt{F1DB}\\
\mSymbol[outlined]{foggy} & \mSymbol[rounded]{foggy} & \mSymbol[sharp]{foggy} & \texttt{\textbackslash mSymbol\{foggy\}} & \texttt{E818}\\
\mSymbol[outlined]{folded-hands} & \mSymbol[rounded]{folded-hands} & \mSymbol[sharp]{folded-hands} & \texttt{\textbackslash mSymbol\{folded-hands\}} & \texttt{F5ED}\\
\mSymbol[outlined]{folder} & \mSymbol[rounded]{folder} & \mSymbol[sharp]{folder} & \texttt{\textbackslash mSymbol\{folder\}} & \texttt{E2C7}\\
\mSymbol[outlined]{folder-copy} & \mSymbol[rounded]{folder-copy} & \mSymbol[sharp]{folder-copy} & \texttt{\textbackslash mSymbol\{folder-copy\}} & \texttt{EBBD}\\
\mSymbol[outlined]{folder-data} & \mSymbol[rounded]{folder-data} & \mSymbol[sharp]{folder-data} & \texttt{\textbackslash mSymbol\{folder-data\}} & \texttt{F586}\\
\mSymbol[outlined]{folder-delete} & \mSymbol[rounded]{folder-delete} & \mSymbol[sharp]{folder-delete} & \texttt{\textbackslash mSymbol\{folder-delete\}} & \texttt{EB34}\\
\mSymbol[outlined]{folder-limited} & \mSymbol[rounded]{folder-limited} & \mSymbol[sharp]{folder-limited} & \texttt{\textbackslash mSymbol\{folder-limited\}} & \texttt{F4E4}\\
\mSymbol[outlined]{folder-managed} & \mSymbol[rounded]{folder-managed} & \mSymbol[sharp]{folder-managed} & \texttt{\textbackslash mSymbol\{folder-managed\}} & \texttt{F775}\\
\mSymbol[outlined]{folder-off} & \mSymbol[rounded]{folder-off} & \mSymbol[sharp]{folder-off} & \texttt{\textbackslash mSymbol\{folder-off\}} & \texttt{EB83}\\
\mSymbol[outlined]{folder-open} & \mSymbol[rounded]{folder-open} & \mSymbol[sharp]{folder-open} & \texttt{\textbackslash mSymbol\{folder-open\}} & \texttt{E2C8}\\
\mSymbol[outlined]{folder-shared} & \mSymbol[rounded]{folder-shared} & \mSymbol[sharp]{folder-shared} & \texttt{\textbackslash mSymbol\{folder-shared\}} & \texttt{E2C9}\\
\mSymbol[outlined]{folder-special} & \mSymbol[rounded]{folder-special} & \mSymbol[sharp]{folder-special} & \texttt{\textbackslash mSymbol\{folder-special\}} & \texttt{E617}\\
\mSymbol[outlined]{folder-supervised} & \mSymbol[rounded]{folder-supervised} & \mSymbol[sharp]{folder-supervised} & \texttt{\textbackslash mSymbol\{folder-supervised\}} & \texttt{F774}\\
\mSymbol[outlined]{folder-zip} & \mSymbol[rounded]{folder-zip} & \mSymbol[sharp]{folder-zip} & \texttt{\textbackslash mSymbol\{folder-zip\}} & \texttt{EB2C}\\
\mSymbol[outlined]{follow-the-signs} & \mSymbol[rounded]{follow-the-signs} & \mSymbol[sharp]{follow-the-signs} & \texttt{\textbackslash mSymbol\{follow-the-signs\}} & \texttt{F222}\\
\mSymbol[outlined]{font-download} & \mSymbol[rounded]{font-download} & \mSymbol[sharp]{font-download} & \texttt{\textbackslash mSymbol\{font-download\}} & \texttt{E167}\\
\mSymbol[outlined]{font-download-off} & \mSymbol[rounded]{font-download-off} & \mSymbol[sharp]{font-download-off} & \texttt{\textbackslash mSymbol\{font-download-off\}} & \texttt{E4F9}\\
\mSymbol[outlined]{food-bank} & \mSymbol[rounded]{food-bank} & \mSymbol[sharp]{food-bank} & \texttt{\textbackslash mSymbol\{food-bank\}} & \texttt{F1F2}\\
\mSymbol[outlined]{foot-bones} & \mSymbol[rounded]{foot-bones} & \mSymbol[sharp]{foot-bones} & \texttt{\textbackslash mSymbol\{foot-bones\}} & \texttt{F893}\\
\mSymbol[outlined]{footprint} & \mSymbol[rounded]{footprint} & \mSymbol[sharp]{footprint} & \texttt{\textbackslash mSymbol\{footprint\}} & \texttt{F87D}\\
\mSymbol[outlined]{for-you} & \mSymbol[rounded]{for-you} & \mSymbol[sharp]{for-you} & \texttt{\textbackslash mSymbol\{for-you\}} & \texttt{E9AC}\\
\mSymbol[outlined]{forest} & \mSymbol[rounded]{forest} & \mSymbol[sharp]{forest} & \texttt{\textbackslash mSymbol\{forest\}} & \texttt{EA99}\\
\mSymbol[outlined]{fork-left} & \mSymbol[rounded]{fork-left} & \mSymbol[sharp]{fork-left} & \texttt{\textbackslash mSymbol\{fork-left\}} & \texttt{EBA0}\\
\mSymbol[outlined]{fork-right} & \mSymbol[rounded]{fork-right} & \mSymbol[sharp]{fork-right} & \texttt{\textbackslash mSymbol\{fork-right\}} & \texttt{EBAC}\\
\mSymbol[outlined]{forklift} & \mSymbol[rounded]{forklift} & \mSymbol[sharp]{forklift} & \texttt{\textbackslash mSymbol\{forklift\}} & \texttt{F868}\\
\mSymbol[outlined]{format-align-center} & \mSymbol[rounded]{format-align-center} & \mSymbol[sharp]{format-align-center} & \texttt{\textbackslash mSymbol\{format-align-center\}} & \texttt{E234}\\
\mSymbol[outlined]{format-align-justify} & \mSymbol[rounded]{format-align-justify} & \mSymbol[sharp]{format-align-justify} & \texttt{\textbackslash mSymbol\{format-align-justify\}} & \texttt{E235}\\
\mSymbol[outlined]{format-align-left} & \mSymbol[rounded]{format-align-left} & \mSymbol[sharp]{format-align-left} & \texttt{\textbackslash mSymbol\{format-align-left\}} & \texttt{E236}\\
\mSymbol[outlined]{format-align-right} & \mSymbol[rounded]{format-align-right} & \mSymbol[sharp]{format-align-right} & \texttt{\textbackslash mSymbol\{format-align-right\}} & \texttt{E237}\\
\mSymbol[outlined]{format-bold} & \mSymbol[rounded]{format-bold} & \mSymbol[sharp]{format-bold} & \texttt{\textbackslash mSymbol\{format-bold\}} & \texttt{E238}\\
\mSymbol[outlined]{format-clear} & \mSymbol[rounded]{format-clear} & \mSymbol[sharp]{format-clear} & \texttt{\textbackslash mSymbol\{format-clear\}} & \texttt{E239}\\
\mSymbol[outlined]{format-color-fill} & \mSymbol[rounded]{format-color-fill} & \mSymbol[sharp]{format-color-fill} & \texttt{\textbackslash mSymbol\{format-color-fill\}} & \texttt{E23A}\\
\mSymbol[outlined]{format-color-reset} & \mSymbol[rounded]{format-color-reset} & \mSymbol[sharp]{format-color-reset} & \texttt{\textbackslash mSymbol\{format-color-reset\}} & \texttt{E23B}\\
\mSymbol[outlined]{format-color-text} & \mSymbol[rounded]{format-color-text} & \mSymbol[sharp]{format-color-text} & \texttt{\textbackslash mSymbol\{format-color-text\}} & \texttt{E23C}\\
\mSymbol[outlined]{format-h1} & \mSymbol[rounded]{format-h1} & \mSymbol[sharp]{format-h1} & \texttt{\textbackslash mSymbol\{format-h1\}} & \texttt{F85D}\\
\mSymbol[outlined]{format-h2} & \mSymbol[rounded]{format-h2} & \mSymbol[sharp]{format-h2} & \texttt{\textbackslash mSymbol\{format-h2\}} & \texttt{F85E}\\
\mSymbol[outlined]{format-h3} & \mSymbol[rounded]{format-h3} & \mSymbol[sharp]{format-h3} & \texttt{\textbackslash mSymbol\{format-h3\}} & \texttt{F85F}\\
\mSymbol[outlined]{format-h4} & \mSymbol[rounded]{format-h4} & \mSymbol[sharp]{format-h4} & \texttt{\textbackslash mSymbol\{format-h4\}} & \texttt{F860}\\
\mSymbol[outlined]{format-h5} & \mSymbol[rounded]{format-h5} & \mSymbol[sharp]{format-h5} & \texttt{\textbackslash mSymbol\{format-h5\}} & \texttt{F861}\\
\mSymbol[outlined]{format-h6} & \mSymbol[rounded]{format-h6} & \mSymbol[sharp]{format-h6} & \texttt{\textbackslash mSymbol\{format-h6\}} & \texttt{F862}\\
\mSymbol[outlined]{format-image-left} & \mSymbol[rounded]{format-image-left} & \mSymbol[sharp]{format-image-left} & \texttt{\textbackslash mSymbol\{format-image-left\}} & \texttt{F863}\\
\mSymbol[outlined]{format-image-right} & \mSymbol[rounded]{format-image-right} & \mSymbol[sharp]{format-image-right} & \texttt{\textbackslash mSymbol\{format-image-right\}} & \texttt{F864}\\
\mSymbol[outlined]{format-indent-decrease} & \mSymbol[rounded]{format-indent-decrease} & \mSymbol[sharp]{format-indent-decrease} & \texttt{\textbackslash mSymbol\{format-indent-decrease\}} & \texttt{E23D}\\
\mSymbol[outlined]{format-indent-increase} & \mSymbol[rounded]{format-indent-increase} & \mSymbol[sharp]{format-indent-increase} & \texttt{\textbackslash mSymbol\{format-indent-increase\}} & \texttt{E23E}\\
\mSymbol[outlined]{format-ink-highlighter} & \mSymbol[rounded]{format-ink-highlighter} & \mSymbol[sharp]{format-ink-highlighter} & \texttt{\textbackslash mSymbol\{format-ink-highlighter\}} & \texttt{F82B}\\
\mSymbol[outlined]{format-italic} & \mSymbol[rounded]{format-italic} & \mSymbol[sharp]{format-italic} & \texttt{\textbackslash mSymbol\{format-italic\}} & \texttt{E23F}\\
\mSymbol[outlined]{format-letter-spacing} & \mSymbol[rounded]{format-letter-spacing} & \mSymbol[sharp]{format-letter-spacing} & \texttt{\textbackslash mSymbol\{format-letter-spacing\}} & \texttt{F773}\\
\mSymbol[outlined]{format-letter-spacing-2} & \mSymbol[rounded]{format-letter-spacing-2} & \mSymbol[sharp]{format-letter-spacing-2} & \texttt{\textbackslash mSymbol\{format-letter-spacing-2\}} & \texttt{F618}\\
\mSymbol[outlined]{format-letter-spacing-standard} & \mSymbol[rounded]{format-letter-spacing-standard} & \mSymbol[sharp]{format-letter-spacing-standard} & \texttt{\textbackslash mSymbol\{format-letter-spacing-standard\}} & \texttt{F617}\\
\mSymbol[outlined]{format-letter-spacing-wide} & \mSymbol[rounded]{format-letter-spacing-wide} & \mSymbol[sharp]{format-letter-spacing-wide} & \texttt{\textbackslash mSymbol\{format-letter-spacing-wide\}} & \texttt{F616}\\
\mSymbol[outlined]{format-letter-spacing-wider} & \mSymbol[rounded]{format-letter-spacing-wider} & \mSymbol[sharp]{format-letter-spacing-wider} & \texttt{\textbackslash mSymbol\{format-letter-spacing-wider\}} & \texttt{F615}\\
\mSymbol[outlined]{format-line-spacing} & \mSymbol[rounded]{format-line-spacing} & \mSymbol[sharp]{format-line-spacing} & \texttt{\textbackslash mSymbol\{format-line-spacing\}} & \texttt{E240}\\
\mSymbol[outlined]{format-list-bulleted} & \mSymbol[rounded]{format-list-bulleted} & \mSymbol[sharp]{format-list-bulleted} & \texttt{\textbackslash mSymbol\{format-list-bulleted\}} & \texttt{E241}\\
\mSymbol[outlined]{format-list-bulleted-add} & \mSymbol[rounded]{format-list-bulleted-add} & \mSymbol[sharp]{format-list-bulleted-add} & \texttt{\textbackslash mSymbol\{format-list-bulleted-add\}} & \texttt{F849}\\
\mSymbol[outlined]{format-list-numbered} & \mSymbol[rounded]{format-list-numbered} & \mSymbol[sharp]{format-list-numbered} & \texttt{\textbackslash mSymbol\{format-list-numbered\}} & \texttt{E242}\\
\mSymbol[outlined]{format-list-numbered-rtl} & \mSymbol[rounded]{format-list-numbered-rtl} & \mSymbol[sharp]{format-list-numbered-rtl} & \texttt{\textbackslash mSymbol\{format-list-numbered-rtl\}} & \texttt{E267}\\
\mSymbol[outlined]{format-overline} & \mSymbol[rounded]{format-overline} & \mSymbol[sharp]{format-overline} & \texttt{\textbackslash mSymbol\{format-overline\}} & \texttt{EB65}\\
\mSymbol[outlined]{format-paint} & \mSymbol[rounded]{format-paint} & \mSymbol[sharp]{format-paint} & \texttt{\textbackslash mSymbol\{format-paint\}} & \texttt{E243}\\
\mSymbol[outlined]{format-paragraph} & \mSymbol[rounded]{format-paragraph} & \mSymbol[sharp]{format-paragraph} & \texttt{\textbackslash mSymbol\{format-paragraph\}} & \texttt{F865}\\
\mSymbol[outlined]{format-quote} & \mSymbol[rounded]{format-quote} & \mSymbol[sharp]{format-quote} & \texttt{\textbackslash mSymbol\{format-quote\}} & \texttt{E244}\\
\mSymbol[outlined]{format-quote-off} & \mSymbol[rounded]{format-quote-off} & \mSymbol[sharp]{format-quote-off} & \texttt{\textbackslash mSymbol\{format-quote-off\}} & \texttt{F413}\\
\mSymbol[outlined]{format-shapes} & \mSymbol[rounded]{format-shapes} & \mSymbol[sharp]{format-shapes} & \texttt{\textbackslash mSymbol\{format-shapes\}} & \texttt{E25E}\\
\mSymbol[outlined]{format-size} & \mSymbol[rounded]{format-size} & \mSymbol[sharp]{format-size} & \texttt{\textbackslash mSymbol\{format-size\}} & \texttt{E245}\\
\mSymbol[outlined]{format-strikethrough} & \mSymbol[rounded]{format-strikethrough} & \mSymbol[sharp]{format-strikethrough} & \texttt{\textbackslash mSymbol\{format-strikethrough\}} & \texttt{E246}\\
\mSymbol[outlined]{format-text-clip} & \mSymbol[rounded]{format-text-clip} & \mSymbol[sharp]{format-text-clip} & \texttt{\textbackslash mSymbol\{format-text-clip\}} & \texttt{F82A}\\
\mSymbol[outlined]{format-text-overflow} & \mSymbol[rounded]{format-text-overflow} & \mSymbol[sharp]{format-text-overflow} & \texttt{\textbackslash mSymbol\{format-text-overflow\}} & \texttt{F829}\\
\mSymbol[outlined]{format-text-wrap} & \mSymbol[rounded]{format-text-wrap} & \mSymbol[sharp]{format-text-wrap} & \texttt{\textbackslash mSymbol\{format-text-wrap\}} & \texttt{F828}\\
\mSymbol[outlined]{format-textdirection-l-to-r} & \mSymbol[rounded]{format-textdirection-l-to-r} & \mSymbol[sharp]{format-textdirection-l-to-r} & \texttt{\textbackslash mSymbol\{format-textdirection-l-to-r\}} & \texttt{E247}\\
\mSymbol[outlined]{format-textdirection-r-to-l} & \mSymbol[rounded]{format-textdirection-r-to-l} & \mSymbol[sharp]{format-textdirection-r-to-l} & \texttt{\textbackslash mSymbol\{format-textdirection-r-to-l\}} & \texttt{E248}\\
\mSymbol[outlined]{format-textdirection-vertical} & \mSymbol[rounded]{format-textdirection-vertical} & \mSymbol[sharp]{format-textdirection-vertical} & \texttt{\textbackslash mSymbol\{format-textdirection-vertical\}} & \texttt{F4B8}\\
\mSymbol[outlined]{format-underlined} & \mSymbol[rounded]{format-underlined} & \mSymbol[sharp]{format-underlined} & \texttt{\textbackslash mSymbol\{format-underlined\}} & \texttt{E249}\\
\mSymbol[outlined]{format-underlined-squiggle} & \mSymbol[rounded]{format-underlined-squiggle} & \mSymbol[sharp]{format-underlined-squiggle} & \texttt{\textbackslash mSymbol\{format-underlined-squiggle\}} & \texttt{F885}\\
\mSymbol[outlined]{forms-add-on} & \mSymbol[rounded]{forms-add-on} & \mSymbol[sharp]{forms-add-on} & \texttt{\textbackslash mSymbol\{forms-add-on\}} & \texttt{F0C7}\\
\mSymbol[outlined]{forms-apps-script} & \mSymbol[rounded]{forms-apps-script} & \mSymbol[sharp]{forms-apps-script} & \texttt{\textbackslash mSymbol\{forms-apps-script\}} & \texttt{F0C8}\\
\mSymbol[outlined]{fort} & \mSymbol[rounded]{fort} & \mSymbol[sharp]{fort} & \texttt{\textbackslash mSymbol\{fort\}} & \texttt{EAAD}\\
\mSymbol[outlined]{forum} & \mSymbol[rounded]{forum} & \mSymbol[sharp]{forum} & \texttt{\textbackslash mSymbol\{forum\}} & \texttt{E8AF}\\
\mSymbol[outlined]{forward} & \mSymbol[rounded]{forward} & \mSymbol[sharp]{forward} & \texttt{\textbackslash mSymbol\{forward\}} & \texttt{F57A}\\
\mSymbol[outlined]{forward-10} & \mSymbol[rounded]{forward-10} & \mSymbol[sharp]{forward-10} & \texttt{\textbackslash mSymbol\{forward-10\}} & \texttt{E056}\\
\mSymbol[outlined]{forward-30} & \mSymbol[rounded]{forward-30} & \mSymbol[sharp]{forward-30} & \texttt{\textbackslash mSymbol\{forward-30\}} & \texttt{E057}\\
\mSymbol[outlined]{forward-5} & \mSymbol[rounded]{forward-5} & \mSymbol[sharp]{forward-5} & \texttt{\textbackslash mSymbol\{forward-5\}} & \texttt{E058}\\
\mSymbol[outlined]{forward-circle} & \mSymbol[rounded]{forward-circle} & \mSymbol[sharp]{forward-circle} & \texttt{\textbackslash mSymbol\{forward-circle\}} & \texttt{F6F5}\\
\mSymbol[outlined]{forward-media} & \mSymbol[rounded]{forward-media} & \mSymbol[sharp]{forward-media} & \texttt{\textbackslash mSymbol\{forward-media\}} & \texttt{F6F4}\\
\mSymbol[outlined]{forward-to-inbox} & \mSymbol[rounded]{forward-to-inbox} & \mSymbol[sharp]{forward-to-inbox} & \texttt{\textbackslash mSymbol\{forward-to-inbox\}} & \texttt{F187}\\
\mSymbol[outlined]{foundation} & \mSymbol[rounded]{foundation} & \mSymbol[sharp]{foundation} & \texttt{\textbackslash mSymbol\{foundation\}} & \texttt{F200}\\
\mSymbol[outlined]{frame-inspect} & \mSymbol[rounded]{frame-inspect} & \mSymbol[sharp]{frame-inspect} & \texttt{\textbackslash mSymbol\{frame-inspect\}} & \texttt{F772}\\
\mSymbol[outlined]{frame-person} & \mSymbol[rounded]{frame-person} & \mSymbol[sharp]{frame-person} & \texttt{\textbackslash mSymbol\{frame-person\}} & \texttt{F8A6}\\
\mSymbol[outlined]{frame-person-mic} & \mSymbol[rounded]{frame-person-mic} & \mSymbol[sharp]{frame-person-mic} & \texttt{\textbackslash mSymbol\{frame-person-mic\}} & \texttt{F4D5}\\
\mSymbol[outlined]{frame-person-off} & \mSymbol[rounded]{frame-person-off} & \mSymbol[sharp]{frame-person-off} & \texttt{\textbackslash mSymbol\{frame-person-off\}} & \texttt{F7D1}\\
\mSymbol[outlined]{frame-reload} & \mSymbol[rounded]{frame-reload} & \mSymbol[sharp]{frame-reload} & \texttt{\textbackslash mSymbol\{frame-reload\}} & \texttt{F771}\\
\mSymbol[outlined]{frame-source} & \mSymbol[rounded]{frame-source} & \mSymbol[sharp]{frame-source} & \texttt{\textbackslash mSymbol\{frame-source\}} & \texttt{F770}\\
\mSymbol[outlined]{free-breakfast} & \mSymbol[rounded]{free-breakfast} & \mSymbol[sharp]{free-breakfast} & \texttt{\textbackslash mSymbol\{free-breakfast\}} & \texttt{EB44}\\
\mSymbol[outlined]{free-cancellation} & \mSymbol[rounded]{free-cancellation} & \mSymbol[sharp]{free-cancellation} & \texttt{\textbackslash mSymbol\{free-cancellation\}} & \texttt{E748}\\
\mSymbol[outlined]{front-hand} & \mSymbol[rounded]{front-hand} & \mSymbol[sharp]{front-hand} & \texttt{\textbackslash mSymbol\{front-hand\}} & \texttt{E769}\\
\mSymbol[outlined]{front-loader} & \mSymbol[rounded]{front-loader} & \mSymbol[sharp]{front-loader} & \texttt{\textbackslash mSymbol\{front-loader\}} & \texttt{F869}\\
\mSymbol[outlined]{full-coverage} & \mSymbol[rounded]{full-coverage} & \mSymbol[sharp]{full-coverage} & \texttt{\textbackslash mSymbol\{full-coverage\}} & \texttt{EB12}\\
\mSymbol[outlined]{full-hd} & \mSymbol[rounded]{full-hd} & \mSymbol[sharp]{full-hd} & \texttt{\textbackslash mSymbol\{full-hd\}} & \texttt{F58B}\\
\mSymbol[outlined]{full-stacked-bar-chart} & \mSymbol[rounded]{full-stacked-bar-chart} & \mSymbol[sharp]{full-stacked-bar-chart} & \texttt{\textbackslash mSymbol\{full-stacked-bar-chart\}} & \texttt{F212}\\
\mSymbol[outlined]{fullscreen} & \mSymbol[rounded]{fullscreen} & \mSymbol[sharp]{fullscreen} & \texttt{\textbackslash mSymbol\{fullscreen\}} & \texttt{E5D0}\\
\mSymbol[outlined]{fullscreen-exit} & \mSymbol[rounded]{fullscreen-exit} & \mSymbol[sharp]{fullscreen-exit} & \texttt{\textbackslash mSymbol\{fullscreen-exit\}} & \texttt{E5D1}\\
\mSymbol[outlined]{fullscreen-portrait} & \mSymbol[rounded]{fullscreen-portrait} & \mSymbol[sharp]{fullscreen-portrait} & \texttt{\textbackslash mSymbol\{fullscreen-portrait\}} & \texttt{F45A}\\
\mSymbol[outlined]{function} & \mSymbol[rounded]{function} & \mSymbol[sharp]{function} & \texttt{\textbackslash mSymbol\{function\}} & \texttt{F866}\\
\mSymbol[outlined]{functions} & \mSymbol[rounded]{functions} & \mSymbol[sharp]{functions} & \texttt{\textbackslash mSymbol\{functions\}} & \texttt{E24A}\\
\mSymbol[outlined]{funicular} & \mSymbol[rounded]{funicular} & \mSymbol[sharp]{funicular} & \texttt{\textbackslash mSymbol\{funicular\}} & \texttt{F477}\\
\mSymbol[outlined]{g-mobiledata} & \mSymbol[rounded]{g-mobiledata} & \mSymbol[sharp]{g-mobiledata} & \texttt{\textbackslash mSymbol\{g-mobiledata\}} & \texttt{F010}\\
\mSymbol[outlined]{g-mobiledata-badge} & \mSymbol[rounded]{g-mobiledata-badge} & \mSymbol[sharp]{g-mobiledata-badge} & \texttt{\textbackslash mSymbol\{g-mobiledata-badge\}} & \texttt{F7E1}\\
\mSymbol[outlined]{g-translate} & \mSymbol[rounded]{g-translate} & \mSymbol[sharp]{g-translate} & \texttt{\textbackslash mSymbol\{g-translate\}} & \texttt{E927}\\
\mSymbol[outlined]{gallery-thumbnail} & \mSymbol[rounded]{gallery-thumbnail} & \mSymbol[sharp]{gallery-thumbnail} & \texttt{\textbackslash mSymbol\{gallery-thumbnail\}} & \texttt{F86F}\\
\mSymbol[outlined]{gamepad} & \mSymbol[rounded]{gamepad} & \mSymbol[sharp]{gamepad} & \texttt{\textbackslash mSymbol\{gamepad\}} & \texttt{E30F}\\
\mSymbol[outlined]{games} & \mSymbol[rounded]{games} & \mSymbol[sharp]{games} & \texttt{\textbackslash mSymbol\{games\}} & \texttt{E30F}\\
\mSymbol[outlined]{garage} & \mSymbol[rounded]{garage} & \mSymbol[sharp]{garage} & \texttt{\textbackslash mSymbol\{garage\}} & \texttt{F011}\\
\mSymbol[outlined]{garage-door} & \mSymbol[rounded]{garage-door} & \mSymbol[sharp]{garage-door} & \texttt{\textbackslash mSymbol\{garage-door\}} & \texttt{E714}\\
\mSymbol[outlined]{garage-home} & \mSymbol[rounded]{garage-home} & \mSymbol[sharp]{garage-home} & \texttt{\textbackslash mSymbol\{garage-home\}} & \texttt{E82D}\\
\mSymbol[outlined]{garden-cart} & \mSymbol[rounded]{garden-cart} & \mSymbol[sharp]{garden-cart} & \texttt{\textbackslash mSymbol\{garden-cart\}} & \texttt{F8A9}\\
\mSymbol[outlined]{gas-meter} & \mSymbol[rounded]{gas-meter} & \mSymbol[sharp]{gas-meter} & \texttt{\textbackslash mSymbol\{gas-meter\}} & \texttt{EC19}\\
\mSymbol[outlined]{gastroenterology} & \mSymbol[rounded]{gastroenterology} & \mSymbol[sharp]{gastroenterology} & \texttt{\textbackslash mSymbol\{gastroenterology\}} & \texttt{E0F1}\\
\mSymbol[outlined]{gate} & \mSymbol[rounded]{gate} & \mSymbol[sharp]{gate} & \texttt{\textbackslash mSymbol\{gate\}} & \texttt{E277}\\
\mSymbol[outlined]{gavel} & \mSymbol[rounded]{gavel} & \mSymbol[sharp]{gavel} & \texttt{\textbackslash mSymbol\{gavel\}} & \texttt{E90E}\\
\mSymbol[outlined]{general-device} & \mSymbol[rounded]{general-device} & \mSymbol[sharp]{general-device} & \texttt{\textbackslash mSymbol\{general-device\}} & \texttt{E6DE}\\
\mSymbol[outlined]{generating-tokens} & \mSymbol[rounded]{generating-tokens} & \mSymbol[sharp]{generating-tokens} & \texttt{\textbackslash mSymbol\{generating-tokens\}} & \texttt{E749}\\
\mSymbol[outlined]{genetics} & \mSymbol[rounded]{genetics} & \mSymbol[sharp]{genetics} & \texttt{\textbackslash mSymbol\{genetics\}} & \texttt{E0F3}\\
\mSymbol[outlined]{genres} & \mSymbol[rounded]{genres} & \mSymbol[sharp]{genres} & \texttt{\textbackslash mSymbol\{genres\}} & \texttt{E6EE}\\
\mSymbol[outlined]{gesture} & \mSymbol[rounded]{gesture} & \mSymbol[sharp]{gesture} & \texttt{\textbackslash mSymbol\{gesture\}} & \texttt{E155}\\
\mSymbol[outlined]{gesture-select} & \mSymbol[rounded]{gesture-select} & \mSymbol[sharp]{gesture-select} & \texttt{\textbackslash mSymbol\{gesture-select\}} & \texttt{F657}\\
\mSymbol[outlined]{get-app} & \mSymbol[rounded]{get-app} & \mSymbol[sharp]{get-app} & \texttt{\textbackslash mSymbol\{get-app\}} & \texttt{F090}\\
\mSymbol[outlined]{gif} & \mSymbol[rounded]{gif} & \mSymbol[sharp]{gif} & \texttt{\textbackslash mSymbol\{gif\}} & \texttt{E908}\\
\mSymbol[outlined]{gif-2} & \mSymbol[rounded]{gif-2} & \mSymbol[sharp]{gif-2} & \texttt{\textbackslash mSymbol\{gif-2\}} & \texttt{F40E}\\
\mSymbol[outlined]{gif-box} & \mSymbol[rounded]{gif-box} & \mSymbol[sharp]{gif-box} & \texttt{\textbackslash mSymbol\{gif-box\}} & \texttt{E7A3}\\
\mSymbol[outlined]{girl} & \mSymbol[rounded]{girl} & \mSymbol[sharp]{girl} & \texttt{\textbackslash mSymbol\{girl\}} & \texttt{EB68}\\
\mSymbol[outlined]{gite} & \mSymbol[rounded]{gite} & \mSymbol[sharp]{gite} & \texttt{\textbackslash mSymbol\{gite\}} & \texttt{E58B}\\
\mSymbol[outlined]{glass-cup} & \mSymbol[rounded]{glass-cup} & \mSymbol[sharp]{glass-cup} & \texttt{\textbackslash mSymbol\{glass-cup\}} & \texttt{F6E3}\\
\mSymbol[outlined]{globe} & \mSymbol[rounded]{globe} & \mSymbol[sharp]{globe} & \texttt{\textbackslash mSymbol\{globe\}} & \texttt{E64C}\\
\mSymbol[outlined]{globe-asia} & \mSymbol[rounded]{globe-asia} & \mSymbol[sharp]{globe-asia} & \texttt{\textbackslash mSymbol\{globe-asia\}} & \texttt{F799}\\
\mSymbol[outlined]{globe-uk} & \mSymbol[rounded]{globe-uk} & \mSymbol[sharp]{globe-uk} & \texttt{\textbackslash mSymbol\{globe-uk\}} & \texttt{F798}\\
\mSymbol[outlined]{glucose} & \mSymbol[rounded]{glucose} & \mSymbol[sharp]{glucose} & \texttt{\textbackslash mSymbol\{glucose\}} & \texttt{E4A0}\\
\mSymbol[outlined]{glyphs} & \mSymbol[rounded]{glyphs} & \mSymbol[sharp]{glyphs} & \texttt{\textbackslash mSymbol\{glyphs\}} & \texttt{F8A3}\\
\mSymbol[outlined]{go-to-line} & \mSymbol[rounded]{go-to-line} & \mSymbol[sharp]{go-to-line} & \texttt{\textbackslash mSymbol\{go-to-line\}} & \texttt{F71D}\\
\mSymbol[outlined]{golf-course} & \mSymbol[rounded]{golf-course} & \mSymbol[sharp]{golf-course} & \texttt{\textbackslash mSymbol\{golf-course\}} & \texttt{EB45}\\
\mSymbol[outlined]{gondola-lift} & \mSymbol[rounded]{gondola-lift} & \mSymbol[sharp]{gondola-lift} & \texttt{\textbackslash mSymbol\{gondola-lift\}} & \texttt{F476}\\
\mSymbol[outlined]{google-home-devices} & \mSymbol[rounded]{google-home-devices} & \mSymbol[sharp]{google-home-devices} & \texttt{\textbackslash mSymbol\{google-home-devices\}} & \texttt{E715}\\
\mSymbol[outlined]{google-plus-reshare} & \mSymbol[rounded]{google-plus-reshare} & \mSymbol[sharp]{google-plus-reshare} & \texttt{\textbackslash mSymbol\{google-plus-reshare\}} & \texttt{F57A}\\
\mSymbol[outlined]{google-tv-remote} & \mSymbol[rounded]{google-tv-remote} & \mSymbol[sharp]{google-tv-remote} & \texttt{\textbackslash mSymbol\{google-tv-remote\}} & \texttt{F5DB}\\
\mSymbol[outlined]{google-wifi} & \mSymbol[rounded]{google-wifi} & \mSymbol[sharp]{google-wifi} & \texttt{\textbackslash mSymbol\{google-wifi\}} & \texttt{F579}\\
\mSymbol[outlined]{gpp-bad} & \mSymbol[rounded]{gpp-bad} & \mSymbol[sharp]{gpp-bad} & \texttt{\textbackslash mSymbol\{gpp-bad\}} & \texttt{F012}\\
\mSymbol[outlined]{gpp-good} & \mSymbol[rounded]{gpp-good} & \mSymbol[sharp]{gpp-good} & \texttt{\textbackslash mSymbol\{gpp-good\}} & \texttt{F013}\\
\mSymbol[outlined]{gpp-maybe} & \mSymbol[rounded]{gpp-maybe} & \mSymbol[sharp]{gpp-maybe} & \texttt{\textbackslash mSymbol\{gpp-maybe\}} & \texttt{F014}\\
\mSymbol[outlined]{gps-fixed} & \mSymbol[rounded]{gps-fixed} & \mSymbol[sharp]{gps-fixed} & \texttt{\textbackslash mSymbol\{gps-fixed\}} & \texttt{E55C}\\
\mSymbol[outlined]{gps-not-fixed} & \mSymbol[rounded]{gps-not-fixed} & \mSymbol[sharp]{gps-not-fixed} & \texttt{\textbackslash mSymbol\{gps-not-fixed\}} & \texttt{E1B7}\\
\mSymbol[outlined]{gps-off} & \mSymbol[rounded]{gps-off} & \mSymbol[sharp]{gps-off} & \texttt{\textbackslash mSymbol\{gps-off\}} & \texttt{E1B6}\\
\mSymbol[outlined]{grade} & \mSymbol[rounded]{grade} & \mSymbol[sharp]{grade} & \texttt{\textbackslash mSymbol\{grade\}} & \texttt{E885}\\
\mSymbol[outlined]{gradient} & \mSymbol[rounded]{gradient} & \mSymbol[sharp]{gradient} & \texttt{\textbackslash mSymbol\{gradient\}} & \texttt{E3E9}\\
\mSymbol[outlined]{grading} & \mSymbol[rounded]{grading} & \mSymbol[sharp]{grading} & \texttt{\textbackslash mSymbol\{grading\}} & \texttt{EA4F}\\
\mSymbol[outlined]{grain} & \mSymbol[rounded]{grain} & \mSymbol[sharp]{grain} & \texttt{\textbackslash mSymbol\{grain\}} & \texttt{E3EA}\\
\mSymbol[outlined]{graphic-eq} & \mSymbol[rounded]{graphic-eq} & \mSymbol[sharp]{graphic-eq} & \texttt{\textbackslash mSymbol\{graphic-eq\}} & \texttt{E1B8}\\
\mSymbol[outlined]{grass} & \mSymbol[rounded]{grass} & \mSymbol[sharp]{grass} & \texttt{\textbackslash mSymbol\{grass\}} & \texttt{F205}\\
\mSymbol[outlined]{grid-3x3} & \mSymbol[rounded]{grid-3x3} & \mSymbol[sharp]{grid-3x3} & \texttt{\textbackslash mSymbol\{grid-3x3\}} & \texttt{F015}\\
\mSymbol[outlined]{grid-3x3-off} & \mSymbol[rounded]{grid-3x3-off} & \mSymbol[sharp]{grid-3x3-off} & \texttt{\textbackslash mSymbol\{grid-3x3-off\}} & \texttt{F67C}\\
\mSymbol[outlined]{grid-4x4} & \mSymbol[rounded]{grid-4x4} & \mSymbol[sharp]{grid-4x4} & \texttt{\textbackslash mSymbol\{grid-4x4\}} & \texttt{F016}\\
\mSymbol[outlined]{grid-goldenratio} & \mSymbol[rounded]{grid-goldenratio} & \mSymbol[sharp]{grid-goldenratio} & \texttt{\textbackslash mSymbol\{grid-goldenratio\}} & \texttt{F017}\\
\mSymbol[outlined]{grid-guides} & \mSymbol[rounded]{grid-guides} & \mSymbol[sharp]{grid-guides} & \texttt{\textbackslash mSymbol\{grid-guides\}} & \texttt{F76F}\\
\mSymbol[outlined]{grid-off} & \mSymbol[rounded]{grid-off} & \mSymbol[sharp]{grid-off} & \texttt{\textbackslash mSymbol\{grid-off\}} & \texttt{E3EB}\\
\mSymbol[outlined]{grid-on} & \mSymbol[rounded]{grid-on} & \mSymbol[sharp]{grid-on} & \texttt{\textbackslash mSymbol\{grid-on\}} & \texttt{E3EC}\\
\mSymbol[outlined]{grid-view} & \mSymbol[rounded]{grid-view} & \mSymbol[sharp]{grid-view} & \texttt{\textbackslash mSymbol\{grid-view\}} & \texttt{E9B0}\\
\mSymbol[outlined]{grocery} & \mSymbol[rounded]{grocery} & \mSymbol[sharp]{grocery} & \texttt{\textbackslash mSymbol\{grocery\}} & \texttt{EF97}\\
\mSymbol[outlined]{group} & \mSymbol[rounded]{group} & \mSymbol[sharp]{group} & \texttt{\textbackslash mSymbol\{group\}} & \texttt{EA21}\\
\mSymbol[outlined]{group-add} & \mSymbol[rounded]{group-add} & \mSymbol[sharp]{group-add} & \texttt{\textbackslash mSymbol\{group-add\}} & \texttt{E7F0}\\
\mSymbol[outlined]{group-off} & \mSymbol[rounded]{group-off} & \mSymbol[sharp]{group-off} & \texttt{\textbackslash mSymbol\{group-off\}} & \texttt{E747}\\
\mSymbol[outlined]{group-remove} & \mSymbol[rounded]{group-remove} & \mSymbol[sharp]{group-remove} & \texttt{\textbackslash mSymbol\{group-remove\}} & \texttt{E7AD}\\
\mSymbol[outlined]{group-work} & \mSymbol[rounded]{group-work} & \mSymbol[sharp]{group-work} & \texttt{\textbackslash mSymbol\{group-work\}} & \texttt{E886}\\
\mSymbol[outlined]{grouped-bar-chart} & \mSymbol[rounded]{grouped-bar-chart} & \mSymbol[sharp]{grouped-bar-chart} & \texttt{\textbackslash mSymbol\{grouped-bar-chart\}} & \texttt{F211}\\
\mSymbol[outlined]{groups} & \mSymbol[rounded]{groups} & \mSymbol[sharp]{groups} & \texttt{\textbackslash mSymbol\{groups\}} & \texttt{F233}\\
\mSymbol[outlined]{groups-2} & \mSymbol[rounded]{groups-2} & \mSymbol[sharp]{groups-2} & \texttt{\textbackslash mSymbol\{groups-2\}} & \texttt{F8DF}\\
\mSymbol[outlined]{groups-3} & \mSymbol[rounded]{groups-3} & \mSymbol[sharp]{groups-3} & \texttt{\textbackslash mSymbol\{groups-3\}} & \texttt{F8E0}\\
\mSymbol[outlined]{guardian} & \mSymbol[rounded]{guardian} & \mSymbol[sharp]{guardian} & \texttt{\textbackslash mSymbol\{guardian\}} & \texttt{F4C1}\\
\mSymbol[outlined]{gynecology} & \mSymbol[rounded]{gynecology} & \mSymbol[sharp]{gynecology} & \texttt{\textbackslash mSymbol\{gynecology\}} & \texttt{E0F4}\\
\mSymbol[outlined]{h-mobiledata} & \mSymbol[rounded]{h-mobiledata} & \mSymbol[sharp]{h-mobiledata} & \texttt{\textbackslash mSymbol\{h-mobiledata\}} & \texttt{F018}\\
\mSymbol[outlined]{h-mobiledata-badge} & \mSymbol[rounded]{h-mobiledata-badge} & \mSymbol[sharp]{h-mobiledata-badge} & \texttt{\textbackslash mSymbol\{h-mobiledata-badge\}} & \texttt{F7E0}\\
\mSymbol[outlined]{h-plus-mobiledata} & \mSymbol[rounded]{h-plus-mobiledata} & \mSymbol[sharp]{h-plus-mobiledata} & \texttt{\textbackslash mSymbol\{h-plus-mobiledata\}} & \texttt{F019}\\
\mSymbol[outlined]{h-plus-mobiledata-badge} & \mSymbol[rounded]{h-plus-mobiledata-badge} & \mSymbol[sharp]{h-plus-mobiledata-badge} & \texttt{\textbackslash mSymbol\{h-plus-mobiledata-badge\}} & \texttt{F7DF}\\
\mSymbol[outlined]{hail} & \mSymbol[rounded]{hail} & \mSymbol[sharp]{hail} & \texttt{\textbackslash mSymbol\{hail\}} & \texttt{E9B1}\\
\mSymbol[outlined]{hallway} & \mSymbol[rounded]{hallway} & \mSymbol[sharp]{hallway} & \texttt{\textbackslash mSymbol\{hallway\}} & \texttt{E6F8}\\
\mSymbol[outlined]{hand-bones} & \mSymbol[rounded]{hand-bones} & \mSymbol[sharp]{hand-bones} & \texttt{\textbackslash mSymbol\{hand-bones\}} & \texttt{F894}\\
\mSymbol[outlined]{hand-gesture} & \mSymbol[rounded]{hand-gesture} & \mSymbol[sharp]{hand-gesture} & \texttt{\textbackslash mSymbol\{hand-gesture\}} & \texttt{EF9C}\\
\mSymbol[outlined]{handheld-controller} & \mSymbol[rounded]{handheld-controller} & \mSymbol[sharp]{handheld-controller} & \texttt{\textbackslash mSymbol\{handheld-controller\}} & \texttt{F4C6}\\
\mSymbol[outlined]{handshake} & \mSymbol[rounded]{handshake} & \mSymbol[sharp]{handshake} & \texttt{\textbackslash mSymbol\{handshake\}} & \texttt{EBCB}\\
\mSymbol[outlined]{handwriting-recognition} & \mSymbol[rounded]{handwriting-recognition} & \mSymbol[sharp]{handwriting-recognition} & \texttt{\textbackslash mSymbol\{handwriting-recognition\}} & \texttt{EB02}\\
\mSymbol[outlined]{handyman} & \mSymbol[rounded]{handyman} & \mSymbol[sharp]{handyman} & \texttt{\textbackslash mSymbol\{handyman\}} & \texttt{F10B}\\
\mSymbol[outlined]{hangout-video} & \mSymbol[rounded]{hangout-video} & \mSymbol[sharp]{hangout-video} & \texttt{\textbackslash mSymbol\{hangout-video\}} & \texttt{E0C1}\\
\mSymbol[outlined]{hangout-video-off} & \mSymbol[rounded]{hangout-video-off} & \mSymbol[sharp]{hangout-video-off} & \texttt{\textbackslash mSymbol\{hangout-video-off\}} & \texttt{E0C2}\\
\mSymbol[outlined]{hard-drive} & \mSymbol[rounded]{hard-drive} & \mSymbol[sharp]{hard-drive} & \texttt{\textbackslash mSymbol\{hard-drive\}} & \texttt{F80E}\\
\mSymbol[outlined]{hard-drive-2} & \mSymbol[rounded]{hard-drive-2} & \mSymbol[sharp]{hard-drive-2} & \texttt{\textbackslash mSymbol\{hard-drive-2\}} & \texttt{F7A4}\\
\mSymbol[outlined]{hardware} & \mSymbol[rounded]{hardware} & \mSymbol[sharp]{hardware} & \texttt{\textbackslash mSymbol\{hardware\}} & \texttt{EA59}\\
\mSymbol[outlined]{hd} & \mSymbol[rounded]{hd} & \mSymbol[sharp]{hd} & \texttt{\textbackslash mSymbol\{hd\}} & \texttt{E052}\\
\mSymbol[outlined]{hdr-auto} & \mSymbol[rounded]{hdr-auto} & \mSymbol[sharp]{hdr-auto} & \texttt{\textbackslash mSymbol\{hdr-auto\}} & \texttt{F01A}\\
\mSymbol[outlined]{hdr-auto-select} & \mSymbol[rounded]{hdr-auto-select} & \mSymbol[sharp]{hdr-auto-select} & \texttt{\textbackslash mSymbol\{hdr-auto-select\}} & \texttt{F01B}\\
\mSymbol[outlined]{hdr-enhanced-select} & \mSymbol[rounded]{hdr-enhanced-select} & \mSymbol[sharp]{hdr-enhanced-select} & \texttt{\textbackslash mSymbol\{hdr-enhanced-select\}} & \texttt{EF51}\\
\mSymbol[outlined]{hdr-off} & \mSymbol[rounded]{hdr-off} & \mSymbol[sharp]{hdr-off} & \texttt{\textbackslash mSymbol\{hdr-off\}} & \texttt{E3ED}\\
\mSymbol[outlined]{hdr-off-select} & \mSymbol[rounded]{hdr-off-select} & \mSymbol[sharp]{hdr-off-select} & \texttt{\textbackslash mSymbol\{hdr-off-select\}} & \texttt{F01C}\\
\mSymbol[outlined]{hdr-on} & \mSymbol[rounded]{hdr-on} & \mSymbol[sharp]{hdr-on} & \texttt{\textbackslash mSymbol\{hdr-on\}} & \texttt{E3EE}\\
\mSymbol[outlined]{hdr-on-select} & \mSymbol[rounded]{hdr-on-select} & \mSymbol[sharp]{hdr-on-select} & \texttt{\textbackslash mSymbol\{hdr-on-select\}} & \texttt{F01D}\\
\mSymbol[outlined]{hdr-plus} & \mSymbol[rounded]{hdr-plus} & \mSymbol[sharp]{hdr-plus} & \texttt{\textbackslash mSymbol\{hdr-plus\}} & \texttt{F01E}\\
\mSymbol[outlined]{hdr-plus-off} & \mSymbol[rounded]{hdr-plus-off} & \mSymbol[sharp]{hdr-plus-off} & \texttt{\textbackslash mSymbol\{hdr-plus-off\}} & \texttt{E3EF}\\
\mSymbol[outlined]{hdr-strong} & \mSymbol[rounded]{hdr-strong} & \mSymbol[sharp]{hdr-strong} & \texttt{\textbackslash mSymbol\{hdr-strong\}} & \texttt{E3F1}\\
\mSymbol[outlined]{hdr-weak} & \mSymbol[rounded]{hdr-weak} & \mSymbol[sharp]{hdr-weak} & \texttt{\textbackslash mSymbol\{hdr-weak\}} & \texttt{E3F2}\\
\mSymbol[outlined]{head-mounted-device} & \mSymbol[rounded]{head-mounted-device} & \mSymbol[sharp]{head-mounted-device} & \texttt{\textbackslash mSymbol\{head-mounted-device\}} & \texttt{F4C5}\\
\mSymbol[outlined]{headphones} & \mSymbol[rounded]{headphones} & \mSymbol[sharp]{headphones} & \texttt{\textbackslash mSymbol\{headphones\}} & \texttt{F01F}\\
\mSymbol[outlined]{headphones-battery} & \mSymbol[rounded]{headphones-battery} & \mSymbol[sharp]{headphones-battery} & \texttt{\textbackslash mSymbol\{headphones-battery\}} & \texttt{F020}\\
\mSymbol[outlined]{headset} & \mSymbol[rounded]{headset} & \mSymbol[sharp]{headset} & \texttt{\textbackslash mSymbol\{headset\}} & \texttt{F01F}\\
\mSymbol[outlined]{headset-mic} & \mSymbol[rounded]{headset-mic} & \mSymbol[sharp]{headset-mic} & \texttt{\textbackslash mSymbol\{headset-mic\}} & \texttt{E311}\\
\mSymbol[outlined]{headset-off} & \mSymbol[rounded]{headset-off} & \mSymbol[sharp]{headset-off} & \texttt{\textbackslash mSymbol\{headset-off\}} & \texttt{E33A}\\
\mSymbol[outlined]{healing} & \mSymbol[rounded]{healing} & \mSymbol[sharp]{healing} & \texttt{\textbackslash mSymbol\{healing\}} & \texttt{E3F3}\\
\mSymbol[outlined]{health-and-beauty} & \mSymbol[rounded]{health-and-beauty} & \mSymbol[sharp]{health-and-beauty} & \texttt{\textbackslash mSymbol\{health-and-beauty\}} & \texttt{EF9D}\\
\mSymbol[outlined]{health-and-safety} & \mSymbol[rounded]{health-and-safety} & \mSymbol[sharp]{health-and-safety} & \texttt{\textbackslash mSymbol\{health-and-safety\}} & \texttt{E1D5}\\
\mSymbol[outlined]{health-metrics} & \mSymbol[rounded]{health-metrics} & \mSymbol[sharp]{health-metrics} & \texttt{\textbackslash mSymbol\{health-metrics\}} & \texttt{F6E2}\\
\mSymbol[outlined]{heap-snapshot-large} & \mSymbol[rounded]{heap-snapshot-large} & \mSymbol[sharp]{heap-snapshot-large} & \texttt{\textbackslash mSymbol\{heap-snapshot-large\}} & \texttt{F76E}\\
\mSymbol[outlined]{heap-snapshot-multiple} & \mSymbol[rounded]{heap-snapshot-multiple} & \mSymbol[sharp]{heap-snapshot-multiple} & \texttt{\textbackslash mSymbol\{heap-snapshot-multiple\}} & \texttt{F76D}\\
\mSymbol[outlined]{heap-snapshot-thumbnail} & \mSymbol[rounded]{heap-snapshot-thumbnail} & \mSymbol[sharp]{heap-snapshot-thumbnail} & \texttt{\textbackslash mSymbol\{heap-snapshot-thumbnail\}} & \texttt{F76C}\\
\mSymbol[outlined]{hearing} & \mSymbol[rounded]{hearing} & \mSymbol[sharp]{hearing} & \texttt{\textbackslash mSymbol\{hearing\}} & \texttt{E023}\\
\mSymbol[outlined]{hearing-aid} & \mSymbol[rounded]{hearing-aid} & \mSymbol[sharp]{hearing-aid} & \texttt{\textbackslash mSymbol\{hearing-aid\}} & \texttt{F464}\\
\mSymbol[outlined]{hearing-disabled} & \mSymbol[rounded]{hearing-disabled} & \mSymbol[sharp]{hearing-disabled} & \texttt{\textbackslash mSymbol\{hearing-disabled\}} & \texttt{F104}\\
\mSymbol[outlined]{heart-broken} & \mSymbol[rounded]{heart-broken} & \mSymbol[sharp]{heart-broken} & \texttt{\textbackslash mSymbol\{heart-broken\}} & \texttt{EAC2}\\
\mSymbol[outlined]{heart-check} & \mSymbol[rounded]{heart-check} & \mSymbol[sharp]{heart-check} & \texttt{\textbackslash mSymbol\{heart-check\}} & \texttt{F60A}\\
\mSymbol[outlined]{heart-minus} & \mSymbol[rounded]{heart-minus} & \mSymbol[sharp]{heart-minus} & \texttt{\textbackslash mSymbol\{heart-minus\}} & \texttt{F883}\\
\mSymbol[outlined]{heart-plus} & \mSymbol[rounded]{heart-plus} & \mSymbol[sharp]{heart-plus} & \texttt{\textbackslash mSymbol\{heart-plus\}} & \texttt{F884}\\
\mSymbol[outlined]{heat} & \mSymbol[rounded]{heat} & \mSymbol[sharp]{heat} & \texttt{\textbackslash mSymbol\{heat\}} & \texttt{F537}\\
\mSymbol[outlined]{heat-pump} & \mSymbol[rounded]{heat-pump} & \mSymbol[sharp]{heat-pump} & \texttt{\textbackslash mSymbol\{heat-pump\}} & \texttt{EC18}\\
\mSymbol[outlined]{heat-pump-balance} & \mSymbol[rounded]{heat-pump-balance} & \mSymbol[sharp]{heat-pump-balance} & \texttt{\textbackslash mSymbol\{heat-pump-balance\}} & \texttt{E27E}\\
\mSymbol[outlined]{height} & \mSymbol[rounded]{height} & \mSymbol[sharp]{height} & \texttt{\textbackslash mSymbol\{height\}} & \texttt{EA16}\\
\mSymbol[outlined]{helicopter} & \mSymbol[rounded]{helicopter} & \mSymbol[sharp]{helicopter} & \texttt{\textbackslash mSymbol\{helicopter\}} & \texttt{F60C}\\
\mSymbol[outlined]{help} & \mSymbol[rounded]{help} & \mSymbol[sharp]{help} & \texttt{\textbackslash mSymbol\{help\}} & \texttt{E8FD}\\
\mSymbol[outlined]{help-center} & \mSymbol[rounded]{help-center} & \mSymbol[sharp]{help-center} & \texttt{\textbackslash mSymbol\{help-center\}} & \texttt{F1C0}\\
\mSymbol[outlined]{help-clinic} & \mSymbol[rounded]{help-clinic} & \mSymbol[sharp]{help-clinic} & \texttt{\textbackslash mSymbol\{help-clinic\}} & \texttt{F810}\\
\mSymbol[outlined]{help-outline} & \mSymbol[rounded]{help-outline} & \mSymbol[sharp]{help-outline} & \texttt{\textbackslash mSymbol\{help-outline\}} & \texttt{E8FD}\\
\mSymbol[outlined]{hematology} & \mSymbol[rounded]{hematology} & \mSymbol[sharp]{hematology} & \texttt{\textbackslash mSymbol\{hematology\}} & \texttt{E0F6}\\
\mSymbol[outlined]{hevc} & \mSymbol[rounded]{hevc} & \mSymbol[sharp]{hevc} & \texttt{\textbackslash mSymbol\{hevc\}} & \texttt{F021}\\
\mSymbol[outlined]{hexagon} & \mSymbol[rounded]{hexagon} & \mSymbol[sharp]{hexagon} & \texttt{\textbackslash mSymbol\{hexagon\}} & \texttt{EB39}\\
\mSymbol[outlined]{hide} & \mSymbol[rounded]{hide} & \mSymbol[sharp]{hide} & \texttt{\textbackslash mSymbol\{hide\}} & \texttt{EF9E}\\
\mSymbol[outlined]{hide-image} & \mSymbol[rounded]{hide-image} & \mSymbol[sharp]{hide-image} & \texttt{\textbackslash mSymbol\{hide-image\}} & \texttt{F022}\\
\mSymbol[outlined]{hide-source} & \mSymbol[rounded]{hide-source} & \mSymbol[sharp]{hide-source} & \texttt{\textbackslash mSymbol\{hide-source\}} & \texttt{F023}\\
\mSymbol[outlined]{high-density} & \mSymbol[rounded]{high-density} & \mSymbol[sharp]{high-density} & \texttt{\textbackslash mSymbol\{high-density\}} & \texttt{F79C}\\
\mSymbol[outlined]{high-quality} & \mSymbol[rounded]{high-quality} & \mSymbol[sharp]{high-quality} & \texttt{\textbackslash mSymbol\{high-quality\}} & \texttt{E024}\\
\mSymbol[outlined]{high-res} & \mSymbol[rounded]{high-res} & \mSymbol[sharp]{high-res} & \texttt{\textbackslash mSymbol\{high-res\}} & \texttt{F54B}\\
\mSymbol[outlined]{highlight} & \mSymbol[rounded]{highlight} & \mSymbol[sharp]{highlight} & \texttt{\textbackslash mSymbol\{highlight\}} & \texttt{E25F}\\
\mSymbol[outlined]{highlight-keyboard-focus} & \mSymbol[rounded]{highlight-keyboard-focus} & \mSymbol[sharp]{highlight-keyboard-focus} & \texttt{\textbackslash mSymbol\{highlight-keyboard-focus\}} & \texttt{F510}\\
\mSymbol[outlined]{highlight-mouse-cursor} & \mSymbol[rounded]{highlight-mouse-cursor} & \mSymbol[sharp]{highlight-mouse-cursor} & \texttt{\textbackslash mSymbol\{highlight-mouse-cursor\}} & \texttt{F511}\\
\mSymbol[outlined]{highlight-off} & \mSymbol[rounded]{highlight-off} & \mSymbol[sharp]{highlight-off} & \texttt{\textbackslash mSymbol\{highlight-off\}} & \texttt{E888}\\
\mSymbol[outlined]{highlight-text-cursor} & \mSymbol[rounded]{highlight-text-cursor} & \mSymbol[sharp]{highlight-text-cursor} & \texttt{\textbackslash mSymbol\{highlight-text-cursor\}} & \texttt{F512}\\
\mSymbol[outlined]{highlighter-size-1} & \mSymbol[rounded]{highlighter-size-1} & \mSymbol[sharp]{highlighter-size-1} & \texttt{\textbackslash mSymbol\{highlighter-size-1\}} & \texttt{F76B}\\
\mSymbol[outlined]{highlighter-size-2} & \mSymbol[rounded]{highlighter-size-2} & \mSymbol[sharp]{highlighter-size-2} & \texttt{\textbackslash mSymbol\{highlighter-size-2\}} & \texttt{F76A}\\
\mSymbol[outlined]{highlighter-size-3} & \mSymbol[rounded]{highlighter-size-3} & \mSymbol[sharp]{highlighter-size-3} & \texttt{\textbackslash mSymbol\{highlighter-size-3\}} & \texttt{F769}\\
\mSymbol[outlined]{highlighter-size-4} & \mSymbol[rounded]{highlighter-size-4} & \mSymbol[sharp]{highlighter-size-4} & \texttt{\textbackslash mSymbol\{highlighter-size-4\}} & \texttt{F768}\\
\mSymbol[outlined]{highlighter-size-5} & \mSymbol[rounded]{highlighter-size-5} & \mSymbol[sharp]{highlighter-size-5} & \texttt{\textbackslash mSymbol\{highlighter-size-5\}} & \texttt{F767}\\
\mSymbol[outlined]{hiking} & \mSymbol[rounded]{hiking} & \mSymbol[sharp]{hiking} & \texttt{\textbackslash mSymbol\{hiking\}} & \texttt{E50A}\\
\mSymbol[outlined]{history} & \mSymbol[rounded]{history} & \mSymbol[sharp]{history} & \texttt{\textbackslash mSymbol\{history\}} & \texttt{E8B3}\\
\mSymbol[outlined]{history-edu} & \mSymbol[rounded]{history-edu} & \mSymbol[sharp]{history-edu} & \texttt{\textbackslash mSymbol\{history-edu\}} & \texttt{EA3E}\\
\mSymbol[outlined]{history-off} & \mSymbol[rounded]{history-off} & \mSymbol[sharp]{history-off} & \texttt{\textbackslash mSymbol\{history-off\}} & \texttt{F4DA}\\
\mSymbol[outlined]{history-toggle-off} & \mSymbol[rounded]{history-toggle-off} & \mSymbol[sharp]{history-toggle-off} & \texttt{\textbackslash mSymbol\{history-toggle-off\}} & \texttt{F17D}\\
\mSymbol[outlined]{hive} & \mSymbol[rounded]{hive} & \mSymbol[sharp]{hive} & \texttt{\textbackslash mSymbol\{hive\}} & \texttt{EAA6}\\
\mSymbol[outlined]{hls} & \mSymbol[rounded]{hls} & \mSymbol[sharp]{hls} & \texttt{\textbackslash mSymbol\{hls\}} & \texttt{EB8A}\\
\mSymbol[outlined]{hls-off} & \mSymbol[rounded]{hls-off} & \mSymbol[sharp]{hls-off} & \texttt{\textbackslash mSymbol\{hls-off\}} & \texttt{EB8C}\\
\mSymbol[outlined]{holiday-village} & \mSymbol[rounded]{holiday-village} & \mSymbol[sharp]{holiday-village} & \texttt{\textbackslash mSymbol\{holiday-village\}} & \texttt{E58A}\\
\mSymbol[outlined]{home} & \mSymbol[rounded]{home} & \mSymbol[sharp]{home} & \texttt{\textbackslash mSymbol\{home\}} & \texttt{E9B2}\\
\mSymbol[outlined]{home-and-garden} & \mSymbol[rounded]{home-and-garden} & \mSymbol[sharp]{home-and-garden} & \texttt{\textbackslash mSymbol\{home-and-garden\}} & \texttt{EF9F}\\
\mSymbol[outlined]{home-app-logo} & \mSymbol[rounded]{home-app-logo} & \mSymbol[sharp]{home-app-logo} & \texttt{\textbackslash mSymbol\{home-app-logo\}} & \texttt{E295}\\
\mSymbol[outlined]{home-filled} & \mSymbol[rounded]{home-filled} & \mSymbol[sharp]{home-filled} & \texttt{\textbackslash mSymbol\{home-filled\}} & \texttt{E9B2}\\
\mSymbol[outlined]{home-health} & \mSymbol[rounded]{home-health} & \mSymbol[sharp]{home-health} & \texttt{\textbackslash mSymbol\{home-health\}} & \texttt{E4B9}\\
\mSymbol[outlined]{home-improvement-and-tools} & \mSymbol[rounded]{home-improvement-and-tools} & \mSymbol[sharp]{home-improvement-and-tools} & \texttt{\textbackslash mSymbol\{home-improvement-and-tools\}} & \texttt{EFA0}\\
\mSymbol[outlined]{home-iot-device} & \mSymbol[rounded]{home-iot-device} & \mSymbol[sharp]{home-iot-device} & \texttt{\textbackslash mSymbol\{home-iot-device\}} & \texttt{E283}\\
\mSymbol[outlined]{home-max} & \mSymbol[rounded]{home-max} & \mSymbol[sharp]{home-max} & \texttt{\textbackslash mSymbol\{home-max\}} & \texttt{F024}\\
\mSymbol[outlined]{home-max-dots} & \mSymbol[rounded]{home-max-dots} & \mSymbol[sharp]{home-max-dots} & \texttt{\textbackslash mSymbol\{home-max-dots\}} & \texttt{E849}\\
\mSymbol[outlined]{home-mini} & \mSymbol[rounded]{home-mini} & \mSymbol[sharp]{home-mini} & \texttt{\textbackslash mSymbol\{home-mini\}} & \texttt{F025}\\
\mSymbol[outlined]{home-pin} & \mSymbol[rounded]{home-pin} & \mSymbol[sharp]{home-pin} & \texttt{\textbackslash mSymbol\{home-pin\}} & \texttt{F14D}\\
\mSymbol[outlined]{home-repair-service} & \mSymbol[rounded]{home-repair-service} & \mSymbol[sharp]{home-repair-service} & \texttt{\textbackslash mSymbol\{home-repair-service\}} & \texttt{F100}\\
\mSymbol[outlined]{home-speaker} & \mSymbol[rounded]{home-speaker} & \mSymbol[sharp]{home-speaker} & \texttt{\textbackslash mSymbol\{home-speaker\}} & \texttt{F11C}\\
\mSymbol[outlined]{home-storage} & \mSymbol[rounded]{home-storage} & \mSymbol[sharp]{home-storage} & \texttt{\textbackslash mSymbol\{home-storage\}} & \texttt{F86C}\\
\mSymbol[outlined]{home-work} & \mSymbol[rounded]{home-work} & \mSymbol[sharp]{home-work} & \texttt{\textbackslash mSymbol\{home-work\}} & \texttt{F030}\\
\mSymbol[outlined]{horizontal-distribute} & \mSymbol[rounded]{horizontal-distribute} & \mSymbol[sharp]{horizontal-distribute} & \texttt{\textbackslash mSymbol\{horizontal-distribute\}} & \texttt{E014}\\
\mSymbol[outlined]{horizontal-rule} & \mSymbol[rounded]{horizontal-rule} & \mSymbol[sharp]{horizontal-rule} & \texttt{\textbackslash mSymbol\{horizontal-rule\}} & \texttt{F108}\\
\mSymbol[outlined]{horizontal-split} & \mSymbol[rounded]{horizontal-split} & \mSymbol[sharp]{horizontal-split} & \texttt{\textbackslash mSymbol\{horizontal-split\}} & \texttt{E947}\\
\mSymbol[outlined]{hot-tub} & \mSymbol[rounded]{hot-tub} & \mSymbol[sharp]{hot-tub} & \texttt{\textbackslash mSymbol\{hot-tub\}} & \texttt{EB46}\\
\mSymbol[outlined]{hotel} & \mSymbol[rounded]{hotel} & \mSymbol[sharp]{hotel} & \texttt{\textbackslash mSymbol\{hotel\}} & \texttt{E549}\\
\mSymbol[outlined]{hotel-class} & \mSymbol[rounded]{hotel-class} & \mSymbol[sharp]{hotel-class} & \texttt{\textbackslash mSymbol\{hotel-class\}} & \texttt{E743}\\
\mSymbol[outlined]{hourglass} & \mSymbol[rounded]{hourglass} & \mSymbol[sharp]{hourglass} & \texttt{\textbackslash mSymbol\{hourglass\}} & \texttt{EBFF}\\
\mSymbol[outlined]{hourglass-bottom} & \mSymbol[rounded]{hourglass-bottom} & \mSymbol[sharp]{hourglass-bottom} & \texttt{\textbackslash mSymbol\{hourglass-bottom\}} & \texttt{EA5C}\\
\mSymbol[outlined]{hourglass-disabled} & \mSymbol[rounded]{hourglass-disabled} & \mSymbol[sharp]{hourglass-disabled} & \texttt{\textbackslash mSymbol\{hourglass-disabled\}} & \texttt{EF53}\\
\mSymbol[outlined]{hourglass-empty} & \mSymbol[rounded]{hourglass-empty} & \mSymbol[sharp]{hourglass-empty} & \texttt{\textbackslash mSymbol\{hourglass-empty\}} & \texttt{E88B}\\
\mSymbol[outlined]{hourglass-full} & \mSymbol[rounded]{hourglass-full} & \mSymbol[sharp]{hourglass-full} & \texttt{\textbackslash mSymbol\{hourglass-full\}} & \texttt{E88C}\\
\mSymbol[outlined]{hourglass-top} & \mSymbol[rounded]{hourglass-top} & \mSymbol[sharp]{hourglass-top} & \texttt{\textbackslash mSymbol\{hourglass-top\}} & \texttt{EA5B}\\
\mSymbol[outlined]{house} & \mSymbol[rounded]{house} & \mSymbol[sharp]{house} & \texttt{\textbackslash mSymbol\{house\}} & \texttt{EA44}\\
\mSymbol[outlined]{house-siding} & \mSymbol[rounded]{house-siding} & \mSymbol[sharp]{house-siding} & \texttt{\textbackslash mSymbol\{house-siding\}} & \texttt{F202}\\
\mSymbol[outlined]{house-with-shield} & \mSymbol[rounded]{house-with-shield} & \mSymbol[sharp]{house-with-shield} & \texttt{\textbackslash mSymbol\{house-with-shield\}} & \texttt{E786}\\
\mSymbol[outlined]{houseboat} & \mSymbol[rounded]{houseboat} & \mSymbol[sharp]{houseboat} & \texttt{\textbackslash mSymbol\{houseboat\}} & \texttt{E584}\\
\mSymbol[outlined]{household-supplies} & \mSymbol[rounded]{household-supplies} & \mSymbol[sharp]{household-supplies} & \texttt{\textbackslash mSymbol\{household-supplies\}} & \texttt{EFA1}\\
\mSymbol[outlined]{hov} & \mSymbol[rounded]{hov} & \mSymbol[sharp]{hov} & \texttt{\textbackslash mSymbol\{hov\}} & \texttt{F475}\\
\mSymbol[outlined]{how-to-reg} & \mSymbol[rounded]{how-to-reg} & \mSymbol[sharp]{how-to-reg} & \texttt{\textbackslash mSymbol\{how-to-reg\}} & \texttt{E174}\\
\mSymbol[outlined]{how-to-vote} & \mSymbol[rounded]{how-to-vote} & \mSymbol[sharp]{how-to-vote} & \texttt{\textbackslash mSymbol\{how-to-vote\}} & \texttt{E175}\\
\mSymbol[outlined]{hr-resting} & \mSymbol[rounded]{hr-resting} & \mSymbol[sharp]{hr-resting} & \texttt{\textbackslash mSymbol\{hr-resting\}} & \texttt{F6BA}\\
\mSymbol[outlined]{html} & \mSymbol[rounded]{html} & \mSymbol[sharp]{html} & \texttt{\textbackslash mSymbol\{html\}} & \texttt{EB7E}\\
\mSymbol[outlined]{http} & \mSymbol[rounded]{http} & \mSymbol[sharp]{http} & \texttt{\textbackslash mSymbol\{http\}} & \texttt{E902}\\
\mSymbol[outlined]{https} & \mSymbol[rounded]{https} & \mSymbol[sharp]{https} & \texttt{\textbackslash mSymbol\{https\}} & \texttt{E899}\\
\mSymbol[outlined]{hub} & \mSymbol[rounded]{hub} & \mSymbol[sharp]{hub} & \texttt{\textbackslash mSymbol\{hub\}} & \texttt{E9F4}\\
\mSymbol[outlined]{humerus} & \mSymbol[rounded]{humerus} & \mSymbol[sharp]{humerus} & \texttt{\textbackslash mSymbol\{humerus\}} & \texttt{F895}\\
\mSymbol[outlined]{humerus-alt} & \mSymbol[rounded]{humerus-alt} & \mSymbol[sharp]{humerus-alt} & \texttt{\textbackslash mSymbol\{humerus-alt\}} & \texttt{F896}\\
\mSymbol[outlined]{humidity-high} & \mSymbol[rounded]{humidity-high} & \mSymbol[sharp]{humidity-high} & \texttt{\textbackslash mSymbol\{humidity-high\}} & \texttt{F163}\\
\mSymbol[outlined]{humidity-indoor} & \mSymbol[rounded]{humidity-indoor} & \mSymbol[sharp]{humidity-indoor} & \texttt{\textbackslash mSymbol\{humidity-indoor\}} & \texttt{F558}\\
\mSymbol[outlined]{humidity-low} & \mSymbol[rounded]{humidity-low} & \mSymbol[sharp]{humidity-low} & \texttt{\textbackslash mSymbol\{humidity-low\}} & \texttt{F164}\\
\mSymbol[outlined]{humidity-mid} & \mSymbol[rounded]{humidity-mid} & \mSymbol[sharp]{humidity-mid} & \texttt{\textbackslash mSymbol\{humidity-mid\}} & \texttt{F165}\\
\mSymbol[outlined]{humidity-percentage} & \mSymbol[rounded]{humidity-percentage} & \mSymbol[sharp]{humidity-percentage} & \texttt{\textbackslash mSymbol\{humidity-percentage\}} & \texttt{F87E}\\
\mSymbol[outlined]{hvac} & \mSymbol[rounded]{hvac} & \mSymbol[sharp]{hvac} & \texttt{\textbackslash mSymbol\{hvac\}} & \texttt{F10E}\\
\mSymbol[outlined]{ice-skating} & \mSymbol[rounded]{ice-skating} & \mSymbol[sharp]{ice-skating} & \texttt{\textbackslash mSymbol\{ice-skating\}} & \texttt{E50B}\\
\mSymbol[outlined]{icecream} & \mSymbol[rounded]{icecream} & \mSymbol[sharp]{icecream} & \texttt{\textbackslash mSymbol\{icecream\}} & \texttt{EA69}\\
\mSymbol[outlined]{id-card} & \mSymbol[rounded]{id-card} & \mSymbol[sharp]{id-card} & \texttt{\textbackslash mSymbol\{id-card\}} & \texttt{F4CA}\\
\mSymbol[outlined]{ifl} & \mSymbol[rounded]{ifl} & \mSymbol[sharp]{ifl} & \texttt{\textbackslash mSymbol\{ifl\}} & \texttt{E025}\\
\mSymbol[outlined]{iframe} & \mSymbol[rounded]{iframe} & \mSymbol[sharp]{iframe} & \texttt{\textbackslash mSymbol\{iframe\}} & \texttt{F71B}\\
\mSymbol[outlined]{iframe-off} & \mSymbol[rounded]{iframe-off} & \mSymbol[sharp]{iframe-off} & \texttt{\textbackslash mSymbol\{iframe-off\}} & \texttt{F71C}\\
\mSymbol[outlined]{image} & \mSymbol[rounded]{image} & \mSymbol[sharp]{image} & \texttt{\textbackslash mSymbol\{image\}} & \texttt{E3F4}\\
\mSymbol[outlined]{image-aspect-ratio} & \mSymbol[rounded]{image-aspect-ratio} & \mSymbol[sharp]{image-aspect-ratio} & \texttt{\textbackslash mSymbol\{image-aspect-ratio\}} & \texttt{E3F5}\\
\mSymbol[outlined]{image-not-supported} & \mSymbol[rounded]{image-not-supported} & \mSymbol[sharp]{image-not-supported} & \texttt{\textbackslash mSymbol\{image-not-supported\}} & \texttt{F116}\\
\mSymbol[outlined]{image-search} & \mSymbol[rounded]{image-search} & \mSymbol[sharp]{image-search} & \texttt{\textbackslash mSymbol\{image-search\}} & \texttt{E43F}\\
\mSymbol[outlined]{imagesearch-roller} & \mSymbol[rounded]{imagesearch-roller} & \mSymbol[sharp]{imagesearch-roller} & \texttt{\textbackslash mSymbol\{imagesearch-roller\}} & \texttt{E9B4}\\
\mSymbol[outlined]{imagesmode} & \mSymbol[rounded]{imagesmode} & \mSymbol[sharp]{imagesmode} & \texttt{\textbackslash mSymbol\{imagesmode\}} & \texttt{EFA2}\\
\mSymbol[outlined]{immunology} & \mSymbol[rounded]{immunology} & \mSymbol[sharp]{immunology} & \texttt{\textbackslash mSymbol\{immunology\}} & \texttt{E0FB}\\
\mSymbol[outlined]{import-contacts} & \mSymbol[rounded]{import-contacts} & \mSymbol[sharp]{import-contacts} & \texttt{\textbackslash mSymbol\{import-contacts\}} & \texttt{E0E0}\\
\mSymbol[outlined]{import-export} & \mSymbol[rounded]{import-export} & \mSymbol[sharp]{import-export} & \texttt{\textbackslash mSymbol\{import-export\}} & \texttt{E8D5}\\
\mSymbol[outlined]{important-devices} & \mSymbol[rounded]{important-devices} & \mSymbol[sharp]{important-devices} & \texttt{\textbackslash mSymbol\{important-devices\}} & \texttt{E912}\\
\mSymbol[outlined]{in-home-mode} & \mSymbol[rounded]{in-home-mode} & \mSymbol[sharp]{in-home-mode} & \texttt{\textbackslash mSymbol\{in-home-mode\}} & \texttt{E833}\\
\mSymbol[outlined]{inactive-order} & \mSymbol[rounded]{inactive-order} & \mSymbol[sharp]{inactive-order} & \texttt{\textbackslash mSymbol\{inactive-order\}} & \texttt{E0FC}\\
\mSymbol[outlined]{inbox} & \mSymbol[rounded]{inbox} & \mSymbol[sharp]{inbox} & \texttt{\textbackslash mSymbol\{inbox\}} & \texttt{E156}\\
\mSymbol[outlined]{inbox-customize} & \mSymbol[rounded]{inbox-customize} & \mSymbol[sharp]{inbox-customize} & \texttt{\textbackslash mSymbol\{inbox-customize\}} & \texttt{F859}\\
\mSymbol[outlined]{incomplete-circle} & \mSymbol[rounded]{incomplete-circle} & \mSymbol[sharp]{incomplete-circle} & \texttt{\textbackslash mSymbol\{incomplete-circle\}} & \texttt{E79B}\\
\mSymbol[outlined]{indeterminate-check-box} & \mSymbol[rounded]{indeterminate-check-box} & \mSymbol[sharp]{indeterminate-check-box} & \texttt{\textbackslash mSymbol\{indeterminate-check-box\}} & \texttt{E909}\\
\mSymbol[outlined]{indeterminate-question-box} & \mSymbol[rounded]{indeterminate-question-box} & \mSymbol[sharp]{indeterminate-question-box} & \texttt{\textbackslash mSymbol\{indeterminate-question-box\}} & \texttt{F56D}\\
\mSymbol[outlined]{info} & \mSymbol[rounded]{info} & \mSymbol[sharp]{info} & \texttt{\textbackslash mSymbol\{info\}} & \texttt{E88E}\\
\mSymbol[outlined]{info-i} & \mSymbol[rounded]{info-i} & \mSymbol[sharp]{info-i} & \texttt{\textbackslash mSymbol\{info-i\}} & \texttt{F59B}\\
\mSymbol[outlined]{infrared} & \mSymbol[rounded]{infrared} & \mSymbol[sharp]{infrared} & \texttt{\textbackslash mSymbol\{infrared\}} & \texttt{F87C}\\
\mSymbol[outlined]{ink-eraser} & \mSymbol[rounded]{ink-eraser} & \mSymbol[sharp]{ink-eraser} & \texttt{\textbackslash mSymbol\{ink-eraser\}} & \texttt{E6D0}\\
\mSymbol[outlined]{ink-eraser-off} & \mSymbol[rounded]{ink-eraser-off} & \mSymbol[sharp]{ink-eraser-off} & \texttt{\textbackslash mSymbol\{ink-eraser-off\}} & \texttt{E7E3}\\
\mSymbol[outlined]{ink-highlighter} & \mSymbol[rounded]{ink-highlighter} & \mSymbol[sharp]{ink-highlighter} & \texttt{\textbackslash mSymbol\{ink-highlighter\}} & \texttt{E6D1}\\
\mSymbol[outlined]{ink-highlighter-move} & \mSymbol[rounded]{ink-highlighter-move} & \mSymbol[sharp]{ink-highlighter-move} & \texttt{\textbackslash mSymbol\{ink-highlighter-move\}} & \texttt{F524}\\
\mSymbol[outlined]{ink-marker} & \mSymbol[rounded]{ink-marker} & \mSymbol[sharp]{ink-marker} & \texttt{\textbackslash mSymbol\{ink-marker\}} & \texttt{E6D2}\\
\mSymbol[outlined]{ink-pen} & \mSymbol[rounded]{ink-pen} & \mSymbol[sharp]{ink-pen} & \texttt{\textbackslash mSymbol\{ink-pen\}} & \texttt{E6D3}\\
\mSymbol[outlined]{inpatient} & \mSymbol[rounded]{inpatient} & \mSymbol[sharp]{inpatient} & \texttt{\textbackslash mSymbol\{inpatient\}} & \texttt{E0FE}\\
\mSymbol[outlined]{input} & \mSymbol[rounded]{input} & \mSymbol[sharp]{input} & \texttt{\textbackslash mSymbol\{input\}} & \texttt{E890}\\
\mSymbol[outlined]{input-circle} & \mSymbol[rounded]{input-circle} & \mSymbol[sharp]{input-circle} & \texttt{\textbackslash mSymbol\{input-circle\}} & \texttt{F71A}\\
\mSymbol[outlined]{insert-chart} & \mSymbol[rounded]{insert-chart} & \mSymbol[sharp]{insert-chart} & \texttt{\textbackslash mSymbol\{insert-chart\}} & \texttt{F0CC}\\
\mSymbol[outlined]{insert-chart-filled} & \mSymbol[rounded]{insert-chart-filled} & \mSymbol[sharp]{insert-chart-filled} & \texttt{\textbackslash mSymbol\{insert-chart-filled\}} & \texttt{F0CC}\\
\mSymbol[outlined]{insert-chart-outlined} & \mSymbol[rounded]{insert-chart-outlined} & \mSymbol[sharp]{insert-chart-outlined} & \texttt{\textbackslash mSymbol\{insert-chart-outlined\}} & \texttt{F0CC}\\
\mSymbol[outlined]{insert-comment} & \mSymbol[rounded]{insert-comment} & \mSymbol[sharp]{insert-comment} & \texttt{\textbackslash mSymbol\{insert-comment\}} & \texttt{E24C}\\
\mSymbol[outlined]{insert-drive-file} & \mSymbol[rounded]{insert-drive-file} & \mSymbol[sharp]{insert-drive-file} & \texttt{\textbackslash mSymbol\{insert-drive-file\}} & \texttt{E66D}\\
\mSymbol[outlined]{insert-emoticon} & \mSymbol[rounded]{insert-emoticon} & \mSymbol[sharp]{insert-emoticon} & \texttt{\textbackslash mSymbol\{insert-emoticon\}} & \texttt{EA22}\\
\mSymbol[outlined]{insert-invitation} & \mSymbol[rounded]{insert-invitation} & \mSymbol[sharp]{insert-invitation} & \texttt{\textbackslash mSymbol\{insert-invitation\}} & \texttt{E878}\\
\mSymbol[outlined]{insert-link} & \mSymbol[rounded]{insert-link} & \mSymbol[sharp]{insert-link} & \texttt{\textbackslash mSymbol\{insert-link\}} & \texttt{E250}\\
\mSymbol[outlined]{insert-page-break} & \mSymbol[rounded]{insert-page-break} & \mSymbol[sharp]{insert-page-break} & \texttt{\textbackslash mSymbol\{insert-page-break\}} & \texttt{EACA}\\
\mSymbol[outlined]{insert-photo} & \mSymbol[rounded]{insert-photo} & \mSymbol[sharp]{insert-photo} & \texttt{\textbackslash mSymbol\{insert-photo\}} & \texttt{E3F4}\\
\mSymbol[outlined]{insert-text} & \mSymbol[rounded]{insert-text} & \mSymbol[sharp]{insert-text} & \texttt{\textbackslash mSymbol\{insert-text\}} & \texttt{F827}\\
\mSymbol[outlined]{insights} & \mSymbol[rounded]{insights} & \mSymbol[sharp]{insights} & \texttt{\textbackslash mSymbol\{insights\}} & \texttt{F092}\\
\mSymbol[outlined]{install-desktop} & \mSymbol[rounded]{install-desktop} & \mSymbol[sharp]{install-desktop} & \texttt{\textbackslash mSymbol\{install-desktop\}} & \texttt{EB71}\\
\mSymbol[outlined]{install-mobile} & \mSymbol[rounded]{install-mobile} & \mSymbol[sharp]{install-mobile} & \texttt{\textbackslash mSymbol\{install-mobile\}} & \texttt{EB72}\\
\mSymbol[outlined]{instant-mix} & \mSymbol[rounded]{instant-mix} & \mSymbol[sharp]{instant-mix} & \texttt{\textbackslash mSymbol\{instant-mix\}} & \texttt{E026}\\
\mSymbol[outlined]{integration-instructions} & \mSymbol[rounded]{integration-instructions} & \mSymbol[sharp]{integration-instructions} & \texttt{\textbackslash mSymbol\{integration-instructions\}} & \texttt{EF54}\\
\mSymbol[outlined]{interactive-space} & \mSymbol[rounded]{interactive-space} & \mSymbol[sharp]{interactive-space} & \texttt{\textbackslash mSymbol\{interactive-space\}} & \texttt{F7FF}\\
\mSymbol[outlined]{interests} & \mSymbol[rounded]{interests} & \mSymbol[sharp]{interests} & \texttt{\textbackslash mSymbol\{interests\}} & \texttt{E7C8}\\
\mSymbol[outlined]{interpreter-mode} & \mSymbol[rounded]{interpreter-mode} & \mSymbol[sharp]{interpreter-mode} & \texttt{\textbackslash mSymbol\{interpreter-mode\}} & \texttt{E83B}\\
\mSymbol[outlined]{inventory} & \mSymbol[rounded]{inventory} & \mSymbol[sharp]{inventory} & \texttt{\textbackslash mSymbol\{inventory\}} & \texttt{E179}\\
\mSymbol[outlined]{inventory-2} & \mSymbol[rounded]{inventory-2} & \mSymbol[sharp]{inventory-2} & \texttt{\textbackslash mSymbol\{inventory-2\}} & \texttt{E1A1}\\
\mSymbol[outlined]{invert-colors} & \mSymbol[rounded]{invert-colors} & \mSymbol[sharp]{invert-colors} & \texttt{\textbackslash mSymbol\{invert-colors\}} & \texttt{E891}\\
\mSymbol[outlined]{invert-colors-off} & \mSymbol[rounded]{invert-colors-off} & \mSymbol[sharp]{invert-colors-off} & \texttt{\textbackslash mSymbol\{invert-colors-off\}} & \texttt{E0C4}\\
\mSymbol[outlined]{ios} & \mSymbol[rounded]{ios} & \mSymbol[sharp]{ios} & \texttt{\textbackslash mSymbol\{ios\}} & \texttt{E027}\\
\mSymbol[outlined]{ios-share} & \mSymbol[rounded]{ios-share} & \mSymbol[sharp]{ios-share} & \texttt{\textbackslash mSymbol\{ios-share\}} & \texttt{E6B8}\\
\mSymbol[outlined]{iron} & \mSymbol[rounded]{iron} & \mSymbol[sharp]{iron} & \texttt{\textbackslash mSymbol\{iron\}} & \texttt{E583}\\
\mSymbol[outlined]{iso} & \mSymbol[rounded]{iso} & \mSymbol[sharp]{iso} & \texttt{\textbackslash mSymbol\{iso\}} & \texttt{E3F6}\\
\mSymbol[outlined]{jamboard-kiosk} & \mSymbol[rounded]{jamboard-kiosk} & \mSymbol[sharp]{jamboard-kiosk} & \texttt{\textbackslash mSymbol\{jamboard-kiosk\}} & \texttt{E9B5}\\
\mSymbol[outlined]{javascript} & \mSymbol[rounded]{javascript} & \mSymbol[sharp]{javascript} & \texttt{\textbackslash mSymbol\{javascript\}} & \texttt{EB7C}\\
\mSymbol[outlined]{join} & \mSymbol[rounded]{join} & \mSymbol[sharp]{join} & \texttt{\textbackslash mSymbol\{join\}} & \texttt{F84F}\\
\mSymbol[outlined]{join-full} & \mSymbol[rounded]{join-full} & \mSymbol[sharp]{join-full} & \texttt{\textbackslash mSymbol\{join-full\}} & \texttt{F84F}\\
\mSymbol[outlined]{join-inner} & \mSymbol[rounded]{join-inner} & \mSymbol[sharp]{join-inner} & \texttt{\textbackslash mSymbol\{join-inner\}} & \texttt{EAF4}\\
\mSymbol[outlined]{join-left} & \mSymbol[rounded]{join-left} & \mSymbol[sharp]{join-left} & \texttt{\textbackslash mSymbol\{join-left\}} & \texttt{EAF2}\\
\mSymbol[outlined]{join-right} & \mSymbol[rounded]{join-right} & \mSymbol[sharp]{join-right} & \texttt{\textbackslash mSymbol\{join-right\}} & \texttt{EAEA}\\
\mSymbol[outlined]{joystick} & \mSymbol[rounded]{joystick} & \mSymbol[sharp]{joystick} & \texttt{\textbackslash mSymbol\{joystick\}} & \texttt{F5EE}\\
\mSymbol[outlined]{jump-to-element} & \mSymbol[rounded]{jump-to-element} & \mSymbol[sharp]{jump-to-element} & \texttt{\textbackslash mSymbol\{jump-to-element\}} & \texttt{F719}\\
\mSymbol[outlined]{kayaking} & \mSymbol[rounded]{kayaking} & \mSymbol[sharp]{kayaking} & \texttt{\textbackslash mSymbol\{kayaking\}} & \texttt{E50C}\\
\mSymbol[outlined]{kebab-dining} & \mSymbol[rounded]{kebab-dining} & \mSymbol[sharp]{kebab-dining} & \texttt{\textbackslash mSymbol\{kebab-dining\}} & \texttt{E842}\\
\mSymbol[outlined]{keep} & \mSymbol[rounded]{keep} & \mSymbol[sharp]{keep} & \texttt{\textbackslash mSymbol\{keep\}} & \texttt{F026}\\
\mSymbol[outlined]{keep-off} & \mSymbol[rounded]{keep-off} & \mSymbol[sharp]{keep-off} & \texttt{\textbackslash mSymbol\{keep-off\}} & \texttt{E6F9}\\
\mSymbol[outlined]{keep-pin} & \mSymbol[rounded]{keep-pin} & \mSymbol[sharp]{keep-pin} & \texttt{\textbackslash mSymbol\{keep-pin\}} & \texttt{F026}\\
\mSymbol[outlined]{keep-public} & \mSymbol[rounded]{keep-public} & \mSymbol[sharp]{keep-public} & \texttt{\textbackslash mSymbol\{keep-public\}} & \texttt{F56F}\\
\mSymbol[outlined]{kettle} & \mSymbol[rounded]{kettle} & \mSymbol[sharp]{kettle} & \texttt{\textbackslash mSymbol\{kettle\}} & \texttt{E2B9}\\
\mSymbol[outlined]{key} & \mSymbol[rounded]{key} & \mSymbol[sharp]{key} & \texttt{\textbackslash mSymbol\{key\}} & \texttt{E73C}\\
\mSymbol[outlined]{key-off} & \mSymbol[rounded]{key-off} & \mSymbol[sharp]{key-off} & \texttt{\textbackslash mSymbol\{key-off\}} & \texttt{EB84}\\
\mSymbol[outlined]{key-vertical} & \mSymbol[rounded]{key-vertical} & \mSymbol[sharp]{key-vertical} & \texttt{\textbackslash mSymbol\{key-vertical\}} & \texttt{F51A}\\
\mSymbol[outlined]{key-visualizer} & \mSymbol[rounded]{key-visualizer} & \mSymbol[sharp]{key-visualizer} & \texttt{\textbackslash mSymbol\{key-visualizer\}} & \texttt{F199}\\
\mSymbol[outlined]{keyboard} & \mSymbol[rounded]{keyboard} & \mSymbol[sharp]{keyboard} & \texttt{\textbackslash mSymbol\{keyboard\}} & \texttt{E312}\\
\mSymbol[outlined]{keyboard-alt} & \mSymbol[rounded]{keyboard-alt} & \mSymbol[sharp]{keyboard-alt} & \texttt{\textbackslash mSymbol\{keyboard-alt\}} & \texttt{F028}\\
\mSymbol[outlined]{keyboard-arrow-down} & \mSymbol[rounded]{keyboard-arrow-down} & \mSymbol[sharp]{keyboard-arrow-down} & \texttt{\textbackslash mSymbol\{keyboard-arrow-down\}} & \texttt{E313}\\
\mSymbol[outlined]{keyboard-arrow-left} & \mSymbol[rounded]{keyboard-arrow-left} & \mSymbol[sharp]{keyboard-arrow-left} & \texttt{\textbackslash mSymbol\{keyboard-arrow-left\}} & \texttt{E314}\\
\mSymbol[outlined]{keyboard-arrow-right} & \mSymbol[rounded]{keyboard-arrow-right} & \mSymbol[sharp]{keyboard-arrow-right} & \texttt{\textbackslash mSymbol\{keyboard-arrow-right\}} & \texttt{E315}\\
\mSymbol[outlined]{keyboard-arrow-up} & \mSymbol[rounded]{keyboard-arrow-up} & \mSymbol[sharp]{keyboard-arrow-up} & \texttt{\textbackslash mSymbol\{keyboard-arrow-up\}} & \texttt{E316}\\
\mSymbol[outlined]{keyboard-backspace} & \mSymbol[rounded]{keyboard-backspace} & \mSymbol[sharp]{keyboard-backspace} & \texttt{\textbackslash mSymbol\{keyboard-backspace\}} & \texttt{E317}\\
\mSymbol[outlined]{keyboard-capslock} & \mSymbol[rounded]{keyboard-capslock} & \mSymbol[sharp]{keyboard-capslock} & \texttt{\textbackslash mSymbol\{keyboard-capslock\}} & \texttt{E318}\\
\mSymbol[outlined]{keyboard-capslock-badge} & \mSymbol[rounded]{keyboard-capslock-badge} & \mSymbol[sharp]{keyboard-capslock-badge} & \texttt{\textbackslash mSymbol\{keyboard-capslock-badge\}} & \texttt{F7DE}\\
\mSymbol[outlined]{keyboard-command-key} & \mSymbol[rounded]{keyboard-command-key} & \mSymbol[sharp]{keyboard-command-key} & \texttt{\textbackslash mSymbol\{keyboard-command-key\}} & \texttt{EAE7}\\
\mSymbol[outlined]{keyboard-control-key} & \mSymbol[rounded]{keyboard-control-key} & \mSymbol[sharp]{keyboard-control-key} & \texttt{\textbackslash mSymbol\{keyboard-control-key\}} & \texttt{EAE6}\\
\mSymbol[outlined]{keyboard-double-arrow-down} & \mSymbol[rounded]{keyboard-double-arrow-down} & \mSymbol[sharp]{keyboard-double-arrow-down} & \texttt{\textbackslash mSymbol\{keyboard-double-arrow-down\}} & \texttt{EAD0}\\
\mSymbol[outlined]{keyboard-double-arrow-left} & \mSymbol[rounded]{keyboard-double-arrow-left} & \mSymbol[sharp]{keyboard-double-arrow-left} & \texttt{\textbackslash mSymbol\{keyboard-double-arrow-left\}} & \texttt{EAC3}\\
\mSymbol[outlined]{keyboard-double-arrow-right} & \mSymbol[rounded]{keyboard-double-arrow-right} & \mSymbol[sharp]{keyboard-double-arrow-right} & \texttt{\textbackslash mSymbol\{keyboard-double-arrow-right\}} & \texttt{EAC9}\\
\mSymbol[outlined]{keyboard-double-arrow-up} & \mSymbol[rounded]{keyboard-double-arrow-up} & \mSymbol[sharp]{keyboard-double-arrow-up} & \texttt{\textbackslash mSymbol\{keyboard-double-arrow-up\}} & \texttt{EACF}\\
\mSymbol[outlined]{keyboard-external-input} & \mSymbol[rounded]{keyboard-external-input} & \mSymbol[sharp]{keyboard-external-input} & \texttt{\textbackslash mSymbol\{keyboard-external-input\}} & \texttt{F7DD}\\
\mSymbol[outlined]{keyboard-full} & \mSymbol[rounded]{keyboard-full} & \mSymbol[sharp]{keyboard-full} & \texttt{\textbackslash mSymbol\{keyboard-full\}} & \texttt{F7DC}\\
\mSymbol[outlined]{keyboard-hide} & \mSymbol[rounded]{keyboard-hide} & \mSymbol[sharp]{keyboard-hide} & \texttt{\textbackslash mSymbol\{keyboard-hide\}} & \texttt{E31A}\\
\mSymbol[outlined]{keyboard-keys} & \mSymbol[rounded]{keyboard-keys} & \mSymbol[sharp]{keyboard-keys} & \texttt{\textbackslash mSymbol\{keyboard-keys\}} & \texttt{F67B}\\
\mSymbol[outlined]{keyboard-lock} & \mSymbol[rounded]{keyboard-lock} & \mSymbol[sharp]{keyboard-lock} & \texttt{\textbackslash mSymbol\{keyboard-lock\}} & \texttt{F492}\\
\mSymbol[outlined]{keyboard-lock-off} & \mSymbol[rounded]{keyboard-lock-off} & \mSymbol[sharp]{keyboard-lock-off} & \texttt{\textbackslash mSymbol\{keyboard-lock-off\}} & \texttt{F491}\\
\mSymbol[outlined]{keyboard-off} & \mSymbol[rounded]{keyboard-off} & \mSymbol[sharp]{keyboard-off} & \texttt{\textbackslash mSymbol\{keyboard-off\}} & \texttt{F67A}\\
\mSymbol[outlined]{keyboard-onscreen} & \mSymbol[rounded]{keyboard-onscreen} & \mSymbol[sharp]{keyboard-onscreen} & \texttt{\textbackslash mSymbol\{keyboard-onscreen\}} & \texttt{F7DB}\\
\mSymbol[outlined]{keyboard-option-key} & \mSymbol[rounded]{keyboard-option-key} & \mSymbol[sharp]{keyboard-option-key} & \texttt{\textbackslash mSymbol\{keyboard-option-key\}} & \texttt{EAE8}\\
\mSymbol[outlined]{keyboard-previous-language} & \mSymbol[rounded]{keyboard-previous-language} & \mSymbol[sharp]{keyboard-previous-language} & \texttt{\textbackslash mSymbol\{keyboard-previous-language\}} & \texttt{F7DA}\\
\mSymbol[outlined]{keyboard-return} & \mSymbol[rounded]{keyboard-return} & \mSymbol[sharp]{keyboard-return} & \texttt{\textbackslash mSymbol\{keyboard-return\}} & \texttt{E31B}\\
\mSymbol[outlined]{keyboard-tab} & \mSymbol[rounded]{keyboard-tab} & \mSymbol[sharp]{keyboard-tab} & \texttt{\textbackslash mSymbol\{keyboard-tab\}} & \texttt{E31C}\\
\mSymbol[outlined]{keyboard-tab-rtl} & \mSymbol[rounded]{keyboard-tab-rtl} & \mSymbol[sharp]{keyboard-tab-rtl} & \texttt{\textbackslash mSymbol\{keyboard-tab-rtl\}} & \texttt{EC73}\\
\mSymbol[outlined]{keyboard-voice} & \mSymbol[rounded]{keyboard-voice} & \mSymbol[sharp]{keyboard-voice} & \texttt{\textbackslash mSymbol\{keyboard-voice\}} & \texttt{E31D}\\
\mSymbol[outlined]{kid-star} & \mSymbol[rounded]{kid-star} & \mSymbol[sharp]{kid-star} & \texttt{\textbackslash mSymbol\{kid-star\}} & \texttt{F526}\\
\mSymbol[outlined]{king-bed} & \mSymbol[rounded]{king-bed} & \mSymbol[sharp]{king-bed} & \texttt{\textbackslash mSymbol\{king-bed\}} & \texttt{EA45}\\
\mSymbol[outlined]{kitchen} & \mSymbol[rounded]{kitchen} & \mSymbol[sharp]{kitchen} & \texttt{\textbackslash mSymbol\{kitchen\}} & \texttt{EB47}\\
\mSymbol[outlined]{kitesurfing} & \mSymbol[rounded]{kitesurfing} & \mSymbol[sharp]{kitesurfing} & \texttt{\textbackslash mSymbol\{kitesurfing\}} & \texttt{E50D}\\
\mSymbol[outlined]{lab-panel} & \mSymbol[rounded]{lab-panel} & \mSymbol[sharp]{lab-panel} & \texttt{\textbackslash mSymbol\{lab-panel\}} & \texttt{E103}\\
\mSymbol[outlined]{lab-profile} & \mSymbol[rounded]{lab-profile} & \mSymbol[sharp]{lab-profile} & \texttt{\textbackslash mSymbol\{lab-profile\}} & \texttt{E104}\\
\mSymbol[outlined]{lab-research} & \mSymbol[rounded]{lab-research} & \mSymbol[sharp]{lab-research} & \texttt{\textbackslash mSymbol\{lab-research\}} & \texttt{F80B}\\
\mSymbol[outlined]{label} & \mSymbol[rounded]{label} & \mSymbol[sharp]{label} & \texttt{\textbackslash mSymbol\{label\}} & \texttt{E893}\\
\mSymbol[outlined]{label-important} & \mSymbol[rounded]{label-important} & \mSymbol[sharp]{label-important} & \texttt{\textbackslash mSymbol\{label-important\}} & \texttt{E948}\\
\mSymbol[outlined]{label-important-outline} & \mSymbol[rounded]{label-important-outline} & \mSymbol[sharp]{label-important-outline} & \texttt{\textbackslash mSymbol\{label-important-outline\}} & \texttt{E948}\\
\mSymbol[outlined]{label-off} & \mSymbol[rounded]{label-off} & \mSymbol[sharp]{label-off} & \texttt{\textbackslash mSymbol\{label-off\}} & \texttt{E9B6}\\
\mSymbol[outlined]{label-outline} & \mSymbol[rounded]{label-outline} & \mSymbol[sharp]{label-outline} & \texttt{\textbackslash mSymbol\{label-outline\}} & \texttt{E893}\\
\mSymbol[outlined]{labs} & \mSymbol[rounded]{labs} & \mSymbol[sharp]{labs} & \texttt{\textbackslash mSymbol\{labs\}} & \texttt{E105}\\
\mSymbol[outlined]{lan} & \mSymbol[rounded]{lan} & \mSymbol[sharp]{lan} & \texttt{\textbackslash mSymbol\{lan\}} & \texttt{EB2F}\\
\mSymbol[outlined]{landscape} & \mSymbol[rounded]{landscape} & \mSymbol[sharp]{landscape} & \texttt{\textbackslash mSymbol\{landscape\}} & \texttt{E564}\\
\mSymbol[outlined]{landscape-2} & \mSymbol[rounded]{landscape-2} & \mSymbol[sharp]{landscape-2} & \texttt{\textbackslash mSymbol\{landscape-2\}} & \texttt{F4C4}\\
\mSymbol[outlined]{landscape-2-off} & \mSymbol[rounded]{landscape-2-off} & \mSymbol[sharp]{landscape-2-off} & \texttt{\textbackslash mSymbol\{landscape-2-off\}} & \texttt{F4C3}\\
\mSymbol[outlined]{landslide} & \mSymbol[rounded]{landslide} & \mSymbol[sharp]{landslide} & \texttt{\textbackslash mSymbol\{landslide\}} & \texttt{EBD7}\\
\mSymbol[outlined]{language} & \mSymbol[rounded]{language} & \mSymbol[sharp]{language} & \texttt{\textbackslash mSymbol\{language\}} & \texttt{E894}\\
\mSymbol[outlined]{language-chinese-array} & \mSymbol[rounded]{language-chinese-array} & \mSymbol[sharp]{language-chinese-array} & \texttt{\textbackslash mSymbol\{language-chinese-array\}} & \texttt{F766}\\
\mSymbol[outlined]{language-chinese-cangjie} & \mSymbol[rounded]{language-chinese-cangjie} & \mSymbol[sharp]{language-chinese-cangjie} & \texttt{\textbackslash mSymbol\{language-chinese-cangjie\}} & \texttt{F765}\\
\mSymbol[outlined]{language-chinese-dayi} & \mSymbol[rounded]{language-chinese-dayi} & \mSymbol[sharp]{language-chinese-dayi} & \texttt{\textbackslash mSymbol\{language-chinese-dayi\}} & \texttt{F764}\\
\mSymbol[outlined]{language-chinese-pinyin} & \mSymbol[rounded]{language-chinese-pinyin} & \mSymbol[sharp]{language-chinese-pinyin} & \texttt{\textbackslash mSymbol\{language-chinese-pinyin\}} & \texttt{F763}\\
\mSymbol[outlined]{language-chinese-quick} & \mSymbol[rounded]{language-chinese-quick} & \mSymbol[sharp]{language-chinese-quick} & \texttt{\textbackslash mSymbol\{language-chinese-quick\}} & \texttt{F762}\\
\mSymbol[outlined]{language-chinese-wubi} & \mSymbol[rounded]{language-chinese-wubi} & \mSymbol[sharp]{language-chinese-wubi} & \texttt{\textbackslash mSymbol\{language-chinese-wubi\}} & \texttt{F761}\\
\mSymbol[outlined]{language-french} & \mSymbol[rounded]{language-french} & \mSymbol[sharp]{language-french} & \texttt{\textbackslash mSymbol\{language-french\}} & \texttt{F760}\\
\mSymbol[outlined]{language-gb-english} & \mSymbol[rounded]{language-gb-english} & \mSymbol[sharp]{language-gb-english} & \texttt{\textbackslash mSymbol\{language-gb-english\}} & \texttt{F75F}\\
\mSymbol[outlined]{language-international} & \mSymbol[rounded]{language-international} & \mSymbol[sharp]{language-international} & \texttt{\textbackslash mSymbol\{language-international\}} & \texttt{F75E}\\
\mSymbol[outlined]{language-japanese-kana} & \mSymbol[rounded]{language-japanese-kana} & \mSymbol[sharp]{language-japanese-kana} & \texttt{\textbackslash mSymbol\{language-japanese-kana\}} & \texttt{F513}\\
\mSymbol[outlined]{language-korean-latin} & \mSymbol[rounded]{language-korean-latin} & \mSymbol[sharp]{language-korean-latin} & \texttt{\textbackslash mSymbol\{language-korean-latin\}} & \texttt{F75D}\\
\mSymbol[outlined]{language-pinyin} & \mSymbol[rounded]{language-pinyin} & \mSymbol[sharp]{language-pinyin} & \texttt{\textbackslash mSymbol\{language-pinyin\}} & \texttt{F75C}\\
\mSymbol[outlined]{language-spanish} & \mSymbol[rounded]{language-spanish} & \mSymbol[sharp]{language-spanish} & \texttt{\textbackslash mSymbol\{language-spanish\}} & \texttt{F5E9}\\
\mSymbol[outlined]{language-us} & \mSymbol[rounded]{language-us} & \mSymbol[sharp]{language-us} & \texttt{\textbackslash mSymbol\{language-us\}} & \texttt{F759}\\
\mSymbol[outlined]{language-us-colemak} & \mSymbol[rounded]{language-us-colemak} & \mSymbol[sharp]{language-us-colemak} & \texttt{\textbackslash mSymbol\{language-us-colemak\}} & \texttt{F75B}\\
\mSymbol[outlined]{language-us-dvorak} & \mSymbol[rounded]{language-us-dvorak} & \mSymbol[sharp]{language-us-dvorak} & \texttt{\textbackslash mSymbol\{language-us-dvorak\}} & \texttt{F75A}\\
\mSymbol[outlined]{laps} & \mSymbol[rounded]{laps} & \mSymbol[sharp]{laps} & \texttt{\textbackslash mSymbol\{laps\}} & \texttt{F6B9}\\
\mSymbol[outlined]{laptop} & \mSymbol[rounded]{laptop} & \mSymbol[sharp]{laptop} & \texttt{\textbackslash mSymbol\{laptop\}} & \texttt{E31E}\\
\mSymbol[outlined]{laptop-chromebook} & \mSymbol[rounded]{laptop-chromebook} & \mSymbol[sharp]{laptop-chromebook} & \texttt{\textbackslash mSymbol\{laptop-chromebook\}} & \texttt{E31F}\\
\mSymbol[outlined]{laptop-mac} & \mSymbol[rounded]{laptop-mac} & \mSymbol[sharp]{laptop-mac} & \texttt{\textbackslash mSymbol\{laptop-mac\}} & \texttt{E320}\\
\mSymbol[outlined]{laptop-windows} & \mSymbol[rounded]{laptop-windows} & \mSymbol[sharp]{laptop-windows} & \texttt{\textbackslash mSymbol\{laptop-windows\}} & \texttt{E321}\\
\mSymbol[outlined]{lasso-select} & \mSymbol[rounded]{lasso-select} & \mSymbol[sharp]{lasso-select} & \texttt{\textbackslash mSymbol\{lasso-select\}} & \texttt{EB03}\\
\mSymbol[outlined]{last-page} & \mSymbol[rounded]{last-page} & \mSymbol[sharp]{last-page} & \texttt{\textbackslash mSymbol\{last-page\}} & \texttt{E5DD}\\
\mSymbol[outlined]{launch} & \mSymbol[rounded]{launch} & \mSymbol[sharp]{launch} & \texttt{\textbackslash mSymbol\{launch\}} & \texttt{E89E}\\
\mSymbol[outlined]{laundry} & \mSymbol[rounded]{laundry} & \mSymbol[sharp]{laundry} & \texttt{\textbackslash mSymbol\{laundry\}} & \texttt{E2A8}\\
\mSymbol[outlined]{layers} & \mSymbol[rounded]{layers} & \mSymbol[sharp]{layers} & \texttt{\textbackslash mSymbol\{layers\}} & \texttt{E53B}\\
\mSymbol[outlined]{layers-clear} & \mSymbol[rounded]{layers-clear} & \mSymbol[sharp]{layers-clear} & \texttt{\textbackslash mSymbol\{layers-clear\}} & \texttt{E53C}\\
\mSymbol[outlined]{lda} & \mSymbol[rounded]{lda} & \mSymbol[sharp]{lda} & \texttt{\textbackslash mSymbol\{lda\}} & \texttt{E106}\\
\mSymbol[outlined]{leaderboard} & \mSymbol[rounded]{leaderboard} & \mSymbol[sharp]{leaderboard} & \texttt{\textbackslash mSymbol\{leaderboard\}} & \texttt{F20C}\\
\mSymbol[outlined]{leak-add} & \mSymbol[rounded]{leak-add} & \mSymbol[sharp]{leak-add} & \texttt{\textbackslash mSymbol\{leak-add\}} & \texttt{E3F8}\\
\mSymbol[outlined]{leak-remove} & \mSymbol[rounded]{leak-remove} & \mSymbol[sharp]{leak-remove} & \texttt{\textbackslash mSymbol\{leak-remove\}} & \texttt{E3F9}\\
\mSymbol[outlined]{left-click} & \mSymbol[rounded]{left-click} & \mSymbol[sharp]{left-click} & \texttt{\textbackslash mSymbol\{left-click\}} & \texttt{F718}\\
\mSymbol[outlined]{left-panel-close} & \mSymbol[rounded]{left-panel-close} & \mSymbol[sharp]{left-panel-close} & \texttt{\textbackslash mSymbol\{left-panel-close\}} & \texttt{F717}\\
\mSymbol[outlined]{left-panel-open} & \mSymbol[rounded]{left-panel-open} & \mSymbol[sharp]{left-panel-open} & \texttt{\textbackslash mSymbol\{left-panel-open\}} & \texttt{F716}\\
\mSymbol[outlined]{legend-toggle} & \mSymbol[rounded]{legend-toggle} & \mSymbol[sharp]{legend-toggle} & \texttt{\textbackslash mSymbol\{legend-toggle\}} & \texttt{F11B}\\
\mSymbol[outlined]{lens} & \mSymbol[rounded]{lens} & \mSymbol[sharp]{lens} & \texttt{\textbackslash mSymbol\{lens\}} & \texttt{E3FA}\\
\mSymbol[outlined]{lens-blur} & \mSymbol[rounded]{lens-blur} & \mSymbol[sharp]{lens-blur} & \texttt{\textbackslash mSymbol\{lens-blur\}} & \texttt{F029}\\
\mSymbol[outlined]{letter-switch} & \mSymbol[rounded]{letter-switch} & \mSymbol[sharp]{letter-switch} & \texttt{\textbackslash mSymbol\{letter-switch\}} & \texttt{F758}\\
\mSymbol[outlined]{library-add} & \mSymbol[rounded]{library-add} & \mSymbol[sharp]{library-add} & \texttt{\textbackslash mSymbol\{library-add\}} & \texttt{E03C}\\
\mSymbol[outlined]{library-add-check} & \mSymbol[rounded]{library-add-check} & \mSymbol[sharp]{library-add-check} & \texttt{\textbackslash mSymbol\{library-add-check\}} & \texttt{E9B7}\\
\mSymbol[outlined]{library-books} & \mSymbol[rounded]{library-books} & \mSymbol[sharp]{library-books} & \texttt{\textbackslash mSymbol\{library-books\}} & \texttt{E02F}\\
\mSymbol[outlined]{library-music} & \mSymbol[rounded]{library-music} & \mSymbol[sharp]{library-music} & \texttt{\textbackslash mSymbol\{library-music\}} & \texttt{E030}\\
\mSymbol[outlined]{license} & \mSymbol[rounded]{license} & \mSymbol[sharp]{license} & \texttt{\textbackslash mSymbol\{license\}} & \texttt{EB04}\\
\mSymbol[outlined]{lift-to-talk} & \mSymbol[rounded]{lift-to-talk} & \mSymbol[sharp]{lift-to-talk} & \texttt{\textbackslash mSymbol\{lift-to-talk\}} & \texttt{EFA3}\\
\mSymbol[outlined]{light} & \mSymbol[rounded]{light} & \mSymbol[sharp]{light} & \texttt{\textbackslash mSymbol\{light\}} & \texttt{F02A}\\
\mSymbol[outlined]{light-group} & \mSymbol[rounded]{light-group} & \mSymbol[sharp]{light-group} & \texttt{\textbackslash mSymbol\{light-group\}} & \texttt{E28B}\\
\mSymbol[outlined]{light-mode} & \mSymbol[rounded]{light-mode} & \mSymbol[sharp]{light-mode} & \texttt{\textbackslash mSymbol\{light-mode\}} & \texttt{E518}\\
\mSymbol[outlined]{light-off} & \mSymbol[rounded]{light-off} & \mSymbol[sharp]{light-off} & \texttt{\textbackslash mSymbol\{light-off\}} & \texttt{E9B8}\\
\mSymbol[outlined]{lightbulb} & \mSymbol[rounded]{lightbulb} & \mSymbol[sharp]{lightbulb} & \texttt{\textbackslash mSymbol\{lightbulb\}} & \texttt{E90F}\\
\mSymbol[outlined]{lightbulb-circle} & \mSymbol[rounded]{lightbulb-circle} & \mSymbol[sharp]{lightbulb-circle} & \texttt{\textbackslash mSymbol\{lightbulb-circle\}} & \texttt{EBFE}\\
\mSymbol[outlined]{lightbulb-outline} & \mSymbol[rounded]{lightbulb-outline} & \mSymbol[sharp]{lightbulb-outline} & \texttt{\textbackslash mSymbol\{lightbulb-outline\}} & \texttt{E90F}\\
\mSymbol[outlined]{lightning-stand} & \mSymbol[rounded]{lightning-stand} & \mSymbol[sharp]{lightning-stand} & \texttt{\textbackslash mSymbol\{lightning-stand\}} & \texttt{EFA4}\\
\mSymbol[outlined]{line-axis} & \mSymbol[rounded]{line-axis} & \mSymbol[sharp]{line-axis} & \texttt{\textbackslash mSymbol\{line-axis\}} & \texttt{EA9A}\\
\mSymbol[outlined]{line-curve} & \mSymbol[rounded]{line-curve} & \mSymbol[sharp]{line-curve} & \texttt{\textbackslash mSymbol\{line-curve\}} & \texttt{F757}\\
\mSymbol[outlined]{line-end} & \mSymbol[rounded]{line-end} & \mSymbol[sharp]{line-end} & \texttt{\textbackslash mSymbol\{line-end\}} & \texttt{F826}\\
\mSymbol[outlined]{line-end-arrow} & \mSymbol[rounded]{line-end-arrow} & \mSymbol[sharp]{line-end-arrow} & \texttt{\textbackslash mSymbol\{line-end-arrow\}} & \texttt{F81D}\\
\mSymbol[outlined]{line-end-arrow-notch} & \mSymbol[rounded]{line-end-arrow-notch} & \mSymbol[sharp]{line-end-arrow-notch} & \texttt{\textbackslash mSymbol\{line-end-arrow-notch\}} & \texttt{F81C}\\
\mSymbol[outlined]{line-end-circle} & \mSymbol[rounded]{line-end-circle} & \mSymbol[sharp]{line-end-circle} & \texttt{\textbackslash mSymbol\{line-end-circle\}} & \texttt{F81B}\\
\mSymbol[outlined]{line-end-diamond} & \mSymbol[rounded]{line-end-diamond} & \mSymbol[sharp]{line-end-diamond} & \texttt{\textbackslash mSymbol\{line-end-diamond\}} & \texttt{F81A}\\
\mSymbol[outlined]{line-end-square} & \mSymbol[rounded]{line-end-square} & \mSymbol[sharp]{line-end-square} & \texttt{\textbackslash mSymbol\{line-end-square\}} & \texttt{F819}\\
\mSymbol[outlined]{line-start} & \mSymbol[rounded]{line-start} & \mSymbol[sharp]{line-start} & \texttt{\textbackslash mSymbol\{line-start\}} & \texttt{F825}\\
\mSymbol[outlined]{line-start-arrow} & \mSymbol[rounded]{line-start-arrow} & \mSymbol[sharp]{line-start-arrow} & \texttt{\textbackslash mSymbol\{line-start-arrow\}} & \texttt{F818}\\
\mSymbol[outlined]{line-start-arrow-notch} & \mSymbol[rounded]{line-start-arrow-notch} & \mSymbol[sharp]{line-start-arrow-notch} & \texttt{\textbackslash mSymbol\{line-start-arrow-notch\}} & \texttt{F817}\\
\mSymbol[outlined]{line-start-circle} & \mSymbol[rounded]{line-start-circle} & \mSymbol[sharp]{line-start-circle} & \texttt{\textbackslash mSymbol\{line-start-circle\}} & \texttt{F816}\\
\mSymbol[outlined]{line-start-diamond} & \mSymbol[rounded]{line-start-diamond} & \mSymbol[sharp]{line-start-diamond} & \texttt{\textbackslash mSymbol\{line-start-diamond\}} & \texttt{F815}\\
\mSymbol[outlined]{line-start-square} & \mSymbol[rounded]{line-start-square} & \mSymbol[sharp]{line-start-square} & \texttt{\textbackslash mSymbol\{line-start-square\}} & \texttt{F814}\\
\mSymbol[outlined]{line-style} & \mSymbol[rounded]{line-style} & \mSymbol[sharp]{line-style} & \texttt{\textbackslash mSymbol\{line-style\}} & \texttt{E919}\\
\mSymbol[outlined]{line-weight} & \mSymbol[rounded]{line-weight} & \mSymbol[sharp]{line-weight} & \texttt{\textbackslash mSymbol\{line-weight\}} & \texttt{E91A}\\
\mSymbol[outlined]{linear-scale} & \mSymbol[rounded]{linear-scale} & \mSymbol[sharp]{linear-scale} & \texttt{\textbackslash mSymbol\{linear-scale\}} & \texttt{E260}\\
\mSymbol[outlined]{link} & \mSymbol[rounded]{link} & \mSymbol[sharp]{link} & \texttt{\textbackslash mSymbol\{link\}} & \texttt{E250}\\
\mSymbol[outlined]{link-off} & \mSymbol[rounded]{link-off} & \mSymbol[sharp]{link-off} & \texttt{\textbackslash mSymbol\{link-off\}} & \texttt{E16F}\\
\mSymbol[outlined]{linked-camera} & \mSymbol[rounded]{linked-camera} & \mSymbol[sharp]{linked-camera} & \texttt{\textbackslash mSymbol\{linked-camera\}} & \texttt{E438}\\
\mSymbol[outlined]{linked-services} & \mSymbol[rounded]{linked-services} & \mSymbol[sharp]{linked-services} & \texttt{\textbackslash mSymbol\{linked-services\}} & \texttt{F535}\\
\mSymbol[outlined]{liquor} & \mSymbol[rounded]{liquor} & \mSymbol[sharp]{liquor} & \texttt{\textbackslash mSymbol\{liquor\}} & \texttt{EA60}\\
\mSymbol[outlined]{list} & \mSymbol[rounded]{list} & \mSymbol[sharp]{list} & \texttt{\textbackslash mSymbol\{list\}} & \texttt{E896}\\
\mSymbol[outlined]{list-alt} & \mSymbol[rounded]{list-alt} & \mSymbol[sharp]{list-alt} & \texttt{\textbackslash mSymbol\{list-alt\}} & \texttt{E0EE}\\
\mSymbol[outlined]{list-alt-add} & \mSymbol[rounded]{list-alt-add} & \mSymbol[sharp]{list-alt-add} & \texttt{\textbackslash mSymbol\{list-alt-add\}} & \texttt{F756}\\
\mSymbol[outlined]{lists} & \mSymbol[rounded]{lists} & \mSymbol[sharp]{lists} & \texttt{\textbackslash mSymbol\{lists\}} & \texttt{E9B9}\\
\mSymbol[outlined]{live-help} & \mSymbol[rounded]{live-help} & \mSymbol[sharp]{live-help} & \texttt{\textbackslash mSymbol\{live-help\}} & \texttt{E0C6}\\
\mSymbol[outlined]{live-tv} & \mSymbol[rounded]{live-tv} & \mSymbol[sharp]{live-tv} & \texttt{\textbackslash mSymbol\{live-tv\}} & \texttt{E63A}\\
\mSymbol[outlined]{living} & \mSymbol[rounded]{living} & \mSymbol[sharp]{living} & \texttt{\textbackslash mSymbol\{living\}} & \texttt{F02B}\\
\mSymbol[outlined]{local-activity} & \mSymbol[rounded]{local-activity} & \mSymbol[sharp]{local-activity} & \texttt{\textbackslash mSymbol\{local-activity\}} & \texttt{E553}\\
\mSymbol[outlined]{local-airport} & \mSymbol[rounded]{local-airport} & \mSymbol[sharp]{local-airport} & \texttt{\textbackslash mSymbol\{local-airport\}} & \texttt{E53D}\\
\mSymbol[outlined]{local-atm} & \mSymbol[rounded]{local-atm} & \mSymbol[sharp]{local-atm} & \texttt{\textbackslash mSymbol\{local-atm\}} & \texttt{E53E}\\
\mSymbol[outlined]{local-bar} & \mSymbol[rounded]{local-bar} & \mSymbol[sharp]{local-bar} & \texttt{\textbackslash mSymbol\{local-bar\}} & \texttt{E540}\\
\mSymbol[outlined]{local-cafe} & \mSymbol[rounded]{local-cafe} & \mSymbol[sharp]{local-cafe} & \texttt{\textbackslash mSymbol\{local-cafe\}} & \texttt{EB44}\\
\mSymbol[outlined]{local-car-wash} & \mSymbol[rounded]{local-car-wash} & \mSymbol[sharp]{local-car-wash} & \texttt{\textbackslash mSymbol\{local-car-wash\}} & \texttt{E542}\\
\mSymbol[outlined]{local-convenience-store} & \mSymbol[rounded]{local-convenience-store} & \mSymbol[sharp]{local-convenience-store} & \texttt{\textbackslash mSymbol\{local-convenience-store\}} & \texttt{E543}\\
\mSymbol[outlined]{local-dining} & \mSymbol[rounded]{local-dining} & \mSymbol[sharp]{local-dining} & \texttt{\textbackslash mSymbol\{local-dining\}} & \texttt{E561}\\
\mSymbol[outlined]{local-drink} & \mSymbol[rounded]{local-drink} & \mSymbol[sharp]{local-drink} & \texttt{\textbackslash mSymbol\{local-drink\}} & \texttt{E544}\\
\mSymbol[outlined]{local-fire-department} & \mSymbol[rounded]{local-fire-department} & \mSymbol[sharp]{local-fire-department} & \texttt{\textbackslash mSymbol\{local-fire-department\}} & \texttt{EF55}\\
\mSymbol[outlined]{local-florist} & \mSymbol[rounded]{local-florist} & \mSymbol[sharp]{local-florist} & \texttt{\textbackslash mSymbol\{local-florist\}} & \texttt{E545}\\
\mSymbol[outlined]{local-gas-station} & \mSymbol[rounded]{local-gas-station} & \mSymbol[sharp]{local-gas-station} & \texttt{\textbackslash mSymbol\{local-gas-station\}} & \texttt{E546}\\
\mSymbol[outlined]{local-grocery-store} & \mSymbol[rounded]{local-grocery-store} & \mSymbol[sharp]{local-grocery-store} & \texttt{\textbackslash mSymbol\{local-grocery-store\}} & \texttt{E8CC}\\
\mSymbol[outlined]{local-hospital} & \mSymbol[rounded]{local-hospital} & \mSymbol[sharp]{local-hospital} & \texttt{\textbackslash mSymbol\{local-hospital\}} & \texttt{E548}\\
\mSymbol[outlined]{local-hotel} & \mSymbol[rounded]{local-hotel} & \mSymbol[sharp]{local-hotel} & \texttt{\textbackslash mSymbol\{local-hotel\}} & \texttt{E549}\\
\mSymbol[outlined]{local-laundry-service} & \mSymbol[rounded]{local-laundry-service} & \mSymbol[sharp]{local-laundry-service} & \texttt{\textbackslash mSymbol\{local-laundry-service\}} & \texttt{E54A}\\
\mSymbol[outlined]{local-library} & \mSymbol[rounded]{local-library} & \mSymbol[sharp]{local-library} & \texttt{\textbackslash mSymbol\{local-library\}} & \texttt{E54B}\\
\mSymbol[outlined]{local-mall} & \mSymbol[rounded]{local-mall} & \mSymbol[sharp]{local-mall} & \texttt{\textbackslash mSymbol\{local-mall\}} & \texttt{E54C}\\
\mSymbol[outlined]{local-movies} & \mSymbol[rounded]{local-movies} & \mSymbol[sharp]{local-movies} & \texttt{\textbackslash mSymbol\{local-movies\}} & \texttt{E8DA}\\
\mSymbol[outlined]{local-offer} & \mSymbol[rounded]{local-offer} & \mSymbol[sharp]{local-offer} & \texttt{\textbackslash mSymbol\{local-offer\}} & \texttt{F05B}\\
\mSymbol[outlined]{local-parking} & \mSymbol[rounded]{local-parking} & \mSymbol[sharp]{local-parking} & \texttt{\textbackslash mSymbol\{local-parking\}} & \texttt{E54F}\\
\mSymbol[outlined]{local-pharmacy} & \mSymbol[rounded]{local-pharmacy} & \mSymbol[sharp]{local-pharmacy} & \texttt{\textbackslash mSymbol\{local-pharmacy\}} & \texttt{E550}\\
\mSymbol[outlined]{local-phone} & \mSymbol[rounded]{local-phone} & \mSymbol[sharp]{local-phone} & \texttt{\textbackslash mSymbol\{local-phone\}} & \texttt{F0D4}\\
\mSymbol[outlined]{local-pizza} & \mSymbol[rounded]{local-pizza} & \mSymbol[sharp]{local-pizza} & \texttt{\textbackslash mSymbol\{local-pizza\}} & \texttt{E552}\\
\mSymbol[outlined]{local-play} & \mSymbol[rounded]{local-play} & \mSymbol[sharp]{local-play} & \texttt{\textbackslash mSymbol\{local-play\}} & \texttt{E553}\\
\mSymbol[outlined]{local-police} & \mSymbol[rounded]{local-police} & \mSymbol[sharp]{local-police} & \texttt{\textbackslash mSymbol\{local-police\}} & \texttt{EF56}\\
\mSymbol[outlined]{local-post-office} & \mSymbol[rounded]{local-post-office} & \mSymbol[sharp]{local-post-office} & \texttt{\textbackslash mSymbol\{local-post-office\}} & \texttt{E554}\\
\mSymbol[outlined]{local-printshop} & \mSymbol[rounded]{local-printshop} & \mSymbol[sharp]{local-printshop} & \texttt{\textbackslash mSymbol\{local-printshop\}} & \texttt{E8AD}\\
\mSymbol[outlined]{local-see} & \mSymbol[rounded]{local-see} & \mSymbol[sharp]{local-see} & \texttt{\textbackslash mSymbol\{local-see\}} & \texttt{E557}\\
\mSymbol[outlined]{local-shipping} & \mSymbol[rounded]{local-shipping} & \mSymbol[sharp]{local-shipping} & \texttt{\textbackslash mSymbol\{local-shipping\}} & \texttt{E558}\\
\mSymbol[outlined]{local-taxi} & \mSymbol[rounded]{local-taxi} & \mSymbol[sharp]{local-taxi} & \texttt{\textbackslash mSymbol\{local-taxi\}} & \texttt{E559}\\
\mSymbol[outlined]{location-automation} & \mSymbol[rounded]{location-automation} & \mSymbol[sharp]{location-automation} & \texttt{\textbackslash mSymbol\{location-automation\}} & \texttt{F14F}\\
\mSymbol[outlined]{location-away} & \mSymbol[rounded]{location-away} & \mSymbol[sharp]{location-away} & \texttt{\textbackslash mSymbol\{location-away\}} & \texttt{F150}\\
\mSymbol[outlined]{location-chip} & \mSymbol[rounded]{location-chip} & \mSymbol[sharp]{location-chip} & \texttt{\textbackslash mSymbol\{location-chip\}} & \texttt{F850}\\
\mSymbol[outlined]{location-city} & \mSymbol[rounded]{location-city} & \mSymbol[sharp]{location-city} & \texttt{\textbackslash mSymbol\{location-city\}} & \texttt{E7F1}\\
\mSymbol[outlined]{location-disabled} & \mSymbol[rounded]{location-disabled} & \mSymbol[sharp]{location-disabled} & \texttt{\textbackslash mSymbol\{location-disabled\}} & \texttt{E1B6}\\
\mSymbol[outlined]{location-home} & \mSymbol[rounded]{location-home} & \mSymbol[sharp]{location-home} & \texttt{\textbackslash mSymbol\{location-home\}} & \texttt{F152}\\
\mSymbol[outlined]{location-off} & \mSymbol[rounded]{location-off} & \mSymbol[sharp]{location-off} & \texttt{\textbackslash mSymbol\{location-off\}} & \texttt{E0C7}\\
\mSymbol[outlined]{location-on} & \mSymbol[rounded]{location-on} & \mSymbol[sharp]{location-on} & \texttt{\textbackslash mSymbol\{location-on\}} & \texttt{F1DB}\\
\mSymbol[outlined]{location-pin} & \mSymbol[rounded]{location-pin} & \mSymbol[sharp]{location-pin} & \texttt{\textbackslash mSymbol\{location-pin\}} & \texttt{F1DB}\\
\mSymbol[outlined]{location-searching} & \mSymbol[rounded]{location-searching} & \mSymbol[sharp]{location-searching} & \texttt{\textbackslash mSymbol\{location-searching\}} & \texttt{E1B7}\\
\mSymbol[outlined]{locator-tag} & \mSymbol[rounded]{locator-tag} & \mSymbol[sharp]{locator-tag} & \texttt{\textbackslash mSymbol\{locator-tag\}} & \texttt{F8C1}\\
\mSymbol[outlined]{lock} & \mSymbol[rounded]{lock} & \mSymbol[sharp]{lock} & \texttt{\textbackslash mSymbol\{lock\}} & \texttt{E899}\\
\mSymbol[outlined]{lock-clock} & \mSymbol[rounded]{lock-clock} & \mSymbol[sharp]{lock-clock} & \texttt{\textbackslash mSymbol\{lock-clock\}} & \texttt{EF57}\\
\mSymbol[outlined]{lock-open} & \mSymbol[rounded]{lock-open} & \mSymbol[sharp]{lock-open} & \texttt{\textbackslash mSymbol\{lock-open\}} & \texttt{E898}\\
\mSymbol[outlined]{lock-open-right} & \mSymbol[rounded]{lock-open-right} & \mSymbol[sharp]{lock-open-right} & \texttt{\textbackslash mSymbol\{lock-open-right\}} & \texttt{F656}\\
\mSymbol[outlined]{lock-outline} & \mSymbol[rounded]{lock-outline} & \mSymbol[sharp]{lock-outline} & \texttt{\textbackslash mSymbol\{lock-outline\}} & \texttt{E899}\\
\mSymbol[outlined]{lock-person} & \mSymbol[rounded]{lock-person} & \mSymbol[sharp]{lock-person} & \texttt{\textbackslash mSymbol\{lock-person\}} & \texttt{F8F3}\\
\mSymbol[outlined]{lock-reset} & \mSymbol[rounded]{lock-reset} & \mSymbol[sharp]{lock-reset} & \texttt{\textbackslash mSymbol\{lock-reset\}} & \texttt{EADE}\\
\mSymbol[outlined]{login} & \mSymbol[rounded]{login} & \mSymbol[sharp]{login} & \texttt{\textbackslash mSymbol\{login\}} & \texttt{EA77}\\
\mSymbol[outlined]{logo-dev} & \mSymbol[rounded]{logo-dev} & \mSymbol[sharp]{logo-dev} & \texttt{\textbackslash mSymbol\{logo-dev\}} & \texttt{EAD6}\\
\mSymbol[outlined]{logout} & \mSymbol[rounded]{logout} & \mSymbol[sharp]{logout} & \texttt{\textbackslash mSymbol\{logout\}} & \texttt{E9BA}\\
\mSymbol[outlined]{looks} & \mSymbol[rounded]{looks} & \mSymbol[sharp]{looks} & \texttt{\textbackslash mSymbol\{looks\}} & \texttt{E3FC}\\
\mSymbol[outlined]{looks-3} & \mSymbol[rounded]{looks-3} & \mSymbol[sharp]{looks-3} & \texttt{\textbackslash mSymbol\{looks-3\}} & \texttt{E3FB}\\
\mSymbol[outlined]{looks-4} & \mSymbol[rounded]{looks-4} & \mSymbol[sharp]{looks-4} & \texttt{\textbackslash mSymbol\{looks-4\}} & \texttt{E3FD}\\
\mSymbol[outlined]{looks-5} & \mSymbol[rounded]{looks-5} & \mSymbol[sharp]{looks-5} & \texttt{\textbackslash mSymbol\{looks-5\}} & \texttt{E3FE}\\
\mSymbol[outlined]{looks-6} & \mSymbol[rounded]{looks-6} & \mSymbol[sharp]{looks-6} & \texttt{\textbackslash mSymbol\{looks-6\}} & \texttt{E3FF}\\
\mSymbol[outlined]{looks-one} & \mSymbol[rounded]{looks-one} & \mSymbol[sharp]{looks-one} & \texttt{\textbackslash mSymbol\{looks-one\}} & \texttt{E400}\\
\mSymbol[outlined]{looks-two} & \mSymbol[rounded]{looks-two} & \mSymbol[sharp]{looks-two} & \texttt{\textbackslash mSymbol\{looks-two\}} & \texttt{E401}\\
\mSymbol[outlined]{loop} & \mSymbol[rounded]{loop} & \mSymbol[sharp]{loop} & \texttt{\textbackslash mSymbol\{loop\}} & \texttt{E863}\\
\mSymbol[outlined]{loupe} & \mSymbol[rounded]{loupe} & \mSymbol[sharp]{loupe} & \texttt{\textbackslash mSymbol\{loupe\}} & \texttt{E402}\\
\mSymbol[outlined]{low-density} & \mSymbol[rounded]{low-density} & \mSymbol[sharp]{low-density} & \texttt{\textbackslash mSymbol\{low-density\}} & \texttt{F79B}\\
\mSymbol[outlined]{low-priority} & \mSymbol[rounded]{low-priority} & \mSymbol[sharp]{low-priority} & \texttt{\textbackslash mSymbol\{low-priority\}} & \texttt{E16D}\\
\mSymbol[outlined]{lowercase} & \mSymbol[rounded]{lowercase} & \mSymbol[sharp]{lowercase} & \texttt{\textbackslash mSymbol\{lowercase\}} & \texttt{F48A}\\
\mSymbol[outlined]{loyalty} & \mSymbol[rounded]{loyalty} & \mSymbol[sharp]{loyalty} & \texttt{\textbackslash mSymbol\{loyalty\}} & \texttt{E89A}\\
\mSymbol[outlined]{lte-mobiledata} & \mSymbol[rounded]{lte-mobiledata} & \mSymbol[sharp]{lte-mobiledata} & \texttt{\textbackslash mSymbol\{lte-mobiledata\}} & \texttt{F02C}\\
\mSymbol[outlined]{lte-mobiledata-badge} & \mSymbol[rounded]{lte-mobiledata-badge} & \mSymbol[sharp]{lte-mobiledata-badge} & \texttt{\textbackslash mSymbol\{lte-mobiledata-badge\}} & \texttt{F7D9}\\
\mSymbol[outlined]{lte-plus-mobiledata} & \mSymbol[rounded]{lte-plus-mobiledata} & \mSymbol[sharp]{lte-plus-mobiledata} & \texttt{\textbackslash mSymbol\{lte-plus-mobiledata\}} & \texttt{F02D}\\
\mSymbol[outlined]{lte-plus-mobiledata-badge} & \mSymbol[rounded]{lte-plus-mobiledata-badge} & \mSymbol[sharp]{lte-plus-mobiledata-badge} & \texttt{\textbackslash mSymbol\{lte-plus-mobiledata-badge\}} & \texttt{F7D8}\\
\mSymbol[outlined]{luggage} & \mSymbol[rounded]{luggage} & \mSymbol[sharp]{luggage} & \texttt{\textbackslash mSymbol\{luggage\}} & \texttt{F235}\\
\mSymbol[outlined]{lunch-dining} & \mSymbol[rounded]{lunch-dining} & \mSymbol[sharp]{lunch-dining} & \texttt{\textbackslash mSymbol\{lunch-dining\}} & \texttt{EA61}\\
\mSymbol[outlined]{lyrics} & \mSymbol[rounded]{lyrics} & \mSymbol[sharp]{lyrics} & \texttt{\textbackslash mSymbol\{lyrics\}} & \texttt{EC0B}\\
\mSymbol[outlined]{macro-auto} & \mSymbol[rounded]{macro-auto} & \mSymbol[sharp]{macro-auto} & \texttt{\textbackslash mSymbol\{macro-auto\}} & \texttt{F6F2}\\
\mSymbol[outlined]{macro-off} & \mSymbol[rounded]{macro-off} & \mSymbol[sharp]{macro-off} & \texttt{\textbackslash mSymbol\{macro-off\}} & \texttt{F8D2}\\
\mSymbol[outlined]{magic-button} & \mSymbol[rounded]{magic-button} & \mSymbol[sharp]{magic-button} & \texttt{\textbackslash mSymbol\{magic-button\}} & \texttt{F136}\\
\mSymbol[outlined]{magic-exchange} & \mSymbol[rounded]{magic-exchange} & \mSymbol[sharp]{magic-exchange} & \texttt{\textbackslash mSymbol\{magic-exchange\}} & \texttt{F7F4}\\
\mSymbol[outlined]{magic-tether} & \mSymbol[rounded]{magic-tether} & \mSymbol[sharp]{magic-tether} & \texttt{\textbackslash mSymbol\{magic-tether\}} & \texttt{F7D7}\\
\mSymbol[outlined]{magnification-large} & \mSymbol[rounded]{magnification-large} & \mSymbol[sharp]{magnification-large} & \texttt{\textbackslash mSymbol\{magnification-large\}} & \texttt{F83D}\\
\mSymbol[outlined]{magnification-small} & \mSymbol[rounded]{magnification-small} & \mSymbol[sharp]{magnification-small} & \texttt{\textbackslash mSymbol\{magnification-small\}} & \texttt{F83C}\\
\mSymbol[outlined]{magnify-docked} & \mSymbol[rounded]{magnify-docked} & \mSymbol[sharp]{magnify-docked} & \texttt{\textbackslash mSymbol\{magnify-docked\}} & \texttt{F7D6}\\
\mSymbol[outlined]{magnify-fullscreen} & \mSymbol[rounded]{magnify-fullscreen} & \mSymbol[sharp]{magnify-fullscreen} & \texttt{\textbackslash mSymbol\{magnify-fullscreen\}} & \texttt{F7D5}\\
\mSymbol[outlined]{mail} & \mSymbol[rounded]{mail} & \mSymbol[sharp]{mail} & \texttt{\textbackslash mSymbol\{mail\}} & \texttt{E159}\\
\mSymbol[outlined]{mail-lock} & \mSymbol[rounded]{mail-lock} & \mSymbol[sharp]{mail-lock} & \texttt{\textbackslash mSymbol\{mail-lock\}} & \texttt{EC0A}\\
\mSymbol[outlined]{mail-off} & \mSymbol[rounded]{mail-off} & \mSymbol[sharp]{mail-off} & \texttt{\textbackslash mSymbol\{mail-off\}} & \texttt{F48B}\\
\mSymbol[outlined]{mail-outline} & \mSymbol[rounded]{mail-outline} & \mSymbol[sharp]{mail-outline} & \texttt{\textbackslash mSymbol\{mail-outline\}} & \texttt{E159}\\
\mSymbol[outlined]{male} & \mSymbol[rounded]{male} & \mSymbol[sharp]{male} & \texttt{\textbackslash mSymbol\{male\}} & \texttt{E58E}\\
\mSymbol[outlined]{man} & \mSymbol[rounded]{man} & \mSymbol[sharp]{man} & \texttt{\textbackslash mSymbol\{man\}} & \texttt{E4EB}\\
\mSymbol[outlined]{man-2} & \mSymbol[rounded]{man-2} & \mSymbol[sharp]{man-2} & \texttt{\textbackslash mSymbol\{man-2\}} & \texttt{F8E1}\\
\mSymbol[outlined]{man-3} & \mSymbol[rounded]{man-3} & \mSymbol[sharp]{man-3} & \texttt{\textbackslash mSymbol\{man-3\}} & \texttt{F8E2}\\
\mSymbol[outlined]{man-4} & \mSymbol[rounded]{man-4} & \mSymbol[sharp]{man-4} & \texttt{\textbackslash mSymbol\{man-4\}} & \texttt{F8E3}\\
\mSymbol[outlined]{manage-accounts} & \mSymbol[rounded]{manage-accounts} & \mSymbol[sharp]{manage-accounts} & \texttt{\textbackslash mSymbol\{manage-accounts\}} & \texttt{F02E}\\
\mSymbol[outlined]{manage-history} & \mSymbol[rounded]{manage-history} & \mSymbol[sharp]{manage-history} & \texttt{\textbackslash mSymbol\{manage-history\}} & \texttt{EBE7}\\
\mSymbol[outlined]{manage-search} & \mSymbol[rounded]{manage-search} & \mSymbol[sharp]{manage-search} & \texttt{\textbackslash mSymbol\{manage-search\}} & \texttt{F02F}\\
\mSymbol[outlined]{manga} & \mSymbol[rounded]{manga} & \mSymbol[sharp]{manga} & \texttt{\textbackslash mSymbol\{manga\}} & \texttt{F5E3}\\
\mSymbol[outlined]{manufacturing} & \mSymbol[rounded]{manufacturing} & \mSymbol[sharp]{manufacturing} & \texttt{\textbackslash mSymbol\{manufacturing\}} & \texttt{E726}\\
\mSymbol[outlined]{map} & \mSymbol[rounded]{map} & \mSymbol[sharp]{map} & \texttt{\textbackslash mSymbol\{map\}} & \texttt{E55B}\\
\mSymbol[outlined]{maps-home-work} & \mSymbol[rounded]{maps-home-work} & \mSymbol[sharp]{maps-home-work} & \texttt{\textbackslash mSymbol\{maps-home-work\}} & \texttt{F030}\\
\mSymbol[outlined]{maps-ugc} & \mSymbol[rounded]{maps-ugc} & \mSymbol[sharp]{maps-ugc} & \texttt{\textbackslash mSymbol\{maps-ugc\}} & \texttt{EF58}\\
\mSymbol[outlined]{margin} & \mSymbol[rounded]{margin} & \mSymbol[sharp]{margin} & \texttt{\textbackslash mSymbol\{margin\}} & \texttt{E9BB}\\
\mSymbol[outlined]{mark-as-unread} & \mSymbol[rounded]{mark-as-unread} & \mSymbol[sharp]{mark-as-unread} & \texttt{\textbackslash mSymbol\{mark-as-unread\}} & \texttt{E9BC}\\
\mSymbol[outlined]{mark-chat-read} & \mSymbol[rounded]{mark-chat-read} & \mSymbol[sharp]{mark-chat-read} & \texttt{\textbackslash mSymbol\{mark-chat-read\}} & \texttt{F18B}\\
\mSymbol[outlined]{mark-chat-unread} & \mSymbol[rounded]{mark-chat-unread} & \mSymbol[sharp]{mark-chat-unread} & \texttt{\textbackslash mSymbol\{mark-chat-unread\}} & \texttt{F189}\\
\mSymbol[outlined]{mark-email-read} & \mSymbol[rounded]{mark-email-read} & \mSymbol[sharp]{mark-email-read} & \texttt{\textbackslash mSymbol\{mark-email-read\}} & \texttt{F18C}\\
\mSymbol[outlined]{mark-email-unread} & \mSymbol[rounded]{mark-email-unread} & \mSymbol[sharp]{mark-email-unread} & \texttt{\textbackslash mSymbol\{mark-email-unread\}} & \texttt{F18A}\\
\mSymbol[outlined]{mark-unread-chat-alt} & \mSymbol[rounded]{mark-unread-chat-alt} & \mSymbol[sharp]{mark-unread-chat-alt} & \texttt{\textbackslash mSymbol\{mark-unread-chat-alt\}} & \texttt{EB9D}\\
\mSymbol[outlined]{markdown} & \mSymbol[rounded]{markdown} & \mSymbol[sharp]{markdown} & \texttt{\textbackslash mSymbol\{markdown\}} & \texttt{F552}\\
\mSymbol[outlined]{markdown-copy} & \mSymbol[rounded]{markdown-copy} & \mSymbol[sharp]{markdown-copy} & \texttt{\textbackslash mSymbol\{markdown-copy\}} & \texttt{F553}\\
\mSymbol[outlined]{markdown-paste} & \mSymbol[rounded]{markdown-paste} & \mSymbol[sharp]{markdown-paste} & \texttt{\textbackslash mSymbol\{markdown-paste\}} & \texttt{F554}\\
\mSymbol[outlined]{markunread} & \mSymbol[rounded]{markunread} & \mSymbol[sharp]{markunread} & \texttt{\textbackslash mSymbol\{markunread\}} & \texttt{E159}\\
\mSymbol[outlined]{markunread-mailbox} & \mSymbol[rounded]{markunread-mailbox} & \mSymbol[sharp]{markunread-mailbox} & \texttt{\textbackslash mSymbol\{markunread-mailbox\}} & \texttt{E89B}\\
\mSymbol[outlined]{masked-transitions} & \mSymbol[rounded]{masked-transitions} & \mSymbol[sharp]{masked-transitions} & \texttt{\textbackslash mSymbol\{masked-transitions\}} & \texttt{E72E}\\
\mSymbol[outlined]{masked-transitions-add} & \mSymbol[rounded]{masked-transitions-add} & \mSymbol[sharp]{masked-transitions-add} & \texttt{\textbackslash mSymbol\{masked-transitions-add\}} & \texttt{F42B}\\
\mSymbol[outlined]{masks} & \mSymbol[rounded]{masks} & \mSymbol[sharp]{masks} & \texttt{\textbackslash mSymbol\{masks\}} & \texttt{F218}\\
\mSymbol[outlined]{match-case} & \mSymbol[rounded]{match-case} & \mSymbol[sharp]{match-case} & \texttt{\textbackslash mSymbol\{match-case\}} & \texttt{F6F1}\\
\mSymbol[outlined]{match-word} & \mSymbol[rounded]{match-word} & \mSymbol[sharp]{match-word} & \texttt{\textbackslash mSymbol\{match-word\}} & \texttt{F6F0}\\
\mSymbol[outlined]{matter} & \mSymbol[rounded]{matter} & \mSymbol[sharp]{matter} & \texttt{\textbackslash mSymbol\{matter\}} & \texttt{E907}\\
\mSymbol[outlined]{maximize} & \mSymbol[rounded]{maximize} & \mSymbol[sharp]{maximize} & \texttt{\textbackslash mSymbol\{maximize\}} & \texttt{E930}\\
\mSymbol[outlined]{measuring-tape} & \mSymbol[rounded]{measuring-tape} & \mSymbol[sharp]{measuring-tape} & \texttt{\textbackslash mSymbol\{measuring-tape\}} & \texttt{F6AF}\\
\mSymbol[outlined]{media-bluetooth-off} & \mSymbol[rounded]{media-bluetooth-off} & \mSymbol[sharp]{media-bluetooth-off} & \texttt{\textbackslash mSymbol\{media-bluetooth-off\}} & \texttt{F031}\\
\mSymbol[outlined]{media-bluetooth-on} & \mSymbol[rounded]{media-bluetooth-on} & \mSymbol[sharp]{media-bluetooth-on} & \texttt{\textbackslash mSymbol\{media-bluetooth-on\}} & \texttt{F032}\\
\mSymbol[outlined]{media-link} & \mSymbol[rounded]{media-link} & \mSymbol[sharp]{media-link} & \texttt{\textbackslash mSymbol\{media-link\}} & \texttt{F83F}\\
\mSymbol[outlined]{media-output} & \mSymbol[rounded]{media-output} & \mSymbol[sharp]{media-output} & \texttt{\textbackslash mSymbol\{media-output\}} & \texttt{F4F2}\\
\mSymbol[outlined]{media-output-off} & \mSymbol[rounded]{media-output-off} & \mSymbol[sharp]{media-output-off} & \texttt{\textbackslash mSymbol\{media-output-off\}} & \texttt{F4F3}\\
\mSymbol[outlined]{mediation} & \mSymbol[rounded]{mediation} & \mSymbol[sharp]{mediation} & \texttt{\textbackslash mSymbol\{mediation\}} & \texttt{EFA7}\\
\mSymbol[outlined]{medical-information} & \mSymbol[rounded]{medical-information} & \mSymbol[sharp]{medical-information} & \texttt{\textbackslash mSymbol\{medical-information\}} & \texttt{EBED}\\
\mSymbol[outlined]{medical-mask} & \mSymbol[rounded]{medical-mask} & \mSymbol[sharp]{medical-mask} & \texttt{\textbackslash mSymbol\{medical-mask\}} & \texttt{F80A}\\
\mSymbol[outlined]{medical-services} & \mSymbol[rounded]{medical-services} & \mSymbol[sharp]{medical-services} & \texttt{\textbackslash mSymbol\{medical-services\}} & \texttt{F109}\\
\mSymbol[outlined]{medication} & \mSymbol[rounded]{medication} & \mSymbol[sharp]{medication} & \texttt{\textbackslash mSymbol\{medication\}} & \texttt{F033}\\
\mSymbol[outlined]{medication-liquid} & \mSymbol[rounded]{medication-liquid} & \mSymbol[sharp]{medication-liquid} & \texttt{\textbackslash mSymbol\{medication-liquid\}} & \texttt{EA87}\\
\mSymbol[outlined]{meeting-room} & \mSymbol[rounded]{meeting-room} & \mSymbol[sharp]{meeting-room} & \texttt{\textbackslash mSymbol\{meeting-room\}} & \texttt{EB4F}\\
\mSymbol[outlined]{memory} & \mSymbol[rounded]{memory} & \mSymbol[sharp]{memory} & \texttt{\textbackslash mSymbol\{memory\}} & \texttt{E322}\\
\mSymbol[outlined]{memory-alt} & \mSymbol[rounded]{memory-alt} & \mSymbol[sharp]{memory-alt} & \texttt{\textbackslash mSymbol\{memory-alt\}} & \texttt{F7A3}\\
\mSymbol[outlined]{menstrual-health} & \mSymbol[rounded]{menstrual-health} & \mSymbol[sharp]{menstrual-health} & \texttt{\textbackslash mSymbol\{menstrual-health\}} & \texttt{F6E1}\\
\mSymbol[outlined]{menu} & \mSymbol[rounded]{menu} & \mSymbol[sharp]{menu} & \texttt{\textbackslash mSymbol\{menu\}} & \texttt{E5D2}\\
\mSymbol[outlined]{menu-book} & \mSymbol[rounded]{menu-book} & \mSymbol[sharp]{menu-book} & \texttt{\textbackslash mSymbol\{menu-book\}} & \texttt{EA19}\\
\mSymbol[outlined]{menu-open} & \mSymbol[rounded]{menu-open} & \mSymbol[sharp]{menu-open} & \texttt{\textbackslash mSymbol\{menu-open\}} & \texttt{E9BD}\\
\mSymbol[outlined]{merge} & \mSymbol[rounded]{merge} & \mSymbol[sharp]{merge} & \texttt{\textbackslash mSymbol\{merge\}} & \texttt{EB98}\\
\mSymbol[outlined]{merge-type} & \mSymbol[rounded]{merge-type} & \mSymbol[sharp]{merge-type} & \texttt{\textbackslash mSymbol\{merge-type\}} & \texttt{E252}\\
\mSymbol[outlined]{message} & \mSymbol[rounded]{message} & \mSymbol[sharp]{message} & \texttt{\textbackslash mSymbol\{message\}} & \texttt{E0C9}\\
\mSymbol[outlined]{metabolism} & \mSymbol[rounded]{metabolism} & \mSymbol[sharp]{metabolism} & \texttt{\textbackslash mSymbol\{metabolism\}} & \texttt{E10B}\\
\mSymbol[outlined]{metro} & \mSymbol[rounded]{metro} & \mSymbol[sharp]{metro} & \texttt{\textbackslash mSymbol\{metro\}} & \texttt{F474}\\
\mSymbol[outlined]{mfg-nest-yale-lock} & \mSymbol[rounded]{mfg-nest-yale-lock} & \mSymbol[sharp]{mfg-nest-yale-lock} & \texttt{\textbackslash mSymbol\{mfg-nest-yale-lock\}} & \texttt{F11D}\\
\mSymbol[outlined]{mic} & \mSymbol[rounded]{mic} & \mSymbol[sharp]{mic} & \texttt{\textbackslash mSymbol\{mic\}} & \texttt{E31D}\\
\mSymbol[outlined]{mic-double} & \mSymbol[rounded]{mic-double} & \mSymbol[sharp]{mic-double} & \texttt{\textbackslash mSymbol\{mic-double\}} & \texttt{F5D1}\\
\mSymbol[outlined]{mic-external-off} & \mSymbol[rounded]{mic-external-off} & \mSymbol[sharp]{mic-external-off} & \texttt{\textbackslash mSymbol\{mic-external-off\}} & \texttt{EF59}\\
\mSymbol[outlined]{mic-external-on} & \mSymbol[rounded]{mic-external-on} & \mSymbol[sharp]{mic-external-on} & \texttt{\textbackslash mSymbol\{mic-external-on\}} & \texttt{EF5A}\\
\mSymbol[outlined]{mic-none} & \mSymbol[rounded]{mic-none} & \mSymbol[sharp]{mic-none} & \texttt{\textbackslash mSymbol\{mic-none\}} & \texttt{E31D}\\
\mSymbol[outlined]{mic-off} & \mSymbol[rounded]{mic-off} & \mSymbol[sharp]{mic-off} & \texttt{\textbackslash mSymbol\{mic-off\}} & \texttt{E02B}\\
\mSymbol[outlined]{microbiology} & \mSymbol[rounded]{microbiology} & \mSymbol[sharp]{microbiology} & \texttt{\textbackslash mSymbol\{microbiology\}} & \texttt{E10C}\\
\mSymbol[outlined]{microwave} & \mSymbol[rounded]{microwave} & \mSymbol[sharp]{microwave} & \texttt{\textbackslash mSymbol\{microwave\}} & \texttt{F204}\\
\mSymbol[outlined]{microwave-gen} & \mSymbol[rounded]{microwave-gen} & \mSymbol[sharp]{microwave-gen} & \texttt{\textbackslash mSymbol\{microwave-gen\}} & \texttt{E847}\\
\mSymbol[outlined]{military-tech} & \mSymbol[rounded]{military-tech} & \mSymbol[sharp]{military-tech} & \texttt{\textbackslash mSymbol\{military-tech\}} & \texttt{EA3F}\\
\mSymbol[outlined]{mimo} & \mSymbol[rounded]{mimo} & \mSymbol[sharp]{mimo} & \texttt{\textbackslash mSymbol\{mimo\}} & \texttt{E9BE}\\
\mSymbol[outlined]{mimo-disconnect} & \mSymbol[rounded]{mimo-disconnect} & \mSymbol[sharp]{mimo-disconnect} & \texttt{\textbackslash mSymbol\{mimo-disconnect\}} & \texttt{E9BF}\\
\mSymbol[outlined]{mindfulness} & \mSymbol[rounded]{mindfulness} & \mSymbol[sharp]{mindfulness} & \texttt{\textbackslash mSymbol\{mindfulness\}} & \texttt{F6E0}\\
\mSymbol[outlined]{minimize} & \mSymbol[rounded]{minimize} & \mSymbol[sharp]{minimize} & \texttt{\textbackslash mSymbol\{minimize\}} & \texttt{E931}\\
\mSymbol[outlined]{minor-crash} & \mSymbol[rounded]{minor-crash} & \mSymbol[sharp]{minor-crash} & \texttt{\textbackslash mSymbol\{minor-crash\}} & \texttt{EBF1}\\
\mSymbol[outlined]{mintmark} & \mSymbol[rounded]{mintmark} & \mSymbol[sharp]{mintmark} & \texttt{\textbackslash mSymbol\{mintmark\}} & \texttt{EFA9}\\
\mSymbol[outlined]{missed-video-call} & \mSymbol[rounded]{missed-video-call} & \mSymbol[sharp]{missed-video-call} & \texttt{\textbackslash mSymbol\{missed-video-call\}} & \texttt{F0CE}\\
\mSymbol[outlined]{missed-video-call-filled} & \mSymbol[rounded]{missed-video-call-filled} & \mSymbol[sharp]{missed-video-call-filled} & \texttt{\textbackslash mSymbol\{missed-video-call-filled\}} & \texttt{F0CE}\\
\mSymbol[outlined]{missing-controller} & \mSymbol[rounded]{missing-controller} & \mSymbol[sharp]{missing-controller} & \texttt{\textbackslash mSymbol\{missing-controller\}} & \texttt{E701}\\
\mSymbol[outlined]{mist} & \mSymbol[rounded]{mist} & \mSymbol[sharp]{mist} & \texttt{\textbackslash mSymbol\{mist\}} & \texttt{E188}\\
\mSymbol[outlined]{mitre} & \mSymbol[rounded]{mitre} & \mSymbol[sharp]{mitre} & \texttt{\textbackslash mSymbol\{mitre\}} & \texttt{F547}\\
\mSymbol[outlined]{mixture-med} & \mSymbol[rounded]{mixture-med} & \mSymbol[sharp]{mixture-med} & \texttt{\textbackslash mSymbol\{mixture-med\}} & \texttt{E4C8}\\
\mSymbol[outlined]{mms} & \mSymbol[rounded]{mms} & \mSymbol[sharp]{mms} & \texttt{\textbackslash mSymbol\{mms\}} & \texttt{E618}\\
\mSymbol[outlined]{mobile-friendly} & \mSymbol[rounded]{mobile-friendly} & \mSymbol[sharp]{mobile-friendly} & \texttt{\textbackslash mSymbol\{mobile-friendly\}} & \texttt{E200}\\
\mSymbol[outlined]{mobile-off} & \mSymbol[rounded]{mobile-off} & \mSymbol[sharp]{mobile-off} & \texttt{\textbackslash mSymbol\{mobile-off\}} & \texttt{E201}\\
\mSymbol[outlined]{mobile-screen-share} & \mSymbol[rounded]{mobile-screen-share} & \mSymbol[sharp]{mobile-screen-share} & \texttt{\textbackslash mSymbol\{mobile-screen-share\}} & \texttt{E0E7}\\
\mSymbol[outlined]{mobiledata-off} & \mSymbol[rounded]{mobiledata-off} & \mSymbol[sharp]{mobiledata-off} & \texttt{\textbackslash mSymbol\{mobiledata-off\}} & \texttt{F034}\\
\mSymbol[outlined]{mode} & \mSymbol[rounded]{mode} & \mSymbol[sharp]{mode} & \texttt{\textbackslash mSymbol\{mode\}} & \texttt{F097}\\
\mSymbol[outlined]{mode-comment} & \mSymbol[rounded]{mode-comment} & \mSymbol[sharp]{mode-comment} & \texttt{\textbackslash mSymbol\{mode-comment\}} & \texttt{E253}\\
\mSymbol[outlined]{mode-cool} & \mSymbol[rounded]{mode-cool} & \mSymbol[sharp]{mode-cool} & \texttt{\textbackslash mSymbol\{mode-cool\}} & \texttt{F166}\\
\mSymbol[outlined]{mode-cool-off} & \mSymbol[rounded]{mode-cool-off} & \mSymbol[sharp]{mode-cool-off} & \texttt{\textbackslash mSymbol\{mode-cool-off\}} & \texttt{F167}\\
\mSymbol[outlined]{mode-dual} & \mSymbol[rounded]{mode-dual} & \mSymbol[sharp]{mode-dual} & \texttt{\textbackslash mSymbol\{mode-dual\}} & \texttt{F557}\\
\mSymbol[outlined]{mode-edit} & \mSymbol[rounded]{mode-edit} & \mSymbol[sharp]{mode-edit} & \texttt{\textbackslash mSymbol\{mode-edit\}} & \texttt{F097}\\
\mSymbol[outlined]{mode-edit-outline} & \mSymbol[rounded]{mode-edit-outline} & \mSymbol[sharp]{mode-edit-outline} & \texttt{\textbackslash mSymbol\{mode-edit-outline\}} & \texttt{F097}\\
\mSymbol[outlined]{mode-fan} & \mSymbol[rounded]{mode-fan} & \mSymbol[sharp]{mode-fan} & \texttt{\textbackslash mSymbol\{mode-fan\}} & \texttt{F168}\\
\mSymbol[outlined]{mode-fan-off} & \mSymbol[rounded]{mode-fan-off} & \mSymbol[sharp]{mode-fan-off} & \texttt{\textbackslash mSymbol\{mode-fan-off\}} & \texttt{EC17}\\
\mSymbol[outlined]{mode-heat} & \mSymbol[rounded]{mode-heat} & \mSymbol[sharp]{mode-heat} & \texttt{\textbackslash mSymbol\{mode-heat\}} & \texttt{F16A}\\
\mSymbol[outlined]{mode-heat-cool} & \mSymbol[rounded]{mode-heat-cool} & \mSymbol[sharp]{mode-heat-cool} & \texttt{\textbackslash mSymbol\{mode-heat-cool\}} & \texttt{F16B}\\
\mSymbol[outlined]{mode-heat-off} & \mSymbol[rounded]{mode-heat-off} & \mSymbol[sharp]{mode-heat-off} & \texttt{\textbackslash mSymbol\{mode-heat-off\}} & \texttt{F16D}\\
\mSymbol[outlined]{mode-night} & \mSymbol[rounded]{mode-night} & \mSymbol[sharp]{mode-night} & \texttt{\textbackslash mSymbol\{mode-night\}} & \texttt{F036}\\
\mSymbol[outlined]{mode-of-travel} & \mSymbol[rounded]{mode-of-travel} & \mSymbol[sharp]{mode-of-travel} & \texttt{\textbackslash mSymbol\{mode-of-travel\}} & \texttt{E7CE}\\
\mSymbol[outlined]{mode-off-on} & \mSymbol[rounded]{mode-off-on} & \mSymbol[sharp]{mode-off-on} & \texttt{\textbackslash mSymbol\{mode-off-on\}} & \texttt{F16F}\\
\mSymbol[outlined]{mode-standby} & \mSymbol[rounded]{mode-standby} & \mSymbol[sharp]{mode-standby} & \texttt{\textbackslash mSymbol\{mode-standby\}} & \texttt{F037}\\
\mSymbol[outlined]{model-training} & \mSymbol[rounded]{model-training} & \mSymbol[sharp]{model-training} & \texttt{\textbackslash mSymbol\{model-training\}} & \texttt{F0CF}\\
\mSymbol[outlined]{monetization-on} & \mSymbol[rounded]{monetization-on} & \mSymbol[sharp]{monetization-on} & \texttt{\textbackslash mSymbol\{monetization-on\}} & \texttt{E263}\\
\mSymbol[outlined]{money} & \mSymbol[rounded]{money} & \mSymbol[sharp]{money} & \texttt{\textbackslash mSymbol\{money\}} & \texttt{E57D}\\
\mSymbol[outlined]{money-off} & \mSymbol[rounded]{money-off} & \mSymbol[sharp]{money-off} & \texttt{\textbackslash mSymbol\{money-off\}} & \texttt{F038}\\
\mSymbol[outlined]{money-off-csred} & \mSymbol[rounded]{money-off-csred} & \mSymbol[sharp]{money-off-csred} & \texttt{\textbackslash mSymbol\{money-off-csred\}} & \texttt{F038}\\
\mSymbol[outlined]{monitor} & \mSymbol[rounded]{monitor} & \mSymbol[sharp]{monitor} & \texttt{\textbackslash mSymbol\{monitor\}} & \texttt{EF5B}\\
\mSymbol[outlined]{monitor-heart} & \mSymbol[rounded]{monitor-heart} & \mSymbol[sharp]{monitor-heart} & \texttt{\textbackslash mSymbol\{monitor-heart\}} & \texttt{EAA2}\\
\mSymbol[outlined]{monitor-weight} & \mSymbol[rounded]{monitor-weight} & \mSymbol[sharp]{monitor-weight} & \texttt{\textbackslash mSymbol\{monitor-weight\}} & \texttt{F039}\\
\mSymbol[outlined]{monitor-weight-gain} & \mSymbol[rounded]{monitor-weight-gain} & \mSymbol[sharp]{monitor-weight-gain} & \texttt{\textbackslash mSymbol\{monitor-weight-gain\}} & \texttt{F6DF}\\
\mSymbol[outlined]{monitor-weight-loss} & \mSymbol[rounded]{monitor-weight-loss} & \mSymbol[sharp]{monitor-weight-loss} & \texttt{\textbackslash mSymbol\{monitor-weight-loss\}} & \texttt{F6DE}\\
\mSymbol[outlined]{monitoring} & \mSymbol[rounded]{monitoring} & \mSymbol[sharp]{monitoring} & \texttt{\textbackslash mSymbol\{monitoring\}} & \texttt{F190}\\
\mSymbol[outlined]{monochrome-photos} & \mSymbol[rounded]{monochrome-photos} & \mSymbol[sharp]{monochrome-photos} & \texttt{\textbackslash mSymbol\{monochrome-photos\}} & \texttt{E403}\\
\mSymbol[outlined]{monorail} & \mSymbol[rounded]{monorail} & \mSymbol[sharp]{monorail} & \texttt{\textbackslash mSymbol\{monorail\}} & \texttt{F473}\\
\mSymbol[outlined]{mood} & \mSymbol[rounded]{mood} & \mSymbol[sharp]{mood} & \texttt{\textbackslash mSymbol\{mood\}} & \texttt{EA22}\\
\mSymbol[outlined]{mood-bad} & \mSymbol[rounded]{mood-bad} & \mSymbol[sharp]{mood-bad} & \texttt{\textbackslash mSymbol\{mood-bad\}} & \texttt{E7F3}\\
\mSymbol[outlined]{mop} & \mSymbol[rounded]{mop} & \mSymbol[sharp]{mop} & \texttt{\textbackslash mSymbol\{mop\}} & \texttt{E28D}\\
\mSymbol[outlined]{more} & \mSymbol[rounded]{more} & \mSymbol[sharp]{more} & \texttt{\textbackslash mSymbol\{more\}} & \texttt{E619}\\
\mSymbol[outlined]{more-down} & \mSymbol[rounded]{more-down} & \mSymbol[sharp]{more-down} & \texttt{\textbackslash mSymbol\{more-down\}} & \texttt{F196}\\
\mSymbol[outlined]{more-horiz} & \mSymbol[rounded]{more-horiz} & \mSymbol[sharp]{more-horiz} & \texttt{\textbackslash mSymbol\{more-horiz\}} & \texttt{E5D3}\\
\mSymbol[outlined]{more-time} & \mSymbol[rounded]{more-time} & \mSymbol[sharp]{more-time} & \texttt{\textbackslash mSymbol\{more-time\}} & \texttt{EA5D}\\
\mSymbol[outlined]{more-up} & \mSymbol[rounded]{more-up} & \mSymbol[sharp]{more-up} & \texttt{\textbackslash mSymbol\{more-up\}} & \texttt{F197}\\
\mSymbol[outlined]{more-vert} & \mSymbol[rounded]{more-vert} & \mSymbol[sharp]{more-vert} & \texttt{\textbackslash mSymbol\{more-vert\}} & \texttt{E5D4}\\
\mSymbol[outlined]{mosque} & \mSymbol[rounded]{mosque} & \mSymbol[sharp]{mosque} & \texttt{\textbackslash mSymbol\{mosque\}} & \texttt{EAB2}\\
\mSymbol[outlined]{motion-blur} & \mSymbol[rounded]{motion-blur} & \mSymbol[sharp]{motion-blur} & \texttt{\textbackslash mSymbol\{motion-blur\}} & \texttt{F0D0}\\
\mSymbol[outlined]{motion-mode} & \mSymbol[rounded]{motion-mode} & \mSymbol[sharp]{motion-mode} & \texttt{\textbackslash mSymbol\{motion-mode\}} & \texttt{F842}\\
\mSymbol[outlined]{motion-photos-auto} & \mSymbol[rounded]{motion-photos-auto} & \mSymbol[sharp]{motion-photos-auto} & \texttt{\textbackslash mSymbol\{motion-photos-auto\}} & \texttt{F03A}\\
\mSymbol[outlined]{motion-photos-off} & \mSymbol[rounded]{motion-photos-off} & \mSymbol[sharp]{motion-photos-off} & \texttt{\textbackslash mSymbol\{motion-photos-off\}} & \texttt{E9C0}\\
\mSymbol[outlined]{motion-photos-on} & \mSymbol[rounded]{motion-photos-on} & \mSymbol[sharp]{motion-photos-on} & \texttt{\textbackslash mSymbol\{motion-photos-on\}} & \texttt{E9C1}\\
\mSymbol[outlined]{motion-photos-pause} & \mSymbol[rounded]{motion-photos-pause} & \mSymbol[sharp]{motion-photos-pause} & \texttt{\textbackslash mSymbol\{motion-photos-pause\}} & \texttt{F227}\\
\mSymbol[outlined]{motion-photos-paused} & \mSymbol[rounded]{motion-photos-paused} & \mSymbol[sharp]{motion-photos-paused} & \texttt{\textbackslash mSymbol\{motion-photos-paused\}} & \texttt{F227}\\
\mSymbol[outlined]{motion-play} & \mSymbol[rounded]{motion-play} & \mSymbol[sharp]{motion-play} & \texttt{\textbackslash mSymbol\{motion-play\}} & \texttt{F40B}\\
\mSymbol[outlined]{motion-sensor-active} & \mSymbol[rounded]{motion-sensor-active} & \mSymbol[sharp]{motion-sensor-active} & \texttt{\textbackslash mSymbol\{motion-sensor-active\}} & \texttt{E792}\\
\mSymbol[outlined]{motion-sensor-alert} & \mSymbol[rounded]{motion-sensor-alert} & \mSymbol[sharp]{motion-sensor-alert} & \texttt{\textbackslash mSymbol\{motion-sensor-alert\}} & \texttt{E784}\\
\mSymbol[outlined]{motion-sensor-idle} & \mSymbol[rounded]{motion-sensor-idle} & \mSymbol[sharp]{motion-sensor-idle} & \texttt{\textbackslash mSymbol\{motion-sensor-idle\}} & \texttt{E783}\\
\mSymbol[outlined]{motion-sensor-urgent} & \mSymbol[rounded]{motion-sensor-urgent} & \mSymbol[sharp]{motion-sensor-urgent} & \texttt{\textbackslash mSymbol\{motion-sensor-urgent\}} & \texttt{E78E}\\
\mSymbol[outlined]{motorcycle} & \mSymbol[rounded]{motorcycle} & \mSymbol[sharp]{motorcycle} & \texttt{\textbackslash mSymbol\{motorcycle\}} & \texttt{E91B}\\
\mSymbol[outlined]{mountain-flag} & \mSymbol[rounded]{mountain-flag} & \mSymbol[sharp]{mountain-flag} & \texttt{\textbackslash mSymbol\{mountain-flag\}} & \texttt{F5E2}\\
\mSymbol[outlined]{mouse} & \mSymbol[rounded]{mouse} & \mSymbol[sharp]{mouse} & \texttt{\textbackslash mSymbol\{mouse\}} & \texttt{E323}\\
\mSymbol[outlined]{mouse-lock} & \mSymbol[rounded]{mouse-lock} & \mSymbol[sharp]{mouse-lock} & \texttt{\textbackslash mSymbol\{mouse-lock\}} & \texttt{F490}\\
\mSymbol[outlined]{mouse-lock-off} & \mSymbol[rounded]{mouse-lock-off} & \mSymbol[sharp]{mouse-lock-off} & \texttt{\textbackslash mSymbol\{mouse-lock-off\}} & \texttt{F48F}\\
\mSymbol[outlined]{move} & \mSymbol[rounded]{move} & \mSymbol[sharp]{move} & \texttt{\textbackslash mSymbol\{move\}} & \texttt{E740}\\
\mSymbol[outlined]{move-down} & \mSymbol[rounded]{move-down} & \mSymbol[sharp]{move-down} & \texttt{\textbackslash mSymbol\{move-down\}} & \texttt{EB61}\\
\mSymbol[outlined]{move-group} & \mSymbol[rounded]{move-group} & \mSymbol[sharp]{move-group} & \texttt{\textbackslash mSymbol\{move-group\}} & \texttt{F715}\\
\mSymbol[outlined]{move-item} & \mSymbol[rounded]{move-item} & \mSymbol[sharp]{move-item} & \texttt{\textbackslash mSymbol\{move-item\}} & \texttt{F1FF}\\
\mSymbol[outlined]{move-location} & \mSymbol[rounded]{move-location} & \mSymbol[sharp]{move-location} & \texttt{\textbackslash mSymbol\{move-location\}} & \texttt{E741}\\
\mSymbol[outlined]{move-selection-down} & \mSymbol[rounded]{move-selection-down} & \mSymbol[sharp]{move-selection-down} & \texttt{\textbackslash mSymbol\{move-selection-down\}} & \texttt{F714}\\
\mSymbol[outlined]{move-selection-left} & \mSymbol[rounded]{move-selection-left} & \mSymbol[sharp]{move-selection-left} & \texttt{\textbackslash mSymbol\{move-selection-left\}} & \texttt{F713}\\
\mSymbol[outlined]{move-selection-right} & \mSymbol[rounded]{move-selection-right} & \mSymbol[sharp]{move-selection-right} & \texttt{\textbackslash mSymbol\{move-selection-right\}} & \texttt{F712}\\
\mSymbol[outlined]{move-selection-up} & \mSymbol[rounded]{move-selection-up} & \mSymbol[sharp]{move-selection-up} & \texttt{\textbackslash mSymbol\{move-selection-up\}} & \texttt{F711}\\
\mSymbol[outlined]{move-to-inbox} & \mSymbol[rounded]{move-to-inbox} & \mSymbol[sharp]{move-to-inbox} & \texttt{\textbackslash mSymbol\{move-to-inbox\}} & \texttt{E168}\\
\mSymbol[outlined]{move-up} & \mSymbol[rounded]{move-up} & \mSymbol[sharp]{move-up} & \texttt{\textbackslash mSymbol\{move-up\}} & \texttt{EB64}\\
\mSymbol[outlined]{moved-location} & \mSymbol[rounded]{moved-location} & \mSymbol[sharp]{moved-location} & \texttt{\textbackslash mSymbol\{moved-location\}} & \texttt{E594}\\
\mSymbol[outlined]{movie} & \mSymbol[rounded]{movie} & \mSymbol[sharp]{movie} & \texttt{\textbackslash mSymbol\{movie\}} & \texttt{E404}\\
\mSymbol[outlined]{movie-creation} & \mSymbol[rounded]{movie-creation} & \mSymbol[sharp]{movie-creation} & \texttt{\textbackslash mSymbol\{movie-creation\}} & \texttt{E404}\\
\mSymbol[outlined]{movie-edit} & \mSymbol[rounded]{movie-edit} & \mSymbol[sharp]{movie-edit} & \texttt{\textbackslash mSymbol\{movie-edit\}} & \texttt{F840}\\
\mSymbol[outlined]{movie-filter} & \mSymbol[rounded]{movie-filter} & \mSymbol[sharp]{movie-filter} & \texttt{\textbackslash mSymbol\{movie-filter\}} & \texttt{E43A}\\
\mSymbol[outlined]{movie-info} & \mSymbol[rounded]{movie-info} & \mSymbol[sharp]{movie-info} & \texttt{\textbackslash mSymbol\{movie-info\}} & \texttt{E02D}\\
\mSymbol[outlined]{movie-off} & \mSymbol[rounded]{movie-off} & \mSymbol[sharp]{movie-off} & \texttt{\textbackslash mSymbol\{movie-off\}} & \texttt{F499}\\
\mSymbol[outlined]{moving} & \mSymbol[rounded]{moving} & \mSymbol[sharp]{moving} & \texttt{\textbackslash mSymbol\{moving\}} & \texttt{E501}\\
\mSymbol[outlined]{moving-beds} & \mSymbol[rounded]{moving-beds} & \mSymbol[sharp]{moving-beds} & \texttt{\textbackslash mSymbol\{moving-beds\}} & \texttt{E73D}\\
\mSymbol[outlined]{moving-ministry} & \mSymbol[rounded]{moving-ministry} & \mSymbol[sharp]{moving-ministry} & \texttt{\textbackslash mSymbol\{moving-ministry\}} & \texttt{E73E}\\
\mSymbol[outlined]{mp} & \mSymbol[rounded]{mp} & \mSymbol[sharp]{mp} & \texttt{\textbackslash mSymbol\{mp\}} & \texttt{E9C3}\\
\mSymbol[outlined]{multicooker} & \mSymbol[rounded]{multicooker} & \mSymbol[sharp]{multicooker} & \texttt{\textbackslash mSymbol\{multicooker\}} & \texttt{E293}\\
\mSymbol[outlined]{multiline-chart} & \mSymbol[rounded]{multiline-chart} & \mSymbol[sharp]{multiline-chart} & \texttt{\textbackslash mSymbol\{multiline-chart\}} & \texttt{E6DF}\\
\mSymbol[outlined]{multimodal-hand-eye} & \mSymbol[rounded]{multimodal-hand-eye} & \mSymbol[sharp]{multimodal-hand-eye} & \texttt{\textbackslash mSymbol\{multimodal-hand-eye\}} & \texttt{F41B}\\
\mSymbol[outlined]{multiple-stop} & \mSymbol[rounded]{multiple-stop} & \mSymbol[sharp]{multiple-stop} & \texttt{\textbackslash mSymbol\{multiple-stop\}} & \texttt{F1B9}\\
\mSymbol[outlined]{museum} & \mSymbol[rounded]{museum} & \mSymbol[sharp]{museum} & \texttt{\textbackslash mSymbol\{museum\}} & \texttt{EA36}\\
\mSymbol[outlined]{music-cast} & \mSymbol[rounded]{music-cast} & \mSymbol[sharp]{music-cast} & \texttt{\textbackslash mSymbol\{music-cast\}} & \texttt{EB1A}\\
\mSymbol[outlined]{music-note} & \mSymbol[rounded]{music-note} & \mSymbol[sharp]{music-note} & \texttt{\textbackslash mSymbol\{music-note\}} & \texttt{E405}\\
\mSymbol[outlined]{music-off} & \mSymbol[rounded]{music-off} & \mSymbol[sharp]{music-off} & \texttt{\textbackslash mSymbol\{music-off\}} & \texttt{E440}\\
\mSymbol[outlined]{music-video} & \mSymbol[rounded]{music-video} & \mSymbol[sharp]{music-video} & \texttt{\textbackslash mSymbol\{music-video\}} & \texttt{E063}\\
\mSymbol[outlined]{my-location} & \mSymbol[rounded]{my-location} & \mSymbol[sharp]{my-location} & \texttt{\textbackslash mSymbol\{my-location\}} & \texttt{E55C}\\
\mSymbol[outlined]{mystery} & \mSymbol[rounded]{mystery} & \mSymbol[sharp]{mystery} & \texttt{\textbackslash mSymbol\{mystery\}} & \texttt{F5E1}\\
\mSymbol[outlined]{nat} & \mSymbol[rounded]{nat} & \mSymbol[sharp]{nat} & \texttt{\textbackslash mSymbol\{nat\}} & \texttt{EF5C}\\
\mSymbol[outlined]{nature} & \mSymbol[rounded]{nature} & \mSymbol[sharp]{nature} & \texttt{\textbackslash mSymbol\{nature\}} & \texttt{E406}\\
\mSymbol[outlined]{nature-people} & \mSymbol[rounded]{nature-people} & \mSymbol[sharp]{nature-people} & \texttt{\textbackslash mSymbol\{nature-people\}} & \texttt{E407}\\
\mSymbol[outlined]{navigate-before} & \mSymbol[rounded]{navigate-before} & \mSymbol[sharp]{navigate-before} & \texttt{\textbackslash mSymbol\{navigate-before\}} & \texttt{E5CB}\\
\mSymbol[outlined]{navigate-next} & \mSymbol[rounded]{navigate-next} & \mSymbol[sharp]{navigate-next} & \texttt{\textbackslash mSymbol\{navigate-next\}} & \texttt{E5CC}\\
\mSymbol[outlined]{navigation} & \mSymbol[rounded]{navigation} & \mSymbol[sharp]{navigation} & \texttt{\textbackslash mSymbol\{navigation\}} & \texttt{E55D}\\
\mSymbol[outlined]{near-me} & \mSymbol[rounded]{near-me} & \mSymbol[sharp]{near-me} & \texttt{\textbackslash mSymbol\{near-me\}} & \texttt{E569}\\
\mSymbol[outlined]{near-me-disabled} & \mSymbol[rounded]{near-me-disabled} & \mSymbol[sharp]{near-me-disabled} & \texttt{\textbackslash mSymbol\{near-me-disabled\}} & \texttt{F1EF}\\
\mSymbol[outlined]{nearby} & \mSymbol[rounded]{nearby} & \mSymbol[sharp]{nearby} & \texttt{\textbackslash mSymbol\{nearby\}} & \texttt{E6B7}\\
\mSymbol[outlined]{nearby-error} & \mSymbol[rounded]{nearby-error} & \mSymbol[sharp]{nearby-error} & \texttt{\textbackslash mSymbol\{nearby-error\}} & \texttt{F03B}\\
\mSymbol[outlined]{nearby-off} & \mSymbol[rounded]{nearby-off} & \mSymbol[sharp]{nearby-off} & \texttt{\textbackslash mSymbol\{nearby-off\}} & \texttt{F03C}\\
\mSymbol[outlined]{nephrology} & \mSymbol[rounded]{nephrology} & \mSymbol[sharp]{nephrology} & \texttt{\textbackslash mSymbol\{nephrology\}} & \texttt{E10D}\\
\mSymbol[outlined]{nest-audio} & \mSymbol[rounded]{nest-audio} & \mSymbol[sharp]{nest-audio} & \texttt{\textbackslash mSymbol\{nest-audio\}} & \texttt{EBBF}\\
\mSymbol[outlined]{nest-cam-floodlight} & \mSymbol[rounded]{nest-cam-floodlight} & \mSymbol[sharp]{nest-cam-floodlight} & \texttt{\textbackslash mSymbol\{nest-cam-floodlight\}} & \texttt{F8B7}\\
\mSymbol[outlined]{nest-cam-indoor} & \mSymbol[rounded]{nest-cam-indoor} & \mSymbol[sharp]{nest-cam-indoor} & \texttt{\textbackslash mSymbol\{nest-cam-indoor\}} & \texttt{F11E}\\
\mSymbol[outlined]{nest-cam-iq} & \mSymbol[rounded]{nest-cam-iq} & \mSymbol[sharp]{nest-cam-iq} & \texttt{\textbackslash mSymbol\{nest-cam-iq\}} & \texttt{F11F}\\
\mSymbol[outlined]{nest-cam-iq-outdoor} & \mSymbol[rounded]{nest-cam-iq-outdoor} & \mSymbol[sharp]{nest-cam-iq-outdoor} & \texttt{\textbackslash mSymbol\{nest-cam-iq-outdoor\}} & \texttt{F120}\\
\mSymbol[outlined]{nest-cam-magnet-mount} & \mSymbol[rounded]{nest-cam-magnet-mount} & \mSymbol[sharp]{nest-cam-magnet-mount} & \texttt{\textbackslash mSymbol\{nest-cam-magnet-mount\}} & \texttt{F8B8}\\
\mSymbol[outlined]{nest-cam-outdoor} & \mSymbol[rounded]{nest-cam-outdoor} & \mSymbol[sharp]{nest-cam-outdoor} & \texttt{\textbackslash mSymbol\{nest-cam-outdoor\}} & \texttt{F121}\\
\mSymbol[outlined]{nest-cam-stand} & \mSymbol[rounded]{nest-cam-stand} & \mSymbol[sharp]{nest-cam-stand} & \texttt{\textbackslash mSymbol\{nest-cam-stand\}} & \texttt{F8B9}\\
\mSymbol[outlined]{nest-cam-wall-mount} & \mSymbol[rounded]{nest-cam-wall-mount} & \mSymbol[sharp]{nest-cam-wall-mount} & \texttt{\textbackslash mSymbol\{nest-cam-wall-mount\}} & \texttt{F8BA}\\
\mSymbol[outlined]{nest-cam-wired-stand} & \mSymbol[rounded]{nest-cam-wired-stand} & \mSymbol[sharp]{nest-cam-wired-stand} & \texttt{\textbackslash mSymbol\{nest-cam-wired-stand\}} & \texttt{EC16}\\
\mSymbol[outlined]{nest-clock-farsight-analog} & \mSymbol[rounded]{nest-clock-farsight-analog} & \mSymbol[sharp]{nest-clock-farsight-analog} & \texttt{\textbackslash mSymbol\{nest-clock-farsight-analog\}} & \texttt{F8BB}\\
\mSymbol[outlined]{nest-clock-farsight-digital} & \mSymbol[rounded]{nest-clock-farsight-digital} & \mSymbol[sharp]{nest-clock-farsight-digital} & \texttt{\textbackslash mSymbol\{nest-clock-farsight-digital\}} & \texttt{F8BC}\\
\mSymbol[outlined]{nest-connect} & \mSymbol[rounded]{nest-connect} & \mSymbol[sharp]{nest-connect} & \texttt{\textbackslash mSymbol\{nest-connect\}} & \texttt{F122}\\
\mSymbol[outlined]{nest-detect} & \mSymbol[rounded]{nest-detect} & \mSymbol[sharp]{nest-detect} & \texttt{\textbackslash mSymbol\{nest-detect\}} & \texttt{F123}\\
\mSymbol[outlined]{nest-display} & \mSymbol[rounded]{nest-display} & \mSymbol[sharp]{nest-display} & \texttt{\textbackslash mSymbol\{nest-display\}} & \texttt{F124}\\
\mSymbol[outlined]{nest-display-max} & \mSymbol[rounded]{nest-display-max} & \mSymbol[sharp]{nest-display-max} & \texttt{\textbackslash mSymbol\{nest-display-max\}} & \texttt{F125}\\
\mSymbol[outlined]{nest-doorbell-visitor} & \mSymbol[rounded]{nest-doorbell-visitor} & \mSymbol[sharp]{nest-doorbell-visitor} & \texttt{\textbackslash mSymbol\{nest-doorbell-visitor\}} & \texttt{F8BD}\\
\mSymbol[outlined]{nest-eco-leaf} & \mSymbol[rounded]{nest-eco-leaf} & \mSymbol[sharp]{nest-eco-leaf} & \texttt{\textbackslash mSymbol\{nest-eco-leaf\}} & \texttt{F8BE}\\
\mSymbol[outlined]{nest-farsight-weather} & \mSymbol[rounded]{nest-farsight-weather} & \mSymbol[sharp]{nest-farsight-weather} & \texttt{\textbackslash mSymbol\{nest-farsight-weather\}} & \texttt{F8BF}\\
\mSymbol[outlined]{nest-found-savings} & \mSymbol[rounded]{nest-found-savings} & \mSymbol[sharp]{nest-found-savings} & \texttt{\textbackslash mSymbol\{nest-found-savings\}} & \texttt{F8C0}\\
\mSymbol[outlined]{nest-gale-wifi} & \mSymbol[rounded]{nest-gale-wifi} & \mSymbol[sharp]{nest-gale-wifi} & \texttt{\textbackslash mSymbol\{nest-gale-wifi\}} & \texttt{F579}\\
\mSymbol[outlined]{nest-heat-link-e} & \mSymbol[rounded]{nest-heat-link-e} & \mSymbol[sharp]{nest-heat-link-e} & \texttt{\textbackslash mSymbol\{nest-heat-link-e\}} & \texttt{F126}\\
\mSymbol[outlined]{nest-heat-link-gen-3} & \mSymbol[rounded]{nest-heat-link-gen-3} & \mSymbol[sharp]{nest-heat-link-gen-3} & \texttt{\textbackslash mSymbol\{nest-heat-link-gen-3\}} & \texttt{F127}\\
\mSymbol[outlined]{nest-hello-doorbell} & \mSymbol[rounded]{nest-hello-doorbell} & \mSymbol[sharp]{nest-hello-doorbell} & \texttt{\textbackslash mSymbol\{nest-hello-doorbell\}} & \texttt{E82C}\\
\mSymbol[outlined]{nest-locator-tag} & \mSymbol[rounded]{nest-locator-tag} & \mSymbol[sharp]{nest-locator-tag} & \texttt{\textbackslash mSymbol\{nest-locator-tag\}} & \texttt{F8C1}\\
\mSymbol[outlined]{nest-mini} & \mSymbol[rounded]{nest-mini} & \mSymbol[sharp]{nest-mini} & \texttt{\textbackslash mSymbol\{nest-mini\}} & \texttt{E789}\\
\mSymbol[outlined]{nest-multi-room} & \mSymbol[rounded]{nest-multi-room} & \mSymbol[sharp]{nest-multi-room} & \texttt{\textbackslash mSymbol\{nest-multi-room\}} & \texttt{F8C2}\\
\mSymbol[outlined]{nest-protect} & \mSymbol[rounded]{nest-protect} & \mSymbol[sharp]{nest-protect} & \texttt{\textbackslash mSymbol\{nest-protect\}} & \texttt{E68E}\\
\mSymbol[outlined]{nest-remote} & \mSymbol[rounded]{nest-remote} & \mSymbol[sharp]{nest-remote} & \texttt{\textbackslash mSymbol\{nest-remote\}} & \texttt{F5DB}\\
\mSymbol[outlined]{nest-remote-comfort-sensor} & \mSymbol[rounded]{nest-remote-comfort-sensor} & \mSymbol[sharp]{nest-remote-comfort-sensor} & \texttt{\textbackslash mSymbol\{nest-remote-comfort-sensor\}} & \texttt{F12A}\\
\mSymbol[outlined]{nest-secure-alarm} & \mSymbol[rounded]{nest-secure-alarm} & \mSymbol[sharp]{nest-secure-alarm} & \texttt{\textbackslash mSymbol\{nest-secure-alarm\}} & \texttt{F12B}\\
\mSymbol[outlined]{nest-sunblock} & \mSymbol[rounded]{nest-sunblock} & \mSymbol[sharp]{nest-sunblock} & \texttt{\textbackslash mSymbol\{nest-sunblock\}} & \texttt{F8C3}\\
\mSymbol[outlined]{nest-tag} & \mSymbol[rounded]{nest-tag} & \mSymbol[sharp]{nest-tag} & \texttt{\textbackslash mSymbol\{nest-tag\}} & \texttt{F8C1}\\
\mSymbol[outlined]{nest-thermostat} & \mSymbol[rounded]{nest-thermostat} & \mSymbol[sharp]{nest-thermostat} & \texttt{\textbackslash mSymbol\{nest-thermostat\}} & \texttt{E68F}\\
\mSymbol[outlined]{nest-thermostat-e-eu} & \mSymbol[rounded]{nest-thermostat-e-eu} & \mSymbol[sharp]{nest-thermostat-e-eu} & \texttt{\textbackslash mSymbol\{nest-thermostat-e-eu\}} & \texttt{F12D}\\
\mSymbol[outlined]{nest-thermostat-gen-3} & \mSymbol[rounded]{nest-thermostat-gen-3} & \mSymbol[sharp]{nest-thermostat-gen-3} & \texttt{\textbackslash mSymbol\{nest-thermostat-gen-3\}} & \texttt{F12E}\\
\mSymbol[outlined]{nest-thermostat-sensor} & \mSymbol[rounded]{nest-thermostat-sensor} & \mSymbol[sharp]{nest-thermostat-sensor} & \texttt{\textbackslash mSymbol\{nest-thermostat-sensor\}} & \texttt{F12F}\\
\mSymbol[outlined]{nest-thermostat-sensor-eu} & \mSymbol[rounded]{nest-thermostat-sensor-eu} & \mSymbol[sharp]{nest-thermostat-sensor-eu} & \texttt{\textbackslash mSymbol\{nest-thermostat-sensor-eu\}} & \texttt{F130}\\
\mSymbol[outlined]{nest-thermostat-zirconium-eu} & \mSymbol[rounded]{nest-thermostat-zirconium-eu} & \mSymbol[sharp]{nest-thermostat-zirconium-eu} & \texttt{\textbackslash mSymbol\{nest-thermostat-zirconium-eu\}} & \texttt{F131}\\
\mSymbol[outlined]{nest-true-radiant} & \mSymbol[rounded]{nest-true-radiant} & \mSymbol[sharp]{nest-true-radiant} & \texttt{\textbackslash mSymbol\{nest-true-radiant\}} & \texttt{F8C4}\\
\mSymbol[outlined]{nest-wake-on-approach} & \mSymbol[rounded]{nest-wake-on-approach} & \mSymbol[sharp]{nest-wake-on-approach} & \texttt{\textbackslash mSymbol\{nest-wake-on-approach\}} & \texttt{F8C5}\\
\mSymbol[outlined]{nest-wake-on-press} & \mSymbol[rounded]{nest-wake-on-press} & \mSymbol[sharp]{nest-wake-on-press} & \texttt{\textbackslash mSymbol\{nest-wake-on-press\}} & \texttt{F8C6}\\
\mSymbol[outlined]{nest-wifi-gale} & \mSymbol[rounded]{nest-wifi-gale} & \mSymbol[sharp]{nest-wifi-gale} & \texttt{\textbackslash mSymbol\{nest-wifi-gale\}} & \texttt{F132}\\
\mSymbol[outlined]{nest-wifi-mistral} & \mSymbol[rounded]{nest-wifi-mistral} & \mSymbol[sharp]{nest-wifi-mistral} & \texttt{\textbackslash mSymbol\{nest-wifi-mistral\}} & \texttt{F133}\\
\mSymbol[outlined]{nest-wifi-point} & \mSymbol[rounded]{nest-wifi-point} & \mSymbol[sharp]{nest-wifi-point} & \texttt{\textbackslash mSymbol\{nest-wifi-point\}} & \texttt{F134}\\
\mSymbol[outlined]{nest-wifi-point-vento} & \mSymbol[rounded]{nest-wifi-point-vento} & \mSymbol[sharp]{nest-wifi-point-vento} & \texttt{\textbackslash mSymbol\{nest-wifi-point-vento\}} & \texttt{F134}\\
\mSymbol[outlined]{nest-wifi-pro} & \mSymbol[rounded]{nest-wifi-pro} & \mSymbol[sharp]{nest-wifi-pro} & \texttt{\textbackslash mSymbol\{nest-wifi-pro\}} & \texttt{F56B}\\
\mSymbol[outlined]{nest-wifi-pro-2} & \mSymbol[rounded]{nest-wifi-pro-2} & \mSymbol[sharp]{nest-wifi-pro-2} & \texttt{\textbackslash mSymbol\{nest-wifi-pro-2\}} & \texttt{F56A}\\
\mSymbol[outlined]{nest-wifi-router} & \mSymbol[rounded]{nest-wifi-router} & \mSymbol[sharp]{nest-wifi-router} & \texttt{\textbackslash mSymbol\{nest-wifi-router\}} & \texttt{F133}\\
\mSymbol[outlined]{network-cell} & \mSymbol[rounded]{network-cell} & \mSymbol[sharp]{network-cell} & \texttt{\textbackslash mSymbol\{network-cell\}} & \texttt{E1B9}\\
\mSymbol[outlined]{network-check} & \mSymbol[rounded]{network-check} & \mSymbol[sharp]{network-check} & \texttt{\textbackslash mSymbol\{network-check\}} & \texttt{E640}\\
\mSymbol[outlined]{network-intelligence-history} & \mSymbol[rounded]{network-intelligence-history} & \mSymbol[sharp]{network-intelligence-history} & \texttt{\textbackslash mSymbol\{network-intelligence-history\}} & \texttt{F5F6}\\
\mSymbol[outlined]{network-intelligence-update} & \mSymbol[rounded]{network-intelligence-update} & \mSymbol[sharp]{network-intelligence-update} & \texttt{\textbackslash mSymbol\{network-intelligence-update\}} & \texttt{F5F5}\\
\mSymbol[outlined]{network-locked} & \mSymbol[rounded]{network-locked} & \mSymbol[sharp]{network-locked} & \texttt{\textbackslash mSymbol\{network-locked\}} & \texttt{E61A}\\
\mSymbol[outlined]{network-manage} & \mSymbol[rounded]{network-manage} & \mSymbol[sharp]{network-manage} & \texttt{\textbackslash mSymbol\{network-manage\}} & \texttt{F7AB}\\
\mSymbol[outlined]{network-node} & \mSymbol[rounded]{network-node} & \mSymbol[sharp]{network-node} & \texttt{\textbackslash mSymbol\{network-node\}} & \texttt{F56E}\\
\mSymbol[outlined]{network-ping} & \mSymbol[rounded]{network-ping} & \mSymbol[sharp]{network-ping} & \texttt{\textbackslash mSymbol\{network-ping\}} & \texttt{EBCA}\\
\mSymbol[outlined]{network-wifi} & \mSymbol[rounded]{network-wifi} & \mSymbol[sharp]{network-wifi} & \texttt{\textbackslash mSymbol\{network-wifi\}} & \texttt{E1BA}\\
\mSymbol[outlined]{network-wifi-1-bar} & \mSymbol[rounded]{network-wifi-1-bar} & \mSymbol[sharp]{network-wifi-1-bar} & \texttt{\textbackslash mSymbol\{network-wifi-1-bar\}} & \texttt{EBE4}\\
\mSymbol[outlined]{network-wifi-1-bar-locked} & \mSymbol[rounded]{network-wifi-1-bar-locked} & \mSymbol[sharp]{network-wifi-1-bar-locked} & \texttt{\textbackslash mSymbol\{network-wifi-1-bar-locked\}} & \texttt{F58F}\\
\mSymbol[outlined]{network-wifi-2-bar} & \mSymbol[rounded]{network-wifi-2-bar} & \mSymbol[sharp]{network-wifi-2-bar} & \texttt{\textbackslash mSymbol\{network-wifi-2-bar\}} & \texttt{EBD6}\\
\mSymbol[outlined]{network-wifi-2-bar-locked} & \mSymbol[rounded]{network-wifi-2-bar-locked} & \mSymbol[sharp]{network-wifi-2-bar-locked} & \texttt{\textbackslash mSymbol\{network-wifi-2-bar-locked\}} & \texttt{F58E}\\
\mSymbol[outlined]{network-wifi-3-bar} & \mSymbol[rounded]{network-wifi-3-bar} & \mSymbol[sharp]{network-wifi-3-bar} & \texttt{\textbackslash mSymbol\{network-wifi-3-bar\}} & \texttt{EBE1}\\
\mSymbol[outlined]{network-wifi-3-bar-locked} & \mSymbol[rounded]{network-wifi-3-bar-locked} & \mSymbol[sharp]{network-wifi-3-bar-locked} & \texttt{\textbackslash mSymbol\{network-wifi-3-bar-locked\}} & \texttt{F58D}\\
\mSymbol[outlined]{network-wifi-locked} & \mSymbol[rounded]{network-wifi-locked} & \mSymbol[sharp]{network-wifi-locked} & \texttt{\textbackslash mSymbol\{network-wifi-locked\}} & \texttt{F532}\\
\mSymbol[outlined]{neurology} & \mSymbol[rounded]{neurology} & \mSymbol[sharp]{neurology} & \texttt{\textbackslash mSymbol\{neurology\}} & \texttt{E10E}\\
\mSymbol[outlined]{new-label} & \mSymbol[rounded]{new-label} & \mSymbol[sharp]{new-label} & \texttt{\textbackslash mSymbol\{new-label\}} & \texttt{E609}\\
\mSymbol[outlined]{new-releases} & \mSymbol[rounded]{new-releases} & \mSymbol[sharp]{new-releases} & \texttt{\textbackslash mSymbol\{new-releases\}} & \texttt{EF76}\\
\mSymbol[outlined]{new-window} & \mSymbol[rounded]{new-window} & \mSymbol[sharp]{new-window} & \texttt{\textbackslash mSymbol\{new-window\}} & \texttt{F710}\\
\mSymbol[outlined]{news} & \mSymbol[rounded]{news} & \mSymbol[sharp]{news} & \texttt{\textbackslash mSymbol\{news\}} & \texttt{E032}\\
\mSymbol[outlined]{newsmode} & \mSymbol[rounded]{newsmode} & \mSymbol[sharp]{newsmode} & \texttt{\textbackslash mSymbol\{newsmode\}} & \texttt{EFAD}\\
\mSymbol[outlined]{newspaper} & \mSymbol[rounded]{newspaper} & \mSymbol[sharp]{newspaper} & \texttt{\textbackslash mSymbol\{newspaper\}} & \texttt{EB81}\\
\mSymbol[outlined]{newsstand} & \mSymbol[rounded]{newsstand} & \mSymbol[sharp]{newsstand} & \texttt{\textbackslash mSymbol\{newsstand\}} & \texttt{E9C4}\\
\mSymbol[outlined]{next-plan} & \mSymbol[rounded]{next-plan} & \mSymbol[sharp]{next-plan} & \texttt{\textbackslash mSymbol\{next-plan\}} & \texttt{EF5D}\\
\mSymbol[outlined]{next-week} & \mSymbol[rounded]{next-week} & \mSymbol[sharp]{next-week} & \texttt{\textbackslash mSymbol\{next-week\}} & \texttt{E16A}\\
\mSymbol[outlined]{nfc} & \mSymbol[rounded]{nfc} & \mSymbol[sharp]{nfc} & \texttt{\textbackslash mSymbol\{nfc\}} & \texttt{E1BB}\\
\mSymbol[outlined]{night-shelter} & \mSymbol[rounded]{night-shelter} & \mSymbol[sharp]{night-shelter} & \texttt{\textbackslash mSymbol\{night-shelter\}} & \texttt{F1F1}\\
\mSymbol[outlined]{night-sight-auto} & \mSymbol[rounded]{night-sight-auto} & \mSymbol[sharp]{night-sight-auto} & \texttt{\textbackslash mSymbol\{night-sight-auto\}} & \texttt{F1D7}\\
\mSymbol[outlined]{night-sight-auto-off} & \mSymbol[rounded]{night-sight-auto-off} & \mSymbol[sharp]{night-sight-auto-off} & \texttt{\textbackslash mSymbol\{night-sight-auto-off\}} & \texttt{F1F9}\\
\mSymbol[outlined]{night-sight-max} & \mSymbol[rounded]{night-sight-max} & \mSymbol[sharp]{night-sight-max} & \texttt{\textbackslash mSymbol\{night-sight-max\}} & \texttt{F6C3}\\
\mSymbol[outlined]{nightlife} & \mSymbol[rounded]{nightlife} & \mSymbol[sharp]{nightlife} & \texttt{\textbackslash mSymbol\{nightlife\}} & \texttt{EA62}\\
\mSymbol[outlined]{nightlight} & \mSymbol[rounded]{nightlight} & \mSymbol[sharp]{nightlight} & \texttt{\textbackslash mSymbol\{nightlight\}} & \texttt{F03D}\\
\mSymbol[outlined]{nightlight-round} & \mSymbol[rounded]{nightlight-round} & \mSymbol[sharp]{nightlight-round} & \texttt{\textbackslash mSymbol\{nightlight-round\}} & \texttt{F03D}\\
\mSymbol[outlined]{nights-stay} & \mSymbol[rounded]{nights-stay} & \mSymbol[sharp]{nights-stay} & \texttt{\textbackslash mSymbol\{nights-stay\}} & \texttt{EA46}\\
\mSymbol[outlined]{no-accounts} & \mSymbol[rounded]{no-accounts} & \mSymbol[sharp]{no-accounts} & \texttt{\textbackslash mSymbol\{no-accounts\}} & \texttt{F03E}\\
\mSymbol[outlined]{no-adult-content} & \mSymbol[rounded]{no-adult-content} & \mSymbol[sharp]{no-adult-content} & \texttt{\textbackslash mSymbol\{no-adult-content\}} & \texttt{F8FE}\\
\mSymbol[outlined]{no-backpack} & \mSymbol[rounded]{no-backpack} & \mSymbol[sharp]{no-backpack} & \texttt{\textbackslash mSymbol\{no-backpack\}} & \texttt{F237}\\
\mSymbol[outlined]{no-crash} & \mSymbol[rounded]{no-crash} & \mSymbol[sharp]{no-crash} & \texttt{\textbackslash mSymbol\{no-crash\}} & \texttt{EBF0}\\
\mSymbol[outlined]{no-drinks} & \mSymbol[rounded]{no-drinks} & \mSymbol[sharp]{no-drinks} & \texttt{\textbackslash mSymbol\{no-drinks\}} & \texttt{F1A5}\\
\mSymbol[outlined]{no-encryption} & \mSymbol[rounded]{no-encryption} & \mSymbol[sharp]{no-encryption} & \texttt{\textbackslash mSymbol\{no-encryption\}} & \texttt{F03F}\\
\mSymbol[outlined]{no-encryption-gmailerrorred} & \mSymbol[rounded]{no-encryption-gmailerrorred} & \mSymbol[sharp]{no-encryption-gmailerrorred} & \texttt{\textbackslash mSymbol\{no-encryption-gmailerrorred\}} & \texttt{F03F}\\
\mSymbol[outlined]{no-flash} & \mSymbol[rounded]{no-flash} & \mSymbol[sharp]{no-flash} & \texttt{\textbackslash mSymbol\{no-flash\}} & \texttt{F1A6}\\
\mSymbol[outlined]{no-food} & \mSymbol[rounded]{no-food} & \mSymbol[sharp]{no-food} & \texttt{\textbackslash mSymbol\{no-food\}} & \texttt{F1A7}\\
\mSymbol[outlined]{no-luggage} & \mSymbol[rounded]{no-luggage} & \mSymbol[sharp]{no-luggage} & \texttt{\textbackslash mSymbol\{no-luggage\}} & \texttt{F23B}\\
\mSymbol[outlined]{no-meals} & \mSymbol[rounded]{no-meals} & \mSymbol[sharp]{no-meals} & \texttt{\textbackslash mSymbol\{no-meals\}} & \texttt{F1D6}\\
\mSymbol[outlined]{no-meeting-room} & \mSymbol[rounded]{no-meeting-room} & \mSymbol[sharp]{no-meeting-room} & \texttt{\textbackslash mSymbol\{no-meeting-room\}} & \texttt{EB4E}\\
\mSymbol[outlined]{no-photography} & \mSymbol[rounded]{no-photography} & \mSymbol[sharp]{no-photography} & \texttt{\textbackslash mSymbol\{no-photography\}} & \texttt{F1A8}\\
\mSymbol[outlined]{no-sim} & \mSymbol[rounded]{no-sim} & \mSymbol[sharp]{no-sim} & \texttt{\textbackslash mSymbol\{no-sim\}} & \texttt{E1CE}\\
\mSymbol[outlined]{no-sound} & \mSymbol[rounded]{no-sound} & \mSymbol[sharp]{no-sound} & \texttt{\textbackslash mSymbol\{no-sound\}} & \texttt{E710}\\
\mSymbol[outlined]{no-stroller} & \mSymbol[rounded]{no-stroller} & \mSymbol[sharp]{no-stroller} & \texttt{\textbackslash mSymbol\{no-stroller\}} & \texttt{F1AF}\\
\mSymbol[outlined]{no-transfer} & \mSymbol[rounded]{no-transfer} & \mSymbol[sharp]{no-transfer} & \texttt{\textbackslash mSymbol\{no-transfer\}} & \texttt{F1D5}\\
\mSymbol[outlined]{noise-aware} & \mSymbol[rounded]{noise-aware} & \mSymbol[sharp]{noise-aware} & \texttt{\textbackslash mSymbol\{noise-aware\}} & \texttt{EBEC}\\
\mSymbol[outlined]{noise-control-off} & \mSymbol[rounded]{noise-control-off} & \mSymbol[sharp]{noise-control-off} & \texttt{\textbackslash mSymbol\{noise-control-off\}} & \texttt{EBF3}\\
\mSymbol[outlined]{noise-control-on} & \mSymbol[rounded]{noise-control-on} & \mSymbol[sharp]{noise-control-on} & \texttt{\textbackslash mSymbol\{noise-control-on\}} & \texttt{F8A8}\\
\mSymbol[outlined]{nordic-walking} & \mSymbol[rounded]{nordic-walking} & \mSymbol[sharp]{nordic-walking} & \texttt{\textbackslash mSymbol\{nordic-walking\}} & \texttt{E50E}\\
\mSymbol[outlined]{north} & \mSymbol[rounded]{north} & \mSymbol[sharp]{north} & \texttt{\textbackslash mSymbol\{north\}} & \texttt{F1E0}\\
\mSymbol[outlined]{north-east} & \mSymbol[rounded]{north-east} & \mSymbol[sharp]{north-east} & \texttt{\textbackslash mSymbol\{north-east\}} & \texttt{F1E1}\\
\mSymbol[outlined]{north-west} & \mSymbol[rounded]{north-west} & \mSymbol[sharp]{north-west} & \texttt{\textbackslash mSymbol\{north-west\}} & \texttt{F1E2}\\
\mSymbol[outlined]{not-accessible} & \mSymbol[rounded]{not-accessible} & \mSymbol[sharp]{not-accessible} & \texttt{\textbackslash mSymbol\{not-accessible\}} & \texttt{F0FE}\\
\mSymbol[outlined]{not-accessible-forward} & \mSymbol[rounded]{not-accessible-forward} & \mSymbol[sharp]{not-accessible-forward} & \texttt{\textbackslash mSymbol\{not-accessible-forward\}} & \texttt{F54A}\\
\mSymbol[outlined]{not-interested} & \mSymbol[rounded]{not-interested} & \mSymbol[sharp]{not-interested} & \texttt{\textbackslash mSymbol\{not-interested\}} & \texttt{F08C}\\
\mSymbol[outlined]{not-listed-location} & \mSymbol[rounded]{not-listed-location} & \mSymbol[sharp]{not-listed-location} & \texttt{\textbackslash mSymbol\{not-listed-location\}} & \texttt{E575}\\
\mSymbol[outlined]{not-started} & \mSymbol[rounded]{not-started} & \mSymbol[sharp]{not-started} & \texttt{\textbackslash mSymbol\{not-started\}} & \texttt{F0D1}\\
\mSymbol[outlined]{note} & \mSymbol[rounded]{note} & \mSymbol[sharp]{note} & \texttt{\textbackslash mSymbol\{note\}} & \texttt{E66D}\\
\mSymbol[outlined]{note-add} & \mSymbol[rounded]{note-add} & \mSymbol[sharp]{note-add} & \texttt{\textbackslash mSymbol\{note-add\}} & \texttt{E89C}\\
\mSymbol[outlined]{note-alt} & \mSymbol[rounded]{note-alt} & \mSymbol[sharp]{note-alt} & \texttt{\textbackslash mSymbol\{note-alt\}} & \texttt{F040}\\
\mSymbol[outlined]{note-stack} & \mSymbol[rounded]{note-stack} & \mSymbol[sharp]{note-stack} & \texttt{\textbackslash mSymbol\{note-stack\}} & \texttt{F562}\\
\mSymbol[outlined]{note-stack-add} & \mSymbol[rounded]{note-stack-add} & \mSymbol[sharp]{note-stack-add} & \texttt{\textbackslash mSymbol\{note-stack-add\}} & \texttt{F563}\\
\mSymbol[outlined]{notes} & \mSymbol[rounded]{notes} & \mSymbol[sharp]{notes} & \texttt{\textbackslash mSymbol\{notes\}} & \texttt{E26C}\\
\mSymbol[outlined]{notification-add} & \mSymbol[rounded]{notification-add} & \mSymbol[sharp]{notification-add} & \texttt{\textbackslash mSymbol\{notification-add\}} & \texttt{E399}\\
\mSymbol[outlined]{notification-important} & \mSymbol[rounded]{notification-important} & \mSymbol[sharp]{notification-important} & \texttt{\textbackslash mSymbol\{notification-important\}} & \texttt{E004}\\
\mSymbol[outlined]{notification-multiple} & \mSymbol[rounded]{notification-multiple} & \mSymbol[sharp]{notification-multiple} & \texttt{\textbackslash mSymbol\{notification-multiple\}} & \texttt{E6C2}\\
\mSymbol[outlined]{notifications} & \mSymbol[rounded]{notifications} & \mSymbol[sharp]{notifications} & \texttt{\textbackslash mSymbol\{notifications\}} & \texttt{E7F5}\\
\mSymbol[outlined]{notifications-active} & \mSymbol[rounded]{notifications-active} & \mSymbol[sharp]{notifications-active} & \texttt{\textbackslash mSymbol\{notifications-active\}} & \texttt{E7F7}\\
\mSymbol[outlined]{notifications-none} & \mSymbol[rounded]{notifications-none} & \mSymbol[sharp]{notifications-none} & \texttt{\textbackslash mSymbol\{notifications-none\}} & \texttt{E7F5}\\
\mSymbol[outlined]{notifications-off} & \mSymbol[rounded]{notifications-off} & \mSymbol[sharp]{notifications-off} & \texttt{\textbackslash mSymbol\{notifications-off\}} & \texttt{E7F6}\\
\mSymbol[outlined]{notifications-paused} & \mSymbol[rounded]{notifications-paused} & \mSymbol[sharp]{notifications-paused} & \texttt{\textbackslash mSymbol\{notifications-paused\}} & \texttt{E7F8}\\
\mSymbol[outlined]{notifications-unread} & \mSymbol[rounded]{notifications-unread} & \mSymbol[sharp]{notifications-unread} & \texttt{\textbackslash mSymbol\{notifications-unread\}} & \texttt{F4FE}\\
\mSymbol[outlined]{numbers} & \mSymbol[rounded]{numbers} & \mSymbol[sharp]{numbers} & \texttt{\textbackslash mSymbol\{numbers\}} & \texttt{EAC7}\\
\mSymbol[outlined]{nutrition} & \mSymbol[rounded]{nutrition} & \mSymbol[sharp]{nutrition} & \texttt{\textbackslash mSymbol\{nutrition\}} & \texttt{E110}\\
\mSymbol[outlined]{ods} & \mSymbol[rounded]{ods} & \mSymbol[sharp]{ods} & \texttt{\textbackslash mSymbol\{ods\}} & \texttt{E6E8}\\
\mSymbol[outlined]{odt} & \mSymbol[rounded]{odt} & \mSymbol[sharp]{odt} & \texttt{\textbackslash mSymbol\{odt\}} & \texttt{E6E9}\\
\mSymbol[outlined]{offline-bolt} & \mSymbol[rounded]{offline-bolt} & \mSymbol[sharp]{offline-bolt} & \texttt{\textbackslash mSymbol\{offline-bolt\}} & \texttt{E932}\\
\mSymbol[outlined]{offline-pin} & \mSymbol[rounded]{offline-pin} & \mSymbol[sharp]{offline-pin} & \texttt{\textbackslash mSymbol\{offline-pin\}} & \texttt{E90A}\\
\mSymbol[outlined]{offline-pin-off} & \mSymbol[rounded]{offline-pin-off} & \mSymbol[sharp]{offline-pin-off} & \texttt{\textbackslash mSymbol\{offline-pin-off\}} & \texttt{F4D0}\\
\mSymbol[outlined]{offline-share} & \mSymbol[rounded]{offline-share} & \mSymbol[sharp]{offline-share} & \texttt{\textbackslash mSymbol\{offline-share\}} & \texttt{E9C5}\\
\mSymbol[outlined]{oil-barrel} & \mSymbol[rounded]{oil-barrel} & \mSymbol[sharp]{oil-barrel} & \texttt{\textbackslash mSymbol\{oil-barrel\}} & \texttt{EC15}\\
\mSymbol[outlined]{on-device-training} & \mSymbol[rounded]{on-device-training} & \mSymbol[sharp]{on-device-training} & \texttt{\textbackslash mSymbol\{on-device-training\}} & \texttt{EBFD}\\
\mSymbol[outlined]{on-hub-device} & \mSymbol[rounded]{on-hub-device} & \mSymbol[sharp]{on-hub-device} & \texttt{\textbackslash mSymbol\{on-hub-device\}} & \texttt{E6C3}\\
\mSymbol[outlined]{oncology} & \mSymbol[rounded]{oncology} & \mSymbol[sharp]{oncology} & \texttt{\textbackslash mSymbol\{oncology\}} & \texttt{E114}\\
\mSymbol[outlined]{ondemand-video} & \mSymbol[rounded]{ondemand-video} & \mSymbol[sharp]{ondemand-video} & \texttt{\textbackslash mSymbol\{ondemand-video\}} & \texttt{E63A}\\
\mSymbol[outlined]{online-prediction} & \mSymbol[rounded]{online-prediction} & \mSymbol[sharp]{online-prediction} & \texttt{\textbackslash mSymbol\{online-prediction\}} & \texttt{F0EB}\\
\mSymbol[outlined]{onsen} & \mSymbol[rounded]{onsen} & \mSymbol[sharp]{onsen} & \texttt{\textbackslash mSymbol\{onsen\}} & \texttt{F6F8}\\
\mSymbol[outlined]{opacity} & \mSymbol[rounded]{opacity} & \mSymbol[sharp]{opacity} & \texttt{\textbackslash mSymbol\{opacity\}} & \texttt{E91C}\\
\mSymbol[outlined]{open-in-browser} & \mSymbol[rounded]{open-in-browser} & \mSymbol[sharp]{open-in-browser} & \texttt{\textbackslash mSymbol\{open-in-browser\}} & \texttt{E89D}\\
\mSymbol[outlined]{open-in-full} & \mSymbol[rounded]{open-in-full} & \mSymbol[sharp]{open-in-full} & \texttt{\textbackslash mSymbol\{open-in-full\}} & \texttt{F1CE}\\
\mSymbol[outlined]{open-in-new} & \mSymbol[rounded]{open-in-new} & \mSymbol[sharp]{open-in-new} & \texttt{\textbackslash mSymbol\{open-in-new\}} & \texttt{E89E}\\
\mSymbol[outlined]{open-in-new-down} & \mSymbol[rounded]{open-in-new-down} & \mSymbol[sharp]{open-in-new-down} & \texttt{\textbackslash mSymbol\{open-in-new-down\}} & \texttt{F70F}\\
\mSymbol[outlined]{open-in-new-off} & \mSymbol[rounded]{open-in-new-off} & \mSymbol[sharp]{open-in-new-off} & \texttt{\textbackslash mSymbol\{open-in-new-off\}} & \texttt{E4F6}\\
\mSymbol[outlined]{open-in-phone} & \mSymbol[rounded]{open-in-phone} & \mSymbol[sharp]{open-in-phone} & \texttt{\textbackslash mSymbol\{open-in-phone\}} & \texttt{E702}\\
\mSymbol[outlined]{open-jam} & \mSymbol[rounded]{open-jam} & \mSymbol[sharp]{open-jam} & \texttt{\textbackslash mSymbol\{open-jam\}} & \texttt{EFAE}\\
\mSymbol[outlined]{open-run} & \mSymbol[rounded]{open-run} & \mSymbol[sharp]{open-run} & \texttt{\textbackslash mSymbol\{open-run\}} & \texttt{F4B7}\\
\mSymbol[outlined]{open-with} & \mSymbol[rounded]{open-with} & \mSymbol[sharp]{open-with} & \texttt{\textbackslash mSymbol\{open-with\}} & \texttt{E89F}\\
\mSymbol[outlined]{ophthalmology} & \mSymbol[rounded]{ophthalmology} & \mSymbol[sharp]{ophthalmology} & \texttt{\textbackslash mSymbol\{ophthalmology\}} & \texttt{E115}\\
\mSymbol[outlined]{oral-disease} & \mSymbol[rounded]{oral-disease} & \mSymbol[sharp]{oral-disease} & \texttt{\textbackslash mSymbol\{oral-disease\}} & \texttt{E116}\\
\mSymbol[outlined]{orbit} & \mSymbol[rounded]{orbit} & \mSymbol[sharp]{orbit} & \texttt{\textbackslash mSymbol\{orbit\}} & \texttt{F426}\\
\mSymbol[outlined]{order-approve} & \mSymbol[rounded]{order-approve} & \mSymbol[sharp]{order-approve} & \texttt{\textbackslash mSymbol\{order-approve\}} & \texttt{F812}\\
\mSymbol[outlined]{order-play} & \mSymbol[rounded]{order-play} & \mSymbol[sharp]{order-play} & \texttt{\textbackslash mSymbol\{order-play\}} & \texttt{F811}\\
\mSymbol[outlined]{orders} & \mSymbol[rounded]{orders} & \mSymbol[sharp]{orders} & \texttt{\textbackslash mSymbol\{orders\}} & \texttt{EB14}\\
\mSymbol[outlined]{orthopedics} & \mSymbol[rounded]{orthopedics} & \mSymbol[sharp]{orthopedics} & \texttt{\textbackslash mSymbol\{orthopedics\}} & \texttt{F897}\\
\mSymbol[outlined]{other-admission} & \mSymbol[rounded]{other-admission} & \mSymbol[sharp]{other-admission} & \texttt{\textbackslash mSymbol\{other-admission\}} & \texttt{E47B}\\
\mSymbol[outlined]{other-houses} & \mSymbol[rounded]{other-houses} & \mSymbol[sharp]{other-houses} & \texttt{\textbackslash mSymbol\{other-houses\}} & \texttt{E58C}\\
\mSymbol[outlined]{outbound} & \mSymbol[rounded]{outbound} & \mSymbol[sharp]{outbound} & \texttt{\textbackslash mSymbol\{outbound\}} & \texttt{E1CA}\\
\mSymbol[outlined]{outbox} & \mSymbol[rounded]{outbox} & \mSymbol[sharp]{outbox} & \texttt{\textbackslash mSymbol\{outbox\}} & \texttt{EF5F}\\
\mSymbol[outlined]{outbox-alt} & \mSymbol[rounded]{outbox-alt} & \mSymbol[sharp]{outbox-alt} & \texttt{\textbackslash mSymbol\{outbox-alt\}} & \texttt{EB17}\\
\mSymbol[outlined]{outdoor-garden} & \mSymbol[rounded]{outdoor-garden} & \mSymbol[sharp]{outdoor-garden} & \texttt{\textbackslash mSymbol\{outdoor-garden\}} & \texttt{E205}\\
\mSymbol[outlined]{outdoor-grill} & \mSymbol[rounded]{outdoor-grill} & \mSymbol[sharp]{outdoor-grill} & \texttt{\textbackslash mSymbol\{outdoor-grill\}} & \texttt{EA47}\\
\mSymbol[outlined]{outgoing-mail} & \mSymbol[rounded]{outgoing-mail} & \mSymbol[sharp]{outgoing-mail} & \texttt{\textbackslash mSymbol\{outgoing-mail\}} & \texttt{F0D2}\\
\mSymbol[outlined]{outlet} & \mSymbol[rounded]{outlet} & \mSymbol[sharp]{outlet} & \texttt{\textbackslash mSymbol\{outlet\}} & \texttt{F1D4}\\
\mSymbol[outlined]{outlined-flag} & \mSymbol[rounded]{outlined-flag} & \mSymbol[sharp]{outlined-flag} & \texttt{\textbackslash mSymbol\{outlined-flag\}} & \texttt{F0C6}\\
\mSymbol[outlined]{outpatient} & \mSymbol[rounded]{outpatient} & \mSymbol[sharp]{outpatient} & \texttt{\textbackslash mSymbol\{outpatient\}} & \texttt{E118}\\
\mSymbol[outlined]{outpatient-med} & \mSymbol[rounded]{outpatient-med} & \mSymbol[sharp]{outpatient-med} & \texttt{\textbackslash mSymbol\{outpatient-med\}} & \texttt{E119}\\
\mSymbol[outlined]{output} & \mSymbol[rounded]{output} & \mSymbol[sharp]{output} & \texttt{\textbackslash mSymbol\{output\}} & \texttt{EBBE}\\
\mSymbol[outlined]{output-circle} & \mSymbol[rounded]{output-circle} & \mSymbol[sharp]{output-circle} & \texttt{\textbackslash mSymbol\{output-circle\}} & \texttt{F70E}\\
\mSymbol[outlined]{oven} & \mSymbol[rounded]{oven} & \mSymbol[sharp]{oven} & \texttt{\textbackslash mSymbol\{oven\}} & \texttt{E9C7}\\
\mSymbol[outlined]{oven-gen} & \mSymbol[rounded]{oven-gen} & \mSymbol[sharp]{oven-gen} & \texttt{\textbackslash mSymbol\{oven-gen\}} & \texttt{E843}\\
\mSymbol[outlined]{overview} & \mSymbol[rounded]{overview} & \mSymbol[sharp]{overview} & \texttt{\textbackslash mSymbol\{overview\}} & \texttt{E4A7}\\
\mSymbol[outlined]{overview-key} & \mSymbol[rounded]{overview-key} & \mSymbol[sharp]{overview-key} & \texttt{\textbackslash mSymbol\{overview-key\}} & \texttt{F7D4}\\
\mSymbol[outlined]{oxygen-saturation} & \mSymbol[rounded]{oxygen-saturation} & \mSymbol[sharp]{oxygen-saturation} & \texttt{\textbackslash mSymbol\{oxygen-saturation\}} & \texttt{E4DE}\\
\mSymbol[outlined]{p2p} & \mSymbol[rounded]{p2p} & \mSymbol[sharp]{p2p} & \texttt{\textbackslash mSymbol\{p2p\}} & \texttt{F52A}\\
\mSymbol[outlined]{pace} & \mSymbol[rounded]{pace} & \mSymbol[sharp]{pace} & \texttt{\textbackslash mSymbol\{pace\}} & \texttt{F6B8}\\
\mSymbol[outlined]{pacemaker} & \mSymbol[rounded]{pacemaker} & \mSymbol[sharp]{pacemaker} & \texttt{\textbackslash mSymbol\{pacemaker\}} & \texttt{E656}\\
\mSymbol[outlined]{package} & \mSymbol[rounded]{package} & \mSymbol[sharp]{package} & \texttt{\textbackslash mSymbol\{package\}} & \texttt{E48F}\\
\mSymbol[outlined]{package-2} & \mSymbol[rounded]{package-2} & \mSymbol[sharp]{package-2} & \texttt{\textbackslash mSymbol\{package-2\}} & \texttt{F569}\\
\mSymbol[outlined]{padding} & \mSymbol[rounded]{padding} & \mSymbol[sharp]{padding} & \texttt{\textbackslash mSymbol\{padding\}} & \texttt{E9C8}\\
\mSymbol[outlined]{page-control} & \mSymbol[rounded]{page-control} & \mSymbol[sharp]{page-control} & \texttt{\textbackslash mSymbol\{page-control\}} & \texttt{E731}\\
\mSymbol[outlined]{page-info} & \mSymbol[rounded]{page-info} & \mSymbol[sharp]{page-info} & \texttt{\textbackslash mSymbol\{page-info\}} & \texttt{F614}\\
\mSymbol[outlined]{pageless} & \mSymbol[rounded]{pageless} & \mSymbol[sharp]{pageless} & \texttt{\textbackslash mSymbol\{pageless\}} & \texttt{F509}\\
\mSymbol[outlined]{pages} & \mSymbol[rounded]{pages} & \mSymbol[sharp]{pages} & \texttt{\textbackslash mSymbol\{pages\}} & \texttt{E7F9}\\
\mSymbol[outlined]{pageview} & \mSymbol[rounded]{pageview} & \mSymbol[sharp]{pageview} & \texttt{\textbackslash mSymbol\{pageview\}} & \texttt{E8A0}\\
\mSymbol[outlined]{paid} & \mSymbol[rounded]{paid} & \mSymbol[sharp]{paid} & \texttt{\textbackslash mSymbol\{paid\}} & \texttt{F041}\\
\mSymbol[outlined]{palette} & \mSymbol[rounded]{palette} & \mSymbol[sharp]{palette} & \texttt{\textbackslash mSymbol\{palette\}} & \texttt{E40A}\\
\mSymbol[outlined]{pallet} & \mSymbol[rounded]{pallet} & \mSymbol[sharp]{pallet} & \texttt{\textbackslash mSymbol\{pallet\}} & \texttt{F86A}\\
\mSymbol[outlined]{pan-tool} & \mSymbol[rounded]{pan-tool} & \mSymbol[sharp]{pan-tool} & \texttt{\textbackslash mSymbol\{pan-tool\}} & \texttt{E925}\\
\mSymbol[outlined]{pan-tool-alt} & \mSymbol[rounded]{pan-tool-alt} & \mSymbol[sharp]{pan-tool-alt} & \texttt{\textbackslash mSymbol\{pan-tool-alt\}} & \texttt{EBB9}\\
\mSymbol[outlined]{pan-zoom} & \mSymbol[rounded]{pan-zoom} & \mSymbol[sharp]{pan-zoom} & \texttt{\textbackslash mSymbol\{pan-zoom\}} & \texttt{F655}\\
\mSymbol[outlined]{panorama} & \mSymbol[rounded]{panorama} & \mSymbol[sharp]{panorama} & \texttt{\textbackslash mSymbol\{panorama\}} & \texttt{E40B}\\
\mSymbol[outlined]{panorama-fish-eye} & \mSymbol[rounded]{panorama-fish-eye} & \mSymbol[sharp]{panorama-fish-eye} & \texttt{\textbackslash mSymbol\{panorama-fish-eye\}} & \texttt{E40C}\\
\mSymbol[outlined]{panorama-horizontal} & \mSymbol[rounded]{panorama-horizontal} & \mSymbol[sharp]{panorama-horizontal} & \texttt{\textbackslash mSymbol\{panorama-horizontal\}} & \texttt{E40D}\\
\mSymbol[outlined]{panorama-photosphere} & \mSymbol[rounded]{panorama-photosphere} & \mSymbol[sharp]{panorama-photosphere} & \texttt{\textbackslash mSymbol\{panorama-photosphere\}} & \texttt{E9C9}\\
\mSymbol[outlined]{panorama-vertical} & \mSymbol[rounded]{panorama-vertical} & \mSymbol[sharp]{panorama-vertical} & \texttt{\textbackslash mSymbol\{panorama-vertical\}} & \texttt{E40E}\\
\mSymbol[outlined]{panorama-wide-angle} & \mSymbol[rounded]{panorama-wide-angle} & \mSymbol[sharp]{panorama-wide-angle} & \texttt{\textbackslash mSymbol\{panorama-wide-angle\}} & \texttt{E40F}\\
\mSymbol[outlined]{paragliding} & \mSymbol[rounded]{paragliding} & \mSymbol[sharp]{paragliding} & \texttt{\textbackslash mSymbol\{paragliding\}} & \texttt{E50F}\\
\mSymbol[outlined]{park} & \mSymbol[rounded]{park} & \mSymbol[sharp]{park} & \texttt{\textbackslash mSymbol\{park\}} & \texttt{EA63}\\
\mSymbol[outlined]{partly-cloudy-day} & \mSymbol[rounded]{partly-cloudy-day} & \mSymbol[sharp]{partly-cloudy-day} & \texttt{\textbackslash mSymbol\{partly-cloudy-day\}} & \texttt{F172}\\
\mSymbol[outlined]{partly-cloudy-night} & \mSymbol[rounded]{partly-cloudy-night} & \mSymbol[sharp]{partly-cloudy-night} & \texttt{\textbackslash mSymbol\{partly-cloudy-night\}} & \texttt{F174}\\
\mSymbol[outlined]{partner-exchange} & \mSymbol[rounded]{partner-exchange} & \mSymbol[sharp]{partner-exchange} & \texttt{\textbackslash mSymbol\{partner-exchange\}} & \texttt{F7F9}\\
\mSymbol[outlined]{partner-reports} & \mSymbol[rounded]{partner-reports} & \mSymbol[sharp]{partner-reports} & \texttt{\textbackslash mSymbol\{partner-reports\}} & \texttt{EFAF}\\
\mSymbol[outlined]{party-mode} & \mSymbol[rounded]{party-mode} & \mSymbol[sharp]{party-mode} & \texttt{\textbackslash mSymbol\{party-mode\}} & \texttt{E7FA}\\
\mSymbol[outlined]{passkey} & \mSymbol[rounded]{passkey} & \mSymbol[sharp]{passkey} & \texttt{\textbackslash mSymbol\{passkey\}} & \texttt{F87F}\\
\mSymbol[outlined]{password} & \mSymbol[rounded]{password} & \mSymbol[sharp]{password} & \texttt{\textbackslash mSymbol\{password\}} & \texttt{F042}\\
\mSymbol[outlined]{password-2} & \mSymbol[rounded]{password-2} & \mSymbol[sharp]{password-2} & \texttt{\textbackslash mSymbol\{password-2\}} & \texttt{F4A9}\\
\mSymbol[outlined]{password-2-off} & \mSymbol[rounded]{password-2-off} & \mSymbol[sharp]{password-2-off} & \texttt{\textbackslash mSymbol\{password-2-off\}} & \texttt{F4A8}\\
\mSymbol[outlined]{patient-list} & \mSymbol[rounded]{patient-list} & \mSymbol[sharp]{patient-list} & \texttt{\textbackslash mSymbol\{patient-list\}} & \texttt{E653}\\
\mSymbol[outlined]{pattern} & \mSymbol[rounded]{pattern} & \mSymbol[sharp]{pattern} & \texttt{\textbackslash mSymbol\{pattern\}} & \texttt{F043}\\
\mSymbol[outlined]{pause} & \mSymbol[rounded]{pause} & \mSymbol[sharp]{pause} & \texttt{\textbackslash mSymbol\{pause\}} & \texttt{E034}\\
\mSymbol[outlined]{pause-circle} & \mSymbol[rounded]{pause-circle} & \mSymbol[sharp]{pause-circle} & \texttt{\textbackslash mSymbol\{pause-circle\}} & \texttt{E1A2}\\
\mSymbol[outlined]{pause-circle-filled} & \mSymbol[rounded]{pause-circle-filled} & \mSymbol[sharp]{pause-circle-filled} & \texttt{\textbackslash mSymbol\{pause-circle-filled\}} & \texttt{E1A2}\\
\mSymbol[outlined]{pause-circle-outline} & \mSymbol[rounded]{pause-circle-outline} & \mSymbol[sharp]{pause-circle-outline} & \texttt{\textbackslash mSymbol\{pause-circle-outline\}} & \texttt{E1A2}\\
\mSymbol[outlined]{pause-presentation} & \mSymbol[rounded]{pause-presentation} & \mSymbol[sharp]{pause-presentation} & \texttt{\textbackslash mSymbol\{pause-presentation\}} & \texttt{E0EA}\\
\mSymbol[outlined]{payment} & \mSymbol[rounded]{payment} & \mSymbol[sharp]{payment} & \texttt{\textbackslash mSymbol\{payment\}} & \texttt{E8A1}\\
\mSymbol[outlined]{payments} & \mSymbol[rounded]{payments} & \mSymbol[sharp]{payments} & \texttt{\textbackslash mSymbol\{payments\}} & \texttt{EF63}\\
\mSymbol[outlined]{pedal-bike} & \mSymbol[rounded]{pedal-bike} & \mSymbol[sharp]{pedal-bike} & \texttt{\textbackslash mSymbol\{pedal-bike\}} & \texttt{EB29}\\
\mSymbol[outlined]{pediatrics} & \mSymbol[rounded]{pediatrics} & \mSymbol[sharp]{pediatrics} & \texttt{\textbackslash mSymbol\{pediatrics\}} & \texttt{E11D}\\
\mSymbol[outlined]{pen-size-1} & \mSymbol[rounded]{pen-size-1} & \mSymbol[sharp]{pen-size-1} & \texttt{\textbackslash mSymbol\{pen-size-1\}} & \texttt{F755}\\
\mSymbol[outlined]{pen-size-2} & \mSymbol[rounded]{pen-size-2} & \mSymbol[sharp]{pen-size-2} & \texttt{\textbackslash mSymbol\{pen-size-2\}} & \texttt{F754}\\
\mSymbol[outlined]{pen-size-3} & \mSymbol[rounded]{pen-size-3} & \mSymbol[sharp]{pen-size-3} & \texttt{\textbackslash mSymbol\{pen-size-3\}} & \texttt{F753}\\
\mSymbol[outlined]{pen-size-4} & \mSymbol[rounded]{pen-size-4} & \mSymbol[sharp]{pen-size-4} & \texttt{\textbackslash mSymbol\{pen-size-4\}} & \texttt{F752}\\
\mSymbol[outlined]{pen-size-5} & \mSymbol[rounded]{pen-size-5} & \mSymbol[sharp]{pen-size-5} & \texttt{\textbackslash mSymbol\{pen-size-5\}} & \texttt{F751}\\
\mSymbol[outlined]{pending} & \mSymbol[rounded]{pending} & \mSymbol[sharp]{pending} & \texttt{\textbackslash mSymbol\{pending\}} & \texttt{EF64}\\
\mSymbol[outlined]{pending-actions} & \mSymbol[rounded]{pending-actions} & \mSymbol[sharp]{pending-actions} & \texttt{\textbackslash mSymbol\{pending-actions\}} & \texttt{F1BB}\\
\mSymbol[outlined]{pentagon} & \mSymbol[rounded]{pentagon} & \mSymbol[sharp]{pentagon} & \texttt{\textbackslash mSymbol\{pentagon\}} & \texttt{EB50}\\
\mSymbol[outlined]{people} & \mSymbol[rounded]{people} & \mSymbol[sharp]{people} & \texttt{\textbackslash mSymbol\{people\}} & \texttt{EA21}\\
\mSymbol[outlined]{people-alt} & \mSymbol[rounded]{people-alt} & \mSymbol[sharp]{people-alt} & \texttt{\textbackslash mSymbol\{people-alt\}} & \texttt{EA21}\\
\mSymbol[outlined]{people-outline} & \mSymbol[rounded]{people-outline} & \mSymbol[sharp]{people-outline} & \texttt{\textbackslash mSymbol\{people-outline\}} & \texttt{EA21}\\
\mSymbol[outlined]{percent} & \mSymbol[rounded]{percent} & \mSymbol[sharp]{percent} & \texttt{\textbackslash mSymbol\{percent\}} & \texttt{EB58}\\
\mSymbol[outlined]{performance-max} & \mSymbol[rounded]{performance-max} & \mSymbol[sharp]{performance-max} & \texttt{\textbackslash mSymbol\{performance-max\}} & \texttt{E51A}\\
\mSymbol[outlined]{pergola} & \mSymbol[rounded]{pergola} & \mSymbol[sharp]{pergola} & \texttt{\textbackslash mSymbol\{pergola\}} & \texttt{E203}\\
\mSymbol[outlined]{perm-camera-mic} & \mSymbol[rounded]{perm-camera-mic} & \mSymbol[sharp]{perm-camera-mic} & \texttt{\textbackslash mSymbol\{perm-camera-mic\}} & \texttt{E8A2}\\
\mSymbol[outlined]{perm-contact-calendar} & \mSymbol[rounded]{perm-contact-calendar} & \mSymbol[sharp]{perm-contact-calendar} & \texttt{\textbackslash mSymbol\{perm-contact-calendar\}} & \texttt{E8A3}\\
\mSymbol[outlined]{perm-data-setting} & \mSymbol[rounded]{perm-data-setting} & \mSymbol[sharp]{perm-data-setting} & \texttt{\textbackslash mSymbol\{perm-data-setting\}} & \texttt{E8A4}\\
\mSymbol[outlined]{perm-device-information} & \mSymbol[rounded]{perm-device-information} & \mSymbol[sharp]{perm-device-information} & \texttt{\textbackslash mSymbol\{perm-device-information\}} & \texttt{E8A5}\\
\mSymbol[outlined]{perm-identity} & \mSymbol[rounded]{perm-identity} & \mSymbol[sharp]{perm-identity} & \texttt{\textbackslash mSymbol\{perm-identity\}} & \texttt{F0D3}\\
\mSymbol[outlined]{perm-media} & \mSymbol[rounded]{perm-media} & \mSymbol[sharp]{perm-media} & \texttt{\textbackslash mSymbol\{perm-media\}} & \texttt{E8A7}\\
\mSymbol[outlined]{perm-phone-msg} & \mSymbol[rounded]{perm-phone-msg} & \mSymbol[sharp]{perm-phone-msg} & \texttt{\textbackslash mSymbol\{perm-phone-msg\}} & \texttt{E8A8}\\
\mSymbol[outlined]{perm-scan-wifi} & \mSymbol[rounded]{perm-scan-wifi} & \mSymbol[sharp]{perm-scan-wifi} & \texttt{\textbackslash mSymbol\{perm-scan-wifi\}} & \texttt{E8A9}\\
\mSymbol[outlined]{person} & \mSymbol[rounded]{person} & \mSymbol[sharp]{person} & \texttt{\textbackslash mSymbol\{person\}} & \texttt{F0D3}\\
\mSymbol[outlined]{person-2} & \mSymbol[rounded]{person-2} & \mSymbol[sharp]{person-2} & \texttt{\textbackslash mSymbol\{person-2\}} & \texttt{F8E4}\\
\mSymbol[outlined]{person-3} & \mSymbol[rounded]{person-3} & \mSymbol[sharp]{person-3} & \texttt{\textbackslash mSymbol\{person-3\}} & \texttt{F8E5}\\
\mSymbol[outlined]{person-4} & \mSymbol[rounded]{person-4} & \mSymbol[sharp]{person-4} & \texttt{\textbackslash mSymbol\{person-4\}} & \texttt{F8E6}\\
\mSymbol[outlined]{person-add} & \mSymbol[rounded]{person-add} & \mSymbol[sharp]{person-add} & \texttt{\textbackslash mSymbol\{person-add\}} & \texttt{EA4D}\\
\mSymbol[outlined]{person-add-alt} & \mSymbol[rounded]{person-add-alt} & \mSymbol[sharp]{person-add-alt} & \texttt{\textbackslash mSymbol\{person-add-alt\}} & \texttt{EA4D}\\
\mSymbol[outlined]{person-add-disabled} & \mSymbol[rounded]{person-add-disabled} & \mSymbol[sharp]{person-add-disabled} & \texttt{\textbackslash mSymbol\{person-add-disabled\}} & \texttt{E9CB}\\
\mSymbol[outlined]{person-alert} & \mSymbol[rounded]{person-alert} & \mSymbol[sharp]{person-alert} & \texttt{\textbackslash mSymbol\{person-alert\}} & \texttt{F567}\\
\mSymbol[outlined]{person-apron} & \mSymbol[rounded]{person-apron} & \mSymbol[sharp]{person-apron} & \texttt{\textbackslash mSymbol\{person-apron\}} & \texttt{F5A3}\\
\mSymbol[outlined]{person-book} & \mSymbol[rounded]{person-book} & \mSymbol[sharp]{person-book} & \texttt{\textbackslash mSymbol\{person-book\}} & \texttt{F5E8}\\
\mSymbol[outlined]{person-cancel} & \mSymbol[rounded]{person-cancel} & \mSymbol[sharp]{person-cancel} & \texttt{\textbackslash mSymbol\{person-cancel\}} & \texttt{F566}\\
\mSymbol[outlined]{person-celebrate} & \mSymbol[rounded]{person-celebrate} & \mSymbol[sharp]{person-celebrate} & \texttt{\textbackslash mSymbol\{person-celebrate\}} & \texttt{F7FE}\\
\mSymbol[outlined]{person-check} & \mSymbol[rounded]{person-check} & \mSymbol[sharp]{person-check} & \texttt{\textbackslash mSymbol\{person-check\}} & \texttt{F565}\\
\mSymbol[outlined]{person-edit} & \mSymbol[rounded]{person-edit} & \mSymbol[sharp]{person-edit} & \texttt{\textbackslash mSymbol\{person-edit\}} & \texttt{F4FA}\\
\mSymbol[outlined]{person-filled} & \mSymbol[rounded]{person-filled} & \mSymbol[sharp]{person-filled} & \texttt{\textbackslash mSymbol\{person-filled\}} & \texttt{F0D3}\\
\mSymbol[outlined]{person-off} & \mSymbol[rounded]{person-off} & \mSymbol[sharp]{person-off} & \texttt{\textbackslash mSymbol\{person-off\}} & \texttt{E510}\\
\mSymbol[outlined]{person-outline} & \mSymbol[rounded]{person-outline} & \mSymbol[sharp]{person-outline} & \texttt{\textbackslash mSymbol\{person-outline\}} & \texttt{F0D3}\\
\mSymbol[outlined]{person-pin} & \mSymbol[rounded]{person-pin} & \mSymbol[sharp]{person-pin} & \texttt{\textbackslash mSymbol\{person-pin\}} & \texttt{E55A}\\
\mSymbol[outlined]{person-pin-circle} & \mSymbol[rounded]{person-pin-circle} & \mSymbol[sharp]{person-pin-circle} & \texttt{\textbackslash mSymbol\{person-pin-circle\}} & \texttt{E56A}\\
\mSymbol[outlined]{person-play} & \mSymbol[rounded]{person-play} & \mSymbol[sharp]{person-play} & \texttt{\textbackslash mSymbol\{person-play\}} & \texttt{F7FD}\\
\mSymbol[outlined]{person-raised-hand} & \mSymbol[rounded]{person-raised-hand} & \mSymbol[sharp]{person-raised-hand} & \texttt{\textbackslash mSymbol\{person-raised-hand\}} & \texttt{F59A}\\
\mSymbol[outlined]{person-remove} & \mSymbol[rounded]{person-remove} & \mSymbol[sharp]{person-remove} & \texttt{\textbackslash mSymbol\{person-remove\}} & \texttt{EF66}\\
\mSymbol[outlined]{person-search} & \mSymbol[rounded]{person-search} & \mSymbol[sharp]{person-search} & \texttt{\textbackslash mSymbol\{person-search\}} & \texttt{F106}\\
\mSymbol[outlined]{personal-bag} & \mSymbol[rounded]{personal-bag} & \mSymbol[sharp]{personal-bag} & \texttt{\textbackslash mSymbol\{personal-bag\}} & \texttt{EB0E}\\
\mSymbol[outlined]{personal-bag-off} & \mSymbol[rounded]{personal-bag-off} & \mSymbol[sharp]{personal-bag-off} & \texttt{\textbackslash mSymbol\{personal-bag-off\}} & \texttt{EB0F}\\
\mSymbol[outlined]{personal-bag-question} & \mSymbol[rounded]{personal-bag-question} & \mSymbol[sharp]{personal-bag-question} & \texttt{\textbackslash mSymbol\{personal-bag-question\}} & \texttt{EB10}\\
\mSymbol[outlined]{personal-injury} & \mSymbol[rounded]{personal-injury} & \mSymbol[sharp]{personal-injury} & \texttt{\textbackslash mSymbol\{personal-injury\}} & \texttt{E6DA}\\
\mSymbol[outlined]{personal-places} & \mSymbol[rounded]{personal-places} & \mSymbol[sharp]{personal-places} & \texttt{\textbackslash mSymbol\{personal-places\}} & \texttt{E703}\\
\mSymbol[outlined]{personal-video} & \mSymbol[rounded]{personal-video} & \mSymbol[sharp]{personal-video} & \texttt{\textbackslash mSymbol\{personal-video\}} & \texttt{E63B}\\
\mSymbol[outlined]{pest-control} & \mSymbol[rounded]{pest-control} & \mSymbol[sharp]{pest-control} & \texttt{\textbackslash mSymbol\{pest-control\}} & \texttt{F0FA}\\
\mSymbol[outlined]{pest-control-rodent} & \mSymbol[rounded]{pest-control-rodent} & \mSymbol[sharp]{pest-control-rodent} & \texttt{\textbackslash mSymbol\{pest-control-rodent\}} & \texttt{F0FD}\\
\mSymbol[outlined]{pet-supplies} & \mSymbol[rounded]{pet-supplies} & \mSymbol[sharp]{pet-supplies} & \texttt{\textbackslash mSymbol\{pet-supplies\}} & \texttt{EFB1}\\
\mSymbol[outlined]{pets} & \mSymbol[rounded]{pets} & \mSymbol[sharp]{pets} & \texttt{\textbackslash mSymbol\{pets\}} & \texttt{E91D}\\
\mSymbol[outlined]{phishing} & \mSymbol[rounded]{phishing} & \mSymbol[sharp]{phishing} & \texttt{\textbackslash mSymbol\{phishing\}} & \texttt{EAD7}\\
\mSymbol[outlined]{phone} & \mSymbol[rounded]{phone} & \mSymbol[sharp]{phone} & \texttt{\textbackslash mSymbol\{phone\}} & \texttt{F0D4}\\
\mSymbol[outlined]{phone-alt} & \mSymbol[rounded]{phone-alt} & \mSymbol[sharp]{phone-alt} & \texttt{\textbackslash mSymbol\{phone-alt\}} & \texttt{F0D4}\\
\mSymbol[outlined]{phone-android} & \mSymbol[rounded]{phone-android} & \mSymbol[sharp]{phone-android} & \texttt{\textbackslash mSymbol\{phone-android\}} & \texttt{E324}\\
\mSymbol[outlined]{phone-bluetooth-speaker} & \mSymbol[rounded]{phone-bluetooth-speaker} & \mSymbol[sharp]{phone-bluetooth-speaker} & \texttt{\textbackslash mSymbol\{phone-bluetooth-speaker\}} & \texttt{E61B}\\
\mSymbol[outlined]{phone-callback} & \mSymbol[rounded]{phone-callback} & \mSymbol[sharp]{phone-callback} & \texttt{\textbackslash mSymbol\{phone-callback\}} & \texttt{E649}\\
\mSymbol[outlined]{phone-disabled} & \mSymbol[rounded]{phone-disabled} & \mSymbol[sharp]{phone-disabled} & \texttt{\textbackslash mSymbol\{phone-disabled\}} & \texttt{E9CC}\\
\mSymbol[outlined]{phone-enabled} & \mSymbol[rounded]{phone-enabled} & \mSymbol[sharp]{phone-enabled} & \texttt{\textbackslash mSymbol\{phone-enabled\}} & \texttt{E9CD}\\
\mSymbol[outlined]{phone-forwarded} & \mSymbol[rounded]{phone-forwarded} & \mSymbol[sharp]{phone-forwarded} & \texttt{\textbackslash mSymbol\{phone-forwarded\}} & \texttt{E61C}\\
\mSymbol[outlined]{phone-in-talk} & \mSymbol[rounded]{phone-in-talk} & \mSymbol[sharp]{phone-in-talk} & \texttt{\textbackslash mSymbol\{phone-in-talk\}} & \texttt{E61D}\\
\mSymbol[outlined]{phone-iphone} & \mSymbol[rounded]{phone-iphone} & \mSymbol[sharp]{phone-iphone} & \texttt{\textbackslash mSymbol\{phone-iphone\}} & \texttt{E325}\\
\mSymbol[outlined]{phone-locked} & \mSymbol[rounded]{phone-locked} & \mSymbol[sharp]{phone-locked} & \texttt{\textbackslash mSymbol\{phone-locked\}} & \texttt{E61E}\\
\mSymbol[outlined]{phone-missed} & \mSymbol[rounded]{phone-missed} & \mSymbol[sharp]{phone-missed} & \texttt{\textbackslash mSymbol\{phone-missed\}} & \texttt{E61F}\\
\mSymbol[outlined]{phone-paused} & \mSymbol[rounded]{phone-paused} & \mSymbol[sharp]{phone-paused} & \texttt{\textbackslash mSymbol\{phone-paused\}} & \texttt{E620}\\
\mSymbol[outlined]{phonelink} & \mSymbol[rounded]{phonelink} & \mSymbol[sharp]{phonelink} & \texttt{\textbackslash mSymbol\{phonelink\}} & \texttt{E326}\\
\mSymbol[outlined]{phonelink-erase} & \mSymbol[rounded]{phonelink-erase} & \mSymbol[sharp]{phonelink-erase} & \texttt{\textbackslash mSymbol\{phonelink-erase\}} & \texttt{E0DB}\\
\mSymbol[outlined]{phonelink-lock} & \mSymbol[rounded]{phonelink-lock} & \mSymbol[sharp]{phonelink-lock} & \texttt{\textbackslash mSymbol\{phonelink-lock\}} & \texttt{E0DC}\\
\mSymbol[outlined]{phonelink-off} & \mSymbol[rounded]{phonelink-off} & \mSymbol[sharp]{phonelink-off} & \texttt{\textbackslash mSymbol\{phonelink-off\}} & \texttt{E327}\\
\mSymbol[outlined]{phonelink-ring} & \mSymbol[rounded]{phonelink-ring} & \mSymbol[sharp]{phonelink-ring} & \texttt{\textbackslash mSymbol\{phonelink-ring\}} & \texttt{E0DD}\\
\mSymbol[outlined]{phonelink-ring-off} & \mSymbol[rounded]{phonelink-ring-off} & \mSymbol[sharp]{phonelink-ring-off} & \texttt{\textbackslash mSymbol\{phonelink-ring-off\}} & \texttt{F7AA}\\
\mSymbol[outlined]{phonelink-setup} & \mSymbol[rounded]{phonelink-setup} & \mSymbol[sharp]{phonelink-setup} & \texttt{\textbackslash mSymbol\{phonelink-setup\}} & \texttt{EF41}\\
\mSymbol[outlined]{photo} & \mSymbol[rounded]{photo} & \mSymbol[sharp]{photo} & \texttt{\textbackslash mSymbol\{photo\}} & \texttt{E432}\\
\mSymbol[outlined]{photo-album} & \mSymbol[rounded]{photo-album} & \mSymbol[sharp]{photo-album} & \texttt{\textbackslash mSymbol\{photo-album\}} & \texttt{E411}\\
\mSymbol[outlined]{photo-auto-merge} & \mSymbol[rounded]{photo-auto-merge} & \mSymbol[sharp]{photo-auto-merge} & \texttt{\textbackslash mSymbol\{photo-auto-merge\}} & \texttt{F530}\\
\mSymbol[outlined]{photo-camera} & \mSymbol[rounded]{photo-camera} & \mSymbol[sharp]{photo-camera} & \texttt{\textbackslash mSymbol\{photo-camera\}} & \texttt{E412}\\
\mSymbol[outlined]{photo-camera-back} & \mSymbol[rounded]{photo-camera-back} & \mSymbol[sharp]{photo-camera-back} & \texttt{\textbackslash mSymbol\{photo-camera-back\}} & \texttt{EF68}\\
\mSymbol[outlined]{photo-camera-front} & \mSymbol[rounded]{photo-camera-front} & \mSymbol[sharp]{photo-camera-front} & \texttt{\textbackslash mSymbol\{photo-camera-front\}} & \texttt{EF69}\\
\mSymbol[outlined]{photo-filter} & \mSymbol[rounded]{photo-filter} & \mSymbol[sharp]{photo-filter} & \texttt{\textbackslash mSymbol\{photo-filter\}} & \texttt{E43B}\\
\mSymbol[outlined]{photo-frame} & \mSymbol[rounded]{photo-frame} & \mSymbol[sharp]{photo-frame} & \texttt{\textbackslash mSymbol\{photo-frame\}} & \texttt{F0D9}\\
\mSymbol[outlined]{photo-library} & \mSymbol[rounded]{photo-library} & \mSymbol[sharp]{photo-library} & \texttt{\textbackslash mSymbol\{photo-library\}} & \texttt{E413}\\
\mSymbol[outlined]{photo-prints} & \mSymbol[rounded]{photo-prints} & \mSymbol[sharp]{photo-prints} & \texttt{\textbackslash mSymbol\{photo-prints\}} & \texttt{EFB2}\\
\mSymbol[outlined]{photo-size-select-actual} & \mSymbol[rounded]{photo-size-select-actual} & \mSymbol[sharp]{photo-size-select-actual} & \texttt{\textbackslash mSymbol\{photo-size-select-actual\}} & \texttt{E432}\\
\mSymbol[outlined]{photo-size-select-large} & \mSymbol[rounded]{photo-size-select-large} & \mSymbol[sharp]{photo-size-select-large} & \texttt{\textbackslash mSymbol\{photo-size-select-large\}} & \texttt{E433}\\
\mSymbol[outlined]{photo-size-select-small} & \mSymbol[rounded]{photo-size-select-small} & \mSymbol[sharp]{photo-size-select-small} & \texttt{\textbackslash mSymbol\{photo-size-select-small\}} & \texttt{E434}\\
\mSymbol[outlined]{php} & \mSymbol[rounded]{php} & \mSymbol[sharp]{php} & \texttt{\textbackslash mSymbol\{php\}} & \texttt{EB8F}\\
\mSymbol[outlined]{physical-therapy} & \mSymbol[rounded]{physical-therapy} & \mSymbol[sharp]{physical-therapy} & \texttt{\textbackslash mSymbol\{physical-therapy\}} & \texttt{E11E}\\
\mSymbol[outlined]{piano} & \mSymbol[rounded]{piano} & \mSymbol[sharp]{piano} & \texttt{\textbackslash mSymbol\{piano\}} & \texttt{E521}\\
\mSymbol[outlined]{piano-off} & \mSymbol[rounded]{piano-off} & \mSymbol[sharp]{piano-off} & \texttt{\textbackslash mSymbol\{piano-off\}} & \texttt{E520}\\
\mSymbol[outlined]{picture-as-pdf} & \mSymbol[rounded]{picture-as-pdf} & \mSymbol[sharp]{picture-as-pdf} & \texttt{\textbackslash mSymbol\{picture-as-pdf\}} & \texttt{E415}\\
\mSymbol[outlined]{picture-in-picture} & \mSymbol[rounded]{picture-in-picture} & \mSymbol[sharp]{picture-in-picture} & \texttt{\textbackslash mSymbol\{picture-in-picture\}} & \texttt{E8AA}\\
\mSymbol[outlined]{picture-in-picture-alt} & \mSymbol[rounded]{picture-in-picture-alt} & \mSymbol[sharp]{picture-in-picture-alt} & \texttt{\textbackslash mSymbol\{picture-in-picture-alt\}} & \texttt{E911}\\
\mSymbol[outlined]{picture-in-picture-center} & \mSymbol[rounded]{picture-in-picture-center} & \mSymbol[sharp]{picture-in-picture-center} & \texttt{\textbackslash mSymbol\{picture-in-picture-center\}} & \texttt{F550}\\
\mSymbol[outlined]{picture-in-picture-large} & \mSymbol[rounded]{picture-in-picture-large} & \mSymbol[sharp]{picture-in-picture-large} & \texttt{\textbackslash mSymbol\{picture-in-picture-large\}} & \texttt{F54F}\\
\mSymbol[outlined]{picture-in-picture-medium} & \mSymbol[rounded]{picture-in-picture-medium} & \mSymbol[sharp]{picture-in-picture-medium} & \texttt{\textbackslash mSymbol\{picture-in-picture-medium\}} & \texttt{F54E}\\
\mSymbol[outlined]{picture-in-picture-mobile} & \mSymbol[rounded]{picture-in-picture-mobile} & \mSymbol[sharp]{picture-in-picture-mobile} & \texttt{\textbackslash mSymbol\{picture-in-picture-mobile\}} & \texttt{F517}\\
\mSymbol[outlined]{picture-in-picture-off} & \mSymbol[rounded]{picture-in-picture-off} & \mSymbol[sharp]{picture-in-picture-off} & \texttt{\textbackslash mSymbol\{picture-in-picture-off\}} & \texttt{F52F}\\
\mSymbol[outlined]{picture-in-picture-small} & \mSymbol[rounded]{picture-in-picture-small} & \mSymbol[sharp]{picture-in-picture-small} & \texttt{\textbackslash mSymbol\{picture-in-picture-small\}} & \texttt{F54D}\\
\mSymbol[outlined]{pie-chart} & \mSymbol[rounded]{pie-chart} & \mSymbol[sharp]{pie-chart} & \texttt{\textbackslash mSymbol\{pie-chart\}} & \texttt{F0DA}\\
\mSymbol[outlined]{pie-chart-filled} & \mSymbol[rounded]{pie-chart-filled} & \mSymbol[sharp]{pie-chart-filled} & \texttt{\textbackslash mSymbol\{pie-chart-filled\}} & \texttt{F0DA}\\
\mSymbol[outlined]{pie-chart-outline} & \mSymbol[rounded]{pie-chart-outline} & \mSymbol[sharp]{pie-chart-outline} & \texttt{\textbackslash mSymbol\{pie-chart-outline\}} & \texttt{F0DA}\\
\mSymbol[outlined]{pie-chart-outlined} & \mSymbol[rounded]{pie-chart-outlined} & \mSymbol[sharp]{pie-chart-outlined} & \texttt{\textbackslash mSymbol\{pie-chart-outlined\}} & \texttt{F0DA}\\
\mSymbol[outlined]{pill} & \mSymbol[rounded]{pill} & \mSymbol[sharp]{pill} & \texttt{\textbackslash mSymbol\{pill\}} & \texttt{E11F}\\
\mSymbol[outlined]{pill-off} & \mSymbol[rounded]{pill-off} & \mSymbol[sharp]{pill-off} & \texttt{\textbackslash mSymbol\{pill-off\}} & \texttt{F809}\\
\mSymbol[outlined]{pin} & \mSymbol[rounded]{pin} & \mSymbol[sharp]{pin} & \texttt{\textbackslash mSymbol\{pin\}} & \texttt{F045}\\
\mSymbol[outlined]{pin-drop} & \mSymbol[rounded]{pin-drop} & \mSymbol[sharp]{pin-drop} & \texttt{\textbackslash mSymbol\{pin-drop\}} & \texttt{E55E}\\
\mSymbol[outlined]{pin-end} & \mSymbol[rounded]{pin-end} & \mSymbol[sharp]{pin-end} & \texttt{\textbackslash mSymbol\{pin-end\}} & \texttt{E767}\\
\mSymbol[outlined]{pin-invoke} & \mSymbol[rounded]{pin-invoke} & \mSymbol[sharp]{pin-invoke} & \texttt{\textbackslash mSymbol\{pin-invoke\}} & \texttt{E763}\\
\mSymbol[outlined]{pinch} & \mSymbol[rounded]{pinch} & \mSymbol[sharp]{pinch} & \texttt{\textbackslash mSymbol\{pinch\}} & \texttt{EB38}\\
\mSymbol[outlined]{pinch-zoom-in} & \mSymbol[rounded]{pinch-zoom-in} & \mSymbol[sharp]{pinch-zoom-in} & \texttt{\textbackslash mSymbol\{pinch-zoom-in\}} & \texttt{F1FA}\\
\mSymbol[outlined]{pinch-zoom-out} & \mSymbol[rounded]{pinch-zoom-out} & \mSymbol[sharp]{pinch-zoom-out} & \texttt{\textbackslash mSymbol\{pinch-zoom-out\}} & \texttt{F1FB}\\
\mSymbol[outlined]{pip} & \mSymbol[rounded]{pip} & \mSymbol[sharp]{pip} & \texttt{\textbackslash mSymbol\{pip\}} & \texttt{F64D}\\
\mSymbol[outlined]{pip-exit} & \mSymbol[rounded]{pip-exit} & \mSymbol[sharp]{pip-exit} & \texttt{\textbackslash mSymbol\{pip-exit\}} & \texttt{F70D}\\
\mSymbol[outlined]{pivot-table-chart} & \mSymbol[rounded]{pivot-table-chart} & \mSymbol[sharp]{pivot-table-chart} & \texttt{\textbackslash mSymbol\{pivot-table-chart\}} & \texttt{E9CE}\\
\mSymbol[outlined]{place} & \mSymbol[rounded]{place} & \mSymbol[sharp]{place} & \texttt{\textbackslash mSymbol\{place\}} & \texttt{F1DB}\\
\mSymbol[outlined]{place-item} & \mSymbol[rounded]{place-item} & \mSymbol[sharp]{place-item} & \texttt{\textbackslash mSymbol\{place-item\}} & \texttt{F1F0}\\
\mSymbol[outlined]{plagiarism} & \mSymbol[rounded]{plagiarism} & \mSymbol[sharp]{plagiarism} & \texttt{\textbackslash mSymbol\{plagiarism\}} & \texttt{EA5A}\\
\mSymbol[outlined]{planner-banner-ad-pt} & \mSymbol[rounded]{planner-banner-ad-pt} & \mSymbol[sharp]{planner-banner-ad-pt} & \texttt{\textbackslash mSymbol\{planner-banner-ad-pt\}} & \texttt{E692}\\
\mSymbol[outlined]{planner-review} & \mSymbol[rounded]{planner-review} & \mSymbol[sharp]{planner-review} & \texttt{\textbackslash mSymbol\{planner-review\}} & \texttt{E694}\\
\mSymbol[outlined]{play-arrow} & \mSymbol[rounded]{play-arrow} & \mSymbol[sharp]{play-arrow} & \texttt{\textbackslash mSymbol\{play-arrow\}} & \texttt{E037}\\
\mSymbol[outlined]{play-circle} & \mSymbol[rounded]{play-circle} & \mSymbol[sharp]{play-circle} & \texttt{\textbackslash mSymbol\{play-circle\}} & \texttt{E1C4}\\
\mSymbol[outlined]{play-disabled} & \mSymbol[rounded]{play-disabled} & \mSymbol[sharp]{play-disabled} & \texttt{\textbackslash mSymbol\{play-disabled\}} & \texttt{EF6A}\\
\mSymbol[outlined]{play-for-work} & \mSymbol[rounded]{play-for-work} & \mSymbol[sharp]{play-for-work} & \texttt{\textbackslash mSymbol\{play-for-work\}} & \texttt{E906}\\
\mSymbol[outlined]{play-lesson} & \mSymbol[rounded]{play-lesson} & \mSymbol[sharp]{play-lesson} & \texttt{\textbackslash mSymbol\{play-lesson\}} & \texttt{F047}\\
\mSymbol[outlined]{play-music} & \mSymbol[rounded]{play-music} & \mSymbol[sharp]{play-music} & \texttt{\textbackslash mSymbol\{play-music\}} & \texttt{E6EE}\\
\mSymbol[outlined]{play-pause} & \mSymbol[rounded]{play-pause} & \mSymbol[sharp]{play-pause} & \texttt{\textbackslash mSymbol\{play-pause\}} & \texttt{F137}\\
\mSymbol[outlined]{play-shapes} & \mSymbol[rounded]{play-shapes} & \mSymbol[sharp]{play-shapes} & \texttt{\textbackslash mSymbol\{play-shapes\}} & \texttt{F7FC}\\
\mSymbol[outlined]{playing-cards} & \mSymbol[rounded]{playing-cards} & \mSymbol[sharp]{playing-cards} & \texttt{\textbackslash mSymbol\{playing-cards\}} & \texttt{F5DC}\\
\mSymbol[outlined]{playlist-add} & \mSymbol[rounded]{playlist-add} & \mSymbol[sharp]{playlist-add} & \texttt{\textbackslash mSymbol\{playlist-add\}} & \texttt{E03B}\\
\mSymbol[outlined]{playlist-add-check} & \mSymbol[rounded]{playlist-add-check} & \mSymbol[sharp]{playlist-add-check} & \texttt{\textbackslash mSymbol\{playlist-add-check\}} & \texttt{E065}\\
\mSymbol[outlined]{playlist-add-check-circle} & \mSymbol[rounded]{playlist-add-check-circle} & \mSymbol[sharp]{playlist-add-check-circle} & \texttt{\textbackslash mSymbol\{playlist-add-check-circle\}} & \texttt{E7E6}\\
\mSymbol[outlined]{playlist-add-circle} & \mSymbol[rounded]{playlist-add-circle} & \mSymbol[sharp]{playlist-add-circle} & \texttt{\textbackslash mSymbol\{playlist-add-circle\}} & \texttt{E7E5}\\
\mSymbol[outlined]{playlist-play} & \mSymbol[rounded]{playlist-play} & \mSymbol[sharp]{playlist-play} & \texttt{\textbackslash mSymbol\{playlist-play\}} & \texttt{E05F}\\
\mSymbol[outlined]{playlist-remove} & \mSymbol[rounded]{playlist-remove} & \mSymbol[sharp]{playlist-remove} & \texttt{\textbackslash mSymbol\{playlist-remove\}} & \texttt{EB80}\\
\mSymbol[outlined]{plumbing} & \mSymbol[rounded]{plumbing} & \mSymbol[sharp]{plumbing} & \texttt{\textbackslash mSymbol\{plumbing\}} & \texttt{F107}\\
\mSymbol[outlined]{plus-one} & \mSymbol[rounded]{plus-one} & \mSymbol[sharp]{plus-one} & \texttt{\textbackslash mSymbol\{plus-one\}} & \texttt{E800}\\
\mSymbol[outlined]{podcasts} & \mSymbol[rounded]{podcasts} & \mSymbol[sharp]{podcasts} & \texttt{\textbackslash mSymbol\{podcasts\}} & \texttt{F048}\\
\mSymbol[outlined]{podiatry} & \mSymbol[rounded]{podiatry} & \mSymbol[sharp]{podiatry} & \texttt{\textbackslash mSymbol\{podiatry\}} & \texttt{E120}\\
\mSymbol[outlined]{podium} & \mSymbol[rounded]{podium} & \mSymbol[sharp]{podium} & \texttt{\textbackslash mSymbol\{podium\}} & \texttt{F7FB}\\
\mSymbol[outlined]{point-of-sale} & \mSymbol[rounded]{point-of-sale} & \mSymbol[sharp]{point-of-sale} & \texttt{\textbackslash mSymbol\{point-of-sale\}} & \texttt{F17E}\\
\mSymbol[outlined]{point-scan} & \mSymbol[rounded]{point-scan} & \mSymbol[sharp]{point-scan} & \texttt{\textbackslash mSymbol\{point-scan\}} & \texttt{F70C}\\
\mSymbol[outlined]{poker-chip} & \mSymbol[rounded]{poker-chip} & \mSymbol[sharp]{poker-chip} & \texttt{\textbackslash mSymbol\{poker-chip\}} & \texttt{F49B}\\
\mSymbol[outlined]{policy} & \mSymbol[rounded]{policy} & \mSymbol[sharp]{policy} & \texttt{\textbackslash mSymbol\{policy\}} & \texttt{EA17}\\
\mSymbol[outlined]{poll} & \mSymbol[rounded]{poll} & \mSymbol[sharp]{poll} & \texttt{\textbackslash mSymbol\{poll\}} & \texttt{F0CC}\\
\mSymbol[outlined]{polyline} & \mSymbol[rounded]{polyline} & \mSymbol[sharp]{polyline} & \texttt{\textbackslash mSymbol\{polyline\}} & \texttt{EBBB}\\
\mSymbol[outlined]{polymer} & \mSymbol[rounded]{polymer} & \mSymbol[sharp]{polymer} & \texttt{\textbackslash mSymbol\{polymer\}} & \texttt{E8AB}\\
\mSymbol[outlined]{pool} & \mSymbol[rounded]{pool} & \mSymbol[sharp]{pool} & \texttt{\textbackslash mSymbol\{pool\}} & \texttt{EB48}\\
\mSymbol[outlined]{portable-wifi-off} & \mSymbol[rounded]{portable-wifi-off} & \mSymbol[sharp]{portable-wifi-off} & \texttt{\textbackslash mSymbol\{portable-wifi-off\}} & \texttt{F087}\\
\mSymbol[outlined]{portrait} & \mSymbol[rounded]{portrait} & \mSymbol[sharp]{portrait} & \texttt{\textbackslash mSymbol\{portrait\}} & \texttt{E851}\\
\mSymbol[outlined]{position-bottom-left} & \mSymbol[rounded]{position-bottom-left} & \mSymbol[sharp]{position-bottom-left} & \texttt{\textbackslash mSymbol\{position-bottom-left\}} & \texttt{F70B}\\
\mSymbol[outlined]{position-bottom-right} & \mSymbol[rounded]{position-bottom-right} & \mSymbol[sharp]{position-bottom-right} & \texttt{\textbackslash mSymbol\{position-bottom-right\}} & \texttt{F70A}\\
\mSymbol[outlined]{position-top-right} & \mSymbol[rounded]{position-top-right} & \mSymbol[sharp]{position-top-right} & \texttt{\textbackslash mSymbol\{position-top-right\}} & \texttt{F709}\\
\mSymbol[outlined]{post} & \mSymbol[rounded]{post} & \mSymbol[sharp]{post} & \texttt{\textbackslash mSymbol\{post\}} & \texttt{E705}\\
\mSymbol[outlined]{post-add} & \mSymbol[rounded]{post-add} & \mSymbol[sharp]{post-add} & \texttt{\textbackslash mSymbol\{post-add\}} & \texttt{EA20}\\
\mSymbol[outlined]{potted-plant} & \mSymbol[rounded]{potted-plant} & \mSymbol[sharp]{potted-plant} & \texttt{\textbackslash mSymbol\{potted-plant\}} & \texttt{F8AA}\\
\mSymbol[outlined]{power} & \mSymbol[rounded]{power} & \mSymbol[sharp]{power} & \texttt{\textbackslash mSymbol\{power\}} & \texttt{E63C}\\
\mSymbol[outlined]{power-input} & \mSymbol[rounded]{power-input} & \mSymbol[sharp]{power-input} & \texttt{\textbackslash mSymbol\{power-input\}} & \texttt{E336}\\
\mSymbol[outlined]{power-off} & \mSymbol[rounded]{power-off} & \mSymbol[sharp]{power-off} & \texttt{\textbackslash mSymbol\{power-off\}} & \texttt{E646}\\
\mSymbol[outlined]{power-rounded} & \mSymbol[rounded]{power-rounded} & \mSymbol[sharp]{power-rounded} & \texttt{\textbackslash mSymbol\{power-rounded\}} & \texttt{F8C7}\\
\mSymbol[outlined]{power-settings-circle} & \mSymbol[rounded]{power-settings-circle} & \mSymbol[sharp]{power-settings-circle} & \texttt{\textbackslash mSymbol\{power-settings-circle\}} & \texttt{F418}\\
\mSymbol[outlined]{power-settings-new} & \mSymbol[rounded]{power-settings-new} & \mSymbol[sharp]{power-settings-new} & \texttt{\textbackslash mSymbol\{power-settings-new\}} & \texttt{F8C7}\\
\mSymbol[outlined]{prayer-times} & \mSymbol[rounded]{prayer-times} & \mSymbol[sharp]{prayer-times} & \texttt{\textbackslash mSymbol\{prayer-times\}} & \texttt{F838}\\
\mSymbol[outlined]{precision-manufacturing} & \mSymbol[rounded]{precision-manufacturing} & \mSymbol[sharp]{precision-manufacturing} & \texttt{\textbackslash mSymbol\{precision-manufacturing\}} & \texttt{F049}\\
\mSymbol[outlined]{pregnancy} & \mSymbol[rounded]{pregnancy} & \mSymbol[sharp]{pregnancy} & \texttt{\textbackslash mSymbol\{pregnancy\}} & \texttt{F5F1}\\
\mSymbol[outlined]{pregnant-woman} & \mSymbol[rounded]{pregnant-woman} & \mSymbol[sharp]{pregnant-woman} & \texttt{\textbackslash mSymbol\{pregnant-woman\}} & \texttt{F5F1}\\
\mSymbol[outlined]{preliminary} & \mSymbol[rounded]{preliminary} & \mSymbol[sharp]{preliminary} & \texttt{\textbackslash mSymbol\{preliminary\}} & \texttt{E7D8}\\
\mSymbol[outlined]{prescriptions} & \mSymbol[rounded]{prescriptions} & \mSymbol[sharp]{prescriptions} & \texttt{\textbackslash mSymbol\{prescriptions\}} & \texttt{E121}\\
\mSymbol[outlined]{present-to-all} & \mSymbol[rounded]{present-to-all} & \mSymbol[sharp]{present-to-all} & \texttt{\textbackslash mSymbol\{present-to-all\}} & \texttt{E0DF}\\
\mSymbol[outlined]{preview} & \mSymbol[rounded]{preview} & \mSymbol[sharp]{preview} & \texttt{\textbackslash mSymbol\{preview\}} & \texttt{F1C5}\\
\mSymbol[outlined]{preview-off} & \mSymbol[rounded]{preview-off} & \mSymbol[sharp]{preview-off} & \texttt{\textbackslash mSymbol\{preview-off\}} & \texttt{F7AF}\\
\mSymbol[outlined]{price-change} & \mSymbol[rounded]{price-change} & \mSymbol[sharp]{price-change} & \texttt{\textbackslash mSymbol\{price-change\}} & \texttt{F04A}\\
\mSymbol[outlined]{price-check} & \mSymbol[rounded]{price-check} & \mSymbol[sharp]{price-check} & \texttt{\textbackslash mSymbol\{price-check\}} & \texttt{F04B}\\
\mSymbol[outlined]{print} & \mSymbol[rounded]{print} & \mSymbol[sharp]{print} & \texttt{\textbackslash mSymbol\{print\}} & \texttt{E8AD}\\
\mSymbol[outlined]{print-add} & \mSymbol[rounded]{print-add} & \mSymbol[sharp]{print-add} & \texttt{\textbackslash mSymbol\{print-add\}} & \texttt{F7A2}\\
\mSymbol[outlined]{print-connect} & \mSymbol[rounded]{print-connect} & \mSymbol[sharp]{print-connect} & \texttt{\textbackslash mSymbol\{print-connect\}} & \texttt{F7A1}\\
\mSymbol[outlined]{print-disabled} & \mSymbol[rounded]{print-disabled} & \mSymbol[sharp]{print-disabled} & \texttt{\textbackslash mSymbol\{print-disabled\}} & \texttt{E9CF}\\
\mSymbol[outlined]{print-error} & \mSymbol[rounded]{print-error} & \mSymbol[sharp]{print-error} & \texttt{\textbackslash mSymbol\{print-error\}} & \texttt{F7A0}\\
\mSymbol[outlined]{print-lock} & \mSymbol[rounded]{print-lock} & \mSymbol[sharp]{print-lock} & \texttt{\textbackslash mSymbol\{print-lock\}} & \texttt{F651}\\
\mSymbol[outlined]{priority} & \mSymbol[rounded]{priority} & \mSymbol[sharp]{priority} & \texttt{\textbackslash mSymbol\{priority\}} & \texttt{E19F}\\
\mSymbol[outlined]{priority-high} & \mSymbol[rounded]{priority-high} & \mSymbol[sharp]{priority-high} & \texttt{\textbackslash mSymbol\{priority-high\}} & \texttt{E645}\\
\mSymbol[outlined]{privacy} & \mSymbol[rounded]{privacy} & \mSymbol[sharp]{privacy} & \texttt{\textbackslash mSymbol\{privacy\}} & \texttt{F148}\\
\mSymbol[outlined]{privacy-tip} & \mSymbol[rounded]{privacy-tip} & \mSymbol[sharp]{privacy-tip} & \texttt{\textbackslash mSymbol\{privacy-tip\}} & \texttt{F0DC}\\
\mSymbol[outlined]{private-connectivity} & \mSymbol[rounded]{private-connectivity} & \mSymbol[sharp]{private-connectivity} & \texttt{\textbackslash mSymbol\{private-connectivity\}} & \texttt{E744}\\
\mSymbol[outlined]{problem} & \mSymbol[rounded]{problem} & \mSymbol[sharp]{problem} & \texttt{\textbackslash mSymbol\{problem\}} & \texttt{E122}\\
\mSymbol[outlined]{procedure} & \mSymbol[rounded]{procedure} & \mSymbol[sharp]{procedure} & \texttt{\textbackslash mSymbol\{procedure\}} & \texttt{E651}\\
\mSymbol[outlined]{process-chart} & \mSymbol[rounded]{process-chart} & \mSymbol[sharp]{process-chart} & \texttt{\textbackslash mSymbol\{process-chart\}} & \texttt{F855}\\
\mSymbol[outlined]{production-quantity-limits} & \mSymbol[rounded]{production-quantity-limits} & \mSymbol[sharp]{production-quantity-limits} & \texttt{\textbackslash mSymbol\{production-quantity-limits\}} & \texttt{E1D1}\\
\mSymbol[outlined]{productivity} & \mSymbol[rounded]{productivity} & \mSymbol[sharp]{productivity} & \texttt{\textbackslash mSymbol\{productivity\}} & \texttt{E296}\\
\mSymbol[outlined]{progress-activity} & \mSymbol[rounded]{progress-activity} & \mSymbol[sharp]{progress-activity} & \texttt{\textbackslash mSymbol\{progress-activity\}} & \texttt{E9D0}\\
\mSymbol[outlined]{prompt-suggestion} & \mSymbol[rounded]{prompt-suggestion} & \mSymbol[sharp]{prompt-suggestion} & \texttt{\textbackslash mSymbol\{prompt-suggestion\}} & \texttt{F4F6}\\
\mSymbol[outlined]{propane} & \mSymbol[rounded]{propane} & \mSymbol[sharp]{propane} & \texttt{\textbackslash mSymbol\{propane\}} & \texttt{EC14}\\
\mSymbol[outlined]{propane-tank} & \mSymbol[rounded]{propane-tank} & \mSymbol[sharp]{propane-tank} & \texttt{\textbackslash mSymbol\{propane-tank\}} & \texttt{EC13}\\
\mSymbol[outlined]{psychiatry} & \mSymbol[rounded]{psychiatry} & \mSymbol[sharp]{psychiatry} & \texttt{\textbackslash mSymbol\{psychiatry\}} & \texttt{E123}\\
\mSymbol[outlined]{psychology} & \mSymbol[rounded]{psychology} & \mSymbol[sharp]{psychology} & \texttt{\textbackslash mSymbol\{psychology\}} & \texttt{EA4A}\\
\mSymbol[outlined]{psychology-alt} & \mSymbol[rounded]{psychology-alt} & \mSymbol[sharp]{psychology-alt} & \texttt{\textbackslash mSymbol\{psychology-alt\}} & \texttt{F8EA}\\
\mSymbol[outlined]{public} & \mSymbol[rounded]{public} & \mSymbol[sharp]{public} & \texttt{\textbackslash mSymbol\{public\}} & \texttt{E80B}\\
\mSymbol[outlined]{public-off} & \mSymbol[rounded]{public-off} & \mSymbol[sharp]{public-off} & \texttt{\textbackslash mSymbol\{public-off\}} & \texttt{F1CA}\\
\mSymbol[outlined]{publish} & \mSymbol[rounded]{publish} & \mSymbol[sharp]{publish} & \texttt{\textbackslash mSymbol\{publish\}} & \texttt{E255}\\
\mSymbol[outlined]{published-with-changes} & \mSymbol[rounded]{published-with-changes} & \mSymbol[sharp]{published-with-changes} & \texttt{\textbackslash mSymbol\{published-with-changes\}} & \texttt{F232}\\
\mSymbol[outlined]{pulmonology} & \mSymbol[rounded]{pulmonology} & \mSymbol[sharp]{pulmonology} & \texttt{\textbackslash mSymbol\{pulmonology\}} & \texttt{E124}\\
\mSymbol[outlined]{pulse-alert} & \mSymbol[rounded]{pulse-alert} & \mSymbol[sharp]{pulse-alert} & \texttt{\textbackslash mSymbol\{pulse-alert\}} & \texttt{F501}\\
\mSymbol[outlined]{punch-clock} & \mSymbol[rounded]{punch-clock} & \mSymbol[sharp]{punch-clock} & \texttt{\textbackslash mSymbol\{punch-clock\}} & \texttt{EAA8}\\
\mSymbol[outlined]{push-pin} & \mSymbol[rounded]{push-pin} & \mSymbol[sharp]{push-pin} & \texttt{\textbackslash mSymbol\{push-pin\}} & \texttt{F10D}\\
\mSymbol[outlined]{qr-code} & \mSymbol[rounded]{qr-code} & \mSymbol[sharp]{qr-code} & \texttt{\textbackslash mSymbol\{qr-code\}} & \texttt{EF6B}\\
\mSymbol[outlined]{qr-code-2} & \mSymbol[rounded]{qr-code-2} & \mSymbol[sharp]{qr-code-2} & \texttt{\textbackslash mSymbol\{qr-code-2\}} & \texttt{E00A}\\
\mSymbol[outlined]{qr-code-2-add} & \mSymbol[rounded]{qr-code-2-add} & \mSymbol[sharp]{qr-code-2-add} & \texttt{\textbackslash mSymbol\{qr-code-2-add\}} & \texttt{F658}\\
\mSymbol[outlined]{qr-code-scanner} & \mSymbol[rounded]{qr-code-scanner} & \mSymbol[sharp]{qr-code-scanner} & \texttt{\textbackslash mSymbol\{qr-code-scanner\}} & \texttt{F206}\\
\mSymbol[outlined]{query-builder} & \mSymbol[rounded]{query-builder} & \mSymbol[sharp]{query-builder} & \texttt{\textbackslash mSymbol\{query-builder\}} & \texttt{EFD6}\\
\mSymbol[outlined]{query-stats} & \mSymbol[rounded]{query-stats} & \mSymbol[sharp]{query-stats} & \texttt{\textbackslash mSymbol\{query-stats\}} & \texttt{E4FC}\\
\mSymbol[outlined]{question-answer} & \mSymbol[rounded]{question-answer} & \mSymbol[sharp]{question-answer} & \texttt{\textbackslash mSymbol\{question-answer\}} & \texttt{E8AF}\\
\mSymbol[outlined]{question-exchange} & \mSymbol[rounded]{question-exchange} & \mSymbol[sharp]{question-exchange} & \texttt{\textbackslash mSymbol\{question-exchange\}} & \texttt{F7F3}\\
\mSymbol[outlined]{question-mark} & \mSymbol[rounded]{question-mark} & \mSymbol[sharp]{question-mark} & \texttt{\textbackslash mSymbol\{question-mark\}} & \texttt{EB8B}\\
\mSymbol[outlined]{queue} & \mSymbol[rounded]{queue} & \mSymbol[sharp]{queue} & \texttt{\textbackslash mSymbol\{queue\}} & \texttt{E03C}\\
\mSymbol[outlined]{queue-music} & \mSymbol[rounded]{queue-music} & \mSymbol[sharp]{queue-music} & \texttt{\textbackslash mSymbol\{queue-music\}} & \texttt{E03D}\\
\mSymbol[outlined]{queue-play-next} & \mSymbol[rounded]{queue-play-next} & \mSymbol[sharp]{queue-play-next} & \texttt{\textbackslash mSymbol\{queue-play-next\}} & \texttt{E066}\\
\mSymbol[outlined]{quick-phrases} & \mSymbol[rounded]{quick-phrases} & \mSymbol[sharp]{quick-phrases} & \texttt{\textbackslash mSymbol\{quick-phrases\}} & \texttt{E7D1}\\
\mSymbol[outlined]{quick-reference} & \mSymbol[rounded]{quick-reference} & \mSymbol[sharp]{quick-reference} & \texttt{\textbackslash mSymbol\{quick-reference\}} & \texttt{E46E}\\
\mSymbol[outlined]{quick-reference-all} & \mSymbol[rounded]{quick-reference-all} & \mSymbol[sharp]{quick-reference-all} & \texttt{\textbackslash mSymbol\{quick-reference-all\}} & \texttt{F801}\\
\mSymbol[outlined]{quick-reorder} & \mSymbol[rounded]{quick-reorder} & \mSymbol[sharp]{quick-reorder} & \texttt{\textbackslash mSymbol\{quick-reorder\}} & \texttt{EB15}\\
\mSymbol[outlined]{quickreply} & \mSymbol[rounded]{quickreply} & \mSymbol[sharp]{quickreply} & \texttt{\textbackslash mSymbol\{quickreply\}} & \texttt{EF6C}\\
\mSymbol[outlined]{quiet-time} & \mSymbol[rounded]{quiet-time} & \mSymbol[sharp]{quiet-time} & \texttt{\textbackslash mSymbol\{quiet-time\}} & \texttt{E1F9}\\
\mSymbol[outlined]{quiet-time-active} & \mSymbol[rounded]{quiet-time-active} & \mSymbol[sharp]{quiet-time-active} & \texttt{\textbackslash mSymbol\{quiet-time-active\}} & \texttt{E291}\\
\mSymbol[outlined]{quiz} & \mSymbol[rounded]{quiz} & \mSymbol[sharp]{quiz} & \texttt{\textbackslash mSymbol\{quiz\}} & \texttt{F04C}\\
\mSymbol[outlined]{r-mobiledata} & \mSymbol[rounded]{r-mobiledata} & \mSymbol[sharp]{r-mobiledata} & \texttt{\textbackslash mSymbol\{r-mobiledata\}} & \texttt{F04D}\\
\mSymbol[outlined]{radar} & \mSymbol[rounded]{radar} & \mSymbol[sharp]{radar} & \texttt{\textbackslash mSymbol\{radar\}} & \texttt{F04E}\\
\mSymbol[outlined]{radio} & \mSymbol[rounded]{radio} & \mSymbol[sharp]{radio} & \texttt{\textbackslash mSymbol\{radio\}} & \texttt{E03E}\\
\mSymbol[outlined]{radio-button-checked} & \mSymbol[rounded]{radio-button-checked} & \mSymbol[sharp]{radio-button-checked} & \texttt{\textbackslash mSymbol\{radio-button-checked\}} & \texttt{E837}\\
\mSymbol[outlined]{radio-button-partial} & \mSymbol[rounded]{radio-button-partial} & \mSymbol[sharp]{radio-button-partial} & \texttt{\textbackslash mSymbol\{radio-button-partial\}} & \texttt{F560}\\
\mSymbol[outlined]{radio-button-unchecked} & \mSymbol[rounded]{radio-button-unchecked} & \mSymbol[sharp]{radio-button-unchecked} & \texttt{\textbackslash mSymbol\{radio-button-unchecked\}} & \texttt{E836}\\
\mSymbol[outlined]{radiology} & \mSymbol[rounded]{radiology} & \mSymbol[sharp]{radiology} & \texttt{\textbackslash mSymbol\{radiology\}} & \texttt{E125}\\
\mSymbol[outlined]{railway-alert} & \mSymbol[rounded]{railway-alert} & \mSymbol[sharp]{railway-alert} & \texttt{\textbackslash mSymbol\{railway-alert\}} & \texttt{E9D1}\\
\mSymbol[outlined]{railway-alert-2} & \mSymbol[rounded]{railway-alert-2} & \mSymbol[sharp]{railway-alert-2} & \texttt{\textbackslash mSymbol\{railway-alert-2\}} & \texttt{F461}\\
\mSymbol[outlined]{rainy} & \mSymbol[rounded]{rainy} & \mSymbol[sharp]{rainy} & \texttt{\textbackslash mSymbol\{rainy\}} & \texttt{F176}\\
\mSymbol[outlined]{rainy-heavy} & \mSymbol[rounded]{rainy-heavy} & \mSymbol[sharp]{rainy-heavy} & \texttt{\textbackslash mSymbol\{rainy-heavy\}} & \texttt{F61F}\\
\mSymbol[outlined]{rainy-light} & \mSymbol[rounded]{rainy-light} & \mSymbol[sharp]{rainy-light} & \texttt{\textbackslash mSymbol\{rainy-light\}} & \texttt{F61E}\\
\mSymbol[outlined]{rainy-snow} & \mSymbol[rounded]{rainy-snow} & \mSymbol[sharp]{rainy-snow} & \texttt{\textbackslash mSymbol\{rainy-snow\}} & \texttt{F61D}\\
\mSymbol[outlined]{ramen-dining} & \mSymbol[rounded]{ramen-dining} & \mSymbol[sharp]{ramen-dining} & \texttt{\textbackslash mSymbol\{ramen-dining\}} & \texttt{EA64}\\
\mSymbol[outlined]{ramp-left} & \mSymbol[rounded]{ramp-left} & \mSymbol[sharp]{ramp-left} & \texttt{\textbackslash mSymbol\{ramp-left\}} & \texttt{EB9C}\\
\mSymbol[outlined]{ramp-right} & \mSymbol[rounded]{ramp-right} & \mSymbol[sharp]{ramp-right} & \texttt{\textbackslash mSymbol\{ramp-right\}} & \texttt{EB96}\\
\mSymbol[outlined]{range-hood} & \mSymbol[rounded]{range-hood} & \mSymbol[sharp]{range-hood} & \texttt{\textbackslash mSymbol\{range-hood\}} & \texttt{E1EA}\\
\mSymbol[outlined]{rate-review} & \mSymbol[rounded]{rate-review} & \mSymbol[sharp]{rate-review} & \texttt{\textbackslash mSymbol\{rate-review\}} & \texttt{E560}\\
\mSymbol[outlined]{raven} & \mSymbol[rounded]{raven} & \mSymbol[sharp]{raven} & \texttt{\textbackslash mSymbol\{raven\}} & \texttt{F555}\\
\mSymbol[outlined]{raw-off} & \mSymbol[rounded]{raw-off} & \mSymbol[sharp]{raw-off} & \texttt{\textbackslash mSymbol\{raw-off\}} & \texttt{F04F}\\
\mSymbol[outlined]{raw-on} & \mSymbol[rounded]{raw-on} & \mSymbol[sharp]{raw-on} & \texttt{\textbackslash mSymbol\{raw-on\}} & \texttt{F050}\\
\mSymbol[outlined]{read-more} & \mSymbol[rounded]{read-more} & \mSymbol[sharp]{read-more} & \texttt{\textbackslash mSymbol\{read-more\}} & \texttt{EF6D}\\
\mSymbol[outlined]{readiness-score} & \mSymbol[rounded]{readiness-score} & \mSymbol[sharp]{readiness-score} & \texttt{\textbackslash mSymbol\{readiness-score\}} & \texttt{F6DD}\\
\mSymbol[outlined]{real-estate-agent} & \mSymbol[rounded]{real-estate-agent} & \mSymbol[sharp]{real-estate-agent} & \texttt{\textbackslash mSymbol\{real-estate-agent\}} & \texttt{E73A}\\
\mSymbol[outlined]{rear-camera} & \mSymbol[rounded]{rear-camera} & \mSymbol[sharp]{rear-camera} & \texttt{\textbackslash mSymbol\{rear-camera\}} & \texttt{F6C2}\\
\mSymbol[outlined]{rebase} & \mSymbol[rounded]{rebase} & \mSymbol[sharp]{rebase} & \texttt{\textbackslash mSymbol\{rebase\}} & \texttt{F845}\\
\mSymbol[outlined]{rebase-edit} & \mSymbol[rounded]{rebase-edit} & \mSymbol[sharp]{rebase-edit} & \texttt{\textbackslash mSymbol\{rebase-edit\}} & \texttt{F846}\\
\mSymbol[outlined]{receipt} & \mSymbol[rounded]{receipt} & \mSymbol[sharp]{receipt} & \texttt{\textbackslash mSymbol\{receipt\}} & \texttt{E8B0}\\
\mSymbol[outlined]{receipt-long} & \mSymbol[rounded]{receipt-long} & \mSymbol[sharp]{receipt-long} & \texttt{\textbackslash mSymbol\{receipt-long\}} & \texttt{EF6E}\\
\mSymbol[outlined]{recent-actors} & \mSymbol[rounded]{recent-actors} & \mSymbol[sharp]{recent-actors} & \texttt{\textbackslash mSymbol\{recent-actors\}} & \texttt{E03F}\\
\mSymbol[outlined]{recent-patient} & \mSymbol[rounded]{recent-patient} & \mSymbol[sharp]{recent-patient} & \texttt{\textbackslash mSymbol\{recent-patient\}} & \texttt{F808}\\
\mSymbol[outlined]{recenter} & \mSymbol[rounded]{recenter} & \mSymbol[sharp]{recenter} & \texttt{\textbackslash mSymbol\{recenter\}} & \texttt{F4C0}\\
\mSymbol[outlined]{recommend} & \mSymbol[rounded]{recommend} & \mSymbol[sharp]{recommend} & \texttt{\textbackslash mSymbol\{recommend\}} & \texttt{E9D2}\\
\mSymbol[outlined]{record-voice-over} & \mSymbol[rounded]{record-voice-over} & \mSymbol[sharp]{record-voice-over} & \texttt{\textbackslash mSymbol\{record-voice-over\}} & \texttt{E91F}\\
\mSymbol[outlined]{rectangle} & \mSymbol[rounded]{rectangle} & \mSymbol[sharp]{rectangle} & \texttt{\textbackslash mSymbol\{rectangle\}} & \texttt{EB54}\\
\mSymbol[outlined]{recycling} & \mSymbol[rounded]{recycling} & \mSymbol[sharp]{recycling} & \texttt{\textbackslash mSymbol\{recycling\}} & \texttt{E760}\\
\mSymbol[outlined]{redeem} & \mSymbol[rounded]{redeem} & \mSymbol[sharp]{redeem} & \texttt{\textbackslash mSymbol\{redeem\}} & \texttt{E8F6}\\
\mSymbol[outlined]{redo} & \mSymbol[rounded]{redo} & \mSymbol[sharp]{redo} & \texttt{\textbackslash mSymbol\{redo\}} & \texttt{E15A}\\
\mSymbol[outlined]{reduce-capacity} & \mSymbol[rounded]{reduce-capacity} & \mSymbol[sharp]{reduce-capacity} & \texttt{\textbackslash mSymbol\{reduce-capacity\}} & \texttt{F21C}\\
\mSymbol[outlined]{refresh} & \mSymbol[rounded]{refresh} & \mSymbol[sharp]{refresh} & \texttt{\textbackslash mSymbol\{refresh\}} & \texttt{E5D5}\\
\mSymbol[outlined]{regular-expression} & \mSymbol[rounded]{regular-expression} & \mSymbol[sharp]{regular-expression} & \texttt{\textbackslash mSymbol\{regular-expression\}} & \texttt{F750}\\
\mSymbol[outlined]{relax} & \mSymbol[rounded]{relax} & \mSymbol[sharp]{relax} & \texttt{\textbackslash mSymbol\{relax\}} & \texttt{F6DC}\\
\mSymbol[outlined]{release-alert} & \mSymbol[rounded]{release-alert} & \mSymbol[sharp]{release-alert} & \texttt{\textbackslash mSymbol\{release-alert\}} & \texttt{F654}\\
\mSymbol[outlined]{remember-me} & \mSymbol[rounded]{remember-me} & \mSymbol[sharp]{remember-me} & \texttt{\textbackslash mSymbol\{remember-me\}} & \texttt{F051}\\
\mSymbol[outlined]{reminder} & \mSymbol[rounded]{reminder} & \mSymbol[sharp]{reminder} & \texttt{\textbackslash mSymbol\{reminder\}} & \texttt{E6C6}\\
\mSymbol[outlined]{reminders-alt} & \mSymbol[rounded]{reminders-alt} & \mSymbol[sharp]{reminders-alt} & \texttt{\textbackslash mSymbol\{reminders-alt\}} & \texttt{E6C6}\\
\mSymbol[outlined]{remote-gen} & \mSymbol[rounded]{remote-gen} & \mSymbol[sharp]{remote-gen} & \texttt{\textbackslash mSymbol\{remote-gen\}} & \texttt{E83E}\\
\mSymbol[outlined]{remove} & \mSymbol[rounded]{remove} & \mSymbol[sharp]{remove} & \texttt{\textbackslash mSymbol\{remove\}} & \texttt{E15B}\\
\mSymbol[outlined]{remove-circle} & \mSymbol[rounded]{remove-circle} & \mSymbol[sharp]{remove-circle} & \texttt{\textbackslash mSymbol\{remove-circle\}} & \texttt{F08F}\\
\mSymbol[outlined]{remove-circle-outline} & \mSymbol[rounded]{remove-circle-outline} & \mSymbol[sharp]{remove-circle-outline} & \texttt{\textbackslash mSymbol\{remove-circle-outline\}} & \texttt{F08F}\\
\mSymbol[outlined]{remove-done} & \mSymbol[rounded]{remove-done} & \mSymbol[sharp]{remove-done} & \texttt{\textbackslash mSymbol\{remove-done\}} & \texttt{E9D3}\\
\mSymbol[outlined]{remove-from-queue} & \mSymbol[rounded]{remove-from-queue} & \mSymbol[sharp]{remove-from-queue} & \texttt{\textbackslash mSymbol\{remove-from-queue\}} & \texttt{E067}\\
\mSymbol[outlined]{remove-moderator} & \mSymbol[rounded]{remove-moderator} & \mSymbol[sharp]{remove-moderator} & \texttt{\textbackslash mSymbol\{remove-moderator\}} & \texttt{E9D4}\\
\mSymbol[outlined]{remove-red-eye} & \mSymbol[rounded]{remove-red-eye} & \mSymbol[sharp]{remove-red-eye} & \texttt{\textbackslash mSymbol\{remove-red-eye\}} & \texttt{E8F4}\\
\mSymbol[outlined]{remove-road} & \mSymbol[rounded]{remove-road} & \mSymbol[sharp]{remove-road} & \texttt{\textbackslash mSymbol\{remove-road\}} & \texttt{EBFC}\\
\mSymbol[outlined]{remove-selection} & \mSymbol[rounded]{remove-selection} & \mSymbol[sharp]{remove-selection} & \texttt{\textbackslash mSymbol\{remove-selection\}} & \texttt{E9D5}\\
\mSymbol[outlined]{remove-shopping-cart} & \mSymbol[rounded]{remove-shopping-cart} & \mSymbol[sharp]{remove-shopping-cart} & \texttt{\textbackslash mSymbol\{remove-shopping-cart\}} & \texttt{E928}\\
\mSymbol[outlined]{reopen-window} & \mSymbol[rounded]{reopen-window} & \mSymbol[sharp]{reopen-window} & \texttt{\textbackslash mSymbol\{reopen-window\}} & \texttt{F708}\\
\mSymbol[outlined]{reorder} & \mSymbol[rounded]{reorder} & \mSymbol[sharp]{reorder} & \texttt{\textbackslash mSymbol\{reorder\}} & \texttt{E8FE}\\
\mSymbol[outlined]{repartition} & \mSymbol[rounded]{repartition} & \mSymbol[sharp]{repartition} & \texttt{\textbackslash mSymbol\{repartition\}} & \texttt{F8E8}\\
\mSymbol[outlined]{repeat} & \mSymbol[rounded]{repeat} & \mSymbol[sharp]{repeat} & \texttt{\textbackslash mSymbol\{repeat\}} & \texttt{E040}\\
\mSymbol[outlined]{repeat-on} & \mSymbol[rounded]{repeat-on} & \mSymbol[sharp]{repeat-on} & \texttt{\textbackslash mSymbol\{repeat-on\}} & \texttt{E9D6}\\
\mSymbol[outlined]{repeat-one} & \mSymbol[rounded]{repeat-one} & \mSymbol[sharp]{repeat-one} & \texttt{\textbackslash mSymbol\{repeat-one\}} & \texttt{E041}\\
\mSymbol[outlined]{repeat-one-on} & \mSymbol[rounded]{repeat-one-on} & \mSymbol[sharp]{repeat-one-on} & \texttt{\textbackslash mSymbol\{repeat-one-on\}} & \texttt{E9D7}\\
\mSymbol[outlined]{replace-audio} & \mSymbol[rounded]{replace-audio} & \mSymbol[sharp]{replace-audio} & \texttt{\textbackslash mSymbol\{replace-audio\}} & \texttt{F451}\\
\mSymbol[outlined]{replace-image} & \mSymbol[rounded]{replace-image} & \mSymbol[sharp]{replace-image} & \texttt{\textbackslash mSymbol\{replace-image\}} & \texttt{F450}\\
\mSymbol[outlined]{replace-video} & \mSymbol[rounded]{replace-video} & \mSymbol[sharp]{replace-video} & \texttt{\textbackslash mSymbol\{replace-video\}} & \texttt{F44F}\\
\mSymbol[outlined]{replay} & \mSymbol[rounded]{replay} & \mSymbol[sharp]{replay} & \texttt{\textbackslash mSymbol\{replay\}} & \texttt{E042}\\
\mSymbol[outlined]{replay-10} & \mSymbol[rounded]{replay-10} & \mSymbol[sharp]{replay-10} & \texttt{\textbackslash mSymbol\{replay-10\}} & \texttt{E059}\\
\mSymbol[outlined]{replay-30} & \mSymbol[rounded]{replay-30} & \mSymbol[sharp]{replay-30} & \texttt{\textbackslash mSymbol\{replay-30\}} & \texttt{E05A}\\
\mSymbol[outlined]{replay-5} & \mSymbol[rounded]{replay-5} & \mSymbol[sharp]{replay-5} & \texttt{\textbackslash mSymbol\{replay-5\}} & \texttt{E05B}\\
\mSymbol[outlined]{replay-circle-filled} & \mSymbol[rounded]{replay-circle-filled} & \mSymbol[sharp]{replay-circle-filled} & \texttt{\textbackslash mSymbol\{replay-circle-filled\}} & \texttt{E9D8}\\
\mSymbol[outlined]{reply} & \mSymbol[rounded]{reply} & \mSymbol[sharp]{reply} & \texttt{\textbackslash mSymbol\{reply\}} & \texttt{E15E}\\
\mSymbol[outlined]{reply-all} & \mSymbol[rounded]{reply-all} & \mSymbol[sharp]{reply-all} & \texttt{\textbackslash mSymbol\{reply-all\}} & \texttt{E15F}\\
\mSymbol[outlined]{report} & \mSymbol[rounded]{report} & \mSymbol[sharp]{report} & \texttt{\textbackslash mSymbol\{report\}} & \texttt{F052}\\
\mSymbol[outlined]{report-gmailerrorred} & \mSymbol[rounded]{report-gmailerrorred} & \mSymbol[sharp]{report-gmailerrorred} & \texttt{\textbackslash mSymbol\{report-gmailerrorred\}} & \texttt{F052}\\
\mSymbol[outlined]{report-off} & \mSymbol[rounded]{report-off} & \mSymbol[sharp]{report-off} & \texttt{\textbackslash mSymbol\{report-off\}} & \texttt{E170}\\
\mSymbol[outlined]{report-problem} & \mSymbol[rounded]{report-problem} & \mSymbol[sharp]{report-problem} & \texttt{\textbackslash mSymbol\{report-problem\}} & \texttt{F083}\\
\mSymbol[outlined]{request-page} & \mSymbol[rounded]{request-page} & \mSymbol[sharp]{request-page} & \texttt{\textbackslash mSymbol\{request-page\}} & \texttt{F22C}\\
\mSymbol[outlined]{request-quote} & \mSymbol[rounded]{request-quote} & \mSymbol[sharp]{request-quote} & \texttt{\textbackslash mSymbol\{request-quote\}} & \texttt{F1B6}\\
\mSymbol[outlined]{reset-brightness} & \mSymbol[rounded]{reset-brightness} & \mSymbol[sharp]{reset-brightness} & \texttt{\textbackslash mSymbol\{reset-brightness\}} & \texttt{F482}\\
\mSymbol[outlined]{reset-focus} & \mSymbol[rounded]{reset-focus} & \mSymbol[sharp]{reset-focus} & \texttt{\textbackslash mSymbol\{reset-focus\}} & \texttt{F481}\\
\mSymbol[outlined]{reset-image} & \mSymbol[rounded]{reset-image} & \mSymbol[sharp]{reset-image} & \texttt{\textbackslash mSymbol\{reset-image\}} & \texttt{F824}\\
\mSymbol[outlined]{reset-iso} & \mSymbol[rounded]{reset-iso} & \mSymbol[sharp]{reset-iso} & \texttt{\textbackslash mSymbol\{reset-iso\}} & \texttt{F480}\\
\mSymbol[outlined]{reset-settings} & \mSymbol[rounded]{reset-settings} & \mSymbol[sharp]{reset-settings} & \texttt{\textbackslash mSymbol\{reset-settings\}} & \texttt{F47F}\\
\mSymbol[outlined]{reset-shadow} & \mSymbol[rounded]{reset-shadow} & \mSymbol[sharp]{reset-shadow} & \texttt{\textbackslash mSymbol\{reset-shadow\}} & \texttt{F47E}\\
\mSymbol[outlined]{reset-shutter-speed} & \mSymbol[rounded]{reset-shutter-speed} & \mSymbol[sharp]{reset-shutter-speed} & \texttt{\textbackslash mSymbol\{reset-shutter-speed\}} & \texttt{F47D}\\
\mSymbol[outlined]{reset-tv} & \mSymbol[rounded]{reset-tv} & \mSymbol[sharp]{reset-tv} & \texttt{\textbackslash mSymbol\{reset-tv\}} & \texttt{E9D9}\\
\mSymbol[outlined]{reset-white-balance} & \mSymbol[rounded]{reset-white-balance} & \mSymbol[sharp]{reset-white-balance} & \texttt{\textbackslash mSymbol\{reset-white-balance\}} & \texttt{F47C}\\
\mSymbol[outlined]{reset-wrench} & \mSymbol[rounded]{reset-wrench} & \mSymbol[sharp]{reset-wrench} & \texttt{\textbackslash mSymbol\{reset-wrench\}} & \texttt{F56C}\\
\mSymbol[outlined]{resize} & \mSymbol[rounded]{resize} & \mSymbol[sharp]{resize} & \texttt{\textbackslash mSymbol\{resize\}} & \texttt{F707}\\
\mSymbol[outlined]{respiratory-rate} & \mSymbol[rounded]{respiratory-rate} & \mSymbol[sharp]{respiratory-rate} & \texttt{\textbackslash mSymbol\{respiratory-rate\}} & \texttt{E127}\\
\mSymbol[outlined]{responsive-layout} & \mSymbol[rounded]{responsive-layout} & \mSymbol[sharp]{responsive-layout} & \texttt{\textbackslash mSymbol\{responsive-layout\}} & \texttt{E9DA}\\
\mSymbol[outlined]{restart-alt} & \mSymbol[rounded]{restart-alt} & \mSymbol[sharp]{restart-alt} & \texttt{\textbackslash mSymbol\{restart-alt\}} & \texttt{F053}\\
\mSymbol[outlined]{restaurant} & \mSymbol[rounded]{restaurant} & \mSymbol[sharp]{restaurant} & \texttt{\textbackslash mSymbol\{restaurant\}} & \texttt{E56C}\\
\mSymbol[outlined]{restaurant-menu} & \mSymbol[rounded]{restaurant-menu} & \mSymbol[sharp]{restaurant-menu} & \texttt{\textbackslash mSymbol\{restaurant-menu\}} & \texttt{E561}\\
\mSymbol[outlined]{restore} & \mSymbol[rounded]{restore} & \mSymbol[sharp]{restore} & \texttt{\textbackslash mSymbol\{restore\}} & \texttt{E8B3}\\
\mSymbol[outlined]{restore-from-trash} & \mSymbol[rounded]{restore-from-trash} & \mSymbol[sharp]{restore-from-trash} & \texttt{\textbackslash mSymbol\{restore-from-trash\}} & \texttt{E938}\\
\mSymbol[outlined]{restore-page} & \mSymbol[rounded]{restore-page} & \mSymbol[sharp]{restore-page} & \texttt{\textbackslash mSymbol\{restore-page\}} & \texttt{E929}\\
\mSymbol[outlined]{resume} & \mSymbol[rounded]{resume} & \mSymbol[sharp]{resume} & \texttt{\textbackslash mSymbol\{resume\}} & \texttt{F7D0}\\
\mSymbol[outlined]{reviews} & \mSymbol[rounded]{reviews} & \mSymbol[sharp]{reviews} & \texttt{\textbackslash mSymbol\{reviews\}} & \texttt{F07C}\\
\mSymbol[outlined]{rewarded-ads} & \mSymbol[rounded]{rewarded-ads} & \mSymbol[sharp]{rewarded-ads} & \texttt{\textbackslash mSymbol\{rewarded-ads\}} & \texttt{EFB6}\\
\mSymbol[outlined]{rheumatology} & \mSymbol[rounded]{rheumatology} & \mSymbol[sharp]{rheumatology} & \texttt{\textbackslash mSymbol\{rheumatology\}} & \texttt{E128}\\
\mSymbol[outlined]{rib-cage} & \mSymbol[rounded]{rib-cage} & \mSymbol[sharp]{rib-cage} & \texttt{\textbackslash mSymbol\{rib-cage\}} & \texttt{F898}\\
\mSymbol[outlined]{rice-bowl} & \mSymbol[rounded]{rice-bowl} & \mSymbol[sharp]{rice-bowl} & \texttt{\textbackslash mSymbol\{rice-bowl\}} & \texttt{F1F5}\\
\mSymbol[outlined]{right-click} & \mSymbol[rounded]{right-click} & \mSymbol[sharp]{right-click} & \texttt{\textbackslash mSymbol\{right-click\}} & \texttt{F706}\\
\mSymbol[outlined]{right-panel-close} & \mSymbol[rounded]{right-panel-close} & \mSymbol[sharp]{right-panel-close} & \texttt{\textbackslash mSymbol\{right-panel-close\}} & \texttt{F705}\\
\mSymbol[outlined]{right-panel-open} & \mSymbol[rounded]{right-panel-open} & \mSymbol[sharp]{right-panel-open} & \texttt{\textbackslash mSymbol\{right-panel-open\}} & \texttt{F704}\\
\mSymbol[outlined]{ring-volume} & \mSymbol[rounded]{ring-volume} & \mSymbol[sharp]{ring-volume} & \texttt{\textbackslash mSymbol\{ring-volume\}} & \texttt{F0DD}\\
\mSymbol[outlined]{ring-volume-filled} & \mSymbol[rounded]{ring-volume-filled} & \mSymbol[sharp]{ring-volume-filled} & \texttt{\textbackslash mSymbol\{ring-volume-filled\}} & \texttt{F0DD}\\
\mSymbol[outlined]{ripples} & \mSymbol[rounded]{ripples} & \mSymbol[sharp]{ripples} & \texttt{\textbackslash mSymbol\{ripples\}} & \texttt{E9DB}\\
\mSymbol[outlined]{road} & \mSymbol[rounded]{road} & \mSymbol[sharp]{road} & \texttt{\textbackslash mSymbol\{road\}} & \texttt{F472}\\
\mSymbol[outlined]{robot} & \mSymbol[rounded]{robot} & \mSymbol[sharp]{robot} & \texttt{\textbackslash mSymbol\{robot\}} & \texttt{F882}\\
\mSymbol[outlined]{robot-2} & \mSymbol[rounded]{robot-2} & \mSymbol[sharp]{robot-2} & \texttt{\textbackslash mSymbol\{robot-2\}} & \texttt{F5D0}\\
\mSymbol[outlined]{rocket} & \mSymbol[rounded]{rocket} & \mSymbol[sharp]{rocket} & \texttt{\textbackslash mSymbol\{rocket\}} & \texttt{EBA5}\\
\mSymbol[outlined]{rocket-launch} & \mSymbol[rounded]{rocket-launch} & \mSymbol[sharp]{rocket-launch} & \texttt{\textbackslash mSymbol\{rocket-launch\}} & \texttt{EB9B}\\
\mSymbol[outlined]{roller-shades} & \mSymbol[rounded]{roller-shades} & \mSymbol[sharp]{roller-shades} & \texttt{\textbackslash mSymbol\{roller-shades\}} & \texttt{EC12}\\
\mSymbol[outlined]{roller-shades-closed} & \mSymbol[rounded]{roller-shades-closed} & \mSymbol[sharp]{roller-shades-closed} & \texttt{\textbackslash mSymbol\{roller-shades-closed\}} & \texttt{EC11}\\
\mSymbol[outlined]{roller-skating} & \mSymbol[rounded]{roller-skating} & \mSymbol[sharp]{roller-skating} & \texttt{\textbackslash mSymbol\{roller-skating\}} & \texttt{EBCD}\\
\mSymbol[outlined]{roofing} & \mSymbol[rounded]{roofing} & \mSymbol[sharp]{roofing} & \texttt{\textbackslash mSymbol\{roofing\}} & \texttt{F201}\\
\mSymbol[outlined]{room} & \mSymbol[rounded]{room} & \mSymbol[sharp]{room} & \texttt{\textbackslash mSymbol\{room\}} & \texttt{F1DB}\\
\mSymbol[outlined]{room-preferences} & \mSymbol[rounded]{room-preferences} & \mSymbol[sharp]{room-preferences} & \texttt{\textbackslash mSymbol\{room-preferences\}} & \texttt{F1B8}\\
\mSymbol[outlined]{room-service} & \mSymbol[rounded]{room-service} & \mSymbol[sharp]{room-service} & \texttt{\textbackslash mSymbol\{room-service\}} & \texttt{EB49}\\
\mSymbol[outlined]{rotate-90-degrees-ccw} & \mSymbol[rounded]{rotate-90-degrees-ccw} & \mSymbol[sharp]{rotate-90-degrees-ccw} & \texttt{\textbackslash mSymbol\{rotate-90-degrees-ccw\}} & \texttt{E418}\\
\mSymbol[outlined]{rotate-90-degrees-cw} & \mSymbol[rounded]{rotate-90-degrees-cw} & \mSymbol[sharp]{rotate-90-degrees-cw} & \texttt{\textbackslash mSymbol\{rotate-90-degrees-cw\}} & \texttt{EAAB}\\
\mSymbol[outlined]{rotate-auto} & \mSymbol[rounded]{rotate-auto} & \mSymbol[sharp]{rotate-auto} & \texttt{\textbackslash mSymbol\{rotate-auto\}} & \texttt{F417}\\
\mSymbol[outlined]{rotate-left} & \mSymbol[rounded]{rotate-left} & \mSymbol[sharp]{rotate-left} & \texttt{\textbackslash mSymbol\{rotate-left\}} & \texttt{E419}\\
\mSymbol[outlined]{rotate-right} & \mSymbol[rounded]{rotate-right} & \mSymbol[sharp]{rotate-right} & \texttt{\textbackslash mSymbol\{rotate-right\}} & \texttt{E41A}\\
\mSymbol[outlined]{roundabout-left} & \mSymbol[rounded]{roundabout-left} & \mSymbol[sharp]{roundabout-left} & \texttt{\textbackslash mSymbol\{roundabout-left\}} & \texttt{EB99}\\
\mSymbol[outlined]{roundabout-right} & \mSymbol[rounded]{roundabout-right} & \mSymbol[sharp]{roundabout-right} & \texttt{\textbackslash mSymbol\{roundabout-right\}} & \texttt{EBA3}\\
\mSymbol[outlined]{rounded-corner} & \mSymbol[rounded]{rounded-corner} & \mSymbol[sharp]{rounded-corner} & \texttt{\textbackslash mSymbol\{rounded-corner\}} & \texttt{E920}\\
\mSymbol[outlined]{route} & \mSymbol[rounded]{route} & \mSymbol[sharp]{route} & \texttt{\textbackslash mSymbol\{route\}} & \texttt{EACD}\\
\mSymbol[outlined]{router} & \mSymbol[rounded]{router} & \mSymbol[sharp]{router} & \texttt{\textbackslash mSymbol\{router\}} & \texttt{E328}\\
\mSymbol[outlined]{routine} & \mSymbol[rounded]{routine} & \mSymbol[sharp]{routine} & \texttt{\textbackslash mSymbol\{routine\}} & \texttt{E20C}\\
\mSymbol[outlined]{rowing} & \mSymbol[rounded]{rowing} & \mSymbol[sharp]{rowing} & \texttt{\textbackslash mSymbol\{rowing\}} & \texttt{E921}\\
\mSymbol[outlined]{rss-feed} & \mSymbol[rounded]{rss-feed} & \mSymbol[sharp]{rss-feed} & \texttt{\textbackslash mSymbol\{rss-feed\}} & \texttt{E0E5}\\
\mSymbol[outlined]{rsvp} & \mSymbol[rounded]{rsvp} & \mSymbol[sharp]{rsvp} & \texttt{\textbackslash mSymbol\{rsvp\}} & \texttt{F055}\\
\mSymbol[outlined]{rtt} & \mSymbol[rounded]{rtt} & \mSymbol[sharp]{rtt} & \texttt{\textbackslash mSymbol\{rtt\}} & \texttt{E9AD}\\
\mSymbol[outlined]{rubric} & \mSymbol[rounded]{rubric} & \mSymbol[sharp]{rubric} & \texttt{\textbackslash mSymbol\{rubric\}} & \texttt{EB27}\\
\mSymbol[outlined]{rule} & \mSymbol[rounded]{rule} & \mSymbol[sharp]{rule} & \texttt{\textbackslash mSymbol\{rule\}} & \texttt{F1C2}\\
\mSymbol[outlined]{rule-folder} & \mSymbol[rounded]{rule-folder} & \mSymbol[sharp]{rule-folder} & \texttt{\textbackslash mSymbol\{rule-folder\}} & \texttt{F1C9}\\
\mSymbol[outlined]{rule-settings} & \mSymbol[rounded]{rule-settings} & \mSymbol[sharp]{rule-settings} & \texttt{\textbackslash mSymbol\{rule-settings\}} & \texttt{F64C}\\
\mSymbol[outlined]{run-circle} & \mSymbol[rounded]{run-circle} & \mSymbol[sharp]{run-circle} & \texttt{\textbackslash mSymbol\{run-circle\}} & \texttt{EF6F}\\
\mSymbol[outlined]{running-with-errors} & \mSymbol[rounded]{running-with-errors} & \mSymbol[sharp]{running-with-errors} & \texttt{\textbackslash mSymbol\{running-with-errors\}} & \texttt{E51D}\\
\mSymbol[outlined]{rv-hookup} & \mSymbol[rounded]{rv-hookup} & \mSymbol[sharp]{rv-hookup} & \texttt{\textbackslash mSymbol\{rv-hookup\}} & \texttt{E642}\\
\mSymbol[outlined]{safety-check} & \mSymbol[rounded]{safety-check} & \mSymbol[sharp]{safety-check} & \texttt{\textbackslash mSymbol\{safety-check\}} & \texttt{EBEF}\\
\mSymbol[outlined]{safety-check-off} & \mSymbol[rounded]{safety-check-off} & \mSymbol[sharp]{safety-check-off} & \texttt{\textbackslash mSymbol\{safety-check-off\}} & \texttt{F59D}\\
\mSymbol[outlined]{safety-divider} & \mSymbol[rounded]{safety-divider} & \mSymbol[sharp]{safety-divider} & \texttt{\textbackslash mSymbol\{safety-divider\}} & \texttt{E1CC}\\
\mSymbol[outlined]{sailing} & \mSymbol[rounded]{sailing} & \mSymbol[sharp]{sailing} & \texttt{\textbackslash mSymbol\{sailing\}} & \texttt{E502}\\
\mSymbol[outlined]{salinity} & \mSymbol[rounded]{salinity} & \mSymbol[sharp]{salinity} & \texttt{\textbackslash mSymbol\{salinity\}} & \texttt{F876}\\
\mSymbol[outlined]{sanitizer} & \mSymbol[rounded]{sanitizer} & \mSymbol[sharp]{sanitizer} & \texttt{\textbackslash mSymbol\{sanitizer\}} & \texttt{F21D}\\
\mSymbol[outlined]{satellite} & \mSymbol[rounded]{satellite} & \mSymbol[sharp]{satellite} & \texttt{\textbackslash mSymbol\{satellite\}} & \texttt{E562}\\
\mSymbol[outlined]{satellite-alt} & \mSymbol[rounded]{satellite-alt} & \mSymbol[sharp]{satellite-alt} & \texttt{\textbackslash mSymbol\{satellite-alt\}} & \texttt{EB3A}\\
\mSymbol[outlined]{sauna} & \mSymbol[rounded]{sauna} & \mSymbol[sharp]{sauna} & \texttt{\textbackslash mSymbol\{sauna\}} & \texttt{F6F7}\\
\mSymbol[outlined]{save} & \mSymbol[rounded]{save} & \mSymbol[sharp]{save} & \texttt{\textbackslash mSymbol\{save\}} & \texttt{E161}\\
\mSymbol[outlined]{save-alt} & \mSymbol[rounded]{save-alt} & \mSymbol[sharp]{save-alt} & \texttt{\textbackslash mSymbol\{save-alt\}} & \texttt{F090}\\
\mSymbol[outlined]{save-as} & \mSymbol[rounded]{save-as} & \mSymbol[sharp]{save-as} & \texttt{\textbackslash mSymbol\{save-as\}} & \texttt{EB60}\\
\mSymbol[outlined]{saved-search} & \mSymbol[rounded]{saved-search} & \mSymbol[sharp]{saved-search} & \texttt{\textbackslash mSymbol\{saved-search\}} & \texttt{EA11}\\
\mSymbol[outlined]{savings} & \mSymbol[rounded]{savings} & \mSymbol[sharp]{savings} & \texttt{\textbackslash mSymbol\{savings\}} & \texttt{E2EB}\\
\mSymbol[outlined]{scale} & \mSymbol[rounded]{scale} & \mSymbol[sharp]{scale} & \texttt{\textbackslash mSymbol\{scale\}} & \texttt{EB5F}\\
\mSymbol[outlined]{scan} & \mSymbol[rounded]{scan} & \mSymbol[sharp]{scan} & \texttt{\textbackslash mSymbol\{scan\}} & \texttt{F74E}\\
\mSymbol[outlined]{scan-delete} & \mSymbol[rounded]{scan-delete} & \mSymbol[sharp]{scan-delete} & \texttt{\textbackslash mSymbol\{scan-delete\}} & \texttt{F74F}\\
\mSymbol[outlined]{scanner} & \mSymbol[rounded]{scanner} & \mSymbol[sharp]{scanner} & \texttt{\textbackslash mSymbol\{scanner\}} & \texttt{E329}\\
\mSymbol[outlined]{scatter-plot} & \mSymbol[rounded]{scatter-plot} & \mSymbol[sharp]{scatter-plot} & \texttt{\textbackslash mSymbol\{scatter-plot\}} & \texttt{E268}\\
\mSymbol[outlined]{scene} & \mSymbol[rounded]{scene} & \mSymbol[sharp]{scene} & \texttt{\textbackslash mSymbol\{scene\}} & \texttt{E2A7}\\
\mSymbol[outlined]{schedule} & \mSymbol[rounded]{schedule} & \mSymbol[sharp]{schedule} & \texttt{\textbackslash mSymbol\{schedule\}} & \texttt{EFD6}\\
\mSymbol[outlined]{schedule-send} & \mSymbol[rounded]{schedule-send} & \mSymbol[sharp]{schedule-send} & \texttt{\textbackslash mSymbol\{schedule-send\}} & \texttt{EA0A}\\
\mSymbol[outlined]{schema} & \mSymbol[rounded]{schema} & \mSymbol[sharp]{schema} & \texttt{\textbackslash mSymbol\{schema\}} & \texttt{E4FD}\\
\mSymbol[outlined]{school} & \mSymbol[rounded]{school} & \mSymbol[sharp]{school} & \texttt{\textbackslash mSymbol\{school\}} & \texttt{E80C}\\
\mSymbol[outlined]{science} & \mSymbol[rounded]{science} & \mSymbol[sharp]{science} & \texttt{\textbackslash mSymbol\{science\}} & \texttt{EA4B}\\
\mSymbol[outlined]{science-off} & \mSymbol[rounded]{science-off} & \mSymbol[sharp]{science-off} & \texttt{\textbackslash mSymbol\{science-off\}} & \texttt{F542}\\
\mSymbol[outlined]{scooter} & \mSymbol[rounded]{scooter} & \mSymbol[sharp]{scooter} & \texttt{\textbackslash mSymbol\{scooter\}} & \texttt{F471}\\
\mSymbol[outlined]{score} & \mSymbol[rounded]{score} & \mSymbol[sharp]{score} & \texttt{\textbackslash mSymbol\{score\}} & \texttt{E269}\\
\mSymbol[outlined]{scoreboard} & \mSymbol[rounded]{scoreboard} & \mSymbol[sharp]{scoreboard} & \texttt{\textbackslash mSymbol\{scoreboard\}} & \texttt{EBD0}\\
\mSymbol[outlined]{screen-lock-landscape} & \mSymbol[rounded]{screen-lock-landscape} & \mSymbol[sharp]{screen-lock-landscape} & \texttt{\textbackslash mSymbol\{screen-lock-landscape\}} & \texttt{E1BE}\\
\mSymbol[outlined]{screen-lock-portrait} & \mSymbol[rounded]{screen-lock-portrait} & \mSymbol[sharp]{screen-lock-portrait} & \texttt{\textbackslash mSymbol\{screen-lock-portrait\}} & \texttt{E1BF}\\
\mSymbol[outlined]{screen-lock-rotation} & \mSymbol[rounded]{screen-lock-rotation} & \mSymbol[sharp]{screen-lock-rotation} & \texttt{\textbackslash mSymbol\{screen-lock-rotation\}} & \texttt{E1C0}\\
\mSymbol[outlined]{screen-record} & \mSymbol[rounded]{screen-record} & \mSymbol[sharp]{screen-record} & \texttt{\textbackslash mSymbol\{screen-record\}} & \texttt{F679}\\
\mSymbol[outlined]{screen-rotation} & \mSymbol[rounded]{screen-rotation} & \mSymbol[sharp]{screen-rotation} & \texttt{\textbackslash mSymbol\{screen-rotation\}} & \texttt{E1C1}\\
\mSymbol[outlined]{screen-rotation-alt} & \mSymbol[rounded]{screen-rotation-alt} & \mSymbol[sharp]{screen-rotation-alt} & \texttt{\textbackslash mSymbol\{screen-rotation-alt\}} & \texttt{EBEE}\\
\mSymbol[outlined]{screen-rotation-up} & \mSymbol[rounded]{screen-rotation-up} & \mSymbol[sharp]{screen-rotation-up} & \texttt{\textbackslash mSymbol\{screen-rotation-up\}} & \texttt{F678}\\
\mSymbol[outlined]{screen-search-desktop} & \mSymbol[rounded]{screen-search-desktop} & \mSymbol[sharp]{screen-search-desktop} & \texttt{\textbackslash mSymbol\{screen-search-desktop\}} & \texttt{EF70}\\
\mSymbol[outlined]{screen-share} & \mSymbol[rounded]{screen-share} & \mSymbol[sharp]{screen-share} & \texttt{\textbackslash mSymbol\{screen-share\}} & \texttt{E0E2}\\
\mSymbol[outlined]{screenshot} & \mSymbol[rounded]{screenshot} & \mSymbol[sharp]{screenshot} & \texttt{\textbackslash mSymbol\{screenshot\}} & \texttt{F056}\\
\mSymbol[outlined]{screenshot-frame} & \mSymbol[rounded]{screenshot-frame} & \mSymbol[sharp]{screenshot-frame} & \texttt{\textbackslash mSymbol\{screenshot-frame\}} & \texttt{F677}\\
\mSymbol[outlined]{screenshot-keyboard} & \mSymbol[rounded]{screenshot-keyboard} & \mSymbol[sharp]{screenshot-keyboard} & \texttt{\textbackslash mSymbol\{screenshot-keyboard\}} & \texttt{F7D3}\\
\mSymbol[outlined]{screenshot-monitor} & \mSymbol[rounded]{screenshot-monitor} & \mSymbol[sharp]{screenshot-monitor} & \texttt{\textbackslash mSymbol\{screenshot-monitor\}} & \texttt{EC08}\\
\mSymbol[outlined]{screenshot-region} & \mSymbol[rounded]{screenshot-region} & \mSymbol[sharp]{screenshot-region} & \texttt{\textbackslash mSymbol\{screenshot-region\}} & \texttt{F7D2}\\
\mSymbol[outlined]{screenshot-tablet} & \mSymbol[rounded]{screenshot-tablet} & \mSymbol[sharp]{screenshot-tablet} & \texttt{\textbackslash mSymbol\{screenshot-tablet\}} & \texttt{F697}\\
\mSymbol[outlined]{script} & \mSymbol[rounded]{script} & \mSymbol[sharp]{script} & \texttt{\textbackslash mSymbol\{script\}} & \texttt{F45F}\\
\mSymbol[outlined]{scrollable-header} & \mSymbol[rounded]{scrollable-header} & \mSymbol[sharp]{scrollable-header} & \texttt{\textbackslash mSymbol\{scrollable-header\}} & \texttt{E9DC}\\
\mSymbol[outlined]{scuba-diving} & \mSymbol[rounded]{scuba-diving} & \mSymbol[sharp]{scuba-diving} & \texttt{\textbackslash mSymbol\{scuba-diving\}} & \texttt{EBCE}\\
\mSymbol[outlined]{sd} & \mSymbol[rounded]{sd} & \mSymbol[sharp]{sd} & \texttt{\textbackslash mSymbol\{sd\}} & \texttt{E9DD}\\
\mSymbol[outlined]{sd-card} & \mSymbol[rounded]{sd-card} & \mSymbol[sharp]{sd-card} & \texttt{\textbackslash mSymbol\{sd-card\}} & \texttt{E623}\\
\mSymbol[outlined]{sd-card-alert} & \mSymbol[rounded]{sd-card-alert} & \mSymbol[sharp]{sd-card-alert} & \texttt{\textbackslash mSymbol\{sd-card-alert\}} & \texttt{F057}\\
\mSymbol[outlined]{sd-storage} & \mSymbol[rounded]{sd-storage} & \mSymbol[sharp]{sd-storage} & \texttt{\textbackslash mSymbol\{sd-storage\}} & \texttt{E623}\\
\mSymbol[outlined]{sdk} & \mSymbol[rounded]{sdk} & \mSymbol[sharp]{sdk} & \texttt{\textbackslash mSymbol\{sdk\}} & \texttt{E720}\\
\mSymbol[outlined]{search} & \mSymbol[rounded]{search} & \mSymbol[sharp]{search} & \texttt{\textbackslash mSymbol\{search\}} & \texttt{E8B6}\\
\mSymbol[outlined]{search-check} & \mSymbol[rounded]{search-check} & \mSymbol[sharp]{search-check} & \texttt{\textbackslash mSymbol\{search-check\}} & \texttt{F800}\\
\mSymbol[outlined]{search-check-2} & \mSymbol[rounded]{search-check-2} & \mSymbol[sharp]{search-check-2} & \texttt{\textbackslash mSymbol\{search-check-2\}} & \texttt{F469}\\
\mSymbol[outlined]{search-hands-free} & \mSymbol[rounded]{search-hands-free} & \mSymbol[sharp]{search-hands-free} & \texttt{\textbackslash mSymbol\{search-hands-free\}} & \texttt{E696}\\
\mSymbol[outlined]{search-insights} & \mSymbol[rounded]{search-insights} & \mSymbol[sharp]{search-insights} & \texttt{\textbackslash mSymbol\{search-insights\}} & \texttt{F4BC}\\
\mSymbol[outlined]{search-off} & \mSymbol[rounded]{search-off} & \mSymbol[sharp]{search-off} & \texttt{\textbackslash mSymbol\{search-off\}} & \texttt{EA76}\\
\mSymbol[outlined]{security} & \mSymbol[rounded]{security} & \mSymbol[sharp]{security} & \texttt{\textbackslash mSymbol\{security\}} & \texttt{E32A}\\
\mSymbol[outlined]{security-key} & \mSymbol[rounded]{security-key} & \mSymbol[sharp]{security-key} & \texttt{\textbackslash mSymbol\{security-key\}} & \texttt{F503}\\
\mSymbol[outlined]{security-update} & \mSymbol[rounded]{security-update} & \mSymbol[sharp]{security-update} & \texttt{\textbackslash mSymbol\{security-update\}} & \texttt{F072}\\
\mSymbol[outlined]{security-update-good} & \mSymbol[rounded]{security-update-good} & \mSymbol[sharp]{security-update-good} & \texttt{\textbackslash mSymbol\{security-update-good\}} & \texttt{F073}\\
\mSymbol[outlined]{security-update-warning} & \mSymbol[rounded]{security-update-warning} & \mSymbol[sharp]{security-update-warning} & \texttt{\textbackslash mSymbol\{security-update-warning\}} & \texttt{F074}\\
\mSymbol[outlined]{segment} & \mSymbol[rounded]{segment} & \mSymbol[sharp]{segment} & \texttt{\textbackslash mSymbol\{segment\}} & \texttt{E94B}\\
\mSymbol[outlined]{select} & \mSymbol[rounded]{select} & \mSymbol[sharp]{select} & \texttt{\textbackslash mSymbol\{select\}} & \texttt{F74D}\\
\mSymbol[outlined]{select-all} & \mSymbol[rounded]{select-all} & \mSymbol[sharp]{select-all} & \texttt{\textbackslash mSymbol\{select-all\}} & \texttt{E162}\\
\mSymbol[outlined]{select-check-box} & \mSymbol[rounded]{select-check-box} & \mSymbol[sharp]{select-check-box} & \texttt{\textbackslash mSymbol\{select-check-box\}} & \texttt{F1FE}\\
\mSymbol[outlined]{select-to-speak} & \mSymbol[rounded]{select-to-speak} & \mSymbol[sharp]{select-to-speak} & \texttt{\textbackslash mSymbol\{select-to-speak\}} & \texttt{F7CF}\\
\mSymbol[outlined]{select-window} & \mSymbol[rounded]{select-window} & \mSymbol[sharp]{select-window} & \texttt{\textbackslash mSymbol\{select-window\}} & \texttt{E6FA}\\
\mSymbol[outlined]{select-window-2} & \mSymbol[rounded]{select-window-2} & \mSymbol[sharp]{select-window-2} & \texttt{\textbackslash mSymbol\{select-window-2\}} & \texttt{F4C8}\\
\mSymbol[outlined]{select-window-off} & \mSymbol[rounded]{select-window-off} & \mSymbol[sharp]{select-window-off} & \texttt{\textbackslash mSymbol\{select-window-off\}} & \texttt{E506}\\
\mSymbol[outlined]{self-care} & \mSymbol[rounded]{self-care} & \mSymbol[sharp]{self-care} & \texttt{\textbackslash mSymbol\{self-care\}} & \texttt{F86D}\\
\mSymbol[outlined]{self-improvement} & \mSymbol[rounded]{self-improvement} & \mSymbol[sharp]{self-improvement} & \texttt{\textbackslash mSymbol\{self-improvement\}} & \texttt{EA78}\\
\mSymbol[outlined]{sell} & \mSymbol[rounded]{sell} & \mSymbol[sharp]{sell} & \texttt{\textbackslash mSymbol\{sell\}} & \texttt{F05B}\\
\mSymbol[outlined]{send} & \mSymbol[rounded]{send} & \mSymbol[sharp]{send} & \texttt{\textbackslash mSymbol\{send\}} & \texttt{E163}\\
\mSymbol[outlined]{send-and-archive} & \mSymbol[rounded]{send-and-archive} & \mSymbol[sharp]{send-and-archive} & \texttt{\textbackslash mSymbol\{send-and-archive\}} & \texttt{EA0C}\\
\mSymbol[outlined]{send-money} & \mSymbol[rounded]{send-money} & \mSymbol[sharp]{send-money} & \texttt{\textbackslash mSymbol\{send-money\}} & \texttt{E8B7}\\
\mSymbol[outlined]{send-time-extension} & \mSymbol[rounded]{send-time-extension} & \mSymbol[sharp]{send-time-extension} & \texttt{\textbackslash mSymbol\{send-time-extension\}} & \texttt{EADB}\\
\mSymbol[outlined]{send-to-mobile} & \mSymbol[rounded]{send-to-mobile} & \mSymbol[sharp]{send-to-mobile} & \texttt{\textbackslash mSymbol\{send-to-mobile\}} & \texttt{F05C}\\
\mSymbol[outlined]{sensor-door} & \mSymbol[rounded]{sensor-door} & \mSymbol[sharp]{sensor-door} & \texttt{\textbackslash mSymbol\{sensor-door\}} & \texttt{F1B5}\\
\mSymbol[outlined]{sensor-occupied} & \mSymbol[rounded]{sensor-occupied} & \mSymbol[sharp]{sensor-occupied} & \texttt{\textbackslash mSymbol\{sensor-occupied\}} & \texttt{EC10}\\
\mSymbol[outlined]{sensor-window} & \mSymbol[rounded]{sensor-window} & \mSymbol[sharp]{sensor-window} & \texttt{\textbackslash mSymbol\{sensor-window\}} & \texttt{F1B4}\\
\mSymbol[outlined]{sensors} & \mSymbol[rounded]{sensors} & \mSymbol[sharp]{sensors} & \texttt{\textbackslash mSymbol\{sensors\}} & \texttt{E51E}\\
\mSymbol[outlined]{sensors-krx} & \mSymbol[rounded]{sensors-krx} & \mSymbol[sharp]{sensors-krx} & \texttt{\textbackslash mSymbol\{sensors-krx\}} & \texttt{F556}\\
\mSymbol[outlined]{sensors-krx-off} & \mSymbol[rounded]{sensors-krx-off} & \mSymbol[sharp]{sensors-krx-off} & \texttt{\textbackslash mSymbol\{sensors-krx-off\}} & \texttt{F515}\\
\mSymbol[outlined]{sensors-off} & \mSymbol[rounded]{sensors-off} & \mSymbol[sharp]{sensors-off} & \texttt{\textbackslash mSymbol\{sensors-off\}} & \texttt{E51F}\\
\mSymbol[outlined]{sentiment-calm} & \mSymbol[rounded]{sentiment-calm} & \mSymbol[sharp]{sentiment-calm} & \texttt{\textbackslash mSymbol\{sentiment-calm\}} & \texttt{F6A7}\\
\mSymbol[outlined]{sentiment-content} & \mSymbol[rounded]{sentiment-content} & \mSymbol[sharp]{sentiment-content} & \texttt{\textbackslash mSymbol\{sentiment-content\}} & \texttt{F6A6}\\
\mSymbol[outlined]{sentiment-dissatisfied} & \mSymbol[rounded]{sentiment-dissatisfied} & \mSymbol[sharp]{sentiment-dissatisfied} & \texttt{\textbackslash mSymbol\{sentiment-dissatisfied\}} & \texttt{E811}\\
\mSymbol[outlined]{sentiment-excited} & \mSymbol[rounded]{sentiment-excited} & \mSymbol[sharp]{sentiment-excited} & \texttt{\textbackslash mSymbol\{sentiment-excited\}} & \texttt{F6A5}\\
\mSymbol[outlined]{sentiment-extremely-dissatisfied} & \mSymbol[rounded]{sentiment-extremely-dissatisfied} & \mSymbol[sharp]{sentiment-extremely-dissatisfied} & \texttt{\textbackslash mSymbol\{sentiment-extremely-dissatisfied\}} & \texttt{F194}\\
\mSymbol[outlined]{sentiment-frustrated} & \mSymbol[rounded]{sentiment-frustrated} & \mSymbol[sharp]{sentiment-frustrated} & \texttt{\textbackslash mSymbol\{sentiment-frustrated\}} & \texttt{F6A4}\\
\mSymbol[outlined]{sentiment-neutral} & \mSymbol[rounded]{sentiment-neutral} & \mSymbol[sharp]{sentiment-neutral} & \texttt{\textbackslash mSymbol\{sentiment-neutral\}} & \texttt{E812}\\
\mSymbol[outlined]{sentiment-sad} & \mSymbol[rounded]{sentiment-sad} & \mSymbol[sharp]{sentiment-sad} & \texttt{\textbackslash mSymbol\{sentiment-sad\}} & \texttt{F6A3}\\
\mSymbol[outlined]{sentiment-satisfied} & \mSymbol[rounded]{sentiment-satisfied} & \mSymbol[sharp]{sentiment-satisfied} & \texttt{\textbackslash mSymbol\{sentiment-satisfied\}} & \texttt{E813}\\
\mSymbol[outlined]{sentiment-satisfied-alt} & \mSymbol[rounded]{sentiment-satisfied-alt} & \mSymbol[sharp]{sentiment-satisfied-alt} & \texttt{\textbackslash mSymbol\{sentiment-satisfied-alt\}} & \texttt{E813}\\
\mSymbol[outlined]{sentiment-stressed} & \mSymbol[rounded]{sentiment-stressed} & \mSymbol[sharp]{sentiment-stressed} & \texttt{\textbackslash mSymbol\{sentiment-stressed\}} & \texttt{F6A2}\\
\mSymbol[outlined]{sentiment-very-dissatisfied} & \mSymbol[rounded]{sentiment-very-dissatisfied} & \mSymbol[sharp]{sentiment-very-dissatisfied} & \texttt{\textbackslash mSymbol\{sentiment-very-dissatisfied\}} & \texttt{E814}\\
\mSymbol[outlined]{sentiment-very-satisfied} & \mSymbol[rounded]{sentiment-very-satisfied} & \mSymbol[sharp]{sentiment-very-satisfied} & \texttt{\textbackslash mSymbol\{sentiment-very-satisfied\}} & \texttt{E815}\\
\mSymbol[outlined]{sentiment-worried} & \mSymbol[rounded]{sentiment-worried} & \mSymbol[sharp]{sentiment-worried} & \texttt{\textbackslash mSymbol\{sentiment-worried\}} & \texttt{F6A1}\\
\mSymbol[outlined]{serif} & \mSymbol[rounded]{serif} & \mSymbol[sharp]{serif} & \texttt{\textbackslash mSymbol\{serif\}} & \texttt{F4AC}\\
\mSymbol[outlined]{service-toolbox} & \mSymbol[rounded]{service-toolbox} & \mSymbol[sharp]{service-toolbox} & \texttt{\textbackslash mSymbol\{service-toolbox\}} & \texttt{E717}\\
\mSymbol[outlined]{set-meal} & \mSymbol[rounded]{set-meal} & \mSymbol[sharp]{set-meal} & \texttt{\textbackslash mSymbol\{set-meal\}} & \texttt{F1EA}\\
\mSymbol[outlined]{settings} & \mSymbol[rounded]{settings} & \mSymbol[sharp]{settings} & \texttt{\textbackslash mSymbol\{settings\}} & \texttt{E8B8}\\
\mSymbol[outlined]{settings-accessibility} & \mSymbol[rounded]{settings-accessibility} & \mSymbol[sharp]{settings-accessibility} & \texttt{\textbackslash mSymbol\{settings-accessibility\}} & \texttt{F05D}\\
\mSymbol[outlined]{settings-account-box} & \mSymbol[rounded]{settings-account-box} & \mSymbol[sharp]{settings-account-box} & \texttt{\textbackslash mSymbol\{settings-account-box\}} & \texttt{F835}\\
\mSymbol[outlined]{settings-alert} & \mSymbol[rounded]{settings-alert} & \mSymbol[sharp]{settings-alert} & \texttt{\textbackslash mSymbol\{settings-alert\}} & \texttt{F143}\\
\mSymbol[outlined]{settings-applications} & \mSymbol[rounded]{settings-applications} & \mSymbol[sharp]{settings-applications} & \texttt{\textbackslash mSymbol\{settings-applications\}} & \texttt{E8B9}\\
\mSymbol[outlined]{settings-b-roll} & \mSymbol[rounded]{settings-b-roll} & \mSymbol[sharp]{settings-b-roll} & \texttt{\textbackslash mSymbol\{settings-b-roll\}} & \texttt{F625}\\
\mSymbol[outlined]{settings-backup-restore} & \mSymbol[rounded]{settings-backup-restore} & \mSymbol[sharp]{settings-backup-restore} & \texttt{\textbackslash mSymbol\{settings-backup-restore\}} & \texttt{E8BA}\\
\mSymbol[outlined]{settings-bluetooth} & \mSymbol[rounded]{settings-bluetooth} & \mSymbol[sharp]{settings-bluetooth} & \texttt{\textbackslash mSymbol\{settings-bluetooth\}} & \texttt{E8BB}\\
\mSymbol[outlined]{settings-brightness} & \mSymbol[rounded]{settings-brightness} & \mSymbol[sharp]{settings-brightness} & \texttt{\textbackslash mSymbol\{settings-brightness\}} & \texttt{E8BD}\\
\mSymbol[outlined]{settings-cell} & \mSymbol[rounded]{settings-cell} & \mSymbol[sharp]{settings-cell} & \texttt{\textbackslash mSymbol\{settings-cell\}} & \texttt{E8BC}\\
\mSymbol[outlined]{settings-cinematic-blur} & \mSymbol[rounded]{settings-cinematic-blur} & \mSymbol[sharp]{settings-cinematic-blur} & \texttt{\textbackslash mSymbol\{settings-cinematic-blur\}} & \texttt{F624}\\
\mSymbol[outlined]{settings-ethernet} & \mSymbol[rounded]{settings-ethernet} & \mSymbol[sharp]{settings-ethernet} & \texttt{\textbackslash mSymbol\{settings-ethernet\}} & \texttt{E8BE}\\
\mSymbol[outlined]{settings-heart} & \mSymbol[rounded]{settings-heart} & \mSymbol[sharp]{settings-heart} & \texttt{\textbackslash mSymbol\{settings-heart\}} & \texttt{F522}\\
\mSymbol[outlined]{settings-input-antenna} & \mSymbol[rounded]{settings-input-antenna} & \mSymbol[sharp]{settings-input-antenna} & \texttt{\textbackslash mSymbol\{settings-input-antenna\}} & \texttt{E8BF}\\
\mSymbol[outlined]{settings-input-component} & \mSymbol[rounded]{settings-input-component} & \mSymbol[sharp]{settings-input-component} & \texttt{\textbackslash mSymbol\{settings-input-component\}} & \texttt{E8C1}\\
\mSymbol[outlined]{settings-input-composite} & \mSymbol[rounded]{settings-input-composite} & \mSymbol[sharp]{settings-input-composite} & \texttt{\textbackslash mSymbol\{settings-input-composite\}} & \texttt{E8C1}\\
\mSymbol[outlined]{settings-input-hdmi} & \mSymbol[rounded]{settings-input-hdmi} & \mSymbol[sharp]{settings-input-hdmi} & \texttt{\textbackslash mSymbol\{settings-input-hdmi\}} & \texttt{E8C2}\\
\mSymbol[outlined]{settings-input-svideo} & \mSymbol[rounded]{settings-input-svideo} & \mSymbol[sharp]{settings-input-svideo} & \texttt{\textbackslash mSymbol\{settings-input-svideo\}} & \texttt{E8C3}\\
\mSymbol[outlined]{settings-motion-mode} & \mSymbol[rounded]{settings-motion-mode} & \mSymbol[sharp]{settings-motion-mode} & \texttt{\textbackslash mSymbol\{settings-motion-mode\}} & \texttt{F833}\\
\mSymbol[outlined]{settings-night-sight} & \mSymbol[rounded]{settings-night-sight} & \mSymbol[sharp]{settings-night-sight} & \texttt{\textbackslash mSymbol\{settings-night-sight\}} & \texttt{F832}\\
\mSymbol[outlined]{settings-overscan} & \mSymbol[rounded]{settings-overscan} & \mSymbol[sharp]{settings-overscan} & \texttt{\textbackslash mSymbol\{settings-overscan\}} & \texttt{E8C4}\\
\mSymbol[outlined]{settings-panorama} & \mSymbol[rounded]{settings-panorama} & \mSymbol[sharp]{settings-panorama} & \texttt{\textbackslash mSymbol\{settings-panorama\}} & \texttt{F831}\\
\mSymbol[outlined]{settings-phone} & \mSymbol[rounded]{settings-phone} & \mSymbol[sharp]{settings-phone} & \texttt{\textbackslash mSymbol\{settings-phone\}} & \texttt{E8C5}\\
\mSymbol[outlined]{settings-photo-camera} & \mSymbol[rounded]{settings-photo-camera} & \mSymbol[sharp]{settings-photo-camera} & \texttt{\textbackslash mSymbol\{settings-photo-camera\}} & \texttt{F834}\\
\mSymbol[outlined]{settings-power} & \mSymbol[rounded]{settings-power} & \mSymbol[sharp]{settings-power} & \texttt{\textbackslash mSymbol\{settings-power\}} & \texttt{E8C6}\\
\mSymbol[outlined]{settings-remote} & \mSymbol[rounded]{settings-remote} & \mSymbol[sharp]{settings-remote} & \texttt{\textbackslash mSymbol\{settings-remote\}} & \texttt{E8C7}\\
\mSymbol[outlined]{settings-slow-motion} & \mSymbol[rounded]{settings-slow-motion} & \mSymbol[sharp]{settings-slow-motion} & \texttt{\textbackslash mSymbol\{settings-slow-motion\}} & \texttt{F623}\\
\mSymbol[outlined]{settings-suggest} & \mSymbol[rounded]{settings-suggest} & \mSymbol[sharp]{settings-suggest} & \texttt{\textbackslash mSymbol\{settings-suggest\}} & \texttt{F05E}\\
\mSymbol[outlined]{settings-system-daydream} & \mSymbol[rounded]{settings-system-daydream} & \mSymbol[sharp]{settings-system-daydream} & \texttt{\textbackslash mSymbol\{settings-system-daydream\}} & \texttt{E1C3}\\
\mSymbol[outlined]{settings-timelapse} & \mSymbol[rounded]{settings-timelapse} & \mSymbol[sharp]{settings-timelapse} & \texttt{\textbackslash mSymbol\{settings-timelapse\}} & \texttt{F622}\\
\mSymbol[outlined]{settings-video-camera} & \mSymbol[rounded]{settings-video-camera} & \mSymbol[sharp]{settings-video-camera} & \texttt{\textbackslash mSymbol\{settings-video-camera\}} & \texttt{F621}\\
\mSymbol[outlined]{settings-voice} & \mSymbol[rounded]{settings-voice} & \mSymbol[sharp]{settings-voice} & \texttt{\textbackslash mSymbol\{settings-voice\}} & \texttt{E8C8}\\
\mSymbol[outlined]{settop-component} & \mSymbol[rounded]{settop-component} & \mSymbol[sharp]{settop-component} & \texttt{\textbackslash mSymbol\{settop-component\}} & \texttt{E2AC}\\
\mSymbol[outlined]{severe-cold} & \mSymbol[rounded]{severe-cold} & \mSymbol[sharp]{severe-cold} & \texttt{\textbackslash mSymbol\{severe-cold\}} & \texttt{EBD3}\\
\mSymbol[outlined]{shadow} & \mSymbol[rounded]{shadow} & \mSymbol[sharp]{shadow} & \texttt{\textbackslash mSymbol\{shadow\}} & \texttt{E9DF}\\
\mSymbol[outlined]{shadow-add} & \mSymbol[rounded]{shadow-add} & \mSymbol[sharp]{shadow-add} & \texttt{\textbackslash mSymbol\{shadow-add\}} & \texttt{F584}\\
\mSymbol[outlined]{shadow-minus} & \mSymbol[rounded]{shadow-minus} & \mSymbol[sharp]{shadow-minus} & \texttt{\textbackslash mSymbol\{shadow-minus\}} & \texttt{F583}\\
\mSymbol[outlined]{shape-line} & \mSymbol[rounded]{shape-line} & \mSymbol[sharp]{shape-line} & \texttt{\textbackslash mSymbol\{shape-line\}} & \texttt{F8D3}\\
\mSymbol[outlined]{shape-recognition} & \mSymbol[rounded]{shape-recognition} & \mSymbol[sharp]{shape-recognition} & \texttt{\textbackslash mSymbol\{shape-recognition\}} & \texttt{EB01}\\
\mSymbol[outlined]{shapes} & \mSymbol[rounded]{shapes} & \mSymbol[sharp]{shapes} & \texttt{\textbackslash mSymbol\{shapes\}} & \texttt{E602}\\
\mSymbol[outlined]{share} & \mSymbol[rounded]{share} & \mSymbol[sharp]{share} & \texttt{\textbackslash mSymbol\{share\}} & \texttt{E80D}\\
\mSymbol[outlined]{share-location} & \mSymbol[rounded]{share-location} & \mSymbol[sharp]{share-location} & \texttt{\textbackslash mSymbol\{share-location\}} & \texttt{F05F}\\
\mSymbol[outlined]{share-off} & \mSymbol[rounded]{share-off} & \mSymbol[sharp]{share-off} & \texttt{\textbackslash mSymbol\{share-off\}} & \texttt{F6CB}\\
\mSymbol[outlined]{share-reviews} & \mSymbol[rounded]{share-reviews} & \mSymbol[sharp]{share-reviews} & \texttt{\textbackslash mSymbol\{share-reviews\}} & \texttt{F8A4}\\
\mSymbol[outlined]{share-windows} & \mSymbol[rounded]{share-windows} & \mSymbol[sharp]{share-windows} & \texttt{\textbackslash mSymbol\{share-windows\}} & \texttt{F613}\\
\mSymbol[outlined]{sheets-rtl} & \mSymbol[rounded]{sheets-rtl} & \mSymbol[sharp]{sheets-rtl} & \texttt{\textbackslash mSymbol\{sheets-rtl\}} & \texttt{F823}\\
\mSymbol[outlined]{shelf-auto-hide} & \mSymbol[rounded]{shelf-auto-hide} & \mSymbol[sharp]{shelf-auto-hide} & \texttt{\textbackslash mSymbol\{shelf-auto-hide\}} & \texttt{F703}\\
\mSymbol[outlined]{shelf-position} & \mSymbol[rounded]{shelf-position} & \mSymbol[sharp]{shelf-position} & \texttt{\textbackslash mSymbol\{shelf-position\}} & \texttt{F702}\\
\mSymbol[outlined]{shelves} & \mSymbol[rounded]{shelves} & \mSymbol[sharp]{shelves} & \texttt{\textbackslash mSymbol\{shelves\}} & \texttt{F86E}\\
\mSymbol[outlined]{shield} & \mSymbol[rounded]{shield} & \mSymbol[sharp]{shield} & \texttt{\textbackslash mSymbol\{shield\}} & \texttt{E9E0}\\
\mSymbol[outlined]{shield-lock} & \mSymbol[rounded]{shield-lock} & \mSymbol[sharp]{shield-lock} & \texttt{\textbackslash mSymbol\{shield-lock\}} & \texttt{F686}\\
\mSymbol[outlined]{shield-locked} & \mSymbol[rounded]{shield-locked} & \mSymbol[sharp]{shield-locked} & \texttt{\textbackslash mSymbol\{shield-locked\}} & \texttt{F592}\\
\mSymbol[outlined]{shield-moon} & \mSymbol[rounded]{shield-moon} & \mSymbol[sharp]{shield-moon} & \texttt{\textbackslash mSymbol\{shield-moon\}} & \texttt{EAA9}\\
\mSymbol[outlined]{shield-person} & \mSymbol[rounded]{shield-person} & \mSymbol[sharp]{shield-person} & \texttt{\textbackslash mSymbol\{shield-person\}} & \texttt{F650}\\
\mSymbol[outlined]{shield-question} & \mSymbol[rounded]{shield-question} & \mSymbol[sharp]{shield-question} & \texttt{\textbackslash mSymbol\{shield-question\}} & \texttt{F529}\\
\mSymbol[outlined]{shield-with-heart} & \mSymbol[rounded]{shield-with-heart} & \mSymbol[sharp]{shield-with-heart} & \texttt{\textbackslash mSymbol\{shield-with-heart\}} & \texttt{E78F}\\
\mSymbol[outlined]{shield-with-house} & \mSymbol[rounded]{shield-with-house} & \mSymbol[sharp]{shield-with-house} & \texttt{\textbackslash mSymbol\{shield-with-house\}} & \texttt{E78D}\\
\mSymbol[outlined]{shift} & \mSymbol[rounded]{shift} & \mSymbol[sharp]{shift} & \texttt{\textbackslash mSymbol\{shift\}} & \texttt{E5F2}\\
\mSymbol[outlined]{shift-lock} & \mSymbol[rounded]{shift-lock} & \mSymbol[sharp]{shift-lock} & \texttt{\textbackslash mSymbol\{shift-lock\}} & \texttt{F7AE}\\
\mSymbol[outlined]{shift-lock-off} & \mSymbol[rounded]{shift-lock-off} & \mSymbol[sharp]{shift-lock-off} & \texttt{\textbackslash mSymbol\{shift-lock-off\}} & \texttt{F483}\\
\mSymbol[outlined]{shop} & \mSymbol[rounded]{shop} & \mSymbol[sharp]{shop} & \texttt{\textbackslash mSymbol\{shop\}} & \texttt{E8C9}\\
\mSymbol[outlined]{shop-2} & \mSymbol[rounded]{shop-2} & \mSymbol[sharp]{shop-2} & \texttt{\textbackslash mSymbol\{shop-2\}} & \texttt{E8CA}\\
\mSymbol[outlined]{shop-two} & \mSymbol[rounded]{shop-two} & \mSymbol[sharp]{shop-two} & \texttt{\textbackslash mSymbol\{shop-two\}} & \texttt{E8CA}\\
\mSymbol[outlined]{shopping-bag} & \mSymbol[rounded]{shopping-bag} & \mSymbol[sharp]{shopping-bag} & \texttt{\textbackslash mSymbol\{shopping-bag\}} & \texttt{F1CC}\\
\mSymbol[outlined]{shopping-basket} & \mSymbol[rounded]{shopping-basket} & \mSymbol[sharp]{shopping-basket} & \texttt{\textbackslash mSymbol\{shopping-basket\}} & \texttt{E8CB}\\
\mSymbol[outlined]{shopping-cart} & \mSymbol[rounded]{shopping-cart} & \mSymbol[sharp]{shopping-cart} & \texttt{\textbackslash mSymbol\{shopping-cart\}} & \texttt{E8CC}\\
\mSymbol[outlined]{shopping-cart-checkout} & \mSymbol[rounded]{shopping-cart-checkout} & \mSymbol[sharp]{shopping-cart-checkout} & \texttt{\textbackslash mSymbol\{shopping-cart-checkout\}} & \texttt{EB88}\\
\mSymbol[outlined]{shopping-cart-off} & \mSymbol[rounded]{shopping-cart-off} & \mSymbol[sharp]{shopping-cart-off} & \texttt{\textbackslash mSymbol\{shopping-cart-off\}} & \texttt{F4F7}\\
\mSymbol[outlined]{shoppingmode} & \mSymbol[rounded]{shoppingmode} & \mSymbol[sharp]{shoppingmode} & \texttt{\textbackslash mSymbol\{shoppingmode\}} & \texttt{EFB7}\\
\mSymbol[outlined]{short-stay} & \mSymbol[rounded]{short-stay} & \mSymbol[sharp]{short-stay} & \texttt{\textbackslash mSymbol\{short-stay\}} & \texttt{E4D0}\\
\mSymbol[outlined]{short-text} & \mSymbol[rounded]{short-text} & \mSymbol[sharp]{short-text} & \texttt{\textbackslash mSymbol\{short-text\}} & \texttt{E261}\\
\mSymbol[outlined]{shortcut} & \mSymbol[rounded]{shortcut} & \mSymbol[sharp]{shortcut} & \texttt{\textbackslash mSymbol\{shortcut\}} & \texttt{F57A}\\
\mSymbol[outlined]{show-chart} & \mSymbol[rounded]{show-chart} & \mSymbol[sharp]{show-chart} & \texttt{\textbackslash mSymbol\{show-chart\}} & \texttt{E6E1}\\
\mSymbol[outlined]{shower} & \mSymbol[rounded]{shower} & \mSymbol[sharp]{shower} & \texttt{\textbackslash mSymbol\{shower\}} & \texttt{F061}\\
\mSymbol[outlined]{shuffle} & \mSymbol[rounded]{shuffle} & \mSymbol[sharp]{shuffle} & \texttt{\textbackslash mSymbol\{shuffle\}} & \texttt{E043}\\
\mSymbol[outlined]{shuffle-on} & \mSymbol[rounded]{shuffle-on} & \mSymbol[sharp]{shuffle-on} & \texttt{\textbackslash mSymbol\{shuffle-on\}} & \texttt{E9E1}\\
\mSymbol[outlined]{shutter-speed} & \mSymbol[rounded]{shutter-speed} & \mSymbol[sharp]{shutter-speed} & \texttt{\textbackslash mSymbol\{shutter-speed\}} & \texttt{E43D}\\
\mSymbol[outlined]{shutter-speed-add} & \mSymbol[rounded]{shutter-speed-add} & \mSymbol[sharp]{shutter-speed-add} & \texttt{\textbackslash mSymbol\{shutter-speed-add\}} & \texttt{F57E}\\
\mSymbol[outlined]{shutter-speed-minus} & \mSymbol[rounded]{shutter-speed-minus} & \mSymbol[sharp]{shutter-speed-minus} & \texttt{\textbackslash mSymbol\{shutter-speed-minus\}} & \texttt{F57D}\\
\mSymbol[outlined]{sick} & \mSymbol[rounded]{sick} & \mSymbol[sharp]{sick} & \texttt{\textbackslash mSymbol\{sick\}} & \texttt{F220}\\
\mSymbol[outlined]{side-navigation} & \mSymbol[rounded]{side-navigation} & \mSymbol[sharp]{side-navigation} & \texttt{\textbackslash mSymbol\{side-navigation\}} & \texttt{E9E2}\\
\mSymbol[outlined]{sign-language} & \mSymbol[rounded]{sign-language} & \mSymbol[sharp]{sign-language} & \texttt{\textbackslash mSymbol\{sign-language\}} & \texttt{EBE5}\\
\mSymbol[outlined]{signal-cellular-0-bar} & \mSymbol[rounded]{signal-cellular-0-bar} & \mSymbol[sharp]{signal-cellular-0-bar} & \texttt{\textbackslash mSymbol\{signal-cellular-0-bar\}} & \texttt{F0A8}\\
\mSymbol[outlined]{signal-cellular-1-bar} & \mSymbol[rounded]{signal-cellular-1-bar} & \mSymbol[sharp]{signal-cellular-1-bar} & \texttt{\textbackslash mSymbol\{signal-cellular-1-bar\}} & \texttt{F0A9}\\
\mSymbol[outlined]{signal-cellular-2-bar} & \mSymbol[rounded]{signal-cellular-2-bar} & \mSymbol[sharp]{signal-cellular-2-bar} & \texttt{\textbackslash mSymbol\{signal-cellular-2-bar\}} & \texttt{F0AA}\\
\mSymbol[outlined]{signal-cellular-3-bar} & \mSymbol[rounded]{signal-cellular-3-bar} & \mSymbol[sharp]{signal-cellular-3-bar} & \texttt{\textbackslash mSymbol\{signal-cellular-3-bar\}} & \texttt{F0AB}\\
\mSymbol[outlined]{signal-cellular-4-bar} & \mSymbol[rounded]{signal-cellular-4-bar} & \mSymbol[sharp]{signal-cellular-4-bar} & \texttt{\textbackslash mSymbol\{signal-cellular-4-bar\}} & \texttt{E1C8}\\
\mSymbol[outlined]{signal-cellular-add} & \mSymbol[rounded]{signal-cellular-add} & \mSymbol[sharp]{signal-cellular-add} & \texttt{\textbackslash mSymbol\{signal-cellular-add\}} & \texttt{F7A9}\\
\mSymbol[outlined]{signal-cellular-alt} & \mSymbol[rounded]{signal-cellular-alt} & \mSymbol[sharp]{signal-cellular-alt} & \texttt{\textbackslash mSymbol\{signal-cellular-alt\}} & \texttt{E202}\\
\mSymbol[outlined]{signal-cellular-alt-1-bar} & \mSymbol[rounded]{signal-cellular-alt-1-bar} & \mSymbol[sharp]{signal-cellular-alt-1-bar} & \texttt{\textbackslash mSymbol\{signal-cellular-alt-1-bar\}} & \texttt{EBDF}\\
\mSymbol[outlined]{signal-cellular-alt-2-bar} & \mSymbol[rounded]{signal-cellular-alt-2-bar} & \mSymbol[sharp]{signal-cellular-alt-2-bar} & \texttt{\textbackslash mSymbol\{signal-cellular-alt-2-bar\}} & \texttt{EBE3}\\
\mSymbol[outlined]{signal-cellular-connected-no-internet-0-bar} & \mSymbol[rounded]{signal-cellular-connected-no-internet-0-bar} & \mSymbol[sharp]{signal-cellular-connected-no-internet-0-bar} & \texttt{\textbackslash mSymbol\{signal-cellular-connected-no-internet-0-bar\}} & \texttt{F0AC}\\
\mSymbol[outlined]{signal-cellular-connected-no-internet-4-bar} & \mSymbol[rounded]{signal-cellular-connected-no-internet-4-bar} & \mSymbol[sharp]{signal-cellular-connected-no-internet-4-bar} & \texttt{\textbackslash mSymbol\{signal-cellular-connected-no-internet-4-bar\}} & \texttt{E1CD}\\
\mSymbol[outlined]{signal-cellular-no-sim} & \mSymbol[rounded]{signal-cellular-no-sim} & \mSymbol[sharp]{signal-cellular-no-sim} & \texttt{\textbackslash mSymbol\{signal-cellular-no-sim\}} & \texttt{E1CE}\\
\mSymbol[outlined]{signal-cellular-nodata} & \mSymbol[rounded]{signal-cellular-nodata} & \mSymbol[sharp]{signal-cellular-nodata} & \texttt{\textbackslash mSymbol\{signal-cellular-nodata\}} & \texttt{F062}\\
\mSymbol[outlined]{signal-cellular-null} & \mSymbol[rounded]{signal-cellular-null} & \mSymbol[sharp]{signal-cellular-null} & \texttt{\textbackslash mSymbol\{signal-cellular-null\}} & \texttt{E1CF}\\
\mSymbol[outlined]{signal-cellular-off} & \mSymbol[rounded]{signal-cellular-off} & \mSymbol[sharp]{signal-cellular-off} & \texttt{\textbackslash mSymbol\{signal-cellular-off\}} & \texttt{E1D0}\\
\mSymbol[outlined]{signal-cellular-pause} & \mSymbol[rounded]{signal-cellular-pause} & \mSymbol[sharp]{signal-cellular-pause} & \texttt{\textbackslash mSymbol\{signal-cellular-pause\}} & \texttt{F5A7}\\
\mSymbol[outlined]{signal-disconnected} & \mSymbol[rounded]{signal-disconnected} & \mSymbol[sharp]{signal-disconnected} & \texttt{\textbackslash mSymbol\{signal-disconnected\}} & \texttt{F239}\\
\mSymbol[outlined]{signal-wifi-0-bar} & \mSymbol[rounded]{signal-wifi-0-bar} & \mSymbol[sharp]{signal-wifi-0-bar} & \texttt{\textbackslash mSymbol\{signal-wifi-0-bar\}} & \texttt{F0B0}\\
\mSymbol[outlined]{signal-wifi-4-bar} & \mSymbol[rounded]{signal-wifi-4-bar} & \mSymbol[sharp]{signal-wifi-4-bar} & \texttt{\textbackslash mSymbol\{signal-wifi-4-bar\}} & \texttt{F065}\\
\mSymbol[outlined]{signal-wifi-4-bar-lock} & \mSymbol[rounded]{signal-wifi-4-bar-lock} & \mSymbol[sharp]{signal-wifi-4-bar-lock} & \texttt{\textbackslash mSymbol\{signal-wifi-4-bar-lock\}} & \texttt{E1E1}\\
\mSymbol[outlined]{signal-wifi-bad} & \mSymbol[rounded]{signal-wifi-bad} & \mSymbol[sharp]{signal-wifi-bad} & \texttt{\textbackslash mSymbol\{signal-wifi-bad\}} & \texttt{F064}\\
\mSymbol[outlined]{signal-wifi-connected-no-internet-4} & \mSymbol[rounded]{signal-wifi-connected-no-internet-4} & \mSymbol[sharp]{signal-wifi-connected-no-internet-4} & \texttt{\textbackslash mSymbol\{signal-wifi-connected-no-internet-4\}} & \texttt{F064}\\
\mSymbol[outlined]{signal-wifi-off} & \mSymbol[rounded]{signal-wifi-off} & \mSymbol[sharp]{signal-wifi-off} & \texttt{\textbackslash mSymbol\{signal-wifi-off\}} & \texttt{E1DA}\\
\mSymbol[outlined]{signal-wifi-statusbar-4-bar} & \mSymbol[rounded]{signal-wifi-statusbar-4-bar} & \mSymbol[sharp]{signal-wifi-statusbar-4-bar} & \texttt{\textbackslash mSymbol\{signal-wifi-statusbar-4-bar\}} & \texttt{F065}\\
\mSymbol[outlined]{signal-wifi-statusbar-not-connected} & \mSymbol[rounded]{signal-wifi-statusbar-not-connected} & \mSymbol[sharp]{signal-wifi-statusbar-not-connected} & \texttt{\textbackslash mSymbol\{signal-wifi-statusbar-not-connected\}} & \texttt{F0EF}\\
\mSymbol[outlined]{signal-wifi-statusbar-null} & \mSymbol[rounded]{signal-wifi-statusbar-null} & \mSymbol[sharp]{signal-wifi-statusbar-null} & \texttt{\textbackslash mSymbol\{signal-wifi-statusbar-null\}} & \texttt{F067}\\
\mSymbol[outlined]{signature} & \mSymbol[rounded]{signature} & \mSymbol[sharp]{signature} & \texttt{\textbackslash mSymbol\{signature\}} & \texttt{F74C}\\
\mSymbol[outlined]{signpost} & \mSymbol[rounded]{signpost} & \mSymbol[sharp]{signpost} & \texttt{\textbackslash mSymbol\{signpost\}} & \texttt{EB91}\\
\mSymbol[outlined]{sim-card} & \mSymbol[rounded]{sim-card} & \mSymbol[sharp]{sim-card} & \texttt{\textbackslash mSymbol\{sim-card\}} & \texttt{E32B}\\
\mSymbol[outlined]{sim-card-alert} & \mSymbol[rounded]{sim-card-alert} & \mSymbol[sharp]{sim-card-alert} & \texttt{\textbackslash mSymbol\{sim-card-alert\}} & \texttt{F057}\\
\mSymbol[outlined]{sim-card-download} & \mSymbol[rounded]{sim-card-download} & \mSymbol[sharp]{sim-card-download} & \texttt{\textbackslash mSymbol\{sim-card-download\}} & \texttt{F068}\\
\mSymbol[outlined]{single-bed} & \mSymbol[rounded]{single-bed} & \mSymbol[sharp]{single-bed} & \texttt{\textbackslash mSymbol\{single-bed\}} & \texttt{EA48}\\
\mSymbol[outlined]{sip} & \mSymbol[rounded]{sip} & \mSymbol[sharp]{sip} & \texttt{\textbackslash mSymbol\{sip\}} & \texttt{F069}\\
\mSymbol[outlined]{skateboarding} & \mSymbol[rounded]{skateboarding} & \mSymbol[sharp]{skateboarding} & \texttt{\textbackslash mSymbol\{skateboarding\}} & \texttt{E511}\\
\mSymbol[outlined]{skeleton} & \mSymbol[rounded]{skeleton} & \mSymbol[sharp]{skeleton} & \texttt{\textbackslash mSymbol\{skeleton\}} & \texttt{F899}\\
\mSymbol[outlined]{skillet} & \mSymbol[rounded]{skillet} & \mSymbol[sharp]{skillet} & \texttt{\textbackslash mSymbol\{skillet\}} & \texttt{F543}\\
\mSymbol[outlined]{skillet-cooktop} & \mSymbol[rounded]{skillet-cooktop} & \mSymbol[sharp]{skillet-cooktop} & \texttt{\textbackslash mSymbol\{skillet-cooktop\}} & \texttt{F544}\\
\mSymbol[outlined]{skip-next} & \mSymbol[rounded]{skip-next} & \mSymbol[sharp]{skip-next} & \texttt{\textbackslash mSymbol\{skip-next\}} & \texttt{E044}\\
\mSymbol[outlined]{skip-previous} & \mSymbol[rounded]{skip-previous} & \mSymbol[sharp]{skip-previous} & \texttt{\textbackslash mSymbol\{skip-previous\}} & \texttt{E045}\\
\mSymbol[outlined]{skull} & \mSymbol[rounded]{skull} & \mSymbol[sharp]{skull} & \texttt{\textbackslash mSymbol\{skull\}} & \texttt{F89A}\\
\mSymbol[outlined]{slab-serif} & \mSymbol[rounded]{slab-serif} & \mSymbol[sharp]{slab-serif} & \texttt{\textbackslash mSymbol\{slab-serif\}} & \texttt{F4AB}\\
\mSymbol[outlined]{sledding} & \mSymbol[rounded]{sledding} & \mSymbol[sharp]{sledding} & \texttt{\textbackslash mSymbol\{sledding\}} & \texttt{E512}\\
\mSymbol[outlined]{sleep} & \mSymbol[rounded]{sleep} & \mSymbol[sharp]{sleep} & \texttt{\textbackslash mSymbol\{sleep\}} & \texttt{E213}\\
\mSymbol[outlined]{sleep-score} & \mSymbol[rounded]{sleep-score} & \mSymbol[sharp]{sleep-score} & \texttt{\textbackslash mSymbol\{sleep-score\}} & \texttt{F6B7}\\
\mSymbol[outlined]{slide-library} & \mSymbol[rounded]{slide-library} & \mSymbol[sharp]{slide-library} & \texttt{\textbackslash mSymbol\{slide-library\}} & \texttt{F822}\\
\mSymbol[outlined]{sliders} & \mSymbol[rounded]{sliders} & \mSymbol[sharp]{sliders} & \texttt{\textbackslash mSymbol\{sliders\}} & \texttt{E9E3}\\
\mSymbol[outlined]{slideshow} & \mSymbol[rounded]{slideshow} & \mSymbol[sharp]{slideshow} & \texttt{\textbackslash mSymbol\{slideshow\}} & \texttt{E41B}\\
\mSymbol[outlined]{slow-motion-video} & \mSymbol[rounded]{slow-motion-video} & \mSymbol[sharp]{slow-motion-video} & \texttt{\textbackslash mSymbol\{slow-motion-video\}} & \texttt{E068}\\
\mSymbol[outlined]{smart-button} & \mSymbol[rounded]{smart-button} & \mSymbol[sharp]{smart-button} & \texttt{\textbackslash mSymbol\{smart-button\}} & \texttt{F1C1}\\
\mSymbol[outlined]{smart-card-reader} & \mSymbol[rounded]{smart-card-reader} & \mSymbol[sharp]{smart-card-reader} & \texttt{\textbackslash mSymbol\{smart-card-reader\}} & \texttt{F4A5}\\
\mSymbol[outlined]{smart-card-reader-off} & \mSymbol[rounded]{smart-card-reader-off} & \mSymbol[sharp]{smart-card-reader-off} & \texttt{\textbackslash mSymbol\{smart-card-reader-off\}} & \texttt{F4A6}\\
\mSymbol[outlined]{smart-display} & \mSymbol[rounded]{smart-display} & \mSymbol[sharp]{smart-display} & \texttt{\textbackslash mSymbol\{smart-display\}} & \texttt{F06A}\\
\mSymbol[outlined]{smart-outlet} & \mSymbol[rounded]{smart-outlet} & \mSymbol[sharp]{smart-outlet} & \texttt{\textbackslash mSymbol\{smart-outlet\}} & \texttt{E844}\\
\mSymbol[outlined]{smart-screen} & \mSymbol[rounded]{smart-screen} & \mSymbol[sharp]{smart-screen} & \texttt{\textbackslash mSymbol\{smart-screen\}} & \texttt{F06B}\\
\mSymbol[outlined]{smart-toy} & \mSymbol[rounded]{smart-toy} & \mSymbol[sharp]{smart-toy} & \texttt{\textbackslash mSymbol\{smart-toy\}} & \texttt{F06C}\\
\mSymbol[outlined]{smartphone} & \mSymbol[rounded]{smartphone} & \mSymbol[sharp]{smartphone} & \texttt{\textbackslash mSymbol\{smartphone\}} & \texttt{E32C}\\
\mSymbol[outlined]{smartphone-camera} & \mSymbol[rounded]{smartphone-camera} & \mSymbol[sharp]{smartphone-camera} & \texttt{\textbackslash mSymbol\{smartphone-camera\}} & \texttt{F44E}\\
\mSymbol[outlined]{smb-share} & \mSymbol[rounded]{smb-share} & \mSymbol[sharp]{smb-share} & \texttt{\textbackslash mSymbol\{smb-share\}} & \texttt{F74B}\\
\mSymbol[outlined]{smoke-free} & \mSymbol[rounded]{smoke-free} & \mSymbol[sharp]{smoke-free} & \texttt{\textbackslash mSymbol\{smoke-free\}} & \texttt{EB4A}\\
\mSymbol[outlined]{smoking-rooms} & \mSymbol[rounded]{smoking-rooms} & \mSymbol[sharp]{smoking-rooms} & \texttt{\textbackslash mSymbol\{smoking-rooms\}} & \texttt{EB4B}\\
\mSymbol[outlined]{sms} & \mSymbol[rounded]{sms} & \mSymbol[sharp]{sms} & \texttt{\textbackslash mSymbol\{sms\}} & \texttt{E625}\\
\mSymbol[outlined]{sms-failed} & \mSymbol[rounded]{sms-failed} & \mSymbol[sharp]{sms-failed} & \texttt{\textbackslash mSymbol\{sms-failed\}} & \texttt{E87F}\\
\mSymbol[outlined]{snippet-folder} & \mSymbol[rounded]{snippet-folder} & \mSymbol[sharp]{snippet-folder} & \texttt{\textbackslash mSymbol\{snippet-folder\}} & \texttt{F1C7}\\
\mSymbol[outlined]{snooze} & \mSymbol[rounded]{snooze} & \mSymbol[sharp]{snooze} & \texttt{\textbackslash mSymbol\{snooze\}} & \texttt{E046}\\
\mSymbol[outlined]{snowboarding} & \mSymbol[rounded]{snowboarding} & \mSymbol[sharp]{snowboarding} & \texttt{\textbackslash mSymbol\{snowboarding\}} & \texttt{E513}\\
\mSymbol[outlined]{snowing} & \mSymbol[rounded]{snowing} & \mSymbol[sharp]{snowing} & \texttt{\textbackslash mSymbol\{snowing\}} & \texttt{E80F}\\
\mSymbol[outlined]{snowing-heavy} & \mSymbol[rounded]{snowing-heavy} & \mSymbol[sharp]{snowing-heavy} & \texttt{\textbackslash mSymbol\{snowing-heavy\}} & \texttt{F61C}\\
\mSymbol[outlined]{snowmobile} & \mSymbol[rounded]{snowmobile} & \mSymbol[sharp]{snowmobile} & \texttt{\textbackslash mSymbol\{snowmobile\}} & \texttt{E503}\\
\mSymbol[outlined]{snowshoeing} & \mSymbol[rounded]{snowshoeing} & \mSymbol[sharp]{snowshoeing} & \texttt{\textbackslash mSymbol\{snowshoeing\}} & \texttt{E514}\\
\mSymbol[outlined]{soap} & \mSymbol[rounded]{soap} & \mSymbol[sharp]{soap} & \texttt{\textbackslash mSymbol\{soap\}} & \texttt{F1B2}\\
\mSymbol[outlined]{social-distance} & \mSymbol[rounded]{social-distance} & \mSymbol[sharp]{social-distance} & \texttt{\textbackslash mSymbol\{social-distance\}} & \texttt{E1CB}\\
\mSymbol[outlined]{social-leaderboard} & \mSymbol[rounded]{social-leaderboard} & \mSymbol[sharp]{social-leaderboard} & \texttt{\textbackslash mSymbol\{social-leaderboard\}} & \texttt{F6A0}\\
\mSymbol[outlined]{solar-power} & \mSymbol[rounded]{solar-power} & \mSymbol[sharp]{solar-power} & \texttt{\textbackslash mSymbol\{solar-power\}} & \texttt{EC0F}\\
\mSymbol[outlined]{sort} & \mSymbol[rounded]{sort} & \mSymbol[sharp]{sort} & \texttt{\textbackslash mSymbol\{sort\}} & \texttt{E164}\\
\mSymbol[outlined]{sort-by-alpha} & \mSymbol[rounded]{sort-by-alpha} & \mSymbol[sharp]{sort-by-alpha} & \texttt{\textbackslash mSymbol\{sort-by-alpha\}} & \texttt{E053}\\
\mSymbol[outlined]{sos} & \mSymbol[rounded]{sos} & \mSymbol[sharp]{sos} & \texttt{\textbackslash mSymbol\{sos\}} & \texttt{EBF7}\\
\mSymbol[outlined]{sound-detection-dog-barking} & \mSymbol[rounded]{sound-detection-dog-barking} & \mSymbol[sharp]{sound-detection-dog-barking} & \texttt{\textbackslash mSymbol\{sound-detection-dog-barking\}} & \texttt{F149}\\
\mSymbol[outlined]{sound-detection-glass-break} & \mSymbol[rounded]{sound-detection-glass-break} & \mSymbol[sharp]{sound-detection-glass-break} & \texttt{\textbackslash mSymbol\{sound-detection-glass-break\}} & \texttt{F14A}\\
\mSymbol[outlined]{sound-detection-loud-sound} & \mSymbol[rounded]{sound-detection-loud-sound} & \mSymbol[sharp]{sound-detection-loud-sound} & \texttt{\textbackslash mSymbol\{sound-detection-loud-sound\}} & \texttt{F14B}\\
\mSymbol[outlined]{sound-sampler} & \mSymbol[rounded]{sound-sampler} & \mSymbol[sharp]{sound-sampler} & \texttt{\textbackslash mSymbol\{sound-sampler\}} & \texttt{F6B4}\\
\mSymbol[outlined]{soup-kitchen} & \mSymbol[rounded]{soup-kitchen} & \mSymbol[sharp]{soup-kitchen} & \texttt{\textbackslash mSymbol\{soup-kitchen\}} & \texttt{E7D3}\\
\mSymbol[outlined]{source} & \mSymbol[rounded]{source} & \mSymbol[sharp]{source} & \texttt{\textbackslash mSymbol\{source\}} & \texttt{F1C8}\\
\mSymbol[outlined]{source-environment} & \mSymbol[rounded]{source-environment} & \mSymbol[sharp]{source-environment} & \texttt{\textbackslash mSymbol\{source-environment\}} & \texttt{E527}\\
\mSymbol[outlined]{source-notes} & \mSymbol[rounded]{source-notes} & \mSymbol[sharp]{source-notes} & \texttt{\textbackslash mSymbol\{source-notes\}} & \texttt{E12D}\\
\mSymbol[outlined]{south} & \mSymbol[rounded]{south} & \mSymbol[sharp]{south} & \texttt{\textbackslash mSymbol\{south\}} & \texttt{F1E3}\\
\mSymbol[outlined]{south-america} & \mSymbol[rounded]{south-america} & \mSymbol[sharp]{south-america} & \texttt{\textbackslash mSymbol\{south-america\}} & \texttt{E7E4}\\
\mSymbol[outlined]{south-east} & \mSymbol[rounded]{south-east} & \mSymbol[sharp]{south-east} & \texttt{\textbackslash mSymbol\{south-east\}} & \texttt{F1E4}\\
\mSymbol[outlined]{south-west} & \mSymbol[rounded]{south-west} & \mSymbol[sharp]{south-west} & \texttt{\textbackslash mSymbol\{south-west\}} & \texttt{F1E5}\\
\mSymbol[outlined]{spa} & \mSymbol[rounded]{spa} & \mSymbol[sharp]{spa} & \texttt{\textbackslash mSymbol\{spa\}} & \texttt{EB4C}\\
\mSymbol[outlined]{space-bar} & \mSymbol[rounded]{space-bar} & \mSymbol[sharp]{space-bar} & \texttt{\textbackslash mSymbol\{space-bar\}} & \texttt{E256}\\
\mSymbol[outlined]{space-dashboard} & \mSymbol[rounded]{space-dashboard} & \mSymbol[sharp]{space-dashboard} & \texttt{\textbackslash mSymbol\{space-dashboard\}} & \texttt{E66B}\\
\mSymbol[outlined]{spatial-audio} & \mSymbol[rounded]{spatial-audio} & \mSymbol[sharp]{spatial-audio} & \texttt{\textbackslash mSymbol\{spatial-audio\}} & \texttt{EBEB}\\
\mSymbol[outlined]{spatial-audio-off} & \mSymbol[rounded]{spatial-audio-off} & \mSymbol[sharp]{spatial-audio-off} & \texttt{\textbackslash mSymbol\{spatial-audio-off\}} & \texttt{EBE8}\\
\mSymbol[outlined]{spatial-speaker} & \mSymbol[rounded]{spatial-speaker} & \mSymbol[sharp]{spatial-speaker} & \texttt{\textbackslash mSymbol\{spatial-speaker\}} & \texttt{F4CF}\\
\mSymbol[outlined]{spatial-tracking} & \mSymbol[rounded]{spatial-tracking} & \mSymbol[sharp]{spatial-tracking} & \texttt{\textbackslash mSymbol\{spatial-tracking\}} & \texttt{EBEA}\\
\mSymbol[outlined]{speaker} & \mSymbol[rounded]{speaker} & \mSymbol[sharp]{speaker} & \texttt{\textbackslash mSymbol\{speaker\}} & \texttt{E32D}\\
\mSymbol[outlined]{speaker-group} & \mSymbol[rounded]{speaker-group} & \mSymbol[sharp]{speaker-group} & \texttt{\textbackslash mSymbol\{speaker-group\}} & \texttt{E32E}\\
\mSymbol[outlined]{speaker-notes} & \mSymbol[rounded]{speaker-notes} & \mSymbol[sharp]{speaker-notes} & \texttt{\textbackslash mSymbol\{speaker-notes\}} & \texttt{E8CD}\\
\mSymbol[outlined]{speaker-notes-off} & \mSymbol[rounded]{speaker-notes-off} & \mSymbol[sharp]{speaker-notes-off} & \texttt{\textbackslash mSymbol\{speaker-notes-off\}} & \texttt{E92A}\\
\mSymbol[outlined]{speaker-phone} & \mSymbol[rounded]{speaker-phone} & \mSymbol[sharp]{speaker-phone} & \texttt{\textbackslash mSymbol\{speaker-phone\}} & \texttt{E0D2}\\
\mSymbol[outlined]{special-character} & \mSymbol[rounded]{special-character} & \mSymbol[sharp]{special-character} & \texttt{\textbackslash mSymbol\{special-character\}} & \texttt{F74A}\\
\mSymbol[outlined]{specific-gravity} & \mSymbol[rounded]{specific-gravity} & \mSymbol[sharp]{specific-gravity} & \texttt{\textbackslash mSymbol\{specific-gravity\}} & \texttt{F872}\\
\mSymbol[outlined]{speech-to-text} & \mSymbol[rounded]{speech-to-text} & \mSymbol[sharp]{speech-to-text} & \texttt{\textbackslash mSymbol\{speech-to-text\}} & \texttt{F8A7}\\
\mSymbol[outlined]{speed} & \mSymbol[rounded]{speed} & \mSymbol[sharp]{speed} & \texttt{\textbackslash mSymbol\{speed\}} & \texttt{E9E4}\\
\mSymbol[outlined]{speed-0-25} & \mSymbol[rounded]{speed-0-25} & \mSymbol[sharp]{speed-0-25} & \texttt{\textbackslash mSymbol\{speed-0-25\}} & \texttt{F4D4}\\
\mSymbol[outlined]{speed-0-2x} & \mSymbol[rounded]{speed-0-2x} & \mSymbol[sharp]{speed-0-2x} & \texttt{\textbackslash mSymbol\{speed-0-2x\}} & \texttt{F498}\\
\mSymbol[outlined]{speed-0-5} & \mSymbol[rounded]{speed-0-5} & \mSymbol[sharp]{speed-0-5} & \texttt{\textbackslash mSymbol\{speed-0-5\}} & \texttt{F4E2}\\
\mSymbol[outlined]{speed-0-5x} & \mSymbol[rounded]{speed-0-5x} & \mSymbol[sharp]{speed-0-5x} & \texttt{\textbackslash mSymbol\{speed-0-5x\}} & \texttt{F497}\\
\mSymbol[outlined]{speed-0-75} & \mSymbol[rounded]{speed-0-75} & \mSymbol[sharp]{speed-0-75} & \texttt{\textbackslash mSymbol\{speed-0-75\}} & \texttt{F4D3}\\
\mSymbol[outlined]{speed-0-7x} & \mSymbol[rounded]{speed-0-7x} & \mSymbol[sharp]{speed-0-7x} & \texttt{\textbackslash mSymbol\{speed-0-7x\}} & \texttt{F496}\\
\mSymbol[outlined]{speed-1-2} & \mSymbol[rounded]{speed-1-2} & \mSymbol[sharp]{speed-1-2} & \texttt{\textbackslash mSymbol\{speed-1-2\}} & \texttt{F4E1}\\
\mSymbol[outlined]{speed-1-25} & \mSymbol[rounded]{speed-1-25} & \mSymbol[sharp]{speed-1-25} & \texttt{\textbackslash mSymbol\{speed-1-25\}} & \texttt{F4D2}\\
\mSymbol[outlined]{speed-1-2x} & \mSymbol[rounded]{speed-1-2x} & \mSymbol[sharp]{speed-1-2x} & \texttt{\textbackslash mSymbol\{speed-1-2x\}} & \texttt{F495}\\
\mSymbol[outlined]{speed-1-5} & \mSymbol[rounded]{speed-1-5} & \mSymbol[sharp]{speed-1-5} & \texttt{\textbackslash mSymbol\{speed-1-5\}} & \texttt{F4E0}\\
\mSymbol[outlined]{speed-1-5x} & \mSymbol[rounded]{speed-1-5x} & \mSymbol[sharp]{speed-1-5x} & \texttt{\textbackslash mSymbol\{speed-1-5x\}} & \texttt{F494}\\
\mSymbol[outlined]{speed-1-75} & \mSymbol[rounded]{speed-1-75} & \mSymbol[sharp]{speed-1-75} & \texttt{\textbackslash mSymbol\{speed-1-75\}} & \texttt{F4D1}\\
\mSymbol[outlined]{speed-1-7x} & \mSymbol[rounded]{speed-1-7x} & \mSymbol[sharp]{speed-1-7x} & \texttt{\textbackslash mSymbol\{speed-1-7x\}} & \texttt{F493}\\
\mSymbol[outlined]{speed-2x} & \mSymbol[rounded]{speed-2x} & \mSymbol[sharp]{speed-2x} & \texttt{\textbackslash mSymbol\{speed-2x\}} & \texttt{F4EB}\\
\mSymbol[outlined]{speed-camera} & \mSymbol[rounded]{speed-camera} & \mSymbol[sharp]{speed-camera} & \texttt{\textbackslash mSymbol\{speed-camera\}} & \texttt{F470}\\
\mSymbol[outlined]{spellcheck} & \mSymbol[rounded]{spellcheck} & \mSymbol[sharp]{spellcheck} & \texttt{\textbackslash mSymbol\{spellcheck\}} & \texttt{E8CE}\\
\mSymbol[outlined]{splitscreen} & \mSymbol[rounded]{splitscreen} & \mSymbol[sharp]{splitscreen} & \texttt{\textbackslash mSymbol\{splitscreen\}} & \texttt{F06D}\\
\mSymbol[outlined]{splitscreen-add} & \mSymbol[rounded]{splitscreen-add} & \mSymbol[sharp]{splitscreen-add} & \texttt{\textbackslash mSymbol\{splitscreen-add\}} & \texttt{F4FD}\\
\mSymbol[outlined]{splitscreen-bottom} & \mSymbol[rounded]{splitscreen-bottom} & \mSymbol[sharp]{splitscreen-bottom} & \texttt{\textbackslash mSymbol\{splitscreen-bottom\}} & \texttt{F676}\\
\mSymbol[outlined]{splitscreen-landscape} & \mSymbol[rounded]{splitscreen-landscape} & \mSymbol[sharp]{splitscreen-landscape} & \texttt{\textbackslash mSymbol\{splitscreen-landscape\}} & \texttt{F459}\\
\mSymbol[outlined]{splitscreen-left} & \mSymbol[rounded]{splitscreen-left} & \mSymbol[sharp]{splitscreen-left} & \texttt{\textbackslash mSymbol\{splitscreen-left\}} & \texttt{F675}\\
\mSymbol[outlined]{splitscreen-portrait} & \mSymbol[rounded]{splitscreen-portrait} & \mSymbol[sharp]{splitscreen-portrait} & \texttt{\textbackslash mSymbol\{splitscreen-portrait\}} & \texttt{F458}\\
\mSymbol[outlined]{splitscreen-right} & \mSymbol[rounded]{splitscreen-right} & \mSymbol[sharp]{splitscreen-right} & \texttt{\textbackslash mSymbol\{splitscreen-right\}} & \texttt{F674}\\
\mSymbol[outlined]{splitscreen-top} & \mSymbol[rounded]{splitscreen-top} & \mSymbol[sharp]{splitscreen-top} & \texttt{\textbackslash mSymbol\{splitscreen-top\}} & \texttt{F673}\\
\mSymbol[outlined]{splitscreen-vertical-add} & \mSymbol[rounded]{splitscreen-vertical-add} & \mSymbol[sharp]{splitscreen-vertical-add} & \texttt{\textbackslash mSymbol\{splitscreen-vertical-add\}} & \texttt{F4FC}\\
\mSymbol[outlined]{spo2} & \mSymbol[rounded]{spo2} & \mSymbol[sharp]{spo2} & \texttt{\textbackslash mSymbol\{spo2\}} & \texttt{F6DB}\\
\mSymbol[outlined]{spoke} & \mSymbol[rounded]{spoke} & \mSymbol[sharp]{spoke} & \texttt{\textbackslash mSymbol\{spoke\}} & \texttt{E9A7}\\
\mSymbol[outlined]{sports} & \mSymbol[rounded]{sports} & \mSymbol[sharp]{sports} & \texttt{\textbackslash mSymbol\{sports\}} & \texttt{EA30}\\
\mSymbol[outlined]{sports-and-outdoors} & \mSymbol[rounded]{sports-and-outdoors} & \mSymbol[sharp]{sports-and-outdoors} & \texttt{\textbackslash mSymbol\{sports-and-outdoors\}} & \texttt{EFB8}\\
\mSymbol[outlined]{sports-bar} & \mSymbol[rounded]{sports-bar} & \mSymbol[sharp]{sports-bar} & \texttt{\textbackslash mSymbol\{sports-bar\}} & \texttt{F1F3}\\
\mSymbol[outlined]{sports-baseball} & \mSymbol[rounded]{sports-baseball} & \mSymbol[sharp]{sports-baseball} & \texttt{\textbackslash mSymbol\{sports-baseball\}} & \texttt{EA51}\\
\mSymbol[outlined]{sports-basketball} & \mSymbol[rounded]{sports-basketball} & \mSymbol[sharp]{sports-basketball} & \texttt{\textbackslash mSymbol\{sports-basketball\}} & \texttt{EA26}\\
\mSymbol[outlined]{sports-cricket} & \mSymbol[rounded]{sports-cricket} & \mSymbol[sharp]{sports-cricket} & \texttt{\textbackslash mSymbol\{sports-cricket\}} & \texttt{EA27}\\
\mSymbol[outlined]{sports-esports} & \mSymbol[rounded]{sports-esports} & \mSymbol[sharp]{sports-esports} & \texttt{\textbackslash mSymbol\{sports-esports\}} & \texttt{EA28}\\
\mSymbol[outlined]{sports-football} & \mSymbol[rounded]{sports-football} & \mSymbol[sharp]{sports-football} & \texttt{\textbackslash mSymbol\{sports-football\}} & \texttt{EA29}\\
\mSymbol[outlined]{sports-golf} & \mSymbol[rounded]{sports-golf} & \mSymbol[sharp]{sports-golf} & \texttt{\textbackslash mSymbol\{sports-golf\}} & \texttt{EA2A}\\
\mSymbol[outlined]{sports-gymnastics} & \mSymbol[rounded]{sports-gymnastics} & \mSymbol[sharp]{sports-gymnastics} & \texttt{\textbackslash mSymbol\{sports-gymnastics\}} & \texttt{EBC4}\\
\mSymbol[outlined]{sports-handball} & \mSymbol[rounded]{sports-handball} & \mSymbol[sharp]{sports-handball} & \texttt{\textbackslash mSymbol\{sports-handball\}} & \texttt{EA33}\\
\mSymbol[outlined]{sports-hockey} & \mSymbol[rounded]{sports-hockey} & \mSymbol[sharp]{sports-hockey} & \texttt{\textbackslash mSymbol\{sports-hockey\}} & \texttt{EA2B}\\
\mSymbol[outlined]{sports-kabaddi} & \mSymbol[rounded]{sports-kabaddi} & \mSymbol[sharp]{sports-kabaddi} & \texttt{\textbackslash mSymbol\{sports-kabaddi\}} & \texttt{EA34}\\
\mSymbol[outlined]{sports-martial-arts} & \mSymbol[rounded]{sports-martial-arts} & \mSymbol[sharp]{sports-martial-arts} & \texttt{\textbackslash mSymbol\{sports-martial-arts\}} & \texttt{EAE9}\\
\mSymbol[outlined]{sports-mma} & \mSymbol[rounded]{sports-mma} & \mSymbol[sharp]{sports-mma} & \texttt{\textbackslash mSymbol\{sports-mma\}} & \texttt{EA2C}\\
\mSymbol[outlined]{sports-motorsports} & \mSymbol[rounded]{sports-motorsports} & \mSymbol[sharp]{sports-motorsports} & \texttt{\textbackslash mSymbol\{sports-motorsports\}} & \texttt{EA2D}\\
\mSymbol[outlined]{sports-rugby} & \mSymbol[rounded]{sports-rugby} & \mSymbol[sharp]{sports-rugby} & \texttt{\textbackslash mSymbol\{sports-rugby\}} & \texttt{EA2E}\\
\mSymbol[outlined]{sports-score} & \mSymbol[rounded]{sports-score} & \mSymbol[sharp]{sports-score} & \texttt{\textbackslash mSymbol\{sports-score\}} & \texttt{F06E}\\
\mSymbol[outlined]{sports-soccer} & \mSymbol[rounded]{sports-soccer} & \mSymbol[sharp]{sports-soccer} & \texttt{\textbackslash mSymbol\{sports-soccer\}} & \texttt{EA2F}\\
\mSymbol[outlined]{sports-tennis} & \mSymbol[rounded]{sports-tennis} & \mSymbol[sharp]{sports-tennis} & \texttt{\textbackslash mSymbol\{sports-tennis\}} & \texttt{EA32}\\
\mSymbol[outlined]{sports-volleyball} & \mSymbol[rounded]{sports-volleyball} & \mSymbol[sharp]{sports-volleyball} & \texttt{\textbackslash mSymbol\{sports-volleyball\}} & \texttt{EA31}\\
\mSymbol[outlined]{sprinkler} & \mSymbol[rounded]{sprinkler} & \mSymbol[sharp]{sprinkler} & \texttt{\textbackslash mSymbol\{sprinkler\}} & \texttt{E29A}\\
\mSymbol[outlined]{sprint} & \mSymbol[rounded]{sprint} & \mSymbol[sharp]{sprint} & \texttt{\textbackslash mSymbol\{sprint\}} & \texttt{F81F}\\
\mSymbol[outlined]{square} & \mSymbol[rounded]{square} & \mSymbol[sharp]{square} & \texttt{\textbackslash mSymbol\{square\}} & \texttt{EB36}\\
\mSymbol[outlined]{square-foot} & \mSymbol[rounded]{square-foot} & \mSymbol[sharp]{square-foot} & \texttt{\textbackslash mSymbol\{square-foot\}} & \texttt{EA49}\\
\mSymbol[outlined]{ssid-chart} & \mSymbol[rounded]{ssid-chart} & \mSymbol[sharp]{ssid-chart} & \texttt{\textbackslash mSymbol\{ssid-chart\}} & \texttt{EB66}\\
\mSymbol[outlined]{stack} & \mSymbol[rounded]{stack} & \mSymbol[sharp]{stack} & \texttt{\textbackslash mSymbol\{stack\}} & \texttt{F609}\\
\mSymbol[outlined]{stack-hexagon} & \mSymbol[rounded]{stack-hexagon} & \mSymbol[sharp]{stack-hexagon} & \texttt{\textbackslash mSymbol\{stack-hexagon\}} & \texttt{F41C}\\
\mSymbol[outlined]{stack-off} & \mSymbol[rounded]{stack-off} & \mSymbol[sharp]{stack-off} & \texttt{\textbackslash mSymbol\{stack-off\}} & \texttt{F608}\\
\mSymbol[outlined]{stack-star} & \mSymbol[rounded]{stack-star} & \mSymbol[sharp]{stack-star} & \texttt{\textbackslash mSymbol\{stack-star\}} & \texttt{F607}\\
\mSymbol[outlined]{stacked-bar-chart} & \mSymbol[rounded]{stacked-bar-chart} & \mSymbol[sharp]{stacked-bar-chart} & \texttt{\textbackslash mSymbol\{stacked-bar-chart\}} & \texttt{E9E6}\\
\mSymbol[outlined]{stacked-email} & \mSymbol[rounded]{stacked-email} & \mSymbol[sharp]{stacked-email} & \texttt{\textbackslash mSymbol\{stacked-email\}} & \texttt{E6C7}\\
\mSymbol[outlined]{stacked-inbox} & \mSymbol[rounded]{stacked-inbox} & \mSymbol[sharp]{stacked-inbox} & \texttt{\textbackslash mSymbol\{stacked-inbox\}} & \texttt{E6C9}\\
\mSymbol[outlined]{stacked-line-chart} & \mSymbol[rounded]{stacked-line-chart} & \mSymbol[sharp]{stacked-line-chart} & \texttt{\textbackslash mSymbol\{stacked-line-chart\}} & \texttt{F22B}\\
\mSymbol[outlined]{stacks} & \mSymbol[rounded]{stacks} & \mSymbol[sharp]{stacks} & \texttt{\textbackslash mSymbol\{stacks\}} & \texttt{F500}\\
\mSymbol[outlined]{stadia-controller} & \mSymbol[rounded]{stadia-controller} & \mSymbol[sharp]{stadia-controller} & \texttt{\textbackslash mSymbol\{stadia-controller\}} & \texttt{F135}\\
\mSymbol[outlined]{stadium} & \mSymbol[rounded]{stadium} & \mSymbol[sharp]{stadium} & \texttt{\textbackslash mSymbol\{stadium\}} & \texttt{EB90}\\
\mSymbol[outlined]{stairs} & \mSymbol[rounded]{stairs} & \mSymbol[sharp]{stairs} & \texttt{\textbackslash mSymbol\{stairs\}} & \texttt{F1A9}\\
\mSymbol[outlined]{stairs-2} & \mSymbol[rounded]{stairs-2} & \mSymbol[sharp]{stairs-2} & \texttt{\textbackslash mSymbol\{stairs-2\}} & \texttt{F46C}\\
\mSymbol[outlined]{star} & \mSymbol[rounded]{star} & \mSymbol[sharp]{star} & \texttt{\textbackslash mSymbol\{star\}} & \texttt{F09A}\\
\mSymbol[outlined]{star-border} & \mSymbol[rounded]{star-border} & \mSymbol[sharp]{star-border} & \texttt{\textbackslash mSymbol\{star-border\}} & \texttt{F09A}\\
\mSymbol[outlined]{star-border-purple500} & \mSymbol[rounded]{star-border-purple500} & \mSymbol[sharp]{star-border-purple500} & \texttt{\textbackslash mSymbol\{star-border-purple500\}} & \texttt{F09A}\\
\mSymbol[outlined]{star-half} & \mSymbol[rounded]{star-half} & \mSymbol[sharp]{star-half} & \texttt{\textbackslash mSymbol\{star-half\}} & \texttt{E839}\\
\mSymbol[outlined]{star-outline} & \mSymbol[rounded]{star-outline} & \mSymbol[sharp]{star-outline} & \texttt{\textbackslash mSymbol\{star-outline\}} & \texttt{F09A}\\
\mSymbol[outlined]{star-purple500} & \mSymbol[rounded]{star-purple500} & \mSymbol[sharp]{star-purple500} & \texttt{\textbackslash mSymbol\{star-purple500\}} & \texttt{F09A}\\
\mSymbol[outlined]{star-rate} & \mSymbol[rounded]{star-rate} & \mSymbol[sharp]{star-rate} & \texttt{\textbackslash mSymbol\{star-rate\}} & \texttt{F0EC}\\
\mSymbol[outlined]{star-rate-half} & \mSymbol[rounded]{star-rate-half} & \mSymbol[sharp]{star-rate-half} & \texttt{\textbackslash mSymbol\{star-rate-half\}} & \texttt{EC45}\\
\mSymbol[outlined]{stars} & \mSymbol[rounded]{stars} & \mSymbol[sharp]{stars} & \texttt{\textbackslash mSymbol\{stars\}} & \texttt{E8D0}\\
\mSymbol[outlined]{start} & \mSymbol[rounded]{start} & \mSymbol[sharp]{start} & \texttt{\textbackslash mSymbol\{start\}} & \texttt{E089}\\
\mSymbol[outlined]{stat-0} & \mSymbol[rounded]{stat-0} & \mSymbol[sharp]{stat-0} & \texttt{\textbackslash mSymbol\{stat-0\}} & \texttt{E697}\\
\mSymbol[outlined]{stat-1} & \mSymbol[rounded]{stat-1} & \mSymbol[sharp]{stat-1} & \texttt{\textbackslash mSymbol\{stat-1\}} & \texttt{E698}\\
\mSymbol[outlined]{stat-2} & \mSymbol[rounded]{stat-2} & \mSymbol[sharp]{stat-2} & \texttt{\textbackslash mSymbol\{stat-2\}} & \texttt{E699}\\
\mSymbol[outlined]{stat-3} & \mSymbol[rounded]{stat-3} & \mSymbol[sharp]{stat-3} & \texttt{\textbackslash mSymbol\{stat-3\}} & \texttt{E69A}\\
\mSymbol[outlined]{stat-minus-1} & \mSymbol[rounded]{stat-minus-1} & \mSymbol[sharp]{stat-minus-1} & \texttt{\textbackslash mSymbol\{stat-minus-1\}} & \texttt{E69B}\\
\mSymbol[outlined]{stat-minus-2} & \mSymbol[rounded]{stat-minus-2} & \mSymbol[sharp]{stat-minus-2} & \texttt{\textbackslash mSymbol\{stat-minus-2\}} & \texttt{E69C}\\
\mSymbol[outlined]{stat-minus-3} & \mSymbol[rounded]{stat-minus-3} & \mSymbol[sharp]{stat-minus-3} & \texttt{\textbackslash mSymbol\{stat-minus-3\}} & \texttt{E69D}\\
\mSymbol[outlined]{stay-current-landscape} & \mSymbol[rounded]{stay-current-landscape} & \mSymbol[sharp]{stay-current-landscape} & \texttt{\textbackslash mSymbol\{stay-current-landscape\}} & \texttt{E0D3}\\
\mSymbol[outlined]{stay-current-portrait} & \mSymbol[rounded]{stay-current-portrait} & \mSymbol[sharp]{stay-current-portrait} & \texttt{\textbackslash mSymbol\{stay-current-portrait\}} & \texttt{E0D4}\\
\mSymbol[outlined]{stay-primary-landscape} & \mSymbol[rounded]{stay-primary-landscape} & \mSymbol[sharp]{stay-primary-landscape} & \texttt{\textbackslash mSymbol\{stay-primary-landscape\}} & \texttt{E0D5}\\
\mSymbol[outlined]{stay-primary-portrait} & \mSymbol[rounded]{stay-primary-portrait} & \mSymbol[sharp]{stay-primary-portrait} & \texttt{\textbackslash mSymbol\{stay-primary-portrait\}} & \texttt{E0D6}\\
\mSymbol[outlined]{step} & \mSymbol[rounded]{step} & \mSymbol[sharp]{step} & \texttt{\textbackslash mSymbol\{step\}} & \texttt{F6FE}\\
\mSymbol[outlined]{step-into} & \mSymbol[rounded]{step-into} & \mSymbol[sharp]{step-into} & \texttt{\textbackslash mSymbol\{step-into\}} & \texttt{F701}\\
\mSymbol[outlined]{step-out} & \mSymbol[rounded]{step-out} & \mSymbol[sharp]{step-out} & \texttt{\textbackslash mSymbol\{step-out\}} & \texttt{F700}\\
\mSymbol[outlined]{step-over} & \mSymbol[rounded]{step-over} & \mSymbol[sharp]{step-over} & \texttt{\textbackslash mSymbol\{step-over\}} & \texttt{F6FF}\\
\mSymbol[outlined]{steppers} & \mSymbol[rounded]{steppers} & \mSymbol[sharp]{steppers} & \texttt{\textbackslash mSymbol\{steppers\}} & \texttt{E9E7}\\
\mSymbol[outlined]{steps} & \mSymbol[rounded]{steps} & \mSymbol[sharp]{steps} & \texttt{\textbackslash mSymbol\{steps\}} & \texttt{F6DA}\\
\mSymbol[outlined]{stethoscope} & \mSymbol[rounded]{stethoscope} & \mSymbol[sharp]{stethoscope} & \texttt{\textbackslash mSymbol\{stethoscope\}} & \texttt{F805}\\
\mSymbol[outlined]{stethoscope-arrow} & \mSymbol[rounded]{stethoscope-arrow} & \mSymbol[sharp]{stethoscope-arrow} & \texttt{\textbackslash mSymbol\{stethoscope-arrow\}} & \texttt{F807}\\
\mSymbol[outlined]{stethoscope-check} & \mSymbol[rounded]{stethoscope-check} & \mSymbol[sharp]{stethoscope-check} & \texttt{\textbackslash mSymbol\{stethoscope-check\}} & \texttt{F806}\\
\mSymbol[outlined]{sticky-note} & \mSymbol[rounded]{sticky-note} & \mSymbol[sharp]{sticky-note} & \texttt{\textbackslash mSymbol\{sticky-note\}} & \texttt{E9E8}\\
\mSymbol[outlined]{sticky-note-2} & \mSymbol[rounded]{sticky-note-2} & \mSymbol[sharp]{sticky-note-2} & \texttt{\textbackslash mSymbol\{sticky-note-2\}} & \texttt{F1FC}\\
\mSymbol[outlined]{stock-media} & \mSymbol[rounded]{stock-media} & \mSymbol[sharp]{stock-media} & \texttt{\textbackslash mSymbol\{stock-media\}} & \texttt{F570}\\
\mSymbol[outlined]{stockpot} & \mSymbol[rounded]{stockpot} & \mSymbol[sharp]{stockpot} & \texttt{\textbackslash mSymbol\{stockpot\}} & \texttt{F545}\\
\mSymbol[outlined]{stop} & \mSymbol[rounded]{stop} & \mSymbol[sharp]{stop} & \texttt{\textbackslash mSymbol\{stop\}} & \texttt{E047}\\
\mSymbol[outlined]{stop-circle} & \mSymbol[rounded]{stop-circle} & \mSymbol[sharp]{stop-circle} & \texttt{\textbackslash mSymbol\{stop-circle\}} & \texttt{EF71}\\
\mSymbol[outlined]{stop-screen-share} & \mSymbol[rounded]{stop-screen-share} & \mSymbol[sharp]{stop-screen-share} & \texttt{\textbackslash mSymbol\{stop-screen-share\}} & \texttt{E0E3}\\
\mSymbol[outlined]{storage} & \mSymbol[rounded]{storage} & \mSymbol[sharp]{storage} & \texttt{\textbackslash mSymbol\{storage\}} & \texttt{E1DB}\\
\mSymbol[outlined]{store} & \mSymbol[rounded]{store} & \mSymbol[sharp]{store} & \texttt{\textbackslash mSymbol\{store\}} & \texttt{E8D1}\\
\mSymbol[outlined]{store-mall-directory} & \mSymbol[rounded]{store-mall-directory} & \mSymbol[sharp]{store-mall-directory} & \texttt{\textbackslash mSymbol\{store-mall-directory\}} & \texttt{E8D1}\\
\mSymbol[outlined]{storefront} & \mSymbol[rounded]{storefront} & \mSymbol[sharp]{storefront} & \texttt{\textbackslash mSymbol\{storefront\}} & \texttt{EA12}\\
\mSymbol[outlined]{storm} & \mSymbol[rounded]{storm} & \mSymbol[sharp]{storm} & \texttt{\textbackslash mSymbol\{storm\}} & \texttt{F070}\\
\mSymbol[outlined]{straight} & \mSymbol[rounded]{straight} & \mSymbol[sharp]{straight} & \texttt{\textbackslash mSymbol\{straight\}} & \texttt{EB95}\\
\mSymbol[outlined]{straighten} & \mSymbol[rounded]{straighten} & \mSymbol[sharp]{straighten} & \texttt{\textbackslash mSymbol\{straighten\}} & \texttt{E41C}\\
\mSymbol[outlined]{strategy} & \mSymbol[rounded]{strategy} & \mSymbol[sharp]{strategy} & \texttt{\textbackslash mSymbol\{strategy\}} & \texttt{F5DF}\\
\mSymbol[outlined]{stream} & \mSymbol[rounded]{stream} & \mSymbol[sharp]{stream} & \texttt{\textbackslash mSymbol\{stream\}} & \texttt{E9E9}\\
\mSymbol[outlined]{stream-apps} & \mSymbol[rounded]{stream-apps} & \mSymbol[sharp]{stream-apps} & \texttt{\textbackslash mSymbol\{stream-apps\}} & \texttt{F79F}\\
\mSymbol[outlined]{streetview} & \mSymbol[rounded]{streetview} & \mSymbol[sharp]{streetview} & \texttt{\textbackslash mSymbol\{streetview\}} & \texttt{E56E}\\
\mSymbol[outlined]{stress-management} & \mSymbol[rounded]{stress-management} & \mSymbol[sharp]{stress-management} & \texttt{\textbackslash mSymbol\{stress-management\}} & \texttt{F6D9}\\
\mSymbol[outlined]{strikethrough-s} & \mSymbol[rounded]{strikethrough-s} & \mSymbol[sharp]{strikethrough-s} & \texttt{\textbackslash mSymbol\{strikethrough-s\}} & \texttt{E257}\\
\mSymbol[outlined]{stroke-full} & \mSymbol[rounded]{stroke-full} & \mSymbol[sharp]{stroke-full} & \texttt{\textbackslash mSymbol\{stroke-full\}} & \texttt{F749}\\
\mSymbol[outlined]{stroke-partial} & \mSymbol[rounded]{stroke-partial} & \mSymbol[sharp]{stroke-partial} & \texttt{\textbackslash mSymbol\{stroke-partial\}} & \texttt{F748}\\
\mSymbol[outlined]{stroller} & \mSymbol[rounded]{stroller} & \mSymbol[sharp]{stroller} & \texttt{\textbackslash mSymbol\{stroller\}} & \texttt{F1AE}\\
\mSymbol[outlined]{style} & \mSymbol[rounded]{style} & \mSymbol[sharp]{style} & \texttt{\textbackslash mSymbol\{style\}} & \texttt{E41D}\\
\mSymbol[outlined]{styler} & \mSymbol[rounded]{styler} & \mSymbol[sharp]{styler} & \texttt{\textbackslash mSymbol\{styler\}} & \texttt{E273}\\
\mSymbol[outlined]{stylus} & \mSymbol[rounded]{stylus} & \mSymbol[sharp]{stylus} & \texttt{\textbackslash mSymbol\{stylus\}} & \texttt{F604}\\
\mSymbol[outlined]{stylus-laser-pointer} & \mSymbol[rounded]{stylus-laser-pointer} & \mSymbol[sharp]{stylus-laser-pointer} & \texttt{\textbackslash mSymbol\{stylus-laser-pointer\}} & \texttt{F747}\\
\mSymbol[outlined]{stylus-note} & \mSymbol[rounded]{stylus-note} & \mSymbol[sharp]{stylus-note} & \texttt{\textbackslash mSymbol\{stylus-note\}} & \texttt{F603}\\
\mSymbol[outlined]{subdirectory-arrow-left} & \mSymbol[rounded]{subdirectory-arrow-left} & \mSymbol[sharp]{subdirectory-arrow-left} & \texttt{\textbackslash mSymbol\{subdirectory-arrow-left\}} & \texttt{E5D9}\\
\mSymbol[outlined]{subdirectory-arrow-right} & \mSymbol[rounded]{subdirectory-arrow-right} & \mSymbol[sharp]{subdirectory-arrow-right} & \texttt{\textbackslash mSymbol\{subdirectory-arrow-right\}} & \texttt{E5DA}\\
\mSymbol[outlined]{subheader} & \mSymbol[rounded]{subheader} & \mSymbol[sharp]{subheader} & \texttt{\textbackslash mSymbol\{subheader\}} & \texttt{E9EA}\\
\mSymbol[outlined]{subject} & \mSymbol[rounded]{subject} & \mSymbol[sharp]{subject} & \texttt{\textbackslash mSymbol\{subject\}} & \texttt{E8D2}\\
\mSymbol[outlined]{subscript} & \mSymbol[rounded]{subscript} & \mSymbol[sharp]{subscript} & \texttt{\textbackslash mSymbol\{subscript\}} & \texttt{F111}\\
\mSymbol[outlined]{subscriptions} & \mSymbol[rounded]{subscriptions} & \mSymbol[sharp]{subscriptions} & \texttt{\textbackslash mSymbol\{subscriptions\}} & \texttt{E064}\\
\mSymbol[outlined]{subtitles} & \mSymbol[rounded]{subtitles} & \mSymbol[sharp]{subtitles} & \texttt{\textbackslash mSymbol\{subtitles\}} & \texttt{E048}\\
\mSymbol[outlined]{subtitles-off} & \mSymbol[rounded]{subtitles-off} & \mSymbol[sharp]{subtitles-off} & \texttt{\textbackslash mSymbol\{subtitles-off\}} & \texttt{EF72}\\
\mSymbol[outlined]{subway} & \mSymbol[rounded]{subway} & \mSymbol[sharp]{subway} & \texttt{\textbackslash mSymbol\{subway\}} & \texttt{E56F}\\
\mSymbol[outlined]{summarize} & \mSymbol[rounded]{summarize} & \mSymbol[sharp]{summarize} & \texttt{\textbackslash mSymbol\{summarize\}} & \texttt{F071}\\
\mSymbol[outlined]{sunny} & \mSymbol[rounded]{sunny} & \mSymbol[sharp]{sunny} & \texttt{\textbackslash mSymbol\{sunny\}} & \texttt{E81A}\\
\mSymbol[outlined]{sunny-snowing} & \mSymbol[rounded]{sunny-snowing} & \mSymbol[sharp]{sunny-snowing} & \texttt{\textbackslash mSymbol\{sunny-snowing\}} & \texttt{E819}\\
\mSymbol[outlined]{superscript} & \mSymbol[rounded]{superscript} & \mSymbol[sharp]{superscript} & \texttt{\textbackslash mSymbol\{superscript\}} & \texttt{F112}\\
\mSymbol[outlined]{supervised-user-circle} & \mSymbol[rounded]{supervised-user-circle} & \mSymbol[sharp]{supervised-user-circle} & \texttt{\textbackslash mSymbol\{supervised-user-circle\}} & \texttt{E939}\\
\mSymbol[outlined]{supervised-user-circle-off} & \mSymbol[rounded]{supervised-user-circle-off} & \mSymbol[sharp]{supervised-user-circle-off} & \texttt{\textbackslash mSymbol\{supervised-user-circle-off\}} & \texttt{F60E}\\
\mSymbol[outlined]{supervisor-account} & \mSymbol[rounded]{supervisor-account} & \mSymbol[sharp]{supervisor-account} & \texttt{\textbackslash mSymbol\{supervisor-account\}} & \texttt{E8D3}\\
\mSymbol[outlined]{support} & \mSymbol[rounded]{support} & \mSymbol[sharp]{support} & \texttt{\textbackslash mSymbol\{support\}} & \texttt{EF73}\\
\mSymbol[outlined]{support-agent} & \mSymbol[rounded]{support-agent} & \mSymbol[sharp]{support-agent} & \texttt{\textbackslash mSymbol\{support-agent\}} & \texttt{F0E2}\\
\mSymbol[outlined]{surfing} & \mSymbol[rounded]{surfing} & \mSymbol[sharp]{surfing} & \texttt{\textbackslash mSymbol\{surfing\}} & \texttt{E515}\\
\mSymbol[outlined]{surgical} & \mSymbol[rounded]{surgical} & \mSymbol[sharp]{surgical} & \texttt{\textbackslash mSymbol\{surgical\}} & \texttt{E131}\\
\mSymbol[outlined]{surround-sound} & \mSymbol[rounded]{surround-sound} & \mSymbol[sharp]{surround-sound} & \texttt{\textbackslash mSymbol\{surround-sound\}} & \texttt{E049}\\
\mSymbol[outlined]{swap-calls} & \mSymbol[rounded]{swap-calls} & \mSymbol[sharp]{swap-calls} & \texttt{\textbackslash mSymbol\{swap-calls\}} & \texttt{E0D7}\\
\mSymbol[outlined]{swap-driving-apps} & \mSymbol[rounded]{swap-driving-apps} & \mSymbol[sharp]{swap-driving-apps} & \texttt{\textbackslash mSymbol\{swap-driving-apps\}} & \texttt{E69E}\\
\mSymbol[outlined]{swap-driving-apps-wheel} & \mSymbol[rounded]{swap-driving-apps-wheel} & \mSymbol[sharp]{swap-driving-apps-wheel} & \texttt{\textbackslash mSymbol\{swap-driving-apps-wheel\}} & \texttt{E69F}\\
\mSymbol[outlined]{swap-horiz} & \mSymbol[rounded]{swap-horiz} & \mSymbol[sharp]{swap-horiz} & \texttt{\textbackslash mSymbol\{swap-horiz\}} & \texttt{E8D4}\\
\mSymbol[outlined]{swap-horizontal-circle} & \mSymbol[rounded]{swap-horizontal-circle} & \mSymbol[sharp]{swap-horizontal-circle} & \texttt{\textbackslash mSymbol\{swap-horizontal-circle\}} & \texttt{E933}\\
\mSymbol[outlined]{swap-vert} & \mSymbol[rounded]{swap-vert} & \mSymbol[sharp]{swap-vert} & \texttt{\textbackslash mSymbol\{swap-vert\}} & \texttt{E8D5}\\
\mSymbol[outlined]{swap-vertical-circle} & \mSymbol[rounded]{swap-vertical-circle} & \mSymbol[sharp]{swap-vertical-circle} & \texttt{\textbackslash mSymbol\{swap-vertical-circle\}} & \texttt{E8D6}\\
\mSymbol[outlined]{sweep} & \mSymbol[rounded]{sweep} & \mSymbol[sharp]{sweep} & \texttt{\textbackslash mSymbol\{sweep\}} & \texttt{E6AC}\\
\mSymbol[outlined]{swipe} & \mSymbol[rounded]{swipe} & \mSymbol[sharp]{swipe} & \texttt{\textbackslash mSymbol\{swipe\}} & \texttt{E9EC}\\
\mSymbol[outlined]{swipe-down} & \mSymbol[rounded]{swipe-down} & \mSymbol[sharp]{swipe-down} & \texttt{\textbackslash mSymbol\{swipe-down\}} & \texttt{EB53}\\
\mSymbol[outlined]{swipe-down-alt} & \mSymbol[rounded]{swipe-down-alt} & \mSymbol[sharp]{swipe-down-alt} & \texttt{\textbackslash mSymbol\{swipe-down-alt\}} & \texttt{EB30}\\
\mSymbol[outlined]{swipe-left} & \mSymbol[rounded]{swipe-left} & \mSymbol[sharp]{swipe-left} & \texttt{\textbackslash mSymbol\{swipe-left\}} & \texttt{EB59}\\
\mSymbol[outlined]{swipe-left-alt} & \mSymbol[rounded]{swipe-left-alt} & \mSymbol[sharp]{swipe-left-alt} & \texttt{\textbackslash mSymbol\{swipe-left-alt\}} & \texttt{EB33}\\
\mSymbol[outlined]{swipe-right} & \mSymbol[rounded]{swipe-right} & \mSymbol[sharp]{swipe-right} & \texttt{\textbackslash mSymbol\{swipe-right\}} & \texttt{EB52}\\
\mSymbol[outlined]{swipe-right-alt} & \mSymbol[rounded]{swipe-right-alt} & \mSymbol[sharp]{swipe-right-alt} & \texttt{\textbackslash mSymbol\{swipe-right-alt\}} & \texttt{EB56}\\
\mSymbol[outlined]{swipe-up} & \mSymbol[rounded]{swipe-up} & \mSymbol[sharp]{swipe-up} & \texttt{\textbackslash mSymbol\{swipe-up\}} & \texttt{EB2E}\\
\mSymbol[outlined]{swipe-up-alt} & \mSymbol[rounded]{swipe-up-alt} & \mSymbol[sharp]{swipe-up-alt} & \texttt{\textbackslash mSymbol\{swipe-up-alt\}} & \texttt{EB35}\\
\mSymbol[outlined]{swipe-vertical} & \mSymbol[rounded]{swipe-vertical} & \mSymbol[sharp]{swipe-vertical} & \texttt{\textbackslash mSymbol\{swipe-vertical\}} & \texttt{EB51}\\
\mSymbol[outlined]{switch} & \mSymbol[rounded]{switch} & \mSymbol[sharp]{switch} & \texttt{\textbackslash mSymbol\{switch\}} & \texttt{E1F4}\\
\mSymbol[outlined]{switch-access} & \mSymbol[rounded]{switch-access} & \mSymbol[sharp]{switch-access} & \texttt{\textbackslash mSymbol\{switch-access\}} & \texttt{F6FD}\\
\mSymbol[outlined]{switch-access-2} & \mSymbol[rounded]{switch-access-2} & \mSymbol[sharp]{switch-access-2} & \texttt{\textbackslash mSymbol\{switch-access-2\}} & \texttt{F506}\\
\mSymbol[outlined]{switch-access-shortcut} & \mSymbol[rounded]{switch-access-shortcut} & \mSymbol[sharp]{switch-access-shortcut} & \texttt{\textbackslash mSymbol\{switch-access-shortcut\}} & \texttt{E7E1}\\
\mSymbol[outlined]{switch-access-shortcut-add} & \mSymbol[rounded]{switch-access-shortcut-add} & \mSymbol[sharp]{switch-access-shortcut-add} & \texttt{\textbackslash mSymbol\{switch-access-shortcut-add\}} & \texttt{E7E2}\\
\mSymbol[outlined]{switch-account} & \mSymbol[rounded]{switch-account} & \mSymbol[sharp]{switch-account} & \texttt{\textbackslash mSymbol\{switch-account\}} & \texttt{E9ED}\\
\mSymbol[outlined]{switch-camera} & \mSymbol[rounded]{switch-camera} & \mSymbol[sharp]{switch-camera} & \texttt{\textbackslash mSymbol\{switch-camera\}} & \texttt{E41E}\\
\mSymbol[outlined]{switch-left} & \mSymbol[rounded]{switch-left} & \mSymbol[sharp]{switch-left} & \texttt{\textbackslash mSymbol\{switch-left\}} & \texttt{F1D1}\\
\mSymbol[outlined]{switch-right} & \mSymbol[rounded]{switch-right} & \mSymbol[sharp]{switch-right} & \texttt{\textbackslash mSymbol\{switch-right\}} & \texttt{F1D2}\\
\mSymbol[outlined]{switch-video} & \mSymbol[rounded]{switch-video} & \mSymbol[sharp]{switch-video} & \texttt{\textbackslash mSymbol\{switch-video\}} & \texttt{E41F}\\
\mSymbol[outlined]{switches} & \mSymbol[rounded]{switches} & \mSymbol[sharp]{switches} & \texttt{\textbackslash mSymbol\{switches\}} & \texttt{E733}\\
\mSymbol[outlined]{sword-rose} & \mSymbol[rounded]{sword-rose} & \mSymbol[sharp]{sword-rose} & \texttt{\textbackslash mSymbol\{sword-rose\}} & \texttt{F5DE}\\
\mSymbol[outlined]{swords} & \mSymbol[rounded]{swords} & \mSymbol[sharp]{swords} & \texttt{\textbackslash mSymbol\{swords\}} & \texttt{F889}\\
\mSymbol[outlined]{symptoms} & \mSymbol[rounded]{symptoms} & \mSymbol[sharp]{symptoms} & \texttt{\textbackslash mSymbol\{symptoms\}} & \texttt{E132}\\
\mSymbol[outlined]{synagogue} & \mSymbol[rounded]{synagogue} & \mSymbol[sharp]{synagogue} & \texttt{\textbackslash mSymbol\{synagogue\}} & \texttt{EAB0}\\
\mSymbol[outlined]{sync} & \mSymbol[rounded]{sync} & \mSymbol[sharp]{sync} & \texttt{\textbackslash mSymbol\{sync\}} & \texttt{E627}\\
\mSymbol[outlined]{sync-alt} & \mSymbol[rounded]{sync-alt} & \mSymbol[sharp]{sync-alt} & \texttt{\textbackslash mSymbol\{sync-alt\}} & \texttt{EA18}\\
\mSymbol[outlined]{sync-desktop} & \mSymbol[rounded]{sync-desktop} & \mSymbol[sharp]{sync-desktop} & \texttt{\textbackslash mSymbol\{sync-desktop\}} & \texttt{F41A}\\
\mSymbol[outlined]{sync-disabled} & \mSymbol[rounded]{sync-disabled} & \mSymbol[sharp]{sync-disabled} & \texttt{\textbackslash mSymbol\{sync-disabled\}} & \texttt{E628}\\
\mSymbol[outlined]{sync-lock} & \mSymbol[rounded]{sync-lock} & \mSymbol[sharp]{sync-lock} & \texttt{\textbackslash mSymbol\{sync-lock\}} & \texttt{EAEE}\\
\mSymbol[outlined]{sync-problem} & \mSymbol[rounded]{sync-problem} & \mSymbol[sharp]{sync-problem} & \texttt{\textbackslash mSymbol\{sync-problem\}} & \texttt{E629}\\
\mSymbol[outlined]{sync-saved-locally} & \mSymbol[rounded]{sync-saved-locally} & \mSymbol[sharp]{sync-saved-locally} & \texttt{\textbackslash mSymbol\{sync-saved-locally\}} & \texttt{F820}\\
\mSymbol[outlined]{syringe} & \mSymbol[rounded]{syringe} & \mSymbol[sharp]{syringe} & \texttt{\textbackslash mSymbol\{syringe\}} & \texttt{E133}\\
\mSymbol[outlined]{system-security-update} & \mSymbol[rounded]{system-security-update} & \mSymbol[sharp]{system-security-update} & \texttt{\textbackslash mSymbol\{system-security-update\}} & \texttt{F072}\\
\mSymbol[outlined]{system-security-update-good} & \mSymbol[rounded]{system-security-update-good} & \mSymbol[sharp]{system-security-update-good} & \texttt{\textbackslash mSymbol\{system-security-update-good\}} & \texttt{F073}\\
\mSymbol[outlined]{system-security-update-warning} & \mSymbol[rounded]{system-security-update-warning} & \mSymbol[sharp]{system-security-update-warning} & \texttt{\textbackslash mSymbol\{system-security-update-warning\}} & \texttt{F074}\\
\mSymbol[outlined]{system-update} & \mSymbol[rounded]{system-update} & \mSymbol[sharp]{system-update} & \texttt{\textbackslash mSymbol\{system-update\}} & \texttt{F072}\\
\mSymbol[outlined]{system-update-alt} & \mSymbol[rounded]{system-update-alt} & \mSymbol[sharp]{system-update-alt} & \texttt{\textbackslash mSymbol\{system-update-alt\}} & \texttt{E8D7}\\
\mSymbol[outlined]{tab} & \mSymbol[rounded]{tab} & \mSymbol[sharp]{tab} & \texttt{\textbackslash mSymbol\{tab\}} & \texttt{E8D8}\\
\mSymbol[outlined]{tab-close} & \mSymbol[rounded]{tab-close} & \mSymbol[sharp]{tab-close} & \texttt{\textbackslash mSymbol\{tab-close\}} & \texttt{F745}\\
\mSymbol[outlined]{tab-close-right} & \mSymbol[rounded]{tab-close-right} & \mSymbol[sharp]{tab-close-right} & \texttt{\textbackslash mSymbol\{tab-close-right\}} & \texttt{F746}\\
\mSymbol[outlined]{tab-duplicate} & \mSymbol[rounded]{tab-duplicate} & \mSymbol[sharp]{tab-duplicate} & \texttt{\textbackslash mSymbol\{tab-duplicate\}} & \texttt{F744}\\
\mSymbol[outlined]{tab-group} & \mSymbol[rounded]{tab-group} & \mSymbol[sharp]{tab-group} & \texttt{\textbackslash mSymbol\{tab-group\}} & \texttt{F743}\\
\mSymbol[outlined]{tab-inactive} & \mSymbol[rounded]{tab-inactive} & \mSymbol[sharp]{tab-inactive} & \texttt{\textbackslash mSymbol\{tab-inactive\}} & \texttt{F43B}\\
\mSymbol[outlined]{tab-move} & \mSymbol[rounded]{tab-move} & \mSymbol[sharp]{tab-move} & \texttt{\textbackslash mSymbol\{tab-move\}} & \texttt{F742}\\
\mSymbol[outlined]{tab-new-right} & \mSymbol[rounded]{tab-new-right} & \mSymbol[sharp]{tab-new-right} & \texttt{\textbackslash mSymbol\{tab-new-right\}} & \texttt{F741}\\
\mSymbol[outlined]{tab-recent} & \mSymbol[rounded]{tab-recent} & \mSymbol[sharp]{tab-recent} & \texttt{\textbackslash mSymbol\{tab-recent\}} & \texttt{F740}\\
\mSymbol[outlined]{tab-unselected} & \mSymbol[rounded]{tab-unselected} & \mSymbol[sharp]{tab-unselected} & \texttt{\textbackslash mSymbol\{tab-unselected\}} & \texttt{E8D9}\\
\mSymbol[outlined]{table} & \mSymbol[rounded]{table} & \mSymbol[sharp]{table} & \texttt{\textbackslash mSymbol\{table\}} & \texttt{F191}\\
\mSymbol[outlined]{table-bar} & \mSymbol[rounded]{table-bar} & \mSymbol[sharp]{table-bar} & \texttt{\textbackslash mSymbol\{table-bar\}} & \texttt{EAD2}\\
\mSymbol[outlined]{table-chart} & \mSymbol[rounded]{table-chart} & \mSymbol[sharp]{table-chart} & \texttt{\textbackslash mSymbol\{table-chart\}} & \texttt{E265}\\
\mSymbol[outlined]{table-chart-view} & \mSymbol[rounded]{table-chart-view} & \mSymbol[sharp]{table-chart-view} & \texttt{\textbackslash mSymbol\{table-chart-view\}} & \texttt{F6EF}\\
\mSymbol[outlined]{table-eye} & \mSymbol[rounded]{table-eye} & \mSymbol[sharp]{table-eye} & \texttt{\textbackslash mSymbol\{table-eye\}} & \texttt{F466}\\
\mSymbol[outlined]{table-lamp} & \mSymbol[rounded]{table-lamp} & \mSymbol[sharp]{table-lamp} & \texttt{\textbackslash mSymbol\{table-lamp\}} & \texttt{E1F2}\\
\mSymbol[outlined]{table-restaurant} & \mSymbol[rounded]{table-restaurant} & \mSymbol[sharp]{table-restaurant} & \texttt{\textbackslash mSymbol\{table-restaurant\}} & \texttt{EAC6}\\
\mSymbol[outlined]{table-rows} & \mSymbol[rounded]{table-rows} & \mSymbol[sharp]{table-rows} & \texttt{\textbackslash mSymbol\{table-rows\}} & \texttt{F101}\\
\mSymbol[outlined]{table-rows-narrow} & \mSymbol[rounded]{table-rows-narrow} & \mSymbol[sharp]{table-rows-narrow} & \texttt{\textbackslash mSymbol\{table-rows-narrow\}} & \texttt{F73F}\\
\mSymbol[outlined]{table-view} & \mSymbol[rounded]{table-view} & \mSymbol[sharp]{table-view} & \texttt{\textbackslash mSymbol\{table-view\}} & \texttt{F1BE}\\
\mSymbol[outlined]{tablet} & \mSymbol[rounded]{tablet} & \mSymbol[sharp]{tablet} & \texttt{\textbackslash mSymbol\{tablet\}} & \texttt{E32F}\\
\mSymbol[outlined]{tablet-android} & \mSymbol[rounded]{tablet-android} & \mSymbol[sharp]{tablet-android} & \texttt{\textbackslash mSymbol\{tablet-android\}} & \texttt{E330}\\
\mSymbol[outlined]{tablet-camera} & \mSymbol[rounded]{tablet-camera} & \mSymbol[sharp]{tablet-camera} & \texttt{\textbackslash mSymbol\{tablet-camera\}} & \texttt{F44D}\\
\mSymbol[outlined]{tablet-mac} & \mSymbol[rounded]{tablet-mac} & \mSymbol[sharp]{tablet-mac} & \texttt{\textbackslash mSymbol\{tablet-mac\}} & \texttt{E331}\\
\mSymbol[outlined]{tabs} & \mSymbol[rounded]{tabs} & \mSymbol[sharp]{tabs} & \texttt{\textbackslash mSymbol\{tabs\}} & \texttt{E9EE}\\
\mSymbol[outlined]{tactic} & \mSymbol[rounded]{tactic} & \mSymbol[sharp]{tactic} & \texttt{\textbackslash mSymbol\{tactic\}} & \texttt{F564}\\
\mSymbol[outlined]{tag} & \mSymbol[rounded]{tag} & \mSymbol[sharp]{tag} & \texttt{\textbackslash mSymbol\{tag\}} & \texttt{E9EF}\\
\mSymbol[outlined]{tag-faces} & \mSymbol[rounded]{tag-faces} & \mSymbol[sharp]{tag-faces} & \texttt{\textbackslash mSymbol\{tag-faces\}} & \texttt{EA22}\\
\mSymbol[outlined]{takeout-dining} & \mSymbol[rounded]{takeout-dining} & \mSymbol[sharp]{takeout-dining} & \texttt{\textbackslash mSymbol\{takeout-dining\}} & \texttt{EA74}\\
\mSymbol[outlined]{tamper-detection-off} & \mSymbol[rounded]{tamper-detection-off} & \mSymbol[sharp]{tamper-detection-off} & \texttt{\textbackslash mSymbol\{tamper-detection-off\}} & \texttt{E82E}\\
\mSymbol[outlined]{tamper-detection-on} & \mSymbol[rounded]{tamper-detection-on} & \mSymbol[sharp]{tamper-detection-on} & \texttt{\textbackslash mSymbol\{tamper-detection-on\}} & \texttt{F8C8}\\
\mSymbol[outlined]{tap-and-play} & \mSymbol[rounded]{tap-and-play} & \mSymbol[sharp]{tap-and-play} & \texttt{\textbackslash mSymbol\{tap-and-play\}} & \texttt{E62B}\\
\mSymbol[outlined]{tapas} & \mSymbol[rounded]{tapas} & \mSymbol[sharp]{tapas} & \texttt{\textbackslash mSymbol\{tapas\}} & \texttt{F1E9}\\
\mSymbol[outlined]{target} & \mSymbol[rounded]{target} & \mSymbol[sharp]{target} & \texttt{\textbackslash mSymbol\{target\}} & \texttt{E719}\\
\mSymbol[outlined]{task} & \mSymbol[rounded]{task} & \mSymbol[sharp]{task} & \texttt{\textbackslash mSymbol\{task\}} & \texttt{F075}\\
\mSymbol[outlined]{task-alt} & \mSymbol[rounded]{task-alt} & \mSymbol[sharp]{task-alt} & \texttt{\textbackslash mSymbol\{task-alt\}} & \texttt{E2E6}\\
\mSymbol[outlined]{taunt} & \mSymbol[rounded]{taunt} & \mSymbol[sharp]{taunt} & \texttt{\textbackslash mSymbol\{taunt\}} & \texttt{F69F}\\
\mSymbol[outlined]{taxi-alert} & \mSymbol[rounded]{taxi-alert} & \mSymbol[sharp]{taxi-alert} & \texttt{\textbackslash mSymbol\{taxi-alert\}} & \texttt{EF74}\\
\mSymbol[outlined]{team-dashboard} & \mSymbol[rounded]{team-dashboard} & \mSymbol[sharp]{team-dashboard} & \texttt{\textbackslash mSymbol\{team-dashboard\}} & \texttt{E013}\\
\mSymbol[outlined]{temp-preferences-custom} & \mSymbol[rounded]{temp-preferences-custom} & \mSymbol[sharp]{temp-preferences-custom} & \texttt{\textbackslash mSymbol\{temp-preferences-custom\}} & \texttt{F8C9}\\
\mSymbol[outlined]{temp-preferences-eco} & \mSymbol[rounded]{temp-preferences-eco} & \mSymbol[sharp]{temp-preferences-eco} & \texttt{\textbackslash mSymbol\{temp-preferences-eco\}} & \texttt{F8CA}\\
\mSymbol[outlined]{temple-buddhist} & \mSymbol[rounded]{temple-buddhist} & \mSymbol[sharp]{temple-buddhist} & \texttt{\textbackslash mSymbol\{temple-buddhist\}} & \texttt{EAB3}\\
\mSymbol[outlined]{temple-hindu} & \mSymbol[rounded]{temple-hindu} & \mSymbol[sharp]{temple-hindu} & \texttt{\textbackslash mSymbol\{temple-hindu\}} & \texttt{EAAF}\\
\mSymbol[outlined]{tenancy} & \mSymbol[rounded]{tenancy} & \mSymbol[sharp]{tenancy} & \texttt{\textbackslash mSymbol\{tenancy\}} & \texttt{F0E3}\\
\mSymbol[outlined]{terminal} & \mSymbol[rounded]{terminal} & \mSymbol[sharp]{terminal} & \texttt{\textbackslash mSymbol\{terminal\}} & \texttt{EB8E}\\
\mSymbol[outlined]{terrain} & \mSymbol[rounded]{terrain} & \mSymbol[sharp]{terrain} & \texttt{\textbackslash mSymbol\{terrain\}} & \texttt{E564}\\
\mSymbol[outlined]{text-ad} & \mSymbol[rounded]{text-ad} & \mSymbol[sharp]{text-ad} & \texttt{\textbackslash mSymbol\{text-ad\}} & \texttt{E728}\\
\mSymbol[outlined]{text-decrease} & \mSymbol[rounded]{text-decrease} & \mSymbol[sharp]{text-decrease} & \texttt{\textbackslash mSymbol\{text-decrease\}} & \texttt{EADD}\\
\mSymbol[outlined]{text-fields} & \mSymbol[rounded]{text-fields} & \mSymbol[sharp]{text-fields} & \texttt{\textbackslash mSymbol\{text-fields\}} & \texttt{E262}\\
\mSymbol[outlined]{text-fields-alt} & \mSymbol[rounded]{text-fields-alt} & \mSymbol[sharp]{text-fields-alt} & \texttt{\textbackslash mSymbol\{text-fields-alt\}} & \texttt{E9F1}\\
\mSymbol[outlined]{text-format} & \mSymbol[rounded]{text-format} & \mSymbol[sharp]{text-format} & \texttt{\textbackslash mSymbol\{text-format\}} & \texttt{E165}\\
\mSymbol[outlined]{text-increase} & \mSymbol[rounded]{text-increase} & \mSymbol[sharp]{text-increase} & \texttt{\textbackslash mSymbol\{text-increase\}} & \texttt{EAE2}\\
\mSymbol[outlined]{text-rotate-up} & \mSymbol[rounded]{text-rotate-up} & \mSymbol[sharp]{text-rotate-up} & \texttt{\textbackslash mSymbol\{text-rotate-up\}} & \texttt{E93A}\\
\mSymbol[outlined]{text-rotate-vertical} & \mSymbol[rounded]{text-rotate-vertical} & \mSymbol[sharp]{text-rotate-vertical} & \texttt{\textbackslash mSymbol\{text-rotate-vertical\}} & \texttt{E93B}\\
\mSymbol[outlined]{text-rotation-angledown} & \mSymbol[rounded]{text-rotation-angledown} & \mSymbol[sharp]{text-rotation-angledown} & \texttt{\textbackslash mSymbol\{text-rotation-angledown\}} & \texttt{E93C}\\
\mSymbol[outlined]{text-rotation-angleup} & \mSymbol[rounded]{text-rotation-angleup} & \mSymbol[sharp]{text-rotation-angleup} & \texttt{\textbackslash mSymbol\{text-rotation-angleup\}} & \texttt{E93D}\\
\mSymbol[outlined]{text-rotation-down} & \mSymbol[rounded]{text-rotation-down} & \mSymbol[sharp]{text-rotation-down} & \texttt{\textbackslash mSymbol\{text-rotation-down\}} & \texttt{E93E}\\
\mSymbol[outlined]{text-rotation-none} & \mSymbol[rounded]{text-rotation-none} & \mSymbol[sharp]{text-rotation-none} & \texttt{\textbackslash mSymbol\{text-rotation-none\}} & \texttt{E93F}\\
\mSymbol[outlined]{text-select-end} & \mSymbol[rounded]{text-select-end} & \mSymbol[sharp]{text-select-end} & \texttt{\textbackslash mSymbol\{text-select-end\}} & \texttt{F73E}\\
\mSymbol[outlined]{text-select-jump-to-beginning} & \mSymbol[rounded]{text-select-jump-to-beginning} & \mSymbol[sharp]{text-select-jump-to-beginning} & \texttt{\textbackslash mSymbol\{text-select-jump-to-beginning\}} & \texttt{F73D}\\
\mSymbol[outlined]{text-select-jump-to-end} & \mSymbol[rounded]{text-select-jump-to-end} & \mSymbol[sharp]{text-select-jump-to-end} & \texttt{\textbackslash mSymbol\{text-select-jump-to-end\}} & \texttt{F73C}\\
\mSymbol[outlined]{text-select-move-back-character} & \mSymbol[rounded]{text-select-move-back-character} & \mSymbol[sharp]{text-select-move-back-character} & \texttt{\textbackslash mSymbol\{text-select-move-back-character\}} & \texttt{F73B}\\
\mSymbol[outlined]{text-select-move-back-word} & \mSymbol[rounded]{text-select-move-back-word} & \mSymbol[sharp]{text-select-move-back-word} & \texttt{\textbackslash mSymbol\{text-select-move-back-word\}} & \texttt{F73A}\\
\mSymbol[outlined]{text-select-move-down} & \mSymbol[rounded]{text-select-move-down} & \mSymbol[sharp]{text-select-move-down} & \texttt{\textbackslash mSymbol\{text-select-move-down\}} & \texttt{F739}\\
\mSymbol[outlined]{text-select-move-forward-character} & \mSymbol[rounded]{text-select-move-forward-character} & \mSymbol[sharp]{text-select-move-forward-character} & \texttt{\textbackslash mSymbol\{text-select-move-forward-character\}} & \texttt{F738}\\
\mSymbol[outlined]{text-select-move-forward-word} & \mSymbol[rounded]{text-select-move-forward-word} & \mSymbol[sharp]{text-select-move-forward-word} & \texttt{\textbackslash mSymbol\{text-select-move-forward-word\}} & \texttt{F737}\\
\mSymbol[outlined]{text-select-move-up} & \mSymbol[rounded]{text-select-move-up} & \mSymbol[sharp]{text-select-move-up} & \texttt{\textbackslash mSymbol\{text-select-move-up\}} & \texttt{F736}\\
\mSymbol[outlined]{text-select-start} & \mSymbol[rounded]{text-select-start} & \mSymbol[sharp]{text-select-start} & \texttt{\textbackslash mSymbol\{text-select-start\}} & \texttt{F735}\\
\mSymbol[outlined]{text-snippet} & \mSymbol[rounded]{text-snippet} & \mSymbol[sharp]{text-snippet} & \texttt{\textbackslash mSymbol\{text-snippet\}} & \texttt{F1C6}\\
\mSymbol[outlined]{text-to-speech} & \mSymbol[rounded]{text-to-speech} & \mSymbol[sharp]{text-to-speech} & \texttt{\textbackslash mSymbol\{text-to-speech\}} & \texttt{F1BC}\\
\mSymbol[outlined]{text-up} & \mSymbol[rounded]{text-up} & \mSymbol[sharp]{text-up} & \texttt{\textbackslash mSymbol\{text-up\}} & \texttt{F49E}\\
\mSymbol[outlined]{textsms} & \mSymbol[rounded]{textsms} & \mSymbol[sharp]{textsms} & \texttt{\textbackslash mSymbol\{textsms\}} & \texttt{E625}\\
\mSymbol[outlined]{texture} & \mSymbol[rounded]{texture} & \mSymbol[sharp]{texture} & \texttt{\textbackslash mSymbol\{texture\}} & \texttt{E421}\\
\mSymbol[outlined]{texture-add} & \mSymbol[rounded]{texture-add} & \mSymbol[sharp]{texture-add} & \texttt{\textbackslash mSymbol\{texture-add\}} & \texttt{F57C}\\
\mSymbol[outlined]{texture-minus} & \mSymbol[rounded]{texture-minus} & \mSymbol[sharp]{texture-minus} & \texttt{\textbackslash mSymbol\{texture-minus\}} & \texttt{F57B}\\
\mSymbol[outlined]{theater-comedy} & \mSymbol[rounded]{theater-comedy} & \mSymbol[sharp]{theater-comedy} & \texttt{\textbackslash mSymbol\{theater-comedy\}} & \texttt{EA66}\\
\mSymbol[outlined]{theaters} & \mSymbol[rounded]{theaters} & \mSymbol[sharp]{theaters} & \texttt{\textbackslash mSymbol\{theaters\}} & \texttt{E8DA}\\
\mSymbol[outlined]{thermometer} & \mSymbol[rounded]{thermometer} & \mSymbol[sharp]{thermometer} & \texttt{\textbackslash mSymbol\{thermometer\}} & \texttt{E846}\\
\mSymbol[outlined]{thermometer-add} & \mSymbol[rounded]{thermometer-add} & \mSymbol[sharp]{thermometer-add} & \texttt{\textbackslash mSymbol\{thermometer-add\}} & \texttt{F582}\\
\mSymbol[outlined]{thermometer-gain} & \mSymbol[rounded]{thermometer-gain} & \mSymbol[sharp]{thermometer-gain} & \texttt{\textbackslash mSymbol\{thermometer-gain\}} & \texttt{F6D8}\\
\mSymbol[outlined]{thermometer-loss} & \mSymbol[rounded]{thermometer-loss} & \mSymbol[sharp]{thermometer-loss} & \texttt{\textbackslash mSymbol\{thermometer-loss\}} & \texttt{F6D7}\\
\mSymbol[outlined]{thermometer-minus} & \mSymbol[rounded]{thermometer-minus} & \mSymbol[sharp]{thermometer-minus} & \texttt{\textbackslash mSymbol\{thermometer-minus\}} & \texttt{F581}\\
\mSymbol[outlined]{thermostat} & \mSymbol[rounded]{thermostat} & \mSymbol[sharp]{thermostat} & \texttt{\textbackslash mSymbol\{thermostat\}} & \texttt{F076}\\
\mSymbol[outlined]{thermostat-auto} & \mSymbol[rounded]{thermostat-auto} & \mSymbol[sharp]{thermostat-auto} & \texttt{\textbackslash mSymbol\{thermostat-auto\}} & \texttt{F077}\\
\mSymbol[outlined]{thermostat-carbon} & \mSymbol[rounded]{thermostat-carbon} & \mSymbol[sharp]{thermostat-carbon} & \texttt{\textbackslash mSymbol\{thermostat-carbon\}} & \texttt{F178}\\
\mSymbol[outlined]{things-to-do} & \mSymbol[rounded]{things-to-do} & \mSymbol[sharp]{things-to-do} & \texttt{\textbackslash mSymbol\{things-to-do\}} & \texttt{EB2A}\\
\mSymbol[outlined]{thread-unread} & \mSymbol[rounded]{thread-unread} & \mSymbol[sharp]{thread-unread} & \texttt{\textbackslash mSymbol\{thread-unread\}} & \texttt{F4F9}\\
\mSymbol[outlined]{thumb-down} & \mSymbol[rounded]{thumb-down} & \mSymbol[sharp]{thumb-down} & \texttt{\textbackslash mSymbol\{thumb-down\}} & \texttt{F578}\\
\mSymbol[outlined]{thumb-down-alt} & \mSymbol[rounded]{thumb-down-alt} & \mSymbol[sharp]{thumb-down-alt} & \texttt{\textbackslash mSymbol\{thumb-down-alt\}} & \texttt{F578}\\
\mSymbol[outlined]{thumb-down-filled} & \mSymbol[rounded]{thumb-down-filled} & \mSymbol[sharp]{thumb-down-filled} & \texttt{\textbackslash mSymbol\{thumb-down-filled\}} & \texttt{F578}\\
\mSymbol[outlined]{thumb-down-off} & \mSymbol[rounded]{thumb-down-off} & \mSymbol[sharp]{thumb-down-off} & \texttt{\textbackslash mSymbol\{thumb-down-off\}} & \texttt{F578}\\
\mSymbol[outlined]{thumb-down-off-alt} & \mSymbol[rounded]{thumb-down-off-alt} & \mSymbol[sharp]{thumb-down-off-alt} & \texttt{\textbackslash mSymbol\{thumb-down-off-alt\}} & \texttt{F578}\\
\mSymbol[outlined]{thumb-up} & \mSymbol[rounded]{thumb-up} & \mSymbol[sharp]{thumb-up} & \texttt{\textbackslash mSymbol\{thumb-up\}} & \texttt{F577}\\
\mSymbol[outlined]{thumb-up-alt} & \mSymbol[rounded]{thumb-up-alt} & \mSymbol[sharp]{thumb-up-alt} & \texttt{\textbackslash mSymbol\{thumb-up-alt\}} & \texttt{F577}\\
\mSymbol[outlined]{thumb-up-filled} & \mSymbol[rounded]{thumb-up-filled} & \mSymbol[sharp]{thumb-up-filled} & \texttt{\textbackslash mSymbol\{thumb-up-filled\}} & \texttt{F577}\\
\mSymbol[outlined]{thumb-up-off} & \mSymbol[rounded]{thumb-up-off} & \mSymbol[sharp]{thumb-up-off} & \texttt{\textbackslash mSymbol\{thumb-up-off\}} & \texttt{F577}\\
\mSymbol[outlined]{thumb-up-off-alt} & \mSymbol[rounded]{thumb-up-off-alt} & \mSymbol[sharp]{thumb-up-off-alt} & \texttt{\textbackslash mSymbol\{thumb-up-off-alt\}} & \texttt{F577}\\
\mSymbol[outlined]{thumbnail-bar} & \mSymbol[rounded]{thumbnail-bar} & \mSymbol[sharp]{thumbnail-bar} & \texttt{\textbackslash mSymbol\{thumbnail-bar\}} & \texttt{F734}\\
\mSymbol[outlined]{thumbs-up-down} & \mSymbol[rounded]{thumbs-up-down} & \mSymbol[sharp]{thumbs-up-down} & \texttt{\textbackslash mSymbol\{thumbs-up-down\}} & \texttt{E8DD}\\
\mSymbol[outlined]{thunderstorm} & \mSymbol[rounded]{thunderstorm} & \mSymbol[sharp]{thunderstorm} & \texttt{\textbackslash mSymbol\{thunderstorm\}} & \texttt{EBDB}\\
\mSymbol[outlined]{tibia} & \mSymbol[rounded]{tibia} & \mSymbol[sharp]{tibia} & \texttt{\textbackslash mSymbol\{tibia\}} & \texttt{F89B}\\
\mSymbol[outlined]{tibia-alt} & \mSymbol[rounded]{tibia-alt} & \mSymbol[sharp]{tibia-alt} & \texttt{\textbackslash mSymbol\{tibia-alt\}} & \texttt{F89C}\\
\mSymbol[outlined]{time-auto} & \mSymbol[rounded]{time-auto} & \mSymbol[sharp]{time-auto} & \texttt{\textbackslash mSymbol\{time-auto\}} & \texttt{F0E4}\\
\mSymbol[outlined]{time-to-leave} & \mSymbol[rounded]{time-to-leave} & \mSymbol[sharp]{time-to-leave} & \texttt{\textbackslash mSymbol\{time-to-leave\}} & \texttt{EFF7}\\
\mSymbol[outlined]{timelapse} & \mSymbol[rounded]{timelapse} & \mSymbol[sharp]{timelapse} & \texttt{\textbackslash mSymbol\{timelapse\}} & \texttt{E422}\\
\mSymbol[outlined]{timeline} & \mSymbol[rounded]{timeline} & \mSymbol[sharp]{timeline} & \texttt{\textbackslash mSymbol\{timeline\}} & \texttt{E922}\\
\mSymbol[outlined]{timer} & \mSymbol[rounded]{timer} & \mSymbol[sharp]{timer} & \texttt{\textbackslash mSymbol\{timer\}} & \texttt{E425}\\
\mSymbol[outlined]{timer-10} & \mSymbol[rounded]{timer-10} & \mSymbol[sharp]{timer-10} & \texttt{\textbackslash mSymbol\{timer-10\}} & \texttt{E423}\\
\mSymbol[outlined]{timer-10-alt-1} & \mSymbol[rounded]{timer-10-alt-1} & \mSymbol[sharp]{timer-10-alt-1} & \texttt{\textbackslash mSymbol\{timer-10-alt-1\}} & \texttt{EFBF}\\
\mSymbol[outlined]{timer-10-select} & \mSymbol[rounded]{timer-10-select} & \mSymbol[sharp]{timer-10-select} & \texttt{\textbackslash mSymbol\{timer-10-select\}} & \texttt{F07A}\\
\mSymbol[outlined]{timer-3} & \mSymbol[rounded]{timer-3} & \mSymbol[sharp]{timer-3} & \texttt{\textbackslash mSymbol\{timer-3\}} & \texttt{E424}\\
\mSymbol[outlined]{timer-3-alt-1} & \mSymbol[rounded]{timer-3-alt-1} & \mSymbol[sharp]{timer-3-alt-1} & \texttt{\textbackslash mSymbol\{timer-3-alt-1\}} & \texttt{EFC0}\\
\mSymbol[outlined]{timer-3-select} & \mSymbol[rounded]{timer-3-select} & \mSymbol[sharp]{timer-3-select} & \texttt{\textbackslash mSymbol\{timer-3-select\}} & \texttt{F07B}\\
\mSymbol[outlined]{timer-5} & \mSymbol[rounded]{timer-5} & \mSymbol[sharp]{timer-5} & \texttt{\textbackslash mSymbol\{timer-5\}} & \texttt{F4B1}\\
\mSymbol[outlined]{timer-5-shutter} & \mSymbol[rounded]{timer-5-shutter} & \mSymbol[sharp]{timer-5-shutter} & \texttt{\textbackslash mSymbol\{timer-5-shutter\}} & \texttt{F4B2}\\
\mSymbol[outlined]{timer-off} & \mSymbol[rounded]{timer-off} & \mSymbol[sharp]{timer-off} & \texttt{\textbackslash mSymbol\{timer-off\}} & \texttt{E426}\\
\mSymbol[outlined]{timer-pause} & \mSymbol[rounded]{timer-pause} & \mSymbol[sharp]{timer-pause} & \texttt{\textbackslash mSymbol\{timer-pause\}} & \texttt{F4BB}\\
\mSymbol[outlined]{timer-play} & \mSymbol[rounded]{timer-play} & \mSymbol[sharp]{timer-play} & \texttt{\textbackslash mSymbol\{timer-play\}} & \texttt{F4BA}\\
\mSymbol[outlined]{tips-and-updates} & \mSymbol[rounded]{tips-and-updates} & \mSymbol[sharp]{tips-and-updates} & \texttt{\textbackslash mSymbol\{tips-and-updates\}} & \texttt{E79A}\\
\mSymbol[outlined]{tire-repair} & \mSymbol[rounded]{tire-repair} & \mSymbol[sharp]{tire-repair} & \texttt{\textbackslash mSymbol\{tire-repair\}} & \texttt{EBC8}\\
\mSymbol[outlined]{title} & \mSymbol[rounded]{title} & \mSymbol[sharp]{title} & \texttt{\textbackslash mSymbol\{title\}} & \texttt{E264}\\
\mSymbol[outlined]{titlecase} & \mSymbol[rounded]{titlecase} & \mSymbol[sharp]{titlecase} & \texttt{\textbackslash mSymbol\{titlecase\}} & \texttt{F489}\\
\mSymbol[outlined]{toast} & \mSymbol[rounded]{toast} & \mSymbol[sharp]{toast} & \texttt{\textbackslash mSymbol\{toast\}} & \texttt{EFC1}\\
\mSymbol[outlined]{toc} & \mSymbol[rounded]{toc} & \mSymbol[sharp]{toc} & \texttt{\textbackslash mSymbol\{toc\}} & \texttt{E8DE}\\
\mSymbol[outlined]{today} & \mSymbol[rounded]{today} & \mSymbol[sharp]{today} & \texttt{\textbackslash mSymbol\{today\}} & \texttt{E8DF}\\
\mSymbol[outlined]{toggle-off} & \mSymbol[rounded]{toggle-off} & \mSymbol[sharp]{toggle-off} & \texttt{\textbackslash mSymbol\{toggle-off\}} & \texttt{E9F5}\\
\mSymbol[outlined]{toggle-on} & \mSymbol[rounded]{toggle-on} & \mSymbol[sharp]{toggle-on} & \texttt{\textbackslash mSymbol\{toggle-on\}} & \texttt{E9F6}\\
\mSymbol[outlined]{token} & \mSymbol[rounded]{token} & \mSymbol[sharp]{token} & \texttt{\textbackslash mSymbol\{token\}} & \texttt{EA25}\\
\mSymbol[outlined]{toll} & \mSymbol[rounded]{toll} & \mSymbol[sharp]{toll} & \texttt{\textbackslash mSymbol\{toll\}} & \texttt{E8E0}\\
\mSymbol[outlined]{tonality} & \mSymbol[rounded]{tonality} & \mSymbol[sharp]{tonality} & \texttt{\textbackslash mSymbol\{tonality\}} & \texttt{E427}\\
\mSymbol[outlined]{toolbar} & \mSymbol[rounded]{toolbar} & \mSymbol[sharp]{toolbar} & \texttt{\textbackslash mSymbol\{toolbar\}} & \texttt{E9F7}\\
\mSymbol[outlined]{tools-flat-head} & \mSymbol[rounded]{tools-flat-head} & \mSymbol[sharp]{tools-flat-head} & \texttt{\textbackslash mSymbol\{tools-flat-head\}} & \texttt{F8CB}\\
\mSymbol[outlined]{tools-installation-kit} & \mSymbol[rounded]{tools-installation-kit} & \mSymbol[sharp]{tools-installation-kit} & \texttt{\textbackslash mSymbol\{tools-installation-kit\}} & \texttt{E2AB}\\
\mSymbol[outlined]{tools-ladder} & \mSymbol[rounded]{tools-ladder} & \mSymbol[sharp]{tools-ladder} & \texttt{\textbackslash mSymbol\{tools-ladder\}} & \texttt{E2CB}\\
\mSymbol[outlined]{tools-level} & \mSymbol[rounded]{tools-level} & \mSymbol[sharp]{tools-level} & \texttt{\textbackslash mSymbol\{tools-level\}} & \texttt{E77B}\\
\mSymbol[outlined]{tools-phillips} & \mSymbol[rounded]{tools-phillips} & \mSymbol[sharp]{tools-phillips} & \texttt{\textbackslash mSymbol\{tools-phillips\}} & \texttt{F8CC}\\
\mSymbol[outlined]{tools-pliers-wire-stripper} & \mSymbol[rounded]{tools-pliers-wire-stripper} & \mSymbol[sharp]{tools-pliers-wire-stripper} & \texttt{\textbackslash mSymbol\{tools-pliers-wire-stripper\}} & \texttt{E2AA}\\
\mSymbol[outlined]{tools-power-drill} & \mSymbol[rounded]{tools-power-drill} & \mSymbol[sharp]{tools-power-drill} & \texttt{\textbackslash mSymbol\{tools-power-drill\}} & \texttt{E1E9}\\
\mSymbol[outlined]{tools-wrench} & \mSymbol[rounded]{tools-wrench} & \mSymbol[sharp]{tools-wrench} & \texttt{\textbackslash mSymbol\{tools-wrench\}} & \texttt{F8CD}\\
\mSymbol[outlined]{tooltip} & \mSymbol[rounded]{tooltip} & \mSymbol[sharp]{tooltip} & \texttt{\textbackslash mSymbol\{tooltip\}} & \texttt{E9F8}\\
\mSymbol[outlined]{top-panel-close} & \mSymbol[rounded]{top-panel-close} & \mSymbol[sharp]{top-panel-close} & \texttt{\textbackslash mSymbol\{top-panel-close\}} & \texttt{F733}\\
\mSymbol[outlined]{top-panel-open} & \mSymbol[rounded]{top-panel-open} & \mSymbol[sharp]{top-panel-open} & \texttt{\textbackslash mSymbol\{top-panel-open\}} & \texttt{F732}\\
\mSymbol[outlined]{topic} & \mSymbol[rounded]{topic} & \mSymbol[sharp]{topic} & \texttt{\textbackslash mSymbol\{topic\}} & \texttt{F1C8}\\
\mSymbol[outlined]{tornado} & \mSymbol[rounded]{tornado} & \mSymbol[sharp]{tornado} & \texttt{\textbackslash mSymbol\{tornado\}} & \texttt{E199}\\
\mSymbol[outlined]{total-dissolved-solids} & \mSymbol[rounded]{total-dissolved-solids} & \mSymbol[sharp]{total-dissolved-solids} & \texttt{\textbackslash mSymbol\{total-dissolved-solids\}} & \texttt{F877}\\
\mSymbol[outlined]{touch-app} & \mSymbol[rounded]{touch-app} & \mSymbol[sharp]{touch-app} & \texttt{\textbackslash mSymbol\{touch-app\}} & \texttt{E913}\\
\mSymbol[outlined]{touchpad-mouse} & \mSymbol[rounded]{touchpad-mouse} & \mSymbol[sharp]{touchpad-mouse} & \texttt{\textbackslash mSymbol\{touchpad-mouse\}} & \texttt{F687}\\
\mSymbol[outlined]{touchpad-mouse-off} & \mSymbol[rounded]{touchpad-mouse-off} & \mSymbol[sharp]{touchpad-mouse-off} & \texttt{\textbackslash mSymbol\{touchpad-mouse-off\}} & \texttt{F4E6}\\
\mSymbol[outlined]{tour} & \mSymbol[rounded]{tour} & \mSymbol[sharp]{tour} & \texttt{\textbackslash mSymbol\{tour\}} & \texttt{EF75}\\
\mSymbol[outlined]{toys} & \mSymbol[rounded]{toys} & \mSymbol[sharp]{toys} & \texttt{\textbackslash mSymbol\{toys\}} & \texttt{E332}\\
\mSymbol[outlined]{toys-and-games} & \mSymbol[rounded]{toys-and-games} & \mSymbol[sharp]{toys-and-games} & \texttt{\textbackslash mSymbol\{toys-and-games\}} & \texttt{EFC2}\\
\mSymbol[outlined]{toys-fan} & \mSymbol[rounded]{toys-fan} & \mSymbol[sharp]{toys-fan} & \texttt{\textbackslash mSymbol\{toys-fan\}} & \texttt{F887}\\
\mSymbol[outlined]{track-changes} & \mSymbol[rounded]{track-changes} & \mSymbol[sharp]{track-changes} & \texttt{\textbackslash mSymbol\{track-changes\}} & \texttt{E8E1}\\
\mSymbol[outlined]{trackpad-input} & \mSymbol[rounded]{trackpad-input} & \mSymbol[sharp]{trackpad-input} & \texttt{\textbackslash mSymbol\{trackpad-input\}} & \texttt{F4C7}\\
\mSymbol[outlined]{traffic} & \mSymbol[rounded]{traffic} & \mSymbol[sharp]{traffic} & \texttt{\textbackslash mSymbol\{traffic\}} & \texttt{E565}\\
\mSymbol[outlined]{traffic-jam} & \mSymbol[rounded]{traffic-jam} & \mSymbol[sharp]{traffic-jam} & \texttt{\textbackslash mSymbol\{traffic-jam\}} & \texttt{F46F}\\
\mSymbol[outlined]{trail-length} & \mSymbol[rounded]{trail-length} & \mSymbol[sharp]{trail-length} & \texttt{\textbackslash mSymbol\{trail-length\}} & \texttt{EB5E}\\
\mSymbol[outlined]{trail-length-medium} & \mSymbol[rounded]{trail-length-medium} & \mSymbol[sharp]{trail-length-medium} & \texttt{\textbackslash mSymbol\{trail-length-medium\}} & \texttt{EB63}\\
\mSymbol[outlined]{trail-length-short} & \mSymbol[rounded]{trail-length-short} & \mSymbol[sharp]{trail-length-short} & \texttt{\textbackslash mSymbol\{trail-length-short\}} & \texttt{EB6D}\\
\mSymbol[outlined]{train} & \mSymbol[rounded]{train} & \mSymbol[sharp]{train} & \texttt{\textbackslash mSymbol\{train\}} & \texttt{E570}\\
\mSymbol[outlined]{tram} & \mSymbol[rounded]{tram} & \mSymbol[sharp]{tram} & \texttt{\textbackslash mSymbol\{tram\}} & \texttt{E571}\\
\mSymbol[outlined]{transcribe} & \mSymbol[rounded]{transcribe} & \mSymbol[sharp]{transcribe} & \texttt{\textbackslash mSymbol\{transcribe\}} & \texttt{F8EC}\\
\mSymbol[outlined]{transfer-within-a-station} & \mSymbol[rounded]{transfer-within-a-station} & \mSymbol[sharp]{transfer-within-a-station} & \texttt{\textbackslash mSymbol\{transfer-within-a-station\}} & \texttt{E572}\\
\mSymbol[outlined]{transform} & \mSymbol[rounded]{transform} & \mSymbol[sharp]{transform} & \texttt{\textbackslash mSymbol\{transform\}} & \texttt{E428}\\
\mSymbol[outlined]{transgender} & \mSymbol[rounded]{transgender} & \mSymbol[sharp]{transgender} & \texttt{\textbackslash mSymbol\{transgender\}} & \texttt{E58D}\\
\mSymbol[outlined]{transit-enterexit} & \mSymbol[rounded]{transit-enterexit} & \mSymbol[sharp]{transit-enterexit} & \texttt{\textbackslash mSymbol\{transit-enterexit\}} & \texttt{E579}\\
\mSymbol[outlined]{transition-chop} & \mSymbol[rounded]{transition-chop} & \mSymbol[sharp]{transition-chop} & \texttt{\textbackslash mSymbol\{transition-chop\}} & \texttt{F50E}\\
\mSymbol[outlined]{transition-dissolve} & \mSymbol[rounded]{transition-dissolve} & \mSymbol[sharp]{transition-dissolve} & \texttt{\textbackslash mSymbol\{transition-dissolve\}} & \texttt{F50D}\\
\mSymbol[outlined]{transition-fade} & \mSymbol[rounded]{transition-fade} & \mSymbol[sharp]{transition-fade} & \texttt{\textbackslash mSymbol\{transition-fade\}} & \texttt{F50C}\\
\mSymbol[outlined]{transition-push} & \mSymbol[rounded]{transition-push} & \mSymbol[sharp]{transition-push} & \texttt{\textbackslash mSymbol\{transition-push\}} & \texttt{F50B}\\
\mSymbol[outlined]{transition-slide} & \mSymbol[rounded]{transition-slide} & \mSymbol[sharp]{transition-slide} & \texttt{\textbackslash mSymbol\{transition-slide\}} & \texttt{F50A}\\
\mSymbol[outlined]{translate} & \mSymbol[rounded]{translate} & \mSymbol[sharp]{translate} & \texttt{\textbackslash mSymbol\{translate\}} & \texttt{E8E2}\\
\mSymbol[outlined]{transportation} & \mSymbol[rounded]{transportation} & \mSymbol[sharp]{transportation} & \texttt{\textbackslash mSymbol\{transportation\}} & \texttt{E21D}\\
\mSymbol[outlined]{travel} & \mSymbol[rounded]{travel} & \mSymbol[sharp]{travel} & \texttt{\textbackslash mSymbol\{travel\}} & \texttt{EF93}\\
\mSymbol[outlined]{travel-explore} & \mSymbol[rounded]{travel-explore} & \mSymbol[sharp]{travel-explore} & \texttt{\textbackslash mSymbol\{travel-explore\}} & \texttt{E2DB}\\
\mSymbol[outlined]{travel-luggage-and-bags} & \mSymbol[rounded]{travel-luggage-and-bags} & \mSymbol[sharp]{travel-luggage-and-bags} & \texttt{\textbackslash mSymbol\{travel-luggage-and-bags\}} & \texttt{EFC3}\\
\mSymbol[outlined]{trending-down} & \mSymbol[rounded]{trending-down} & \mSymbol[sharp]{trending-down} & \texttt{\textbackslash mSymbol\{trending-down\}} & \texttt{E8E3}\\
\mSymbol[outlined]{trending-flat} & \mSymbol[rounded]{trending-flat} & \mSymbol[sharp]{trending-flat} & \texttt{\textbackslash mSymbol\{trending-flat\}} & \texttt{E8E4}\\
\mSymbol[outlined]{trending-up} & \mSymbol[rounded]{trending-up} & \mSymbol[sharp]{trending-up} & \texttt{\textbackslash mSymbol\{trending-up\}} & \texttt{E8E5}\\
\mSymbol[outlined]{trip} & \mSymbol[rounded]{trip} & \mSymbol[sharp]{trip} & \texttt{\textbackslash mSymbol\{trip\}} & \texttt{E6FB}\\
\mSymbol[outlined]{trip-origin} & \mSymbol[rounded]{trip-origin} & \mSymbol[sharp]{trip-origin} & \texttt{\textbackslash mSymbol\{trip-origin\}} & \texttt{E57B}\\
\mSymbol[outlined]{trolley} & \mSymbol[rounded]{trolley} & \mSymbol[sharp]{trolley} & \texttt{\textbackslash mSymbol\{trolley\}} & \texttt{F86B}\\
\mSymbol[outlined]{trolley-cable-car} & \mSymbol[rounded]{trolley-cable-car} & \mSymbol[sharp]{trolley-cable-car} & \texttt{\textbackslash mSymbol\{trolley-cable-car\}} & \texttt{F46E}\\
\mSymbol[outlined]{trophy} & \mSymbol[rounded]{trophy} & \mSymbol[sharp]{trophy} & \texttt{\textbackslash mSymbol\{trophy\}} & \texttt{EA23}\\
\mSymbol[outlined]{troubleshoot} & \mSymbol[rounded]{troubleshoot} & \mSymbol[sharp]{troubleshoot} & \texttt{\textbackslash mSymbol\{troubleshoot\}} & \texttt{E1D2}\\
\mSymbol[outlined]{try} & \mSymbol[rounded]{try} & \mSymbol[sharp]{try} & \texttt{\textbackslash mSymbol\{try\}} & \texttt{F07C}\\
\mSymbol[outlined]{tsunami} & \mSymbol[rounded]{tsunami} & \mSymbol[sharp]{tsunami} & \texttt{\textbackslash mSymbol\{tsunami\}} & \texttt{EBD8}\\
\mSymbol[outlined]{tsv} & \mSymbol[rounded]{tsv} & \mSymbol[sharp]{tsv} & \texttt{\textbackslash mSymbol\{tsv\}} & \texttt{E6D6}\\
\mSymbol[outlined]{tty} & \mSymbol[rounded]{tty} & \mSymbol[sharp]{tty} & \texttt{\textbackslash mSymbol\{tty\}} & \texttt{F1AA}\\
\mSymbol[outlined]{tune} & \mSymbol[rounded]{tune} & \mSymbol[sharp]{tune} & \texttt{\textbackslash mSymbol\{tune\}} & \texttt{E429}\\
\mSymbol[outlined]{tungsten} & \mSymbol[rounded]{tungsten} & \mSymbol[sharp]{tungsten} & \texttt{\textbackslash mSymbol\{tungsten\}} & \texttt{F07D}\\
\mSymbol[outlined]{turn-left} & \mSymbol[rounded]{turn-left} & \mSymbol[sharp]{turn-left} & \texttt{\textbackslash mSymbol\{turn-left\}} & \texttt{EBA6}\\
\mSymbol[outlined]{turn-right} & \mSymbol[rounded]{turn-right} & \mSymbol[sharp]{turn-right} & \texttt{\textbackslash mSymbol\{turn-right\}} & \texttt{EBAB}\\
\mSymbol[outlined]{turn-sharp-left} & \mSymbol[rounded]{turn-sharp-left} & \mSymbol[sharp]{turn-sharp-left} & \texttt{\textbackslash mSymbol\{turn-sharp-left\}} & \texttt{EBA7}\\
\mSymbol[outlined]{turn-sharp-right} & \mSymbol[rounded]{turn-sharp-right} & \mSymbol[sharp]{turn-sharp-right} & \texttt{\textbackslash mSymbol\{turn-sharp-right\}} & \texttt{EBAA}\\
\mSymbol[outlined]{turn-slight-left} & \mSymbol[rounded]{turn-slight-left} & \mSymbol[sharp]{turn-slight-left} & \texttt{\textbackslash mSymbol\{turn-slight-left\}} & \texttt{EBA4}\\
\mSymbol[outlined]{turn-slight-right} & \mSymbol[rounded]{turn-slight-right} & \mSymbol[sharp]{turn-slight-right} & \texttt{\textbackslash mSymbol\{turn-slight-right\}} & \texttt{EB9A}\\
\mSymbol[outlined]{turned-in} & \mSymbol[rounded]{turned-in} & \mSymbol[sharp]{turned-in} & \texttt{\textbackslash mSymbol\{turned-in\}} & \texttt{E8E7}\\
\mSymbol[outlined]{turned-in-not} & \mSymbol[rounded]{turned-in-not} & \mSymbol[sharp]{turned-in-not} & \texttt{\textbackslash mSymbol\{turned-in-not\}} & \texttt{E8E7}\\
\mSymbol[outlined]{tv} & \mSymbol[rounded]{tv} & \mSymbol[sharp]{tv} & \texttt{\textbackslash mSymbol\{tv\}} & \texttt{E63B}\\
\mSymbol[outlined]{tv-gen} & \mSymbol[rounded]{tv-gen} & \mSymbol[sharp]{tv-gen} & \texttt{\textbackslash mSymbol\{tv-gen\}} & \texttt{E830}\\
\mSymbol[outlined]{tv-guide} & \mSymbol[rounded]{tv-guide} & \mSymbol[sharp]{tv-guide} & \texttt{\textbackslash mSymbol\{tv-guide\}} & \texttt{E1DC}\\
\mSymbol[outlined]{tv-off} & \mSymbol[rounded]{tv-off} & \mSymbol[sharp]{tv-off} & \texttt{\textbackslash mSymbol\{tv-off\}} & \texttt{E647}\\
\mSymbol[outlined]{tv-options-edit-channels} & \mSymbol[rounded]{tv-options-edit-channels} & \mSymbol[sharp]{tv-options-edit-channels} & \texttt{\textbackslash mSymbol\{tv-options-edit-channels\}} & \texttt{E1DD}\\
\mSymbol[outlined]{tv-options-input-settings} & \mSymbol[rounded]{tv-options-input-settings} & \mSymbol[sharp]{tv-options-input-settings} & \texttt{\textbackslash mSymbol\{tv-options-input-settings\}} & \texttt{E1DE}\\
\mSymbol[outlined]{tv-remote} & \mSymbol[rounded]{tv-remote} & \mSymbol[sharp]{tv-remote} & \texttt{\textbackslash mSymbol\{tv-remote\}} & \texttt{F5D9}\\
\mSymbol[outlined]{tv-signin} & \mSymbol[rounded]{tv-signin} & \mSymbol[sharp]{tv-signin} & \texttt{\textbackslash mSymbol\{tv-signin\}} & \texttt{E71B}\\
\mSymbol[outlined]{tv-with-assistant} & \mSymbol[rounded]{tv-with-assistant} & \mSymbol[sharp]{tv-with-assistant} & \texttt{\textbackslash mSymbol\{tv-with-assistant\}} & \texttt{E785}\\
\mSymbol[outlined]{two-pager} & \mSymbol[rounded]{two-pager} & \mSymbol[sharp]{two-pager} & \texttt{\textbackslash mSymbol\{two-pager\}} & \texttt{F51F}\\
\mSymbol[outlined]{two-wheeler} & \mSymbol[rounded]{two-wheeler} & \mSymbol[sharp]{two-wheeler} & \texttt{\textbackslash mSymbol\{two-wheeler\}} & \texttt{E9F9}\\
\mSymbol[outlined]{type-specimen} & \mSymbol[rounded]{type-specimen} & \mSymbol[sharp]{type-specimen} & \texttt{\textbackslash mSymbol\{type-specimen\}} & \texttt{F8F0}\\
\mSymbol[outlined]{u-turn-left} & \mSymbol[rounded]{u-turn-left} & \mSymbol[sharp]{u-turn-left} & \texttt{\textbackslash mSymbol\{u-turn-left\}} & \texttt{EBA1}\\
\mSymbol[outlined]{u-turn-right} & \mSymbol[rounded]{u-turn-right} & \mSymbol[sharp]{u-turn-right} & \texttt{\textbackslash mSymbol\{u-turn-right\}} & \texttt{EBA2}\\
\mSymbol[outlined]{ulna-radius} & \mSymbol[rounded]{ulna-radius} & \mSymbol[sharp]{ulna-radius} & \texttt{\textbackslash mSymbol\{ulna-radius\}} & \texttt{F89D}\\
\mSymbol[outlined]{ulna-radius-alt} & \mSymbol[rounded]{ulna-radius-alt} & \mSymbol[sharp]{ulna-radius-alt} & \texttt{\textbackslash mSymbol\{ulna-radius-alt\}} & \texttt{F89E}\\
\mSymbol[outlined]{umbrella} & \mSymbol[rounded]{umbrella} & \mSymbol[sharp]{umbrella} & \texttt{\textbackslash mSymbol\{umbrella\}} & \texttt{F1AD}\\
\mSymbol[outlined]{unarchive} & \mSymbol[rounded]{unarchive} & \mSymbol[sharp]{unarchive} & \texttt{\textbackslash mSymbol\{unarchive\}} & \texttt{E169}\\
\mSymbol[outlined]{undo} & \mSymbol[rounded]{undo} & \mSymbol[sharp]{undo} & \texttt{\textbackslash mSymbol\{undo\}} & \texttt{E166}\\
\mSymbol[outlined]{unfold-less} & \mSymbol[rounded]{unfold-less} & \mSymbol[sharp]{unfold-less} & \texttt{\textbackslash mSymbol\{unfold-less\}} & \texttt{E5D6}\\
\mSymbol[outlined]{unfold-less-double} & \mSymbol[rounded]{unfold-less-double} & \mSymbol[sharp]{unfold-less-double} & \texttt{\textbackslash mSymbol\{unfold-less-double\}} & \texttt{F8CF}\\
\mSymbol[outlined]{unfold-more} & \mSymbol[rounded]{unfold-more} & \mSymbol[sharp]{unfold-more} & \texttt{\textbackslash mSymbol\{unfold-more\}} & \texttt{E5D7}\\
\mSymbol[outlined]{unfold-more-double} & \mSymbol[rounded]{unfold-more-double} & \mSymbol[sharp]{unfold-more-double} & \texttt{\textbackslash mSymbol\{unfold-more-double\}} & \texttt{F8D0}\\
\mSymbol[outlined]{ungroup} & \mSymbol[rounded]{ungroup} & \mSymbol[sharp]{ungroup} & \texttt{\textbackslash mSymbol\{ungroup\}} & \texttt{F731}\\
\mSymbol[outlined]{universal-currency} & \mSymbol[rounded]{universal-currency} & \mSymbol[sharp]{universal-currency} & \texttt{\textbackslash mSymbol\{universal-currency\}} & \texttt{E9FA}\\
\mSymbol[outlined]{universal-currency-alt} & \mSymbol[rounded]{universal-currency-alt} & \mSymbol[sharp]{universal-currency-alt} & \texttt{\textbackslash mSymbol\{universal-currency-alt\}} & \texttt{E734}\\
\mSymbol[outlined]{universal-local} & \mSymbol[rounded]{universal-local} & \mSymbol[sharp]{universal-local} & \texttt{\textbackslash mSymbol\{universal-local\}} & \texttt{E9FB}\\
\mSymbol[outlined]{unknown-2} & \mSymbol[rounded]{unknown-2} & \mSymbol[sharp]{unknown-2} & \texttt{\textbackslash mSymbol\{unknown-2\}} & \texttt{F49F}\\
\mSymbol[outlined]{unknown-5} & \mSymbol[rounded]{unknown-5} & \mSymbol[sharp]{unknown-5} & \texttt{\textbackslash mSymbol\{unknown-5\}} & \texttt{E6A5}\\
\mSymbol[outlined]{unknown-7} & \mSymbol[rounded]{unknown-7} & \mSymbol[sharp]{unknown-7} & \texttt{\textbackslash mSymbol\{unknown-7\}} & \texttt{F49E}\\
\mSymbol[outlined]{unknown-document} & \mSymbol[rounded]{unknown-document} & \mSymbol[sharp]{unknown-document} & \texttt{\textbackslash mSymbol\{unknown-document\}} & \texttt{F804}\\
\mSymbol[outlined]{unknown-med} & \mSymbol[rounded]{unknown-med} & \mSymbol[sharp]{unknown-med} & \texttt{\textbackslash mSymbol\{unknown-med\}} & \texttt{EABD}\\
\mSymbol[outlined]{unlicense} & \mSymbol[rounded]{unlicense} & \mSymbol[sharp]{unlicense} & \texttt{\textbackslash mSymbol\{unlicense\}} & \texttt{EB05}\\
\mSymbol[outlined]{unpaved-road} & \mSymbol[rounded]{unpaved-road} & \mSymbol[sharp]{unpaved-road} & \texttt{\textbackslash mSymbol\{unpaved-road\}} & \texttt{F46D}\\
\mSymbol[outlined]{unpin} & \mSymbol[rounded]{unpin} & \mSymbol[sharp]{unpin} & \texttt{\textbackslash mSymbol\{unpin\}} & \texttt{E6F9}\\
\mSymbol[outlined]{unpublished} & \mSymbol[rounded]{unpublished} & \mSymbol[sharp]{unpublished} & \texttt{\textbackslash mSymbol\{unpublished\}} & \texttt{F236}\\
\mSymbol[outlined]{unsubscribe} & \mSymbol[rounded]{unsubscribe} & \mSymbol[sharp]{unsubscribe} & \texttt{\textbackslash mSymbol\{unsubscribe\}} & \texttt{E0EB}\\
\mSymbol[outlined]{upcoming} & \mSymbol[rounded]{upcoming} & \mSymbol[sharp]{upcoming} & \texttt{\textbackslash mSymbol\{upcoming\}} & \texttt{F07E}\\
\mSymbol[outlined]{update} & \mSymbol[rounded]{update} & \mSymbol[sharp]{update} & \texttt{\textbackslash mSymbol\{update\}} & \texttt{E923}\\
\mSymbol[outlined]{update-disabled} & \mSymbol[rounded]{update-disabled} & \mSymbol[sharp]{update-disabled} & \texttt{\textbackslash mSymbol\{update-disabled\}} & \texttt{E075}\\
\mSymbol[outlined]{upgrade} & \mSymbol[rounded]{upgrade} & \mSymbol[sharp]{upgrade} & \texttt{\textbackslash mSymbol\{upgrade\}} & \texttt{F0FB}\\
\mSymbol[outlined]{upload} & \mSymbol[rounded]{upload} & \mSymbol[sharp]{upload} & \texttt{\textbackslash mSymbol\{upload\}} & \texttt{F09B}\\
\mSymbol[outlined]{upload-2} & \mSymbol[rounded]{upload-2} & \mSymbol[sharp]{upload-2} & \texttt{\textbackslash mSymbol\{upload-2\}} & \texttt{F521}\\
\mSymbol[outlined]{upload-file} & \mSymbol[rounded]{upload-file} & \mSymbol[sharp]{upload-file} & \texttt{\textbackslash mSymbol\{upload-file\}} & \texttt{E9FC}\\
\mSymbol[outlined]{uppercase} & \mSymbol[rounded]{uppercase} & \mSymbol[sharp]{uppercase} & \texttt{\textbackslash mSymbol\{uppercase\}} & \texttt{F488}\\
\mSymbol[outlined]{urology} & \mSymbol[rounded]{urology} & \mSymbol[sharp]{urology} & \texttt{\textbackslash mSymbol\{urology\}} & \texttt{E137}\\
\mSymbol[outlined]{usb} & \mSymbol[rounded]{usb} & \mSymbol[sharp]{usb} & \texttt{\textbackslash mSymbol\{usb\}} & \texttt{E1E0}\\
\mSymbol[outlined]{usb-off} & \mSymbol[rounded]{usb-off} & \mSymbol[sharp]{usb-off} & \texttt{\textbackslash mSymbol\{usb-off\}} & \texttt{E4FA}\\
\mSymbol[outlined]{user-attributes} & \mSymbol[rounded]{user-attributes} & \mSymbol[sharp]{user-attributes} & \texttt{\textbackslash mSymbol\{user-attributes\}} & \texttt{E708}\\
\mSymbol[outlined]{vaccines} & \mSymbol[rounded]{vaccines} & \mSymbol[sharp]{vaccines} & \texttt{\textbackslash mSymbol\{vaccines\}} & \texttt{E138}\\
\mSymbol[outlined]{vacuum} & \mSymbol[rounded]{vacuum} & \mSymbol[sharp]{vacuum} & \texttt{\textbackslash mSymbol\{vacuum\}} & \texttt{EFC5}\\
\mSymbol[outlined]{valve} & \mSymbol[rounded]{valve} & \mSymbol[sharp]{valve} & \texttt{\textbackslash mSymbol\{valve\}} & \texttt{E224}\\
\mSymbol[outlined]{vape-free} & \mSymbol[rounded]{vape-free} & \mSymbol[sharp]{vape-free} & \texttt{\textbackslash mSymbol\{vape-free\}} & \texttt{EBC6}\\
\mSymbol[outlined]{vaping-rooms} & \mSymbol[rounded]{vaping-rooms} & \mSymbol[sharp]{vaping-rooms} & \texttt{\textbackslash mSymbol\{vaping-rooms\}} & \texttt{EBCF}\\
\mSymbol[outlined]{variable-add} & \mSymbol[rounded]{variable-add} & \mSymbol[sharp]{variable-add} & \texttt{\textbackslash mSymbol\{variable-add\}} & \texttt{F51E}\\
\mSymbol[outlined]{variable-insert} & \mSymbol[rounded]{variable-insert} & \mSymbol[sharp]{variable-insert} & \texttt{\textbackslash mSymbol\{variable-insert\}} & \texttt{F51D}\\
\mSymbol[outlined]{variable-remove} & \mSymbol[rounded]{variable-remove} & \mSymbol[sharp]{variable-remove} & \texttt{\textbackslash mSymbol\{variable-remove\}} & \texttt{F51C}\\
\mSymbol[outlined]{variables} & \mSymbol[rounded]{variables} & \mSymbol[sharp]{variables} & \texttt{\textbackslash mSymbol\{variables\}} & \texttt{F851}\\
\mSymbol[outlined]{ventilator} & \mSymbol[rounded]{ventilator} & \mSymbol[sharp]{ventilator} & \texttt{\textbackslash mSymbol\{ventilator\}} & \texttt{E139}\\
\mSymbol[outlined]{verified} & \mSymbol[rounded]{verified} & \mSymbol[sharp]{verified} & \texttt{\textbackslash mSymbol\{verified\}} & \texttt{EF76}\\
\mSymbol[outlined]{verified-user} & \mSymbol[rounded]{verified-user} & \mSymbol[sharp]{verified-user} & \texttt{\textbackslash mSymbol\{verified-user\}} & \texttt{F013}\\
\mSymbol[outlined]{vertical-align-bottom} & \mSymbol[rounded]{vertical-align-bottom} & \mSymbol[sharp]{vertical-align-bottom} & \texttt{\textbackslash mSymbol\{vertical-align-bottom\}} & \texttt{E258}\\
\mSymbol[outlined]{vertical-align-center} & \mSymbol[rounded]{vertical-align-center} & \mSymbol[sharp]{vertical-align-center} & \texttt{\textbackslash mSymbol\{vertical-align-center\}} & \texttt{E259}\\
\mSymbol[outlined]{vertical-align-top} & \mSymbol[rounded]{vertical-align-top} & \mSymbol[sharp]{vertical-align-top} & \texttt{\textbackslash mSymbol\{vertical-align-top\}} & \texttt{E25A}\\
\mSymbol[outlined]{vertical-distribute} & \mSymbol[rounded]{vertical-distribute} & \mSymbol[sharp]{vertical-distribute} & \texttt{\textbackslash mSymbol\{vertical-distribute\}} & \texttt{E076}\\
\mSymbol[outlined]{vertical-shades} & \mSymbol[rounded]{vertical-shades} & \mSymbol[sharp]{vertical-shades} & \texttt{\textbackslash mSymbol\{vertical-shades\}} & \texttt{EC0E}\\
\mSymbol[outlined]{vertical-shades-closed} & \mSymbol[rounded]{vertical-shades-closed} & \mSymbol[sharp]{vertical-shades-closed} & \texttt{\textbackslash mSymbol\{vertical-shades-closed\}} & \texttt{EC0D}\\
\mSymbol[outlined]{vertical-split} & \mSymbol[rounded]{vertical-split} & \mSymbol[sharp]{vertical-split} & \texttt{\textbackslash mSymbol\{vertical-split\}} & \texttt{E949}\\
\mSymbol[outlined]{vibration} & \mSymbol[rounded]{vibration} & \mSymbol[sharp]{vibration} & \texttt{\textbackslash mSymbol\{vibration\}} & \texttt{E62D}\\
\mSymbol[outlined]{video-call} & \mSymbol[rounded]{video-call} & \mSymbol[sharp]{video-call} & \texttt{\textbackslash mSymbol\{video-call\}} & \texttt{E070}\\
\mSymbol[outlined]{video-camera-back} & \mSymbol[rounded]{video-camera-back} & \mSymbol[sharp]{video-camera-back} & \texttt{\textbackslash mSymbol\{video-camera-back\}} & \texttt{F07F}\\
\mSymbol[outlined]{video-camera-back-add} & \mSymbol[rounded]{video-camera-back-add} & \mSymbol[sharp]{video-camera-back-add} & \texttt{\textbackslash mSymbol\{video-camera-back-add\}} & \texttt{F40C}\\
\mSymbol[outlined]{video-camera-front} & \mSymbol[rounded]{video-camera-front} & \mSymbol[sharp]{video-camera-front} & \texttt{\textbackslash mSymbol\{video-camera-front\}} & \texttt{F080}\\
\mSymbol[outlined]{video-camera-front-off} & \mSymbol[rounded]{video-camera-front-off} & \mSymbol[sharp]{video-camera-front-off} & \texttt{\textbackslash mSymbol\{video-camera-front-off\}} & \texttt{F83B}\\
\mSymbol[outlined]{video-chat} & \mSymbol[rounded]{video-chat} & \mSymbol[sharp]{video-chat} & \texttt{\textbackslash mSymbol\{video-chat\}} & \texttt{F8A0}\\
\mSymbol[outlined]{video-file} & \mSymbol[rounded]{video-file} & \mSymbol[sharp]{video-file} & \texttt{\textbackslash mSymbol\{video-file\}} & \texttt{EB87}\\
\mSymbol[outlined]{video-label} & \mSymbol[rounded]{video-label} & \mSymbol[sharp]{video-label} & \texttt{\textbackslash mSymbol\{video-label\}} & \texttt{E071}\\
\mSymbol[outlined]{video-library} & \mSymbol[rounded]{video-library} & \mSymbol[sharp]{video-library} & \texttt{\textbackslash mSymbol\{video-library\}} & \texttt{E04A}\\
\mSymbol[outlined]{video-search} & \mSymbol[rounded]{video-search} & \mSymbol[sharp]{video-search} & \texttt{\textbackslash mSymbol\{video-search\}} & \texttt{EFC6}\\
\mSymbol[outlined]{video-settings} & \mSymbol[rounded]{video-settings} & \mSymbol[sharp]{video-settings} & \texttt{\textbackslash mSymbol\{video-settings\}} & \texttt{EA75}\\
\mSymbol[outlined]{video-stable} & \mSymbol[rounded]{video-stable} & \mSymbol[sharp]{video-stable} & \texttt{\textbackslash mSymbol\{video-stable\}} & \texttt{F081}\\
\mSymbol[outlined]{videocam} & \mSymbol[rounded]{videocam} & \mSymbol[sharp]{videocam} & \texttt{\textbackslash mSymbol\{videocam\}} & \texttt{E04B}\\
\mSymbol[outlined]{videocam-off} & \mSymbol[rounded]{videocam-off} & \mSymbol[sharp]{videocam-off} & \texttt{\textbackslash mSymbol\{videocam-off\}} & \texttt{E04C}\\
\mSymbol[outlined]{videogame-asset} & \mSymbol[rounded]{videogame-asset} & \mSymbol[sharp]{videogame-asset} & \texttt{\textbackslash mSymbol\{videogame-asset\}} & \texttt{E338}\\
\mSymbol[outlined]{videogame-asset-off} & \mSymbol[rounded]{videogame-asset-off} & \mSymbol[sharp]{videogame-asset-off} & \texttt{\textbackslash mSymbol\{videogame-asset-off\}} & \texttt{E500}\\
\mSymbol[outlined]{view-agenda} & \mSymbol[rounded]{view-agenda} & \mSymbol[sharp]{view-agenda} & \texttt{\textbackslash mSymbol\{view-agenda\}} & \texttt{E8E9}\\
\mSymbol[outlined]{view-array} & \mSymbol[rounded]{view-array} & \mSymbol[sharp]{view-array} & \texttt{\textbackslash mSymbol\{view-array\}} & \texttt{E8EA}\\
\mSymbol[outlined]{view-carousel} & \mSymbol[rounded]{view-carousel} & \mSymbol[sharp]{view-carousel} & \texttt{\textbackslash mSymbol\{view-carousel\}} & \texttt{E8EB}\\
\mSymbol[outlined]{view-column} & \mSymbol[rounded]{view-column} & \mSymbol[sharp]{view-column} & \texttt{\textbackslash mSymbol\{view-column\}} & \texttt{E8EC}\\
\mSymbol[outlined]{view-column-2} & \mSymbol[rounded]{view-column-2} & \mSymbol[sharp]{view-column-2} & \texttt{\textbackslash mSymbol\{view-column-2\}} & \texttt{F847}\\
\mSymbol[outlined]{view-comfy} & \mSymbol[rounded]{view-comfy} & \mSymbol[sharp]{view-comfy} & \texttt{\textbackslash mSymbol\{view-comfy\}} & \texttt{E42A}\\
\mSymbol[outlined]{view-comfy-alt} & \mSymbol[rounded]{view-comfy-alt} & \mSymbol[sharp]{view-comfy-alt} & \texttt{\textbackslash mSymbol\{view-comfy-alt\}} & \texttt{EB73}\\
\mSymbol[outlined]{view-compact} & \mSymbol[rounded]{view-compact} & \mSymbol[sharp]{view-compact} & \texttt{\textbackslash mSymbol\{view-compact\}} & \texttt{E42B}\\
\mSymbol[outlined]{view-compact-alt} & \mSymbol[rounded]{view-compact-alt} & \mSymbol[sharp]{view-compact-alt} & \texttt{\textbackslash mSymbol\{view-compact-alt\}} & \texttt{EB74}\\
\mSymbol[outlined]{view-cozy} & \mSymbol[rounded]{view-cozy} & \mSymbol[sharp]{view-cozy} & \texttt{\textbackslash mSymbol\{view-cozy\}} & \texttt{EB75}\\
\mSymbol[outlined]{view-day} & \mSymbol[rounded]{view-day} & \mSymbol[sharp]{view-day} & \texttt{\textbackslash mSymbol\{view-day\}} & \texttt{E8ED}\\
\mSymbol[outlined]{view-headline} & \mSymbol[rounded]{view-headline} & \mSymbol[sharp]{view-headline} & \texttt{\textbackslash mSymbol\{view-headline\}} & \texttt{E8EE}\\
\mSymbol[outlined]{view-in-ar} & \mSymbol[rounded]{view-in-ar} & \mSymbol[sharp]{view-in-ar} & \texttt{\textbackslash mSymbol\{view-in-ar\}} & \texttt{EFC9}\\
\mSymbol[outlined]{view-in-ar-new} & \mSymbol[rounded]{view-in-ar-new} & \mSymbol[sharp]{view-in-ar-new} & \texttt{\textbackslash mSymbol\{view-in-ar-new\}} & \texttt{EFC9}\\
\mSymbol[outlined]{view-in-ar-off} & \mSymbol[rounded]{view-in-ar-off} & \mSymbol[sharp]{view-in-ar-off} & \texttt{\textbackslash mSymbol\{view-in-ar-off\}} & \texttt{F61B}\\
\mSymbol[outlined]{view-kanban} & \mSymbol[rounded]{view-kanban} & \mSymbol[sharp]{view-kanban} & \texttt{\textbackslash mSymbol\{view-kanban\}} & \texttt{EB7F}\\
\mSymbol[outlined]{view-list} & \mSymbol[rounded]{view-list} & \mSymbol[sharp]{view-list} & \texttt{\textbackslash mSymbol\{view-list\}} & \texttt{E8EF}\\
\mSymbol[outlined]{view-module} & \mSymbol[rounded]{view-module} & \mSymbol[sharp]{view-module} & \texttt{\textbackslash mSymbol\{view-module\}} & \texttt{E8F0}\\
\mSymbol[outlined]{view-object-track} & \mSymbol[rounded]{view-object-track} & \mSymbol[sharp]{view-object-track} & \texttt{\textbackslash mSymbol\{view-object-track\}} & \texttt{F432}\\
\mSymbol[outlined]{view-quilt} & \mSymbol[rounded]{view-quilt} & \mSymbol[sharp]{view-quilt} & \texttt{\textbackslash mSymbol\{view-quilt\}} & \texttt{E8F1}\\
\mSymbol[outlined]{view-real-size} & \mSymbol[rounded]{view-real-size} & \mSymbol[sharp]{view-real-size} & \texttt{\textbackslash mSymbol\{view-real-size\}} & \texttt{F4C2}\\
\mSymbol[outlined]{view-sidebar} & \mSymbol[rounded]{view-sidebar} & \mSymbol[sharp]{view-sidebar} & \texttt{\textbackslash mSymbol\{view-sidebar\}} & \texttt{F114}\\
\mSymbol[outlined]{view-stream} & \mSymbol[rounded]{view-stream} & \mSymbol[sharp]{view-stream} & \texttt{\textbackslash mSymbol\{view-stream\}} & \texttt{E8F2}\\
\mSymbol[outlined]{view-timeline} & \mSymbol[rounded]{view-timeline} & \mSymbol[sharp]{view-timeline} & \texttt{\textbackslash mSymbol\{view-timeline\}} & \texttt{EB85}\\
\mSymbol[outlined]{view-week} & \mSymbol[rounded]{view-week} & \mSymbol[sharp]{view-week} & \texttt{\textbackslash mSymbol\{view-week\}} & \texttt{E8F3}\\
\mSymbol[outlined]{vignette} & \mSymbol[rounded]{vignette} & \mSymbol[sharp]{vignette} & \texttt{\textbackslash mSymbol\{vignette\}} & \texttt{E435}\\
\mSymbol[outlined]{villa} & \mSymbol[rounded]{villa} & \mSymbol[sharp]{villa} & \texttt{\textbackslash mSymbol\{villa\}} & \texttt{E586}\\
\mSymbol[outlined]{visibility} & \mSymbol[rounded]{visibility} & \mSymbol[sharp]{visibility} & \texttt{\textbackslash mSymbol\{visibility\}} & \texttt{E8F4}\\
\mSymbol[outlined]{visibility-lock} & \mSymbol[rounded]{visibility-lock} & \mSymbol[sharp]{visibility-lock} & \texttt{\textbackslash mSymbol\{visibility-lock\}} & \texttt{F653}\\
\mSymbol[outlined]{visibility-off} & \mSymbol[rounded]{visibility-off} & \mSymbol[sharp]{visibility-off} & \texttt{\textbackslash mSymbol\{visibility-off\}} & \texttt{E8F5}\\
\mSymbol[outlined]{vital-signs} & \mSymbol[rounded]{vital-signs} & \mSymbol[sharp]{vital-signs} & \texttt{\textbackslash mSymbol\{vital-signs\}} & \texttt{E650}\\
\mSymbol[outlined]{vitals} & \mSymbol[rounded]{vitals} & \mSymbol[sharp]{vitals} & \texttt{\textbackslash mSymbol\{vitals\}} & \texttt{E13B}\\
\mSymbol[outlined]{vo2-max} & \mSymbol[rounded]{vo2-max} & \mSymbol[sharp]{vo2-max} & \texttt{\textbackslash mSymbol\{vo2-max\}} & \texttt{F4AA}\\
\mSymbol[outlined]{voice-chat} & \mSymbol[rounded]{voice-chat} & \mSymbol[sharp]{voice-chat} & \texttt{\textbackslash mSymbol\{voice-chat\}} & \texttt{E62E}\\
\mSymbol[outlined]{voice-over-off} & \mSymbol[rounded]{voice-over-off} & \mSymbol[sharp]{voice-over-off} & \texttt{\textbackslash mSymbol\{voice-over-off\}} & \texttt{E94A}\\
\mSymbol[outlined]{voice-selection} & \mSymbol[rounded]{voice-selection} & \mSymbol[sharp]{voice-selection} & \texttt{\textbackslash mSymbol\{voice-selection\}} & \texttt{F58A}\\
\mSymbol[outlined]{voice-selection-off} & \mSymbol[rounded]{voice-selection-off} & \mSymbol[sharp]{voice-selection-off} & \texttt{\textbackslash mSymbol\{voice-selection-off\}} & \texttt{F42C}\\
\mSymbol[outlined]{voicemail} & \mSymbol[rounded]{voicemail} & \mSymbol[sharp]{voicemail} & \texttt{\textbackslash mSymbol\{voicemail\}} & \texttt{E0D9}\\
\mSymbol[outlined]{volcano} & \mSymbol[rounded]{volcano} & \mSymbol[sharp]{volcano} & \texttt{\textbackslash mSymbol\{volcano\}} & \texttt{EBDA}\\
\mSymbol[outlined]{volume-down} & \mSymbol[rounded]{volume-down} & \mSymbol[sharp]{volume-down} & \texttt{\textbackslash mSymbol\{volume-down\}} & \texttt{E04D}\\
\mSymbol[outlined]{volume-down-alt} & \mSymbol[rounded]{volume-down-alt} & \mSymbol[sharp]{volume-down-alt} & \texttt{\textbackslash mSymbol\{volume-down-alt\}} & \texttt{E79C}\\
\mSymbol[outlined]{volume-mute} & \mSymbol[rounded]{volume-mute} & \mSymbol[sharp]{volume-mute} & \texttt{\textbackslash mSymbol\{volume-mute\}} & \texttt{E04E}\\
\mSymbol[outlined]{volume-off} & \mSymbol[rounded]{volume-off} & \mSymbol[sharp]{volume-off} & \texttt{\textbackslash mSymbol\{volume-off\}} & \texttt{E04F}\\
\mSymbol[outlined]{volume-up} & \mSymbol[rounded]{volume-up} & \mSymbol[sharp]{volume-up} & \texttt{\textbackslash mSymbol\{volume-up\}} & \texttt{E050}\\
\mSymbol[outlined]{volunteer-activism} & \mSymbol[rounded]{volunteer-activism} & \mSymbol[sharp]{volunteer-activism} & \texttt{\textbackslash mSymbol\{volunteer-activism\}} & \texttt{EA70}\\
\mSymbol[outlined]{voting-chip} & \mSymbol[rounded]{voting-chip} & \mSymbol[sharp]{voting-chip} & \texttt{\textbackslash mSymbol\{voting-chip\}} & \texttt{F852}\\
\mSymbol[outlined]{vpn-key} & \mSymbol[rounded]{vpn-key} & \mSymbol[sharp]{vpn-key} & \texttt{\textbackslash mSymbol\{vpn-key\}} & \texttt{E0DA}\\
\mSymbol[outlined]{vpn-key-alert} & \mSymbol[rounded]{vpn-key-alert} & \mSymbol[sharp]{vpn-key-alert} & \texttt{\textbackslash mSymbol\{vpn-key-alert\}} & \texttt{F6CC}\\
\mSymbol[outlined]{vpn-key-off} & \mSymbol[rounded]{vpn-key-off} & \mSymbol[sharp]{vpn-key-off} & \texttt{\textbackslash mSymbol\{vpn-key-off\}} & \texttt{EB7A}\\
\mSymbol[outlined]{vpn-lock} & \mSymbol[rounded]{vpn-lock} & \mSymbol[sharp]{vpn-lock} & \texttt{\textbackslash mSymbol\{vpn-lock\}} & \texttt{E62F}\\
\mSymbol[outlined]{vr180-create2d} & \mSymbol[rounded]{vr180-create2d} & \mSymbol[sharp]{vr180-create2d} & \texttt{\textbackslash mSymbol\{vr180-create2d\}} & \texttt{EFCA}\\
\mSymbol[outlined]{vr180-create2d-off} & \mSymbol[rounded]{vr180-create2d-off} & \mSymbol[sharp]{vr180-create2d-off} & \texttt{\textbackslash mSymbol\{vr180-create2d-off\}} & \texttt{F571}\\
\mSymbol[outlined]{vrpano} & \mSymbol[rounded]{vrpano} & \mSymbol[sharp]{vrpano} & \texttt{\textbackslash mSymbol\{vrpano\}} & \texttt{F082}\\
\mSymbol[outlined]{wall-art} & \mSymbol[rounded]{wall-art} & \mSymbol[sharp]{wall-art} & \texttt{\textbackslash mSymbol\{wall-art\}} & \texttt{EFCB}\\
\mSymbol[outlined]{wall-lamp} & \mSymbol[rounded]{wall-lamp} & \mSymbol[sharp]{wall-lamp} & \texttt{\textbackslash mSymbol\{wall-lamp\}} & \texttt{E2B4}\\
\mSymbol[outlined]{wallet} & \mSymbol[rounded]{wallet} & \mSymbol[sharp]{wallet} & \texttt{\textbackslash mSymbol\{wallet\}} & \texttt{F8FF}\\
\mSymbol[outlined]{wallpaper} & \mSymbol[rounded]{wallpaper} & \mSymbol[sharp]{wallpaper} & \texttt{\textbackslash mSymbol\{wallpaper\}} & \texttt{E1BC}\\
\mSymbol[outlined]{wallpaper-slideshow} & \mSymbol[rounded]{wallpaper-slideshow} & \mSymbol[sharp]{wallpaper-slideshow} & \texttt{\textbackslash mSymbol\{wallpaper-slideshow\}} & \texttt{F672}\\
\mSymbol[outlined]{ward} & \mSymbol[rounded]{ward} & \mSymbol[sharp]{ward} & \texttt{\textbackslash mSymbol\{ward\}} & \texttt{E13C}\\
\mSymbol[outlined]{warehouse} & \mSymbol[rounded]{warehouse} & \mSymbol[sharp]{warehouse} & \texttt{\textbackslash mSymbol\{warehouse\}} & \texttt{EBB8}\\
\mSymbol[outlined]{warning} & \mSymbol[rounded]{warning} & \mSymbol[sharp]{warning} & \texttt{\textbackslash mSymbol\{warning\}} & \texttt{F083}\\
\mSymbol[outlined]{warning-amber} & \mSymbol[rounded]{warning-amber} & \mSymbol[sharp]{warning-amber} & \texttt{\textbackslash mSymbol\{warning-amber\}} & \texttt{F083}\\
\mSymbol[outlined]{warning-off} & \mSymbol[rounded]{warning-off} & \mSymbol[sharp]{warning-off} & \texttt{\textbackslash mSymbol\{warning-off\}} & \texttt{F7AD}\\
\mSymbol[outlined]{wash} & \mSymbol[rounded]{wash} & \mSymbol[sharp]{wash} & \texttt{\textbackslash mSymbol\{wash\}} & \texttt{F1B1}\\
\mSymbol[outlined]{watch} & \mSymbol[rounded]{watch} & \mSymbol[sharp]{watch} & \texttt{\textbackslash mSymbol\{watch\}} & \texttt{E334}\\
\mSymbol[outlined]{watch-button-press} & \mSymbol[rounded]{watch-button-press} & \mSymbol[sharp]{watch-button-press} & \texttt{\textbackslash mSymbol\{watch-button-press\}} & \texttt{F6AA}\\
\mSymbol[outlined]{watch-check} & \mSymbol[rounded]{watch-check} & \mSymbol[sharp]{watch-check} & \texttt{\textbackslash mSymbol\{watch-check\}} & \texttt{F468}\\
\mSymbol[outlined]{watch-later} & \mSymbol[rounded]{watch-later} & \mSymbol[sharp]{watch-later} & \texttt{\textbackslash mSymbol\{watch-later\}} & \texttt{EFD6}\\
\mSymbol[outlined]{watch-off} & \mSymbol[rounded]{watch-off} & \mSymbol[sharp]{watch-off} & \texttt{\textbackslash mSymbol\{watch-off\}} & \texttt{EAE3}\\
\mSymbol[outlined]{watch-screentime} & \mSymbol[rounded]{watch-screentime} & \mSymbol[sharp]{watch-screentime} & \texttt{\textbackslash mSymbol\{watch-screentime\}} & \texttt{F6AE}\\
\mSymbol[outlined]{watch-vibration} & \mSymbol[rounded]{watch-vibration} & \mSymbol[sharp]{watch-vibration} & \texttt{\textbackslash mSymbol\{watch-vibration\}} & \texttt{F467}\\
\mSymbol[outlined]{watch-wake} & \mSymbol[rounded]{watch-wake} & \mSymbol[sharp]{watch-wake} & \texttt{\textbackslash mSymbol\{watch-wake\}} & \texttt{F6A9}\\
\mSymbol[outlined]{water} & \mSymbol[rounded]{water} & \mSymbol[sharp]{water} & \texttt{\textbackslash mSymbol\{water\}} & \texttt{F084}\\
\mSymbol[outlined]{water-bottle} & \mSymbol[rounded]{water-bottle} & \mSymbol[sharp]{water-bottle} & \texttt{\textbackslash mSymbol\{water-bottle\}} & \texttt{F69D}\\
\mSymbol[outlined]{water-bottle-large} & \mSymbol[rounded]{water-bottle-large} & \mSymbol[sharp]{water-bottle-large} & \texttt{\textbackslash mSymbol\{water-bottle-large\}} & \texttt{F69E}\\
\mSymbol[outlined]{water-damage} & \mSymbol[rounded]{water-damage} & \mSymbol[sharp]{water-damage} & \texttt{\textbackslash mSymbol\{water-damage\}} & \texttt{F203}\\
\mSymbol[outlined]{water-do} & \mSymbol[rounded]{water-do} & \mSymbol[sharp]{water-do} & \texttt{\textbackslash mSymbol\{water-do\}} & \texttt{F870}\\
\mSymbol[outlined]{water-drop} & \mSymbol[rounded]{water-drop} & \mSymbol[sharp]{water-drop} & \texttt{\textbackslash mSymbol\{water-drop\}} & \texttt{E798}\\
\mSymbol[outlined]{water-ec} & \mSymbol[rounded]{water-ec} & \mSymbol[sharp]{water-ec} & \texttt{\textbackslash mSymbol\{water-ec\}} & \texttt{F875}\\
\mSymbol[outlined]{water-full} & \mSymbol[rounded]{water-full} & \mSymbol[sharp]{water-full} & \texttt{\textbackslash mSymbol\{water-full\}} & \texttt{F6D6}\\
\mSymbol[outlined]{water-heater} & \mSymbol[rounded]{water-heater} & \mSymbol[sharp]{water-heater} & \texttt{\textbackslash mSymbol\{water-heater\}} & \texttt{E284}\\
\mSymbol[outlined]{water-lock} & \mSymbol[rounded]{water-lock} & \mSymbol[sharp]{water-lock} & \texttt{\textbackslash mSymbol\{water-lock\}} & \texttt{F6AD}\\
\mSymbol[outlined]{water-loss} & \mSymbol[rounded]{water-loss} & \mSymbol[sharp]{water-loss} & \texttt{\textbackslash mSymbol\{water-loss\}} & \texttt{F6D5}\\
\mSymbol[outlined]{water-lux} & \mSymbol[rounded]{water-lux} & \mSymbol[sharp]{water-lux} & \texttt{\textbackslash mSymbol\{water-lux\}} & \texttt{F874}\\
\mSymbol[outlined]{water-medium} & \mSymbol[rounded]{water-medium} & \mSymbol[sharp]{water-medium} & \texttt{\textbackslash mSymbol\{water-medium\}} & \texttt{F6D4}\\
\mSymbol[outlined]{water-orp} & \mSymbol[rounded]{water-orp} & \mSymbol[sharp]{water-orp} & \texttt{\textbackslash mSymbol\{water-orp\}} & \texttt{F878}\\
\mSymbol[outlined]{water-ph} & \mSymbol[rounded]{water-ph} & \mSymbol[sharp]{water-ph} & \texttt{\textbackslash mSymbol\{water-ph\}} & \texttt{F87A}\\
\mSymbol[outlined]{water-pump} & \mSymbol[rounded]{water-pump} & \mSymbol[sharp]{water-pump} & \texttt{\textbackslash mSymbol\{water-pump\}} & \texttt{F5D8}\\
\mSymbol[outlined]{water-voc} & \mSymbol[rounded]{water-voc} & \mSymbol[sharp]{water-voc} & \texttt{\textbackslash mSymbol\{water-voc\}} & \texttt{F87B}\\
\mSymbol[outlined]{waterfall-chart} & \mSymbol[rounded]{waterfall-chart} & \mSymbol[sharp]{waterfall-chart} & \texttt{\textbackslash mSymbol\{waterfall-chart\}} & \texttt{EA00}\\
\mSymbol[outlined]{waves} & \mSymbol[rounded]{waves} & \mSymbol[sharp]{waves} & \texttt{\textbackslash mSymbol\{waves\}} & \texttt{E176}\\
\mSymbol[outlined]{waving-hand} & \mSymbol[rounded]{waving-hand} & \mSymbol[sharp]{waving-hand} & \texttt{\textbackslash mSymbol\{waving-hand\}} & \texttt{E766}\\
\mSymbol[outlined]{wb-auto} & \mSymbol[rounded]{wb-auto} & \mSymbol[sharp]{wb-auto} & \texttt{\textbackslash mSymbol\{wb-auto\}} & \texttt{E42C}\\
\mSymbol[outlined]{wb-cloudy} & \mSymbol[rounded]{wb-cloudy} & \mSymbol[sharp]{wb-cloudy} & \texttt{\textbackslash mSymbol\{wb-cloudy\}} & \texttt{F15C}\\
\mSymbol[outlined]{wb-incandescent} & \mSymbol[rounded]{wb-incandescent} & \mSymbol[sharp]{wb-incandescent} & \texttt{\textbackslash mSymbol\{wb-incandescent\}} & \texttt{E42E}\\
\mSymbol[outlined]{wb-iridescent} & \mSymbol[rounded]{wb-iridescent} & \mSymbol[sharp]{wb-iridescent} & \texttt{\textbackslash mSymbol\{wb-iridescent\}} & \texttt{F07D}\\
\mSymbol[outlined]{wb-shade} & \mSymbol[rounded]{wb-shade} & \mSymbol[sharp]{wb-shade} & \texttt{\textbackslash mSymbol\{wb-shade\}} & \texttt{EA01}\\
\mSymbol[outlined]{wb-sunny} & \mSymbol[rounded]{wb-sunny} & \mSymbol[sharp]{wb-sunny} & \texttt{\textbackslash mSymbol\{wb-sunny\}} & \texttt{E430}\\
\mSymbol[outlined]{wb-twilight} & \mSymbol[rounded]{wb-twilight} & \mSymbol[sharp]{wb-twilight} & \texttt{\textbackslash mSymbol\{wb-twilight\}} & \texttt{E1C6}\\
\mSymbol[outlined]{wc} & \mSymbol[rounded]{wc} & \mSymbol[sharp]{wc} & \texttt{\textbackslash mSymbol\{wc\}} & \texttt{E63D}\\
\mSymbol[outlined]{weather-hail} & \mSymbol[rounded]{weather-hail} & \mSymbol[sharp]{weather-hail} & \texttt{\textbackslash mSymbol\{weather-hail\}} & \texttt{F67F}\\
\mSymbol[outlined]{weather-mix} & \mSymbol[rounded]{weather-mix} & \mSymbol[sharp]{weather-mix} & \texttt{\textbackslash mSymbol\{weather-mix\}} & \texttt{F60B}\\
\mSymbol[outlined]{weather-snowy} & \mSymbol[rounded]{weather-snowy} & \mSymbol[sharp]{weather-snowy} & \texttt{\textbackslash mSymbol\{weather-snowy\}} & \texttt{E2CD}\\
\mSymbol[outlined]{web} & \mSymbol[rounded]{web} & \mSymbol[sharp]{web} & \texttt{\textbackslash mSymbol\{web\}} & \texttt{E051}\\
\mSymbol[outlined]{web-asset} & \mSymbol[rounded]{web-asset} & \mSymbol[sharp]{web-asset} & \texttt{\textbackslash mSymbol\{web-asset\}} & \texttt{E069}\\
\mSymbol[outlined]{web-asset-off} & \mSymbol[rounded]{web-asset-off} & \mSymbol[sharp]{web-asset-off} & \texttt{\textbackslash mSymbol\{web-asset-off\}} & \texttt{EF47}\\
\mSymbol[outlined]{web-stories} & \mSymbol[rounded]{web-stories} & \mSymbol[sharp]{web-stories} & \texttt{\textbackslash mSymbol\{web-stories\}} & \texttt{E595}\\
\mSymbol[outlined]{web-traffic} & \mSymbol[rounded]{web-traffic} & \mSymbol[sharp]{web-traffic} & \texttt{\textbackslash mSymbol\{web-traffic\}} & \texttt{EA03}\\
\mSymbol[outlined]{webhook} & \mSymbol[rounded]{webhook} & \mSymbol[sharp]{webhook} & \texttt{\textbackslash mSymbol\{webhook\}} & \texttt{EB92}\\
\mSymbol[outlined]{weekend} & \mSymbol[rounded]{weekend} & \mSymbol[sharp]{weekend} & \texttt{\textbackslash mSymbol\{weekend\}} & \texttt{E16B}\\
\mSymbol[outlined]{weight} & \mSymbol[rounded]{weight} & \mSymbol[sharp]{weight} & \texttt{\textbackslash mSymbol\{weight\}} & \texttt{E13D}\\
\mSymbol[outlined]{west} & \mSymbol[rounded]{west} & \mSymbol[sharp]{west} & \texttt{\textbackslash mSymbol\{west\}} & \texttt{F1E6}\\
\mSymbol[outlined]{whatshot} & \mSymbol[rounded]{whatshot} & \mSymbol[sharp]{whatshot} & \texttt{\textbackslash mSymbol\{whatshot\}} & \texttt{E80E}\\
\mSymbol[outlined]{wheelchair-pickup} & \mSymbol[rounded]{wheelchair-pickup} & \mSymbol[sharp]{wheelchair-pickup} & \texttt{\textbackslash mSymbol\{wheelchair-pickup\}} & \texttt{F1AB}\\
\mSymbol[outlined]{where-to-vote} & \mSymbol[rounded]{where-to-vote} & \mSymbol[sharp]{where-to-vote} & \texttt{\textbackslash mSymbol\{where-to-vote\}} & \texttt{E177}\\
\mSymbol[outlined]{widgets} & \mSymbol[rounded]{widgets} & \mSymbol[sharp]{widgets} & \texttt{\textbackslash mSymbol\{widgets\}} & \texttt{E1BD}\\
\mSymbol[outlined]{width} & \mSymbol[rounded]{width} & \mSymbol[sharp]{width} & \texttt{\textbackslash mSymbol\{width\}} & \texttt{F730}\\
\mSymbol[outlined]{width-full} & \mSymbol[rounded]{width-full} & \mSymbol[sharp]{width-full} & \texttt{\textbackslash mSymbol\{width-full\}} & \texttt{F8F5}\\
\mSymbol[outlined]{width-normal} & \mSymbol[rounded]{width-normal} & \mSymbol[sharp]{width-normal} & \texttt{\textbackslash mSymbol\{width-normal\}} & \texttt{F8F6}\\
\mSymbol[outlined]{width-wide} & \mSymbol[rounded]{width-wide} & \mSymbol[sharp]{width-wide} & \texttt{\textbackslash mSymbol\{width-wide\}} & \texttt{F8F7}\\
\mSymbol[outlined]{wifi} & \mSymbol[rounded]{wifi} & \mSymbol[sharp]{wifi} & \texttt{\textbackslash mSymbol\{wifi\}} & \texttt{E63E}\\
\mSymbol[outlined]{wifi-1-bar} & \mSymbol[rounded]{wifi-1-bar} & \mSymbol[sharp]{wifi-1-bar} & \texttt{\textbackslash mSymbol\{wifi-1-bar\}} & \texttt{E4CA}\\
\mSymbol[outlined]{wifi-2-bar} & \mSymbol[rounded]{wifi-2-bar} & \mSymbol[sharp]{wifi-2-bar} & \texttt{\textbackslash mSymbol\{wifi-2-bar\}} & \texttt{E4D9}\\
\mSymbol[outlined]{wifi-add} & \mSymbol[rounded]{wifi-add} & \mSymbol[sharp]{wifi-add} & \texttt{\textbackslash mSymbol\{wifi-add\}} & \texttt{F7A8}\\
\mSymbol[outlined]{wifi-calling} & \mSymbol[rounded]{wifi-calling} & \mSymbol[sharp]{wifi-calling} & \texttt{\textbackslash mSymbol\{wifi-calling\}} & \texttt{EF77}\\
\mSymbol[outlined]{wifi-calling-1} & \mSymbol[rounded]{wifi-calling-1} & \mSymbol[sharp]{wifi-calling-1} & \texttt{\textbackslash mSymbol\{wifi-calling-1\}} & \texttt{F0E7}\\
\mSymbol[outlined]{wifi-calling-2} & \mSymbol[rounded]{wifi-calling-2} & \mSymbol[sharp]{wifi-calling-2} & \texttt{\textbackslash mSymbol\{wifi-calling-2\}} & \texttt{F0F6}\\
\mSymbol[outlined]{wifi-calling-3} & \mSymbol[rounded]{wifi-calling-3} & \mSymbol[sharp]{wifi-calling-3} & \texttt{\textbackslash mSymbol\{wifi-calling-3\}} & \texttt{F0E7}\\
\mSymbol[outlined]{wifi-calling-bar-1} & \mSymbol[rounded]{wifi-calling-bar-1} & \mSymbol[sharp]{wifi-calling-bar-1} & \texttt{\textbackslash mSymbol\{wifi-calling-bar-1\}} & \texttt{F44C}\\
\mSymbol[outlined]{wifi-calling-bar-2} & \mSymbol[rounded]{wifi-calling-bar-2} & \mSymbol[sharp]{wifi-calling-bar-2} & \texttt{\textbackslash mSymbol\{wifi-calling-bar-2\}} & \texttt{F44B}\\
\mSymbol[outlined]{wifi-calling-bar-3} & \mSymbol[rounded]{wifi-calling-bar-3} & \mSymbol[sharp]{wifi-calling-bar-3} & \texttt{\textbackslash mSymbol\{wifi-calling-bar-3\}} & \texttt{F44A}\\
\mSymbol[outlined]{wifi-channel} & \mSymbol[rounded]{wifi-channel} & \mSymbol[sharp]{wifi-channel} & \texttt{\textbackslash mSymbol\{wifi-channel\}} & \texttt{EB6A}\\
\mSymbol[outlined]{wifi-find} & \mSymbol[rounded]{wifi-find} & \mSymbol[sharp]{wifi-find} & \texttt{\textbackslash mSymbol\{wifi-find\}} & \texttt{EB31}\\
\mSymbol[outlined]{wifi-home} & \mSymbol[rounded]{wifi-home} & \mSymbol[sharp]{wifi-home} & \texttt{\textbackslash mSymbol\{wifi-home\}} & \texttt{F671}\\
\mSymbol[outlined]{wifi-lock} & \mSymbol[rounded]{wifi-lock} & \mSymbol[sharp]{wifi-lock} & \texttt{\textbackslash mSymbol\{wifi-lock\}} & \texttt{E1E1}\\
\mSymbol[outlined]{wifi-notification} & \mSymbol[rounded]{wifi-notification} & \mSymbol[sharp]{wifi-notification} & \texttt{\textbackslash mSymbol\{wifi-notification\}} & \texttt{F670}\\
\mSymbol[outlined]{wifi-off} & \mSymbol[rounded]{wifi-off} & \mSymbol[sharp]{wifi-off} & \texttt{\textbackslash mSymbol\{wifi-off\}} & \texttt{E648}\\
\mSymbol[outlined]{wifi-password} & \mSymbol[rounded]{wifi-password} & \mSymbol[sharp]{wifi-password} & \texttt{\textbackslash mSymbol\{wifi-password\}} & \texttt{EB6B}\\
\mSymbol[outlined]{wifi-protected-setup} & \mSymbol[rounded]{wifi-protected-setup} & \mSymbol[sharp]{wifi-protected-setup} & \texttt{\textbackslash mSymbol\{wifi-protected-setup\}} & \texttt{F0FC}\\
\mSymbol[outlined]{wifi-proxy} & \mSymbol[rounded]{wifi-proxy} & \mSymbol[sharp]{wifi-proxy} & \texttt{\textbackslash mSymbol\{wifi-proxy\}} & \texttt{F7A7}\\
\mSymbol[outlined]{wifi-tethering} & \mSymbol[rounded]{wifi-tethering} & \mSymbol[sharp]{wifi-tethering} & \texttt{\textbackslash mSymbol\{wifi-tethering\}} & \texttt{E1E2}\\
\mSymbol[outlined]{wifi-tethering-error} & \mSymbol[rounded]{wifi-tethering-error} & \mSymbol[sharp]{wifi-tethering-error} & \texttt{\textbackslash mSymbol\{wifi-tethering-error\}} & \texttt{EAD9}\\
\mSymbol[outlined]{wifi-tethering-off} & \mSymbol[rounded]{wifi-tethering-off} & \mSymbol[sharp]{wifi-tethering-off} & \texttt{\textbackslash mSymbol\{wifi-tethering-off\}} & \texttt{F087}\\
\mSymbol[outlined]{wind-power} & \mSymbol[rounded]{wind-power} & \mSymbol[sharp]{wind-power} & \texttt{\textbackslash mSymbol\{wind-power\}} & \texttt{EC0C}\\
\mSymbol[outlined]{window} & \mSymbol[rounded]{window} & \mSymbol[sharp]{window} & \texttt{\textbackslash mSymbol\{window\}} & \texttt{F088}\\
\mSymbol[outlined]{window-closed} & \mSymbol[rounded]{window-closed} & \mSymbol[sharp]{window-closed} & \texttt{\textbackslash mSymbol\{window-closed\}} & \texttt{E77E}\\
\mSymbol[outlined]{window-open} & \mSymbol[rounded]{window-open} & \mSymbol[sharp]{window-open} & \texttt{\textbackslash mSymbol\{window-open\}} & \texttt{E78C}\\
\mSymbol[outlined]{window-sensor} & \mSymbol[rounded]{window-sensor} & \mSymbol[sharp]{window-sensor} & \texttt{\textbackslash mSymbol\{window-sensor\}} & \texttt{E2BB}\\
\mSymbol[outlined]{wine-bar} & \mSymbol[rounded]{wine-bar} & \mSymbol[sharp]{wine-bar} & \texttt{\textbackslash mSymbol\{wine-bar\}} & \texttt{F1E8}\\
\mSymbol[outlined]{woman} & \mSymbol[rounded]{woman} & \mSymbol[sharp]{woman} & \texttt{\textbackslash mSymbol\{woman\}} & \texttt{E13E}\\
\mSymbol[outlined]{woman-2} & \mSymbol[rounded]{woman-2} & \mSymbol[sharp]{woman-2} & \texttt{\textbackslash mSymbol\{woman-2\}} & \texttt{F8E7}\\
\mSymbol[outlined]{work} & \mSymbol[rounded]{work} & \mSymbol[sharp]{work} & \texttt{\textbackslash mSymbol\{work\}} & \texttt{E943}\\
\mSymbol[outlined]{work-alert} & \mSymbol[rounded]{work-alert} & \mSymbol[sharp]{work-alert} & \texttt{\textbackslash mSymbol\{work-alert\}} & \texttt{F5F7}\\
\mSymbol[outlined]{work-history} & \mSymbol[rounded]{work-history} & \mSymbol[sharp]{work-history} & \texttt{\textbackslash mSymbol\{work-history\}} & \texttt{EC09}\\
\mSymbol[outlined]{work-off} & \mSymbol[rounded]{work-off} & \mSymbol[sharp]{work-off} & \texttt{\textbackslash mSymbol\{work-off\}} & \texttt{E942}\\
\mSymbol[outlined]{work-outline} & \mSymbol[rounded]{work-outline} & \mSymbol[sharp]{work-outline} & \texttt{\textbackslash mSymbol\{work-outline\}} & \texttt{E943}\\
\mSymbol[outlined]{work-update} & \mSymbol[rounded]{work-update} & \mSymbol[sharp]{work-update} & \texttt{\textbackslash mSymbol\{work-update\}} & \texttt{F5F8}\\
\mSymbol[outlined]{workflow} & \mSymbol[rounded]{workflow} & \mSymbol[sharp]{workflow} & \texttt{\textbackslash mSymbol\{workflow\}} & \texttt{EA04}\\
\mSymbol[outlined]{workspace-premium} & \mSymbol[rounded]{workspace-premium} & \mSymbol[sharp]{workspace-premium} & \texttt{\textbackslash mSymbol\{workspace-premium\}} & \texttt{E7AF}\\
\mSymbol[outlined]{workspaces} & \mSymbol[rounded]{workspaces} & \mSymbol[sharp]{workspaces} & \texttt{\textbackslash mSymbol\{workspaces\}} & \texttt{EA0F}\\
\mSymbol[outlined]{workspaces-outline} & \mSymbol[rounded]{workspaces-outline} & \mSymbol[sharp]{workspaces-outline} & \texttt{\textbackslash mSymbol\{workspaces-outline\}} & \texttt{EA0F}\\
\mSymbol[outlined]{wounds-injuries} & \mSymbol[rounded]{wounds-injuries} & \mSymbol[sharp]{wounds-injuries} & \texttt{\textbackslash mSymbol\{wounds-injuries\}} & \texttt{E13F}\\
\mSymbol[outlined]{wrap-text} & \mSymbol[rounded]{wrap-text} & \mSymbol[sharp]{wrap-text} & \texttt{\textbackslash mSymbol\{wrap-text\}} & \texttt{E25B}\\
\mSymbol[outlined]{wrist} & \mSymbol[rounded]{wrist} & \mSymbol[sharp]{wrist} & \texttt{\textbackslash mSymbol\{wrist\}} & \texttt{F69C}\\
\mSymbol[outlined]{wrong-location} & \mSymbol[rounded]{wrong-location} & \mSymbol[sharp]{wrong-location} & \texttt{\textbackslash mSymbol\{wrong-location\}} & \texttt{EF78}\\
\mSymbol[outlined]{wysiwyg} & \mSymbol[rounded]{wysiwyg} & \mSymbol[sharp]{wysiwyg} & \texttt{\textbackslash mSymbol\{wysiwyg\}} & \texttt{F1C3}\\
\mSymbol[outlined]{yard} & \mSymbol[rounded]{yard} & \mSymbol[sharp]{yard} & \texttt{\textbackslash mSymbol\{yard\}} & \texttt{F089}\\
\mSymbol[outlined]{your-trips} & \mSymbol[rounded]{your-trips} & \mSymbol[sharp]{your-trips} & \texttt{\textbackslash mSymbol\{your-trips\}} & \texttt{EB2B}\\
\mSymbol[outlined]{youtube-activity} & \mSymbol[rounded]{youtube-activity} & \mSymbol[sharp]{youtube-activity} & \texttt{\textbackslash mSymbol\{youtube-activity\}} & \texttt{F85A}\\
\mSymbol[outlined]{youtube-searched-for} & \mSymbol[rounded]{youtube-searched-for} & \mSymbol[sharp]{youtube-searched-for} & \texttt{\textbackslash mSymbol\{youtube-searched-for\}} & \texttt{E8FA}\\
\mSymbol[outlined]{zone-person-alert} & \mSymbol[rounded]{zone-person-alert} & \mSymbol[sharp]{zone-person-alert} & \texttt{\textbackslash mSymbol\{zone-person-alert\}} & \texttt{E781}\\
\mSymbol[outlined]{zone-person-idle} & \mSymbol[rounded]{zone-person-idle} & \mSymbol[sharp]{zone-person-idle} & \texttt{\textbackslash mSymbol\{zone-person-idle\}} & \texttt{E77A}\\
\mSymbol[outlined]{zone-person-urgent} & \mSymbol[rounded]{zone-person-urgent} & \mSymbol[sharp]{zone-person-urgent} & \texttt{\textbackslash mSymbol\{zone-person-urgent\}} & \texttt{E788}\\
\mSymbol[outlined]{zoom-in} & \mSymbol[rounded]{zoom-in} & \mSymbol[sharp]{zoom-in} & \texttt{\textbackslash mSymbol\{zoom-in\}} & \texttt{E8FF}\\
\mSymbol[outlined]{zoom-in-map} & \mSymbol[rounded]{zoom-in-map} & \mSymbol[sharp]{zoom-in-map} & \texttt{\textbackslash mSymbol\{zoom-in-map\}} & \texttt{EB2D}\\
\mSymbol[outlined]{zoom-out} & \mSymbol[rounded]{zoom-out} & \mSymbol[sharp]{zoom-out} & \texttt{\textbackslash mSymbol\{zoom-out\}} & \texttt{E900}\\
\mSymbol[outlined]{zoom-out-map} & \mSymbol[rounded]{zoom-out-map} & \mSymbol[sharp]{zoom-out-map} & \texttt{\textbackslash mSymbol\{zoom-out-map\}} & \texttt{E56B}\\

\end{longtable}



\end{document}
